\begin{abstract}
The Groth-Sahai proof system, introduced by Groth and Sahai at Eurocrypt 2007,
  provides a proof system for satisfiability of equations over the integers modulo a prime $q$ and over bilinear groups.
  This language suffices to capture almost all of the 
  statements which appear 
  in practice when designing cryptographic schemes over bilinear groups.  
Although Groth-Sahai proofs are quite efficient, proving satisfiability of $m$ equations in $n$ variables requires 
sending the solutions, requiring $\Theta(n)$ elements of a bilinear group, and a proof that the solutions satisfy the equation, requiring $\Theta(m)$ group elements. Although linear in both $m$ and $n$, the constants are on the order of $\sim 10$ group elements, that is 10*64 bytes = 640 bytes which limits $m+n$ to be less than $1600$ whenever we want the proof to less that $1$ Megabyte.

In this thesis we show that \textbf{all linear equations} and \textbf{all quadratic equations over $\Z_q$} admit an aggregated proof of size $\Theta(1)$. We also show that \textbf{all set-membership proofs} over $\Z_q$, that is a proof that $x=s$ for some $s\in S$, admit aggregated proofs of size $\Theta(\log \ell)$. We show that this results can be extended to linear equations over $\GG_1$ and/or $\GG_2$ and to set-membership proofs over $\GG_1$ or $\GG_2$ meeting some restrictions. We use this results to improve the efficiency of several protocols. 
\end{abstract}
