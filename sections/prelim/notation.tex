Let $\advA$ a Turing Machine. We denote by $x:=\advA(y)$ the asignment of $x$ to the output of $\advA$ when run on input $y$. Given a distribution $\dist$ we write $x\gets\dist$ when $x$ is sampled following distribution $\dist$, and given a set $S$ we denote $x\gets S$ when $x$ is sampled uniformly from the set $S$. We denote by $x\gets\advA(y)$ the assignment of $x$ to the output of $\advA$ when run on input $y$ and random coins $r\gets\bits^\ell$, for $\ell$ long enough, which can be equivalently written as $x:=\advA(y;r)$. 

We say that a function $f:\mathbb{N}\to\mathbb{R}$ is negligible if for any $c\in\mathbb{N}$ there exists an integer $n_c\in\mathbb{N}$ such that for any $n> n_c$, $f(n)<1/n^c$. We write $f(n)=\negl(n)$ as shorthand for ``$f$ is negligible'' and we write $f(n)\approx g(n)$ when $|f(n)-g(n)|=\negl(n)$. We say that a function $f:\mathbb{N}\to\mathbb{R}$ is polynomial if there exists $c\in\mathbb{N},n_c\in\mathbb{N}$ such that for any $n\geq n_c$, $f(n)\leq n^c$. We write $f(n)=\poly(n)$ as a shorthand for ``$f$ is polynomial''.

Cryptographic schemes are constituted by many algorithms and among them there is usually a key generation algorithm, which receives the security parameter and return a set of keys. In all the schemes used in this work the security parameter is used to choose a (bilinear) group of size polynomially related to the security parameter, and then the security parameter is never used again. For this reason the key generation algorithms used in this work will receive the group description instead of the security parameter.
