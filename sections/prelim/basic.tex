Perhaps the most groundbreaking achievement in modern Cryptography was the work of Diffie and Hellman \cite{DifHel76} where they introduce \emph{Public Key Cryptography}. Diffie and Hellman conceived systems where each entity publish a \emph{public key} $\mathcal{X}$ while keeping her \emph{secret key} $x$.  It should be computationally infeasible to compute the secret key from the public, which was supposed to be true when $x$ is the \emph{discrete logarithm} of $\mathcal{X}$ on a \emph{Cyclic Abelian Group}.
%\footnote{A couple years after Diffie and Hellman seminal paper,  Rivest, Shamir, and Adlelman came up with the first realization of a public key cryptosystem, the RSA cryptosystem. The RSA cryptosystem was build on top of the hardness of factorizing big numbers, and thus it rely on \emph{composite order} groups with unknown factorization. Although such groups have many interesting applications, this work only focus on prime order groups and we will not refer to composite order groups anymore.{\color{red} MmMmMmMm ni idea si esto esta del todo bien}}
\begin{definition}[Abelian Group]\footnote{Historically, the discrete logarithm problem was defined in multiplicative groups. However in this work we will be using additive notation to avoid tangled expressions in the exponent.} 
An Abelian group is a set $\GG$ together with a map $+:\GG\times\GG\to\GG$ (written in infix notation) and  it holds that
\begin{description}
\item[Associativity]
For all $\mathcal{X}, \mathcal{Y}$, and $\mathcal{Z}$ in $\GG$, $(\mathcal{X} + \mathcal{Y}) + \mathcal{Z}= \mathcal{X} + (\mathcal{Y} + \mathcal{Z})$ holds.
\item[Identity element]
There exists an element $0$ in $\GG$, such that for all $\mathcal{X}$ in $\GG$, $0 + \mathcal{X} = \mathcal{X} + 0 = \mathcal{X}$ holds.
\item[Inverse element]
For each $\mathcal{X}$ in $\GG$, there exists an element $-\mathcal{X}$ in $\GG$ such that $\mathcal{X}+ (-\mathcal{X}) = (-\mathcal{X}) +\mathcal{X} = 0$.
\item[Commutativity]
For all $\mathcal{X}, \mathcal{Y}$ in $\GG$, $\mathcal{X} + \mathcal{Y}=\mathcal{Y}+\mathcal{X}$.
\end{description}
We say that $\GG$ is cyclic if there exists an element $\mathcal{P}\in\GG$, a generator of $\GG$, such that $\GG=\{\mathcal{P}, 2\mathcal{P}, 3\mathcal{P},\ldots\}$. We say that $|\GG|$ is the order of $\GG$.

We denote by $\G(1^\lambda)$ a randomized algorithm which on input the security parameter $\lambda$ outputs $gk:=(\GG,\mathcal{P},q)$, $q=|\GG|$, the description of a cyclic group of order $q$.
\end{definition}

\begin{definition}[Discrete Logarithm Assumption (DL)]
We say that the discrete logarithm assumption holds relative to $\G$ if for any adversary $\advA$
$$
\Pr[gk\gets\G(1^\lambda); x\gets\Z_q; \mathcal{X}:=x\mathcal{P}:\advA(gk,\mathcal{X})=x]\approx 0.
$$
\end{definition}
 
Diffie and Hellman also introduced a novel \emph{Key-Exchange} protocol, which was later known as the \emph{Diffie-Hellman Key-Exchange}, based on the following assumption

\begin{definition}[Computational Diffie-Hellman Assumption (CDH)]
We say that the computational Diffie-Hellman assumption holds relative to $\G$ if for any adversary $\advA$
$$
\Pr[gk\gets\G(1^\lambda); x,y\gets\Z_q; \mathcal{X}:=x\mathcal{P};\mathcal{Y}:=y\mathcal{P}:\advA(gk,\mathcal{X},\mathcal{Y})=xy\mathcal{P}]\approx 0.
$$
\end{definition}

The Diffie-Hellman key-exchange allows two parties $A$, in possession of random secret $x\in\Z_q$ and $\mathcal{Y}=y\mathcal{P}$, and $B$, in possession of random secret $y\in\Z_q$ and $\mathcal{X}=x\mathcal{P}$, to compute the shared secret key $\mathcal{Z}=x\mathcal{Y}=y\mathcal{X}=xy\mathcal{P}$. The computational Diffie-Hellman assumption says that the only way to compute the shared secret key is to know one of the secrets. The \emph{Decisional Diffie-Hellman} assumptions goes a step beyond and says that the shared key ``looks random'' to any other than $A$ and $B$

\begin{definition}[Decisional Diffie-Hellman Assumption (DDH)]
We say that the decisional Diffie-Hellman assumption holds relative to $\G$ if for any adversary $\advA$
$$
\Pr\left[\begin{array}{l}
gk\gets\G(1^\lambda); x,y,z\gets\Z_q,b\gets\bits;\mathcal{X}:=x\mathcal{P};\mathcal{Y}:=y\mathcal{P};\\
\mathcal{Z}:=(bxy+(1-b)z)\mathcal{P}:\advA(gk,\mathcal{X},\mathcal{Y},\mathcal{Z})=b
\end{array}\right]\approx 1/2
$$
\end{definition}

We finish this section introducing \emph{Public-Key Encryption} and then ElGamal encryption scheme as an instantiation based on the DDH assumption.

A public-key encryption scheme is a tuple of 3 algorithms $(\KG,\Enc,\Dec)$. $\G$ is a randomized algorithm which on input a group key $gk$ generates a public/secret key pair. $\Enc$ is a randomized algorithm which on input the public key and a \emph{plaintext}, which is an element from the set $\mathcal{M}_{gk}$, returns a \emph{ciphertext}, which is an element from the set $\mathcal{C}_{gk}$. $\Dec$ is a deterministic algorithm which on input a ciphertext and the secret key returns a plaintext. It is required that for every pair $(pk,sk)$ output by $\G$ and any $m\in\mathcal{M}_{gk}$, $\Dec_{sk}(\Enc_{pk}(m))=m$.

ElGamal introduced the first \emph{Semantically Secure Encryption Scheme} based on the DDH assumption \cite{ElGamal85}. The idea was simple and clean: encrypt a message $\mathcal{M}$ under public key $pk:=\mathcal{X}$ picking $r\gets\Z_q$ and computing the ciphertext $\Enc_{pk}(\mathcal{M};r):=(\mathcal{C}_1,\mathcal{C}_2)=(r\mathcal{P},\mathcal{M}+r\mathcal{X})$; and decrypt a ciphertext using the secret key $x$ and computing $\Dec_{sk}(\mathcal{C}_1,\mathcal{C}_2)=\mathcal{C}_2-x\mathcal{C}_1$. It follows that $(\mathcal{C}_1,\mathcal{C}_2)$ hides $\mathcal{M}$ since $r\mathcal{X}$, by the DDH assumption, looks like fresh random value, independent of $\mathcal{C}_1$, making $\mathcal{M}+r\mathcal{X}$ independent of $\mathcal{M}$. Formally, it can be shown that ElGamal cryptosystem is \emph{Indistinguishable under Chosen Plaintext Attacks} (also called semantically secure and IND-CPA for short).

\begin{definition}[IND-CPA \cite{GolMic84}]
We say that $(\KG,\Enc,\Dec)$ is IND-CPA secure if for any $\advA_1,\advA_2$
$$
\Pr\left[\begin{array}{l}
gk\gets\G;(pk,sk)\gets\KG(gk);b\gets\bits;\\
(m_0,m_1)\gets\advA_1(gk,pk);
c_b\gets\Enc_{pk}(m_b):\advA_2(gk,pk,c_b)=b
\end{array}\right]\approx 1/2.
$$
\end{definition}
 
