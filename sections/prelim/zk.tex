Zero-Knowledge Proofs are proofs that reveal nothing beyond their validity. Since their introduction by Goldreich et al.~\cite{GolMicRac89}, Zero-Knowledge proofs have played a central role in cryptography and complexity theory from both the theoretical side --they have been the inspiration of \emph{Probabilistic Checkable Proofs} and the groundbreaking results on \emph{Hardness of Approximation}-- and the practical side --applications ranges from \emph{Multi-Party Computation} to \emph{Electronic voting} and \emph{E-commerce}.

In a Zero-Knowledge proof a \emph{prover} $\algP$ in possession of a secret $w$ wants to convince a \emph{verifier} $\algV$ that some statement $x$ is true, namely that $x$ belongs to some language $\Lang$. To do so the prover and the verifier engage on an \emph{interactive protocol}: the prover starts with a message $a_1$, the verifier answers with $b_1$, and so on. At the end the verifier outputs a bit $b\in\bits$ indicating whether it rejects or accepts the proofs.

There are two basic requirements for a Zero-Knowledge proof: \emph{Completeness}, which says that the prover should be successful when convincing the verifier about a true statement, and \emph{Soundness}, which says that the verifier rejects false statements with \emph{high probability}. There is also a third requirement which  gives the name to Zero-Knowledge proofs and it aims to require that the verifier does not learn nothing beyond the fact that $x$ is true. This is done by requiring the existence of an efficient algorithm $\algS$, the simulator, which, for any true statement, it is able to construct a \emph{transcript} of the interactive execution of $\algP$ and $\algV$ even without knowing any secret $w$.
A weaker variant of a Zero-Knowledge proof is a \emph{Witness-Indistinguishable Proof}, this proofs are not necessarily simulatable and only guarantee to reveal the same information when using two different secrets $w,w'$. 

The focus of this work are \emph{Non-Interactive Zero-Knowledge Proofs} (NIZK), where the prover sends a single message, the \emph{proof}, to the verifier. Since their introduction by Blum et al.~\cite{STOC:BluFelMic88}, it was known that a ``pre-shared'' information, known as the \emph{Common Reference String} (CRS), allows to construct NIZK proof systems (later it was shown a necessary condition if the statement is not trivial \cite{JC:GolOre94}). Thereby, the simulator is allowed to simulate the CRS and thus is able to compute \emph{trapdoors} associated to it. The knowledge of such trapdoors enhance the possibilities of $\algS$ to successfully simulate proofs.

Syntactically, a NIZK proof system consists of three probabilistic polynomial time algorithms: a CRS
generation algorithm $\algK$, a prover $\algP$, and a verifier $\algV$.
The CRS generation algorithm takes a group description $gk$ as input and produces a CRS $\sigma$ (which we assume includes $gk$). The prover takes as input $(\sigma, x, w)$ and produces a proof $\pi$. The verifier takes as input $(\sigma, x, \pi)$ and outputs 1 if the proof is acceptable and 0 if rejecting the proof.

In this work we will consider two particular cases of NIZK: \emph{Composable} NIZK \cite{EC:GroSah08} and  \emph{Quasi-Adaptive} NIZK \cite{AC:JutRoy13}. Both deals with the case of \emph{CRS dependent languages}, say the language language is parameterized by some values which are (randomly) sampled within the CRS. For example, the CRS might contain the group key $gk$ defining a bilinear group and the language might be some set of satisfiable equations over that group. Quasi-Adaptive NIZK goes further and the language might also depend on group constants defined in the CRS.
 
\subsection{Composable Witness-Indistinguishability and Zero-Knowledge}
The following definitions are taken from \cite{SIAMJC:GroSah12}.

\begin{definition}[Group dependent languages] Let $\mathcal{R}$ be an efficiently computable ternary relation.
For triplets $(gk, x, w)\in \mathcal{R}$ we call $gk$ the group key, $x$ the statement, and $w$ the witness.
Given some $gk$, we let $\Lang$ be the language consisting of statements $x$ that have a
witness $w$ so $(gk, x, w) \in \mathcal{R}$. For a relation that ignores $gk$ this is, of course, the
standard definition of an NP-language. We will be more interested in the case where
$gk$ describes a bilinear group, though.
\end{definition}

\begin{definition}[Composable NIZK proof system] We say that a non-interactive proof system $(\algK,\algP,\algV)$ is a composable NIZK proof system with respect to $\G_a$ if
\begin{description}
\item[Perfect Completeness:] For any $x,w$
$$
\Pr\left[
\begin{array}{l}
gk\gets\G_a(1^\lambda);\sigma\gets\algK(gk);\pi\gets\algP(\sigma,x,w):\\
V(\sigma,x,\pi)=1 \text{ if } (gk,x,w)\in\mathcal{R}
\end{array}
\right]=1.
$$
\item[Perfect Soundness:] For any adversary $\advA$
$$
\Pr\left[
\begin{array}{l}
gk\gets\G_a(1^\lambda);\sigma\gets\algK(gk);(x,\pi)\gets\advA(gk,\sigma):\\
V(\sigma,x,\pi)=0 \text{ if } x\notin\Lang
\end{array}
\right]=1.
$$
\item[Composable Zero-Knowledge:] There exist efficient algorithms $\algS_1,\algS_2$ such that for any adversaries $\advA_1,\advA_2$
\begin{align*}
&\Pr\left[
gk\gets\G_a(1^\lambda);\sigma\gets\algK(gk):
\advA_1(gk,\sigma)=1
\right]\approx\\
&\Pr\left[
gk\gets\G_a(1^\lambda);(\sigma,\tau)\gets\algS_1(gk):
\advA_1(gk,\sigma)=1
\right]
\end{align*}
and
\begin{align*}
&\Pr\left[
\begin{array}{l}
gk\gets\G_a(1^\lambda);(\sigma,\tau)\gets\algS_1(gk);(x,w)\gets\advA_2(gk,\sigma,\tau);\\
\pi\gets\algP(\sigma,x,w):\advA_2(\pi)=1\text{ if }(gk,x,w)\in\mathcal{R}
\end{array}
\right]\approx\\
&\Pr\left[
\begin{array}{l}
gk\gets\G_a(1^\lambda);(\sigma,\tau)\gets\algS_1(gk);(x,w)\gets\advA_2(gk,\sigma,\tau);\\
\pi\gets\algS_2(\sigma,\tau,x):\advA_2(\pi)=1\text{ if }(gk,x,w)\in\mathcal{R}
\end{array}
\right].
\end{align*}
\end{description}
\end{definition}

\begin{definition}[Composable NIWI proof system] We say that a non-interactive proof system $(\algK,\algP,\algV)$ is a composable NIWI proof system with respect to $\G_a$ if it have Perfect Completeness and Soundness as defined above and also
\begin{description}
\item[Composable Witness-Indistinguishability:] There exists and efficient algorithm $\algS_1$ such that for any adversaries $\advA_1,\advA_2$
\begin{align*}
&\Pr\left[
gk\gets\G_a(1^\lambda);\sigma\gets\algK(gk):
\advA_1(gk,\sigma)=1
\right]\approx\\
&\Pr\left[
gk\gets\G_a(1^\lambda);(\sigma,\tau)\gets\algS_1(gk):
\advA_1(gk,\sigma)=1
\right]
\end{align*}
and
\begin{align*}
&\Pr\left[
\begin{array}{l}
gk\gets\G_a(1^\lambda);(\sigma,\tau)\gets\algS_1(gk);(x,w,w')\gets\advA_2(gk,\sigma,\tau);\\
\pi\gets\algP(\sigma,x,w):\advA_2(\pi)=1\text{ if }(gk,x,w),(gk,x,w')\in\mathcal{R}
\end{array}
\right]\approx\\
&\Pr\left[
\begin{array}{l}
gk\gets\G_a(1^\lambda);(\sigma,\tau)\gets\algS_1(gk);(x,w,w')\gets\advA_2(gk,\sigma,\tau);\\
\pi\gets\algP(\sigma,x,w'):\advA_2(\pi)=1\text{ if }(gk,x,w),(gk,x,w')\in\mathcal{R}
\end{array}
\right].
\end{align*}
\end{description}
\end{definition}

\subsection{Quasi-Adaptive Non-Interactive Zero-Knowledge Proofs}
A Quasi-Adaptive NIZK proof system \cite{AC:JutRoy13} enables
to prove membership in a language defined by a relation $\R_\rho$, which in turn is completely determined by some parameter
$\rho$ sampled from a distribution $\dist_\gk$.
We say that $\dist_\gk$ is \emph{witness samplable} if there exists an efficient
algorithm that samples $(\rho,\omega)$ from a distribution $\dist_\gk^{\mathsf{par}}$ such that $\rho$ is distributed according to $\dist_\gk$, and membership of $\rho$
in the \emph{parameter language} $\Lang_\mathsf{par}$ can be efficiently verified with $\omega$.
While the Common Reference String can be set based on $\rho$, the zero-knowledge simulator is required to be a single probabilistic polynomial time
algorithm that works for the whole collection of relations $\R_\gk:=\{\R_\rho\}_{\rho \in \mathrm{sup}(\dist_\gk)}$. 

\begin{definition}[Quasi-Adaptive NIZK (QA-NIZK) proof system]
A non-interactive proof system  $(\algK,\algP,\algV)$ is called a QA-NIZK proof system with respect to $\G_a$ for witness-relations
$\R_\gk = \{\R_\rho\}_{\rho \in \mathrm{sup}(\dist_\gk)}$,
with parameters sampled from a distribution $\dist_\gk$ over associated parameter language
$\Lang_\mathsf{par}$, if there exists a probabilistic polynomial time simulator $(\algS_1, \algS_2)$,
such that for all non-uniform PPT adversaries $\advA_1$, $\advA_2$, $\advA_3$ we have:

\begin{description}
\item[Quasi-Adaptive Completeness:]
$$\Pr\begin{bmatrix*}[l]
    \gk \gets \G_a(1^\lambda);
    \rho \gets \dist_\gk;
    \psi \gets \algK(\gk, \rho);
    (x, w) \gets \advA_1(\gk, \psi);\\
    \pi \gets \algP(\psi, x, w) :
        \algV(\psi, x, \pi) = 1 \text{ if } \R_\rho(x, w)
\end{bmatrix*} = 1.$$
\item[Computational Quasi-Adaptive Soundness:]
$$\Pr\begin{bmatrix*}[l]
    \gk \gets \G_a(1^\lambda);
    \rho \gets \dist_\gk;
    \psi \gets \algK(\gk, \rho); \\
    (x, \pi) \gets \advA_2(\gk, \psi) :
        \algV(\psi, x, \pi) = 1 \text{ and } \neg (\exists w : \R_\rho(x, w))
\end{bmatrix*} \approx 0.$$ 
\item[Perfect Quasi-Adaptive Zero-Knowledge:]
\begin{eqnarray*}
\Pr[
    \gk \gets \G_a(1^\lambda);
    \rho \gets \dist_\gk;
    \psi \gets \algK(\gk, \rho):
        \advA_3^{\algP(\psi,\cdot, \cdot)}(\gk,\psi) = 1]
=\;\;\;\\
\Pr[
    \gk \gets \G_a(1^\lambda);
    \rho \gets \dist_\gk;
    (\psi,\tau) \gets \algS_1(\gk, \rho):
        \advA_3^{\algS(\psi,\tau,\cdot,\cdot)}(\gk,\psi)=1]
\end{eqnarray*}
where
\begin{itemize}
\item $\algP(\psi, \cdot, \cdot)$ emulates the actual prover. It takes input $(x,w)$ and outputs a 
proof $\pi$ if $(x,w)\in\R_\rho$. Otherwise, it outputs $\perp$.
\item $\algS(\psi,\tau, \cdot, \cdot)$ is an oracle that takes input $(x,w)$. It outputs a simulated proof
$\algS_2(\psi,\tau, x)$ if $(x, w) \in \R_\rho$ and $\perp$ if $(x, w) \notin \R_\rho$.
\end{itemize}
\end{description}
Note that $\psi$ is the CRS in the above definitions.
We assume that $\psi$ contains an encoding of $\rho$, which is thus available to $\algV$.
\end{definition}

For witness samplable distributions, a stronger notion of soundness, where the adversary has also access to a witness of the parameter $\rho$, is defined in the full version of \cite{AC:GonHevRaf15}.
\begin{description}
\item[Computational Quasi-Adaptive Strong Soundness:]
$$\Pr\begin{bmatrix*}[l]
    \gk \gets \G_a(1^\lambda);
    (\rho,\omega) \gets \dist_\gk^\mathsf{par};
    \psi \gets \algK(\gk, \rho); \\
    (x, \pi) \gets \advA_2(\gk, \omega,\psi) :
        \algV(\psi, x, \pi) = 1 \text{ and } \neg (\exists w : \R_\rho(x, w))
\end{bmatrix*} \approx 0.$$ 

\end{description}
