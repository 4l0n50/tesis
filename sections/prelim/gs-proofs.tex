The GS proof system allows to prove satisfiability of a set of quadratic equations in a bilinear group. In general, the proof is witness-indistinguishable but for most equations it is also zero-knowledge.

The admissible equation types must be in the following form:
\begin{equation}\label{gseq}
\sum_{j=1}^{m_y} f(\alpha_j, \vary_j)+\sum_{i=1}^{m_x} f(\varx_i, \beta_i)+\sum_{i=1}^{m_x} \sum_{j=1}^{m_y}  f(\varx_i,\escQE_{i,j} \vary_j)=t,
\end{equation}
 where $\Am_1,\Am_2,\Am_T$ are $\Z_q$-vector spaces equipped with some bilinear map $f:\Am_1\times \Am_2 \rightarrow \Am_T$, $\boldsymbol \alpha  \in \Am_1^{m_y}$, $\boldsymbol \beta  \in \Am_2^{m_x}$, $\matr{\EscQE}=(\escQE_{i,j}) \in \Z_q^{m_x\times m_y}$, $t \in \Am_T$. The vector spaces and the map $f$ can be defined in different ways as:
\begin{itemize}
\item[(a)] in pairing-product equations (PPEs), $\Am_1=\Gr$, $\Am_2=\Hr$, $\Am_T=\GG_T$, $f([x]_1,[y]_2)=[x]_1 [y]_2 \in \GG_T$,
\item[(b1)] in multi-scalar multiplication equations in $\Gr$ (MMEs), $\Am_1=\Gr$, $\Am_2=\Z_q$, $\Am_T=\Gr$, $f([x]_1,y)=y [x]_1 \in \Gr$,
\item[(b2)] MMEs in $\Hr$ (MMEs),  $\Am_1=\Z_q$, $\Am_2=\Hr$, $\Am_T=\Hr$, $f(x,[y]_2)=x [y]_2 \in \Hr$, and
\item[(c)] in quadratic equations in $\Z_q$ (QEs), $\Am_1=\Am_2=\Am_T=\Z_q$, $f(x,y)=xy \in \Z_q$.
\end{itemize} 
 An equation is linear if $\matr{\EscQE}=\vecb{0}$, 
 it is \textit{two-sided linear} if both $\boldsymbol \alpha \neq \vecb{0}$ and $\boldsymbol \beta \neq \vecb{0}$, and \textit{one-sided} otherwise.

When $t=f(t_1,1)$ or $t=f(1,t_2)$, for some efficiently computable $t_1\in\Am_1$ or $t_2\in\Am_2$, we say that the equation allows simulation (see \cite[Sect. 11]{SIAMJC:GroSah12}) and the proof is zero-knowledge (rather than just witness-indistinguishable). Note that this is always the case for equations other than PPEs.

\subsection{Commit-and-Prove} The GS Proof system works as a commit-and-prove scheme: first the prover commits to 
all the variables in an equation with the \emph{Groth-Sahai commitment scheme}, defined in the next section, and then it ``proves'' that the committed values satisfy the equation. The commitment to an element in the vector space $A_i$ lives in another vector space $B_i$ of larger dimension where interesting decisional assumptions exist. To prove that the equation is satisfied the verifier checks some equations in these larger fields.

\subsection{Groth-Sahai Commitments} 
 
\begin{definition} The Groth-Sahai commitment scheme in the group $\GG_\gamma$, $\gamma\in\{1,2\}$, is  specified by the following three algorithms 
	$(\mathsf{GS}.\algK,\mathsf{GS}.\Com,\mathsf{GS}.\algVrfy)$ such that:
	\begin{itemize} 
		\item  $\mathsf{GS}.\algK(gk,\dist_{2,2})$ is a randomized algorithm, which on input the group key $gk$ and the description of some matrix distribution $\dist_{2,2}$, outputs a commitment key $ck:=[\matr{U}]_\gamma=[(\vecb{u}_1||\vecb{u}_2)]_\gamma \in\GG_\gamma^{2\times 2}$, where $\matr{U}\gets\dist_{2,2}$.
		\item $\mathsf{GS}.\Com_{ck}(\mathsf{m};r)$ is a randomized algorithm which, on input a commitment key $ck=\bmatr{U}_\gamma$, and a message 
		$\mathsf{m}$ in the message space $\mathcal{M}_{ck}=A_\gamma$, it proceeds as follows. If $\mathsf{m}=m\in\Z_q$, it samples $r \gets \Z_q$ and outputs a commitment $\bvecb{c}_\gamma := m[\vecb{e}_2+\vecb{u}_1]_\gamma+r[\vecb{u}_2]_\gamma$ in the commitment space $\mathcal{C}_{ck}=\GG_\gamma^2$ and an opening $Op=r$. If $\mathsf{m}=[m]_\gamma\in\GG_\gamma$, it samples $\vecb{r} \gets \Z_q^2$ and outputs a commitment $\bvecb{c}_\gamma := [m]_\gamma\vecb{e}_2+[\matr{U}]_\gamma\vecb{r}$ in the commitment space $\mathcal{C}_{ck}=\GG_\gamma^2$ and an opening $Op=\vecb{r}$.
		\item $\mathsf{GS}.\algVrfy_{ck}([\vecb{c}]_\gamma,Op)$ is a deterministic algorithm which, on input the commitment key $ck=\bmatr{U}_\gamma$, a commitment $\bvecb{c}_\gamma$,  a message 
		$m \in \mathcal{M}_{ck}$ and an opening $Op$, outputs $1$ if $\bvecb{c}_\gamma=\GS.\Com_{ck}(m;Op)$
		and $0$ otherwise.
	\end{itemize}
\end{definition}

We will instantiate this commitment scheme with two different matrix distributions which give rise to two different commitment keys: the \emph{Perfectly Binding} and the \emph{Perfectly Hiding} commitment keys. The matrix distribution for perfectly binding commitment keys is defined as
$$
\mathcal{B}:\matr{U}=(\vecb{u}_1||\vecb{u}_2),\text{ where }\vecb{u}_2\gets\distlin_1\text{ and } \vecb{u}_1 :=\mu\vecb{u}_2, \mu\gets\Z_q,
$$
and the matrix distribution for perfectly hiding commitment keys is defined as
$$
\mathcal{H}:\matr{U}=(\vecb{u}_1||\vecb{u}_2),\text{ where } \vecb{u}_2\gets\distlin_1\text{ and } \vecb{u}_1 :=\mu\vecb{u}_2-\vecb{e}_2, \mu\gets\Z_q.
$$

\begin{theorem}[\cite{SIAMJC:GroSah12}] If $ck\gets\algK(gk,\mathcal{B})$ (resp. $ck\gets\algK(gk,\mathcal{H})$), where $gk\gets\G_a(1^\lambda)$ , the Groth-Sahai commitment scheme is perfectly binding (resp. computationally binding if the DL assumption relative to $\G_a$ holds) and computationally hiding if the SXDH assumption relative to $\G_a$ holds (resp. perfectly hiding).

\end{theorem}

\subsection{The Scheme}
Next, we give a description of the Groth-Sahai proof system in the SXDH instantiation.
\begin{description}
\item[$\algK(gk)$:]  On input the group key $gk$ pick $\matr{U},\matr{V}\gets\mathcal{B}$ and define $ck_1 := [\matr{U}]_1$ and $ck_2:=[\matr{V}]_2$.
   The common reference string is:
   $\crs_\GS:=(gk,[\vecb{u}_1]_1,[\vecb{u}_2]_1,[\vecb{v}_1]_2,[\vecb{v}_2]_2)$ and is known as the \emph{Perfectly Binding CRS}.
The CRS defines some associated maps:
\begin{align*}
&\iota_1: \Gr \cup \Z_q \rightarrow \Gr^2,& &\iota_1([x]_1):=([x]_1,[0]_1)^\top,& &\iota_1(x):=x[\vecb{u}_1]_1. \\
&\iota_2: \Hr \cup \Z_q \rightarrow \Hr^2,& &\iota_2([y]_2):=([y]_2,[0]_2)^\top,& &\iota_2(y):=y[\vecb{v}_1]_2.\\
&\iota_T: \begin{matrix}\Z_q\cup\Gr\cup\\\Hr\cup\GG_T\end{matrix} \rightarrow \GG_T^{2\times2}, &
    &\iota_T(t):=\pmatri{t & 0 \\ 0 & 0},& & \iota_T(x) := \iota_1(x)\iota_2(1)^\top,\\
&                                       & &\iota_T([x]_1):=\iota_1([x]_1)\iota_2(1)^\top,& &\iota_T([y]_2) := \iota_1(1)\iota_2([y]_2)^\top.
\end{align*}
The maps $\iota_X$ $X \in \{1,2\}$ can be naturally extended to column vectors of arbitrary length, and we write $\iota_X(\boldsymbol \delta^{\top})$ for 
$(\iota_X(\delta_1)|| \ldots ||\iota_X(\delta_r))$.

\item[$\algP(\crs_\GS,\mathsf{eq},(\varx_1,\ldots,\varx_{m_x}),(\vary_1,\ldots,\vary_{m_y}))$:] Given some equation $\mathsf{eq}$ of the form \ref{gseq} and solutions to the equation
$\varx_1,\ldots,\varx_{m_x}$ and $\vary_1,\ldots,\vary_{m_y}$, the prover proceeds as follows:
\begin{itemize}
\item Commit to all $\varx_i \in A_1$ computing $([\vecb{c}_i]_1,\vecb{r}_i)\gets\GS.\Com_{ck_1}(\varx_i)$ and commit to all $\vary_i \in A_2$ computing $([\vecb{d}_i]_2,\vecb{s}_i)\gets\GS.\Com_{ck_2}(\vary_i)$.
\item Let $\matr{R} := (\vecb{r}_1||\ldots||\vecb{r}_{m_x})$ and $\matr{S} := (\vecb{s}_1||\ldots||\vecb{s}_{m_y})$. Compute 
\begin{eqnarray*}
[\matr{\Pi}]_2 
& := & \iota_2(\boldsymbol \beta^{\top}) \matr{R}^\top +\iota_2(\vecb{y}^{\top}) \matr{\EscQE}^\top  \matr{R}^{\top}+
[\matr{V}]_2\matr{S} \matr{\EscQE}^\top \matr{R}^\top - [\matr{V}]_2 \matr{T}^\top,\\
\  [\matr{\Theta}]_1 &  := &\iota_1(\boldsymbol \alpha^{\top}) \matr{S}^\top +\iota_1(\vecb{x}^{\top}) \matr{\EscQE} \matr{S}^{\top}+ [\matr{U}]_1 \matr{T}.
 \end{eqnarray*}
\end{itemize}
\item[{$\algV(\crs_\GS,\mathsf{eq},\{[\vecb{c}_i]_1:i\in[m_x]\},\{[\vecb{d}_i]_2:i\in[m_y]\},[\matr{\Theta}]_1,[\matr{\Pi}]_2)$}:] Check if
\begin{eqnarray*}
\sum_{i \in m_x} [\vecb{c}_i]_1 \iota_2(\beta_i)^{\top} + \sum_{j \in m_y} \iota_1(\alpha_j) [\vecb{d}_j]_2^{\top} 
+ \sum_{i \in m_x}\sum_{j \in m_y}\escQE_{i,j} [\vecb{c}_i]_1 [\vecb{d}_j]_2^{\top} = \\ \iota_T(t) +[\matr{\Theta}]_1[\matr{V}]_2^\top + [\matr{U}]_1[\matr{\Pi}]_2^\top.
\end{eqnarray*}
\item[$\algS_1(gk)$:]  On input the group key $gk$ pick $\matr{U},\matr{V}\gets\mathcal{H}$ and define $ck_1 := [\matr{U}]_1$ and $ck_2:=[\matr{V}]_2$.
   The common reference string is:
   $\crs_\GS:=(gk,[\vecb{u}_1]_1,[\vecb{u}_2]_1,[\vecb{v}_1]_2,[\vecb{v}_2]_2)$ and is known as the \emph{Perfectly Hiding CRS}. This algorithm also outputs the trapdoor $\tau:=(\matr{U},\matr{V})$.
\item[$\algS_2(\crs_\GS,\mathsf{eq},\tau)$:] If $\mathsf{eq}$ allows simulation, it defines the new equation
\begin{equation}
\mathsf{eq}':=\sum_{j=1}^{m_y} f(\alpha_j, \vary_j)+\sum_{i=1}^{m_x} f(\varx_i, \beta_i)-f(\varx',t)+\sum_{i=1}^{m_x} \sum_{j=1}^{m_y}  f(\varx_i,\escQE_{i,j} \vary_j)=0,
\end{equation}
and runs the GS prover for equation $\mathsf{eq}'$ for solutions $\varx_1=\varx_2=\ldots=\varx_{m_x}=\vary_1=\ldots=\vary_{m_y}=0$ and $\varx'=1$. The simulated proof is $([\matr{\Theta}]_1,[\matr{\Pi}]_2)$, the proof of the GS prover for the modified equation.

\end{description}
The reference string $\crs_\GS$ chosen by algorithm $\algK$ defines perfectly binding commitments and the proof system has perfect soundness. Furthermore, there exists some extraction trapdoor which allows to compute a function of the witness (this is the perfect \emph{F-Knowledge} property, see \cite{PKC:EscGro14}). More specifically, the extraction trapdoor
allows to compute maps $p_1: \Gr^2\to \Gr$, $p_2:\Hr^2 \to \Hr$, and $p_T:\GG_T^{2 \times 2} \to  \GG_T$, such that:
\begin{enumerate}[label=(\alph*)]
\item for each $X\in\{1,2,T\}$, $p_X\circ\iota_X$ is the identity map,
\item for all $\varx\in\Am_1,\vary\in\Am_2$, $\iota_1(\varx)\iota_2(\vary)^\top = \iota_T(f(\varx,\vary))$ and for all $[\vecb{x}]_1\in\Gr^2,[\vecb{y}]_2\in\Hr^2$,
$f(p_1([\vecb{x}]_1),p_2([\vecb{y}]_2))=p_T([\vecb{x}]_1[\vecb{y}]_2^\top)$,
\item for each $i\in[2]$, $p_1([\vecb{u}_i]_1)=0$ and $p_2([\vecb{v}_i]_2)=0$.
\end{enumerate}

\begin{theorem}[\cite{EC:GroSah08}] If $\eq$ allows simulation then the Groth-Sahai proof system is a composable zero-knowledge proof system with perfect completeness, perfect soundness, and computational zero-knowledge based on the SXDH assumption. Otherwise, is a composable witness-indistinguishable proof system with perfect completeness, perfect soundness, and computational witness indistinguishability based the SXDH assumption.
\end{theorem}

