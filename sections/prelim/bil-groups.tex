Groups where the DDH assumption (as well as the discrete logarithm assumption) is believed to be hard were usually constructed as multiplicative subgroups of a finite field of order $n$, $\mathbb{F}_n$ -- for example the set of \emph{quadratic residues} of $\Z_p$ when $p$ is a safe prime (i.e. $(p-1)/2$ is also a prime number).
Koblitz and (independently) Miller proposed as an alternative the usage of \emph{Elliptic Curves} over $\mathbb{F}_n$ \cite{MOC:Koblitz87,C:Miller85}, that is the set of points $(x,y)$ over $\mathbb{F}_n^2$ which are solutions to the equation $E:y^2=x^3+ax+b$, where $a,b\in\mathbb{F}_n$, together with an especial point $\mathcal{O}$. The solutions to $E$ together with $\mathcal{O}$ form an abelian group, denoted by $E(\mathbb{F}_n)$, where $\mathcal{O}$ is the identity element and the group operation is defined as a geometrical operation between the group elements (called the cord-and-tangent method). For the appropiate choise of $a$ and $b$ the discrete logarithm is beleived to require exponencial time to be computed, while for subgroups of $\mathbb{F}_n$  there are known sub-exponencial attacks. As consequence, the size of the elements of $E(\mathbb{F}_n)$ is much smaler than the size of elements of traditional finite fields at an equivalent security level.

However not all elliptic curves offer the same security. For example in some elliptic curves, which we will call bilinear groups for short, one can compute the \emph{Weil Pairing} or the \emph{Tate Pairing}. These \emph{pairings} are functions of the form $e:\GG_1\times\GG_2\to\GG_T$ where $\GG_1$ and $\GG_2$ are cyclic subgroups of $E(\mathbb{F}_n)$ and $\GG_T$ is another cyclic group known as the \emph{target group}. The function $e$ is \emph{bilinear}, because $e(a\mathcal{X},b\mathcal{Y})=ab\cdot e(\mathcal{X},\mathcal{Y})$ for any $\mathcal{X}\in\GG_1$ and any $\mathcal{Y}\in\GG_2$, and \emph{non-degenerate}, because $e(\mathcal{P}_1,\mathcal{P}_2)$ is a generator of $\GG_T$ when $\mathcal{P}_s$ is a generator of $\GG_s$, $s\in\{1,2\}$. For simplicity assume that $\GG_1=\GG_2=\GG$ (this is the case of the Weil pairing) and let $\mathcal{P}$ the generator of $\GG$. Intuitively, the function $e$ is ``solving CDH in $\GG$ but giving the answer in $\GG_T$'', that is, given $\mathcal{X}=a\mathcal{P}$ and $\mathcal{Y}:=b\mathcal{P}$ one can compute $t=e(\mathcal{X},\mathcal{Y})=ab\cdot e(\mathcal{P},\mathcal{P})$. Although this does not gives a solution to the CDH problem in $\GG$, it does solves the DDH problem in $\GG$. Indeed, $e(\mathcal{X},\mathcal{Y})=e(\mathcal{Z},\mathcal{P})$ is satisfied by $\mathcal{Z}=ab\mathcal{P}$ but only with negligible probability by a random element of $\GG$. 
Moreover, the MOV and the FR reductions compile the DL problem in $E(\mathbb{F}_n)$ into the finite field $\mathbb{F}_{n^\alpha}$, for some $\alpha$ known as the \emph{embedding degree}, where knwon sub-exponencial algorithms for solving the DL exist \cite{STOC:MenVanOka91,MOC:FreRuc94}.
For this reasons elliptic curves where a pairing function can be efficiently computed were initially considered unsafe for cryptographic purposes.

But this changed with the work of Joux \cite{ANTS:Joux00}, where he showed that pairings can be also used to construct cryptographic schemes rather than to destroy them. Joux noted that pairings can be used to construct a non-interactive three party Diffie-Hellman key exchange \cite{ANTS:Joux00}. Given public keys $\mathcal{X}=x\mathcal{P}$, $\mathcal{Y}=y\mathcal{P}$, and $\mathcal{Z}=z\mathcal{P}$ and one secret key from $x,y,$ or $z$, each party can compute the shared key $x\cdot e(\mathcal{Y},\mathcal{Z})= y\cdot e(\mathcal{X},\mathcal{Z})= z\cdot e(\mathcal{X},\mathcal{Y})=xyz\cdot e(\mathcal{P},\mathcal{P})$. The critical observation was that, although the DDH might be easy in bilinear groups, the DL might be made hard if the size of the group is appropiately incresed in order to rule out the sub-exponencial attacks implied by the MOV and FR reductions. Moreover, Joux protocol can be proven secure under the \emph{Bilinear Decisional Diffie-Hellman} assumption, which is believed to be a hard problem in some bilinear groups, and also CDH and many other assumptions are believed to hold on some bilinear groups. Furthermore, from Joux seminal work many open problems have been solved using bilinear groups, being \emph{Identity Based Encryption} \cite{C:BonFra01} the more outstanding example, and many new applications have been found. 

%\subsection{Intuition}
%The structure of abelian groups allows to compute the addition of two group elements and, if used together with the DDH assumption, allows to compute \emph{linear functions} of a hidden value. This is the case of ElGamal encryption, where for any integer $a$ and any group element $\mathcal{B}$, given $(\mathcal{C}_1,\mathcal{C}_2)=\Enc_{pk}(\mathcal{X};r)$ one can compute $a(\mathcal{C}_1,\mathcal{C}_2)+(0,\mathcal{B})=\Enc_{pk}(a\mathcal{X}+\mathcal{B};ar)$. 
%%This is interesting because using the techniques of Groth and Sahai \cite{EC:GroSah08} one can give a NIZK proof of the satisfiability of the linear equation.
%
%One can also try to generalize this and evaluate a set of quadratic equations. \footnote{This generalization is quite powerful since includes all the statements which can be proven in NP, a facts that follows from the NP-completeness of the satisfiability of a set of quadratic equations over a finite field.}
%Following the example of ElGamal, given $\mathcal{X}$ and $\mathcal{Y}$ we want to be able to ``homomorphically'' compute a third value $\mathcal{Z}$ which encodes the multiplication of $\mathcal{X}$ and $\mathcal{Y}$. That is, $\mathcal{X}=x\mathcal{P}_1$, $\mathcal{Y}=y\mathcal{P}$, and $\mathcal{Z}=xy\mathcal{P}_T$, where $\mathcal{P}_T$ is the generator of the group $\GG_T$. This is exactly what can be done with a pairing such as the Weil pairing or the Tate pairing. 
%
\subsection{Definition}

\begin{definition}[Bilinear Groups] \footnote{While the usual notation for the target group has been multiplicative, we write it in additive notation. The reason is just to elude cumbersome expressions in the exponent.}
Let $\GG_1$ and $\GG_2$ be cyclic groups of prime order $q$ and $\mathcal{P}_s$ the generator of $\GG_s$, $s\in\{1,2\}$, and $\GG_T$ be another cyclic group of order $q$. We say that $\GG_1,\GG_2,\GG_T$ form a bilinear group if there exists a map $e:\GG_1\times\GG_2\to\GG_T$ such that:
\begin{description}
\item[Bilinearity:] For all $a,b\in\Z_q$, for all $\mathcal{X}\in\GG_1$, and all $\mathcal{Y}\in\GG_2$
$$
e(a\mathcal{X},b\mathcal{Y})=ab\cdot e(\mathcal{X},\mathcal{Y}),
$$
\item[Non-degeneracy:] If $\mathcal{X},\mathcal{Y}\neq0$, then $e(\mathcal{X},\mathcal{Y})\neq 0$,
\item[Computability:] $e(\mathcal{X},\mathcal{Y})$ is efficiently computable.
\end{description} 
\end{definition} 

The first property says that $e$ allows to ``homomorphically'' compute degree 2 expressions in the field $(\Z_q,+,\cdot)$. Since any $\mathcal{X}$ can be written as $x\mathcal{P}_1$, where $x$ is some element of $\Z_q$ (and the same can be done in $\GG_2$), $e(\mathcal{X},\mathcal{Y})=xy\cdot e(\mathcal{P}_1,\mathcal{P}_2)$. Non-degeneracy says that $e$ is does not map everything to 0. While the third property says that computing $e$ is practical, the reality is that it is still an expensive operation and is one of the critical performance measures of cryptographic constructions.

Galbraith et al.~classify bilinear groups in three types \cite{DAM:GalPatSma08}:

\begin{description}
\item[Type I:] $\GG_1=\GG_2$ and also known as \emph{symmetric} groups. 
\item[Type II:] $\GG_1\neq\GG_2$ but there is an efficiently computable homomorphism $\phi:\GG_1\to\GG_2$.
\item[Type III:] $\GG_1\neq\GG_2$ and no efficiently computable homomorphism is known. Also known as \emph{asymmetric} groups.
\end{description}

Type III bilinear groups are the most attractive from the perspective
of a performance and security trade off, specially since the recent attacks on discrete logarithms in
finite fields by Joux \cite{SAC:Joux13} and subsequent improvements.

We denote by $\ggen_a$ the probabilistic polynomial time algorithm which on input $1^{\lambda}$, where $\lambda$ is the security parameter, returns the \emph{group key} which is the description of an asymmetric bilinear group $gk:=(q,\GG_1,\GG_2,\GG_T,e,\mathcal{P}_1,\mathcal{P}_2)$, where $\GG_1,\GG_2$
and $\GG_T$ are groups of prime order $q$, the elements $\mathcal{P}_1, \mathcal{P}_2$ are generators of 
$\GG_1,\GG_2$ respectively, and $e:\GG_1\times\GG_2\to\GG_T$ is an efficiently
computable, non-degenerate bilinear map. 

\subsubsection{Implicit Representation of Group Elements}

Elements in $\GG_s$, are denoted implicitly as $[a]_s:=a \Pt_s$, where $s \in \{1,2,T\}$ and $\Pt_T:=e(\Pt_1,\Pt_2)$. 
The pairing operation will be written as a product $\cdot$, that is $[a]_1 \cdot [b]_2=[a]_1 [b]_2=e([a]_1,[b]_2)=[ab]_T$. Given a matrix $\matr{T}=(t_{i,j})$, $[\matr{T}]_s$ is
the natural embedding of $\matr{T}$ in $\GG_s$, that is, the matrix whose $(i,j)$th entry
is $t_{i,j}\mathcal{P}_s$. We denote by $|\GG_s|$ the bit-size of the elements of $\GG_s$.

