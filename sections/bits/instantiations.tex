We discuss in detail two particular cases of languages $\mathcal{\mathcal{L}}_{ck,\sfbits}$. First, in Section~\ref{sec:bits-scheme} we discuss the case when 
\begin{itemize}
\item[(a)] $[\vecb{c}]_1$ is a vector in $\Gr^{n+1}$,  $\vecb{u}_{n+1} \leftarrow \distlin_{n+1,1}$ and
 $\matr{U}_1:=\begin{pmatrix}{\matr{I}_{n\times n}}\\{\matr{0}_{1\times n}}\end{pmatrix} \in \Z_q^{(n+1) \times n}, \matr{U}_2:=\vecb{u}_{n+1} \in \Z_q^{n+1}$, $\matr{U}:=(\matr{U}_1\cat \matr{U}_2)$.    
\end{itemize}
In this case, the vectors $[\vecb{g}_i]_1$ in the intuition are defined as $[\vecb{g}_i]_1=\matr{\Delta} [\vecb{u}_i]_1$, where $\matr{\Delta}\gets\Z_q^{2\times(n+1)}$, and the commitment 
to $\vecb{b}$ is computed as $[\vecb{c}]_1:=\sum_{i\in[n]}b_i[\vecb{u}_i]_1+w[\vecb{u}_{n+1}]_1$.
Then in Section~\ref{sec:bits-extensions} we discuss how to generalize the construction for a) to 
\begin{itemize}
 \item[(b)] $[\vecb{c}]_1$ is the concatenation of $n$ GS commitments. That is, given the  GS CRS   $\mathsf{crs}_\GS=(gk,[\vecb{u}_1]_1,[\vecb{u}_2]_1,[\vecb{v}_1]_2,[\vecb{v}_2]_2)$, we define,
$$\matr{U}_1:=  \begin{pmatrix} \vecb{u}_1 & \ldots & \vecb{0}\\ \vdots & \ddots & \vdots \\   \vecb{0} & \ldots & \vecb{u}_1  \end{pmatrix} \in \Z_q^{2n \times n},  \matr{U}_2:= \begin{pmatrix} \vecb{u}_2 & \ldots & \vecb{0}\\ \vdots & \ddots & \vdots \\  \vecb{0} & \ldots & \vecb{u}_2  \end{pmatrix} \in \Z_q^{2n \times n}.$$ 
\end{itemize}

%\paragraph{Remark.}
Although the proof size is constant, in both of our instantiations the commitment size is $\Theta(n)$. Specifically, $(n+1)\sG$ for case a) and $2n\sG$ for case b).
%Furthermore, one might relax the requirement of perfectly binding commitments as long as one is able to extract a witness from $[\vecb{c}]_1$. However, if for example $[\vecb{U}]_1$ defines a distribution over hard instances of the Subset-Sum language, impossibility results from $\cite{STOC:GenWic11}$ imply that $[\vecb{c}]_1$ cannot be of size $\mathsf{poly}(\lambda)n^{o(1)}$.

%Although this is a limitation for some applications, \textit{e.g.} range proofs, this is a requirement for some others (like proving membership in a list).   

