
\begin{table}[h]
\begin{center}
\begin{minipage}{\textwidth}
\begin{center}
\begin{tiny}
             %   proof  % CRS   %
% group type %  Gr % Hr % Gr Hr % #Pairings
\begin{tabular}{|l||c|c|c|c|c|}
\hline
                                                    & Comms     & Proof           & CK                   & CRS($\rho$)       & \#Pairings
\\ \hline\hline
\rule{0pt}{2.5ex}GS \cite{EC:GroSah08}              & $(2n,2n)$    & $(4n,4n)$          & $(4,4)$                & $0$                 & $28n$   \\ \hline
\rule{0pt}{2.5ex}GS + $\QANIZKcomms$                & $(2n,2n)$    & $(2n+2,2n+2)$      & $(4,4)$                & \begin{tabular}{c}
                                $(10n+4,$\\
                                $10n+4)$
                              \end{tabular}         & $20n+8$  \\ \hline
\rule{0pt}{2.5ex}$\Pi_\bit$ $m=1$                   & $(n+1,0)$& $(10,10)$          & $(n + 1,0)$         & \begin{tabular}{c}
                        $(6n^2 +11n+34,$\\
                        $6n^2+11n+34)$
                      \end{tabular}                 & $n+55$   \\ \hline
\rule{0pt}{2.5ex}$\Pi_\bit$ $m=n$ (i)               & $(2n,0)$    & $(10,10)$         & $(4,0)$               & \begin{tabular}{c}
                                                                                                             $(12n^2+14n+22,$\\
                                                                                                             $12n^2+13n+24)$
                                                                                                           \end{tabular}       & $2n+52$  \\ \hline
\rule{0pt}{2.5ex}$\Pi_\bit$ $m=n$ (ii)              & $(2n,0)$    & $(10,10)$         & $(4,0)$               & \begin{tabular}{c}
                                                                                                             $(6n^2+16n+32,$\\
                                                                                                             $6n^2+12n+32)$
                                                                                                           \end{tabular}       & $4n+52$  \\ \hline

\rule{0pt}{2.5ex}$\Pi_\bit$ weight 1, $m=1$         & $(n+1,0)$& $(10,10)$          & $(n+1,0)$           & \begin{tabular}{c}
                                                                                                             $(18n+32,$\\
                                                                                                             $19n+34)$
                                                                                                           \end{tabular}       & $n+55$   \\ \hline
\rule{0pt}{2.5ex}$\Pi_\bit$ weight 1, $m=n$         & $(2n,0)$   & $(10,10)$          & $(4,0)$               & \begin{tabular}{c}
                                                                                                             $(20n+32,$\\
                                                                                                             $18n+32)$
                                                                                                           \end{tabular}       & $4n+52$   \\ \hline
%\rule{0pt}{2.5ex}Partial Sat. \cite{TCC:Rafols15}   &           &                 &                      &                     &          \\ \hline
\end{tabular}
\end{tiny}
\end{center}
\caption{Comparison for proofs of membership in $\Lang_{ck,\sfbits}$ between GS proofs 
 and our different constructions. Our NIZK construction for bit-strings is denoted by $\Pi_\bit$ and the construction for proving that two sets of commitments open to the same value $\QANIZKcomms$. Row ``$\Pi_\bit$ $m=1$'' is for our construction for a single commitment of size $n+1$ to a bit-string of size $n$. Rows ``$\Pi_\bit$ $m=n$ (i)'' and ``$\Pi_\bit$ $m=n$ (ii)'' are for our construction for $n$ concatenated GS commitments, using the two different CRS distributions described in Sect. \ref{sec:bits-instantiations}. Rows ``$\Pi_\bit$ weight 1, $m=1$'' and ``$\Pi_\bit$ weight 1, $m=n$'' are 
for our constructions for bit-strings of weight 1 with $m=1$ and $m=n$, respectively.
Column ``Comms'' contains
the size of the commitments, ``CK'' the size of the commitment keys in the CRS, and ``CRS($\rho$)''
the size of the language dependent part of the CRS. Notation $(a,b)$ means $a$ elements of $\GG_1$ and $b$ elements of $\GG_2$. The table is computed for $\dist_k=\distlin_2$, the 2-Linear matrix distribution. \label{table:eff1}  } 
\end{minipage}
\vspace{-0.54cm}

%\begin{minipage}{\textwidth}
%\medskip
%\begin{center}
%\begin{tabular}{|l||c|c|c|c|c|}
%\hline
%                                                    & Comms                 & Proof           & CK                   & CRS($\rho$)       & \#Pairings
%\\ \hline\hline
%\rule{0pt}{2.5ex}GS \cite{EC:GroSah08}              & $2m_x\sG+2m_y\sH$     & $2n\s$          & $4\s$                & $nm_y\sG+nm_x\sH$   & $16n$   \\ \hline
%\rule{0pt}{2.5ex}$\Pi_\sfts$                        & $2m_x\sG+2m_y\sH$     & $2\s$           & $4\s$                & $2nm_y\sG+2nm_y\sH$ & $8n+4$  \\ \hline
%\end{tabular}
%\end{center}
%\caption{Comparison for two-sided Multi-scalar Multiplication Equations $\cvecb{y}^\top\hgrkb{\alpha}_\ell+\hvecb{x}^\top\cgrkb{\beta} = 0$,
%for $\ell\in[n]$, $\hvecb{x}\in\Gr^{m_x}$, $\cvecb{y}\in\Hr^{m_y}$, $\hgrkb{\alpha}\in{\Gr}^{m_y}$, and $\cgrkb{\beta}\in\Hr^{m_y}$,
%between GS proofs and our aggregation techniques.\label{table:eff2}}
%\end{minipage}
\end{center}
\end{table}
%\begin{table}\begin{center}
%\end{minipage}\end{center}
% Column ``Comms'' contain
%the size of the commitments, ``CK'' the size of the commitment keys in the CRS, and ``CRS($\Gamma$)'' the size of the language dependent part of the CRS.
%The size of elements in $\Gr$ and $\Hr$ is $\sG$ and $\sH$ respectively, and $\s = \sG+\sH$.
%. \label{table:eff}}
%\end{table}
