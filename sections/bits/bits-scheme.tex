%\input{bitstrings/proofasym-fig.tex}
\begin{description}

\item[$\algK_0(1^\lambda)$:]  Return $\Gamma := (q,\Gr,\Hr,\T,e,\hat{g},\check{h}) \leftarrow \ggen_a(1^{\lambda})$.

\item[$\dist_\Gamma$:] The distribution $\dist_\Gamma$ over $\Gr^{(n+1) \times (n+1)}$ is some witness samplable distribution which 
defines the relation $\R_\Gamma = \{\R_{\hmatr{U}}\} 
\subseteq \Gr^{n+1}\times(\bits^{n}\times\Z_q)$,
where $\hmatr{U}\gets\dist_\Gamma$,
such that $(\hvecb{c},\langle\vecb{b}, w\rangle)\in\R_{\hmatr{U}}$ iff
$\hvecb{c}=\hmatr{U}\binom{\vecb{b}}{w}$. The relation $\R_{par}$ consists of pairs $(\hmatr{U},\matr{U})$ where $\hmatr{U} \gets \dist_{\Gamma}$.
\item[$\algK_1(\Gamma, \hmatr{U})$:]
Let $\lrck_{n+1}\gets \Z_q^2$
and for all $i \in [n]$, $\lrck_{i}:=\epsilon_{i}\lrck_{n+1}$, where
$\epsilon_{i} \leftarrow \Z_q$. Define
$\Rck := (\rck_1|| \ldots ||\rck_{n+1})$.
Choose 
$\matr{\Delta} \leftarrow \Z_q^{2 \times (n+1)}$,
define $\Lck := \matr{\Delta}\hmatr{U}$
and $\hat{\llck}_{i}:=\matr{\Delta} \hat{\vecb{u}}_i \in \Gr^2$, for all $i \in [n+1]$. 
Let $\vecb{a} \leftarrow \distlin_{1}$ and define $\cvecb{a}_{\Delta}:=\matr{\Delta}^\top\cvecb{a} \in \Hr^{n+1}$. 
For any pair $(i,j) \in \indexSet{n}{1}$, let 
$\matr{T}_{i,j}\gets\Z_q^{2\times2}$ and set:
$$\hmatr{C}_{i,j}:=\lck_i \lrck_j^{\top} - \hmatr{T}_{i,j}  \in \Gr^{2 \times 2},
\qquad \qquad 
\cmatr{D}_{i,j}:=\cmatr{T}_{i,j} \in \Hr^{2 \times 2}.$$ 
Note that $\hmatr{C}_{i,j}$ can be efficiently computed 
as $\lrck_j \in \Z_q^{2}$ is the vector of discrete logarithms of $\check{\lrck}_j$.

Let $\QANIZKsum$ be the proof system for Sum in Subspace 
(Sect. \ref{sec:QANIZKsum}) and $\QANIZKcomms$
%=(\algK_0,\algK_1,\algP,\algV,\algS_1,\algS_2)$
be an instance of our proof system for Equal Opening (Sect. \ref{sec:aggcomms}).

Let
$\crs_\QANIZKsum \gets \algK_1(\Gamma, \{\hmatr{C}_{i,j},\cmatr{D}_{i,j}\}_{(i,j)\in\indexSet{n}{1}})$ and \footnote{We identify
matrices in $\Gr^{2 \times 2}$ (resp. in $\Hr^{2 \times 2}$) with vectors in $\Gr^{4}$ (resp. in $\Hr^{4}$).}  
 $\crs_\QANIZKcomms \gets \algK_1(\Gamma, \Lck,\Rck,n)$. The common reference string is given by:
\begin{eqnarray*}
\mathsf{crs}_P&:=&\left( \hat{\matr{U}},  \Lck,
    \check{\Lrck}, \{\hmatr{C}_{i,j},\cmatr{D}_{i,j} \}_{(i,j) \in \indexSet{n}{1}},\crs_\QANIZKsum,\crs_\QANIZKcomms \right), \\
\mathsf{crs}_V&:=&\left(\cvecb{a}, \cvecb{a}_\Delta, \crs_\QANIZKsum,\crs_\QANIZKcomms \right). 
 \end{eqnarray*}
\item[$\algP(\mathsf{crs}_P, \hvecb{c}, \langle \vecb{b}, w_g \rangle)$:]
Pick $w_h \gets \Z_q$,  $\matr{R} \gets \Z_q^{2\times 2}$ and then: 
\begin{enumerate}
\item Define 
$$\hvecb{c}_{\Delta} := \Lck \begin{pmatrix} \vecb{b} \\ w_g \end{pmatrix},
\qquad \cvecb{d} := \Rck \begin{pmatrix} \vecb{b} \\ w_h \end{pmatrix}.$$ 
\item Compute 
 $(\hmatr{\Theta}_{b(\overline{b}-1)}, \cmatr{\Pi}_{b(\overline{b}-1)})\, :=$
\begin{eqnarray} \label{eq:ThetaPi}
%(\hmatr{\Theta}_{b(\overline{b}-1)}, \cmatr{\Pi}_{b(\overline{b}-1)})
%& := &
%    \sum_{i \in [n]}\left(
%        b_i w_h (\hmatr{C}_{i,n+1},\cmatr{D}_{i,n+1})+
%        w_g(b_i-1) (\hmatr{C}_{n+1,i}, \cmatr{D}_{n+1,i})\right)
%        \nonumber\\ & &           +
%       \sum_{i \in [n]}  \sum_{\substack{j \in [n]\\ j\neq i}} b_i (b_j-1) (\hmatr{C}_{i,j}, \cmatr{D}_{i,j})\nonumber\\
%       & &
%     +
%    w_gw_h (\hmatr{C}_{n+1,n+1}, \cmatr{D}_{n+1,n+1}) +  (\hmatr{R},-\cmatr{R}).
& &
    \sum_{i \in [n]}\left(
        b_i w_h (\hmatr{C}_{i,n+1},\cmatr{D}_{i,n+1})+
        w_g(b_i-1) (\hmatr{C}_{n+1,i}, \cmatr{D}_{n+1,i})\right)
        \nonumber\\ & &           +
       \sum_{i \in [n]}  \sum_{\substack{j \in [n]\\ j\neq i}} b_i (b_j-1) (\hmatr{C}_{i,j}, \cmatr{D}_{i,j})\nonumber\\
       & &
     +
    w_gw_h (\hmatr{C}_{n+1,n+1}, \cmatr{D}_{n+1,n+1}) +  (\hmatr{R},-\cmatr{R}).
%\\
%(\hmatr{\Theta}_{b-\overline{b}}, 
%\cmatr{\Pi}_{b-\overline{b}})
%  & := &
%    \sum_{i\in[n]}\left(
%        (b_i-w_h)(\hmatr{C}_{i,n+1},\cmatr{D}_{i,n+1})
%         +
%        (w_g-b_i)(\hmatr{C}_{n+1,i}, \cmatr{D}_{n+1,i})\right)\\
%     & &  + \sum_{i\in[n]}\sum_{\substack{j\in[n]\\j\neq i}} (b_i-b_j)(\hmatr{C}_{i,j}, \cmatr{D}_{i,j}) 
%\\ & &+
%    (w_g-w_h)(\hmatr{C}_{n+1,n+1},\cmatr{D}_{n+1,n+1}) + (\hmatr{S},-\cmatr{S}).
 \end{eqnarray}

\item Compute a proof $(\hat{\boldsymbol \rho}_{b(\overline{b}-1)},\check{\boldsymbol \sigma}_{b(\overline{b}-1)})$
that $\matr{\Theta}_{b(\overline{b}-1)}+\cmatr{\Pi}_{b(\overline{b}-1)}$
belongs to the space spanned by $\{\matr{C}_{i,j}+\matr{D}_{i,j}\}_{(i,j)\in\indexSet{n}{1}}$,
 and a proof 
$(\hgrkb{\rho}_{b-\overline{b}}, \cgrkb{\sigma}_{b-\overline{b}})$
that
$(\hvecb{c}_\Delta,\cvecb{d})$ open to the same value,
using $\vecb{b},w_g$, and $w_h$. 
\end{enumerate}

\item[$\algV(
    \mathsf{crs}_V,
    \hvecb{c},
    \langle
        \hvecb{c}_{\Delta}, \cvecb{d},
        (\hmatr{\Theta}_{b(\overline{b}-1)}, \cmatr{\Pi}_{b(\overline{b}-1)}), 
        \{(\hat{\boldsymbol \rho}_{X}, \check{\boldsymbol \sigma}_{X})\}_{X \in \{b(\overline{b}-1), b-\overline{b}\}} \rangle)$:] ~
%   
\begin{enumerate}
\item  Check if $\hvecb{c}^\top\cvecb{a}_\Delta = \hvecb{c}_\Delta^\top\cvecb{a}$. 
\item Check if 
\begin{equation}\label{eq:ver1}\hvecb{c}_{\Delta}
\left(
    \check{\vecb{d}}-
    \sum_{j \in [n]} \rck_{j}
\right)^{\top} =
    \hmatr{\Theta}_{b(\overline{b}-1)} \cmatr{I}_{2 \times 2} +
    \hmatr{I}_{2 \times 2}\cmatr{\Pi}_{b(\overline{b}-1)}.
    \end{equation}  
  \item Verify that $(\hgrkb{\rho}_{b(\overline{b}-1)}, \cgrkb{\sigma}_{b(\overline{b}-1)}),(\hgrkb{\rho}_{b-\overline{b}},\cgrkb{\sigma}_{b-\overline{b}})$ are valid proofs for %\linebreak 
  $(\hmatr{\Theta}_{b(\overline{b}-1)},$ $\cmatr{\Pi}_{b(\overline{b}-1)})$
        and $(\hvecb{c}_\Delta,\cvecb{d})$ using $\crs_\QANIZKsum$ and $\crs_\QANIZKcomms$ respectively.
\end{enumerate}
If any of these checks fails, the verifier outputs $0$, else it outputs $1$.
%\item[$\mathsf{S}_1(\Gamma,\hat{\matr{U}})$:] The simulator receives as input a description of an asymmetric bilinear group $\Gamma$ and a matrix $\hat{\matr{U}} \in \Gr^{(n+1) \times (n+1)}$ sampled according to distribution $\dist_{\Gamma}$. It generates and outputs the CRS in the same way as $\algK_1$, but additionally it also  outputs the simulation trapdoor 
%$$\tau=\left(\Lrck, \matr{\Delta}, \tau_\QANIZKsum, \tau_\QANIZKcomms\right),$$
%where $\tau_\QANIZKsum$ and $\tau_\QANIZKcomms$ are, respectively, $\QANIZKsum$'s and $\QANIZKcomms$'s simulation trapdoors.
%\item[$\mathsf{S}_2(\crs_P,\vc,\tau)$:] Compute $\vc_{\Delta}:=\matr{\Delta} \vc$.
%      Then pick random $\overline{w}_h \gets \Z_q$, $\matr{R} \gets \Z_q^{2 \times 2}$ and define 
% $\vecb{d}:= \overline{w}_{h} \lrck_{n+1}.$
% Then set:
%\begin{align*} 
%\hmatr{\Theta}_{b(\overline{b}-1)} & :=  \hvecb{c}_\Delta \left(\vecb{d}-\sum_{i \in [n]} \lrck_i\right)^\top + \hmatr{R},
%    &
%    \cmatr{\Pi}_{b(\overline{b}-1)} & := - \cmatr{R}.
%%\\
%%\hmatr{\Theta}_{b-\overline{b}} & :=  \hvecb{c}_\Delta \left( \sum_{i\in[n]}\lrck_{i}^\top \right) -
%%                          \left( \sum_{i\in[n]}\hvecb{u}_{i} \right)\vecb{d}^\top + \hmatr{S},
%%    &
%%    \cmatr{\Pi}_{b-\overline{b}}& := - \cmatr{S}.
%\end{align*}
%Finally, simulate proofs $(\hat{\boldsymbol \rho}_{X},
%  \check{\boldsymbol \sigma}_{X})$
%%for $(\hmatr{\Theta}_{X}, \cmatr{\Pi}_{X})$
%for $X \in \{b(\overline{b}-1),  b-\overline{b}
%\}$  using $\tau_\QANIZKsum$ and $\tau_\QANIZKcomms$.
%
%\end{description} 
%The simulators $\algS_1$ and $\algS_2$ are defined as follows.
%\begin{description}
\item[$\mathsf{S}_1(\Gamma,\hat{\matr{U}})$:] The simulator receives as input a description of an asymmetric bilinear group $\Gamma$ and a matrix $\hat{\matr{U}} \in \Gr^{(n+1) \times (n+1)}$ sampled according to distribution $\dist_{\Gamma}$. It generates and outputs the CRS in the same way as $\algK_1$, but additionally it also  outputs the simulation trapdoor 
$$\tau=\left(\Lrck, \matr{\Delta}, \tau_\QANIZKsum, \tau_\QANIZKcomms\right),$$
where $\tau_\QANIZKsum$ and $\tau_\QANIZKcomms$ are, respectively, $\QANIZKsum$'s and $\QANIZKcomms$'s simulation trapdoors.
\item[$\mathsf{S}_2(\crs_P,\vc,\tau)$:] Compute $\vc_{\Delta}:=\matr{\Delta} \vc$.
      Then pick random $\overline{w}_h \gets \Z_q$, $\matr{R} \gets \Z_q^{2 \times 2}$ and define 
 $\vecb{d}:= \overline{w}_{h} \lrck_{n+1}.$
 Then set:
\begin{align*} 
\hmatr{\Theta}_{b(\overline{b}-1)} & :=  \hvecb{c}_\Delta \left(\vecb{d}-\sum_{i \in [n]} \lrck_i\right)^\top + \hmatr{R},
    &
    \cmatr{\Pi}_{b(\overline{b}-1)} & := - \cmatr{R}.
%\\
%\hmatr{\Theta}_{b-\overline{b}} & :=  \hvecb{c}_\Delta \left( \sum_{i\in[n]}\lrck_{i}^\top \right) -
%                          \left( \sum_{i\in[n]}\hvecb{u}_{i} \right)\vecb{d}^\top + \hmatr{S},
%    &
%    \cmatr{\Pi}_{b-\overline{b}}& := - \cmatr{S}.
\end{align*}
Finally, simulate proofs $(\hat{\boldsymbol \rho}_{X},
  \check{\boldsymbol \sigma}_{X})$
%for $(\hmatr{\Theta}_{X}, \cmatr{\Pi}_{X})$
for $X \in \{b(\overline{b}-1),  b-\overline{b}
\}$  using $\tau_\QANIZKsum$ and $\tau_\QANIZKcomms$.
\end{description}



\subsection{Proof of Security}

\begin{theorem}
The proof system from Sect. \ref{bits-proof-system} is QA-NIZK proof system for the language $\Lang_{ck,\sfbits}$ with perfect completeness, computational soundness, and perfect zero-knowledge
\end{theorem}

\begin{proof}
(Completeness.) It is obvious by definition that for any $[\vecb{c}]_1\in \Lang_{ck,\sfbits}$
the vector $[\vecb{c}_\Delta]_1$
generated by an honest prover passes the verification test described in 1).

Note that,
by definition of $[\matr{C}_{i,j}]_1$ and $[\matr{D}_{i,j}]_2$, 
$[\matr{C}_{i,j}]_1[\matr{I}_{2\times2}]_2 + [\matr{I}_{2\times2}]_1 [\matr{D}_{i,j}]_2
= [\vecb{g}_{i}]_1[\vecb{h}_j^\top]_2$.  Since $b_i(b_i-1) = 0$ for each $i\in[n]$,
\begin{align*}
&[\vecb{c}_{\Delta}]_1 \left( [\vecb{d}]_2 - \sum_{i\in[n]} [\vecb{h}_{i}]_2 \right)^\top \\
  =& 
    \sum_{i \in [n]}\left(
        b_i w_h[\vecb{g}_{i}]_1[\vecb{h}_{n+1}^{\top}]_2 +w_g(b_i-1) [\vecb{g}_{n+1}]_1[\vecb{h}_i^{\top}]_2 +
        \sum_{j \in [n]} b_i (b_j-1) [\vecb{g}_{i}]_1[\vecb{h}_{j}^{\top}]_2
    \right) +\\
&
    w_gw_h [\vecb{g}_{n+1}][\vecb{h}_{n+1}^{\top}]_2
\\  = & 
    \sum_{i \in [n]}\left(
        b_i w_h[\vecb{g}_{i}]_1[\vecb{h}_{n+1}^\top]_2 +
        w_g(b_i-1) [\vecb{g}_{n+1}]_1[\vecb{h}_{i}^\top]_2 +
        \sum_{\substack{j \in [n]\\ j\neq i}} b_i (b_j-1) [\vecb{g}_{i}]_1[\vecb{h}_{j}^\top]_2
    \right)
\\  &
    + w_gw_h [\vecb{g}_{n+1}]_1[\vecb{h}_{n+1}^\top]_2 +
    [\matr{R}]_1[\matr{I}_{2\times2}]_2 - [\matr{I}_{2\times2}]_1[\matr{R}]_2
\\  = &
    [\matr{\Theta}]_1[\matr{I}_{2\times2}]_2 +
    [\matr{I}_{2\times2}]_1[\matr{\Pi}]_2.
\end{align*}
Finally, the rest of the proof follows from completeness of ${\Psi_\sfcom}$ and ${\Psi_\sfsum}$. 

(Soundness.) Soundness is proven in Theorem \ref{teo:bitstr-soundness}.

(Zero-Knowledge.) First, note that the vector $[\vecb{d}]_2 \in \Hr^2$ output by the prover and the vector output by $\mathsf{S}_2$ follow exactly the same distribution. This is because the rank of $\Rck$ is $1$. In particular, although the simulator $\mathsf{S}_2$ does not know the opening of $[\vecb{c}]_1$, which is some $\vecb{b} \in \{0,1\}^{n}$, 
there exists $w_h \in \Z_q$ such that $[\vecb{d}]_2=\Rck\smallpmatrix{\vecb{b}\\ w_h}$. 
Since $\matr{R}$ is chosen uniformly at random in $\Z_q^{2 \times 2}$, the proof $([\matr{\Theta}]_1, [\matr{\Pi}]_2)$ is uniformly distributed conditioned on satisfying check 2) of algorithm $\algV$.
Therefore, these elements of the simulated proof have the same distribution as in a real proof. This fact combined with the perfect zero-knowledge property of ${\Psi_\sfsum}$  and ${\Psi_\sfcom}$ concludes the proof. 
\end{proof}

The following theorem guarantees Soundness. 
 
\begin{theorem} Let $\mathsf{Adv}_{\mathcal{PS}}(\advA)$ 
be the advantage of an adversary $\advA$ against the soundness of 
the proof system  described above. There exist PPT adversaries
$\advB_1,\advB_2,\advB_3,\advSound_1,\advSound_2$ such that 
\begin{eqnarray*}
\mathsf{Adv}_{\mathcal{PS}}(\advA) & \leq 
n& \left(6/q+ \mathsf{Adv}_{\mathcal{U}_1,\Gr}(\advB_1)
+  \mathsf{Adv}_{\mathcal{U}_1,\Hr}(\advB_2)
+  \mathsf{Adv}_{\SP_{\Hr}}(\advB_3)\right. \\
& & \mbox{ } 
+  \left.\mathsf{Adv}_{{\Psi_\sfsum}}(\advSound_1)
+
 \mathsf{Adv}_{{\Psi_\sfcom}}(\advSound_2)\right).
\end{eqnarray*}
\label{teo:bitstr-soundness}
\end{theorem}

The proof follows from the indistinguishability of the following games:
\begin{itemize}
\item [$\mathsf{Real}$] This is the real soundness game. 
 The output is $1$ if  the adversary breaks the soundness,
i.e. the adversary submits
some $[\vecb{c}]_1 = [\matr{U}]_1\left(\begin{smallmatrix}\vecb{b}\\ w_g\end{smallmatrix}\right)$, for some
$\vecb{b}\notin \bits^n$ and $w \in \Z_q$, and
the corresponding proof which is accepted by the verifier.
\item[$\mathsf{Game}_0$] This game is identical to 
$\mathsf{Real}$ except that algorithm $\algK_1$ does not receive $[\matr{U}]_1$ as a input but it samples
$([\matr{U}]_1,\matr{U}) \in \mathcal{R}_{par}$
itself according to $\dist_{{gk}}$.
\item[$\mathsf{Game}_1$] This game is identical to 
$\mathsf{Game_0}$ except that the simulator picks a random $i^* \in [n]$, and uses $\matr{U}$ to check  
    if the output of the adversary $\advA$ is such that 
    $b_{i^*}\in \bits$.  It aborts if  $b_{i^*}\in \bits$.
\item[$\mathsf{Game}_{2}$] This game is identical to 
$\mathsf{Game}_1$ except that now the vectors $[\vecb{g}_{i}]_1$, $i \in [n]$ and $i \neq i^*$,
are uniform vectors in the space spanned by $[\vecb{g}_{n+1}]_1$.   
\item[$\mathsf{Game}_{3}$] This game is identical to 
$\mathsf{Game}_2$ except that now the vector $[\vecb{h}_{i^*}]_2$ is 
a uniform vector in $\Hr^2$, sampled independently of 
$[\vecb{h}_{n+1}]_2$.     
\end{itemize}
It is obvious that the first two games are indistinguishable. 
The rest of the argument goes as follows. 

\begin{lemma} $\Pr\left[ \mathsf{Game}_1(\advA)=1\right]\geq\dfrac{1}{n}\Pr\left[\mathsf{Game}_0(\advA)=1\right].$
\label{lemma:bits1}
\end{lemma}

\begin{proof}  The probability that
 $\mathsf{Game}_1(\advA)=1$ is the probability that  a) $\mathsf{Game}_0(\advA)=1$ and
b)  $b_{i^*} \notin \bits$. The view of adversary $\advA$ is independent of $i^*$, while, if $\mathsf{Game_0}(\advA)=1$, then there is at least one index $\ell \in [n]$ such that  
such that  $b_{\ell} \notin \bits$. Thus, 
the probability that the event described in b) occurs conditioned on $\mathsf{Game_0}(\advA)=1$, is greater than or equal to $1/n$ and the lemma follows.
\end{proof}

\begin{lemma} There exists a~$\mathcal{U}_1$-$\mddh_{\Gr}$ adversary $\advB$ such that
$|\Pr\left[\mathsf{Game}_{1}(\advA)=1\right]$ $-\Pr\left[\mathsf{Game}_{2}(\advA)=1\right]|$ $\leq
    \mathsf{Adv}_{\mathcal{U}_1,\Gr}(\advB) + 2/q.$
\label{lemma:bits2}
\end{lemma}
\begin{proof}
The adversary $\advB$ receives $([\vecb{s}]_1, [\vecb{t}]_1)$ an instance of the $\mathcal{U}_1$-$\mddh_{\Gr}$ problem.
$\advB$ defines all the parameters honestly except that
it embeds the $\mathcal{U}_1$-$\mddh_{\Gr}$ challenge in the matrix 
$\Lck$.

Let $[\matr{E}]_1:=([\vecb{s}]_1||[\vecb{t}]_1)$. $\advB$ picks $i^*\gets[n]$, $\matr{W}_0\gets\Z_q^{2\times(i^*-1)}$,
$\matr{W}_1\gets\Z_q^{2\times(n-i^*)}$,
$[\vecb{g}_{i^*}]_1\gets\Gr^{2}$,
and defines $\Lck := ([\matr{E}]_1\matr{W}_0||[\vecb{g}_{i^*}]_1||[\matr{E}]_1\matr{W}_1|| [\vecb{s}]_1)$. 
In the real algorithm $\mathsf{K}_1$, the generator picks the matrix $\matr{\Delta} \in \Z_q^{2 \times (n+1)}$.
Although $\advB$ does not know $\matr{\Delta}$,  it can compute $[\matr{\Delta}]_1$ as $[\matr{\Delta}]_1= \Lck\matr{U}^{-1}$,
given that $\matr{U}$ is full rank and was  sampled 
by $\advB$, so it can compute the rest of the elements of the
common reference string  using the discrete logarithms of $[\matr{U}]_1$, $\Rck$ and $[\vecb{a}]_2$.  

In case $[\vecb{t}]_1$ is uniform over $\Gr^{2}$, by the Schwartz-Zippel lemma $\det([\matr{E}]_1) = 0$ with probability at most $2/q$.
Thus, with probability at least $1-2/q$, the matrix $[\matr{E}]_1$ is full-rank and $\Lck$ is uniform over $\Gr^{2\times(n+1)}$ as in
$\sfGame_1$.
On the other hand, in case $[\vecb{t}]_1=\gamma [\vecb{s}]_1$, all of $[\vecb{g}_{i}]_1$, $i\neq i^*$, are in the space
spanned by $[\vecb{g}_{n+1}]_1$ as in $\sfGame_2$.
\end{proof}

\begin{lemma} There exists a $\mathcal{U}_1$-$\mddh_{\Hr}$ adversary $\advB$ such that
$|\Pr\left[\mathsf{Game}_{2}(\advA)=1\right]$ $-\Pr\left[\mathsf{Game}_{3}(\advA)=1\right]|$ $\leq
\mathsf{Adv}_{\mathcal{U}_1,\Hr}(\advB).$
\label{lemma:bits3}
\end{lemma}

\begin{proof}
The adversary $\advB$ receives an instance of the $\mathcal{U}_1$-$\mddh_{\GG_2}$ problem, which is a pair
$([\vecb{s}]_2, [\vecb{t}]_2)$, where $[\vecb{s}]_2$ is a uniform vector 
of $\Hr^{2}$ and $[\vecb{t}]_2$ is either a uniform vector in $\Hr^2$ or 
$[\vecb{t}]_2=\gamma[\vecb{s}]_2$, for random $\gamma\in\Z_q$.      
 
Adversary $\advB$ defines 
$[\vecb{h}_{n+1}]_2:= [\vecb{s}]_2$ and the rest of the columns of $\Rck$ are honestly sampled
with the sole exception of $[\vecb{h}_{i^*}]_2$, which is set to $[\vecb{t}]_2$.

Given that adversary $\advB$ can only compute $\llck_{i}[\vecb{h}_j^\top]_2\in\Hr^{2\times2}$,
it defines $[\matr{D}_{i,j}]_2 := \llck_{i}[\vecb{h}_j^\top]_2 - [\matr{T}_{i,j}]_2$ and
$[\matr{C}_{i,j}]_1:=[\matr{T}_{i,j}]_1$, for $\matr{T}_{i,j}\gets\Z_q^{2\times 2}$ and $(i,j)\in\indexSet{n}{1}$. Note 
that this does not change the distribution of $([\matr{D}_{i,j}]_2,[\matr{C}_{i,j}]_1)$, which is the uniform one conditioned
on $\matr{C}_{i,j}+\matr{D}_{i,j}= \llck_i\lrck_j^\top.$

The rest of the parameters are computed using $\vecb{a}\gets\distlin_1$,
the matrix $\matr{\Delta} \in \Z_q^{2 \times (n+1)}$ and the discrete logarithms
of $\Lck$.
It is immediate to see that adversary $\advB$ perfectly simulates $\sfGame_2$ when $[\vecb{t}]_2=\gamma[\vecb{s}]_2$ and $\sfGame_3$ when $[\vecb{t}]_2$ is uniform.  
\end{proof}

\begin{lemma} There exists a $\SP_{\Hr}$ adversary $\mathcal{\advB}$, a soundness adversary $\advSound_1$  for ${\Psi_\sfsum}$ and 
a strong soundness adversary $\advSound_2$ for ${\Psi_\sfcom}$  such that
$$\Pr\left[\mathsf{Game}_3(\advA)=1\right] \leq 4/q + \adv_{\SP_{\Hr}}(\advB) +
\adv_{\Psi_\sfsum}(\advSound_1)+\adv_{\Psi_\sfcom}(\advSound_2).$$  
\label{lemma:G3}
\end{lemma}
\begin{proof}
$\Pr[\det((\llck_{i^*}||\llck_{n+1}))=0]=\Pr[\det((\lrck_{i^*}||\lrck_{n+1}))=0]\leq2/q$, by the Schwartz-Zippel lemma. Then, with probability at least $1-4/q$, $\llck_{i^*}\lrck_{i^*}^\top$ is linearly independent from
$\{\llck_{i}\lrck_j^\top:(i,j)\in[n+1]^2\setminus\{(i^*, i^*)\}\}$ which implies that $\llck_{i^*}\lrck_{i^*}^\top\notin\Span(\{\matr{C}_{i,j}+\matr{D}_{i,j}:(i,j)\in\indexSet{n}{1}\})$. 
Additionally  $\sfGame_3(\advA)=1$ implies that $b_{i^*} \notin \{0,1\}$
while the verifier accepts the proof  produced by $\advA$, which is
$ (
        [\vecb{c}_{\Delta}]_1, [\vecb{d}]_2,
        ([\matr{\Theta}]_1, [\matr{\Pi}]_2), 
        \pi_\sfsum,\pi_\sfcom
).$ Since $\{[\vecb{h}_{i^*}]_2,[\vecb{h}_{n+1}]_2\}$ is a basis of $\Hr^2$,
we can define $\overline{w}_h,\overline{b}_{i^*}$ as the unique coefficients in $\Z_q$ such that $[\vecb{d}]_2= \overline{b}_{i^*} [\vecb{h}_{i^*}]_2 + \overline{w}_h [\vecb{h}_{n+1}]_2$.
We distinguish three cases:
\begin{itemize}
\item[1)] If $[\vecb{c}_{\Delta}]_1 \neq \matr{\Delta} [\vecb{c}]_1$, we can construct an adversary 
$\advB$ against the $\SP_{\Hr}$ Assumption that outputs 
$[\vecb{c}_{\Delta}]_1-\matr{\Delta} [\vecb{c}]_1\in\ker([\matr{a}]_2^\top)$.
\item[2)] If $[\vecb{c}_{\Delta}]_1 = \matr{\Delta} [\vecb{c}]_1$ but $b_{i^*} \neq \overline{b}_{i^*}$. Given that $[\vecb{c}]_1$ is perfectly binding and that $\bb_{i^*}\neq b_{i^*}$ is the unique opening of $[\vecb{c}_\Delta]_1$ at coordinate $i^*$, both commitments can not be opened to the same value. Therefore, the adversary $\advSound_2$ against the strong soundness of ${\Psi_\sfcom}$
outputs $\pi_\sfcom$ which is a fake proof for 
$([\vecb{c}_\Delta]_1,[\vecb{d}]_2)$. Note that strong soundness is required since, in order to compute $\{[\matr{C}_{i,j}]_1,[\matr{D}_{i,j}]_2:(i,j)\in\indexSet{n}{1}\}$, $\advSound_2$ requires the discrete logs of either $[\matr{G}]_1$ or $[\matr{D}]_2$.
\item[3)] If $[\vecb{c}_{\Delta}]_1 = \matr{\Delta} [\vecb{c}]_1$ and $b_{i^*} = \overline{b}_{i^*}$, then 
$b_{i^*}(\overline{b}_{i^*} -1) \neq 0$.
If we express $\matr{\Theta}+\matr{\Pi}$
as a linear combination of $\llck_{i}\lrck_{j}^{\top}$, the coordinate of
$\llck_{i^*}\lrck_{i^*}^\top$ is $b_{i^*}(\overline{b}_{i^*}-1)\neq 0$ and thus $\matr{\Theta}+\matr{\Pi}\notin\Span(\{\matr{C}_{i,j}+\matr{D}_{i,j}:(i,j)\in\indexSet{n}{1}\})$.
The adversary $\advSound_1$ against ${\Psi_\sfsum}$  outputs  $\pi_\sfsum$
which is a fake proof for $([\matr{\Theta}]_1, [\matr{\Pi}]_2)$. \footnote{The proof system $\Psi_\sfsum$ is constructed for matrices $\{(\matr{C}_{i,j},\matr{D}_{i,j}):(i,j)\in\indexSet{n}{1}\}$ sampled from some distribution $\dist_{{gk}}$, which in this case depends on the distribution of $\matr{G}$ and $\matr{H}$. We assume that the adversary $\advSound_2$ against $\Psi_\sfsum$ receives the common reference string of $\Psi_\sfsum$ as described in Section \ref{sec:sum} and additionally the matrices $[\matr{G}]_1$ and $[\matr{H}]_2$ which defines the language, i.e. the distribution of $\matr{C}_{i,j}$, $\matr{D}_{i,j}$ (this is necessary so that $\advSound_2$ can simulate the crs for adversary $\advA$). We stress this additional information to describe the language does not affect the soundness proof for Theorem \ref{theo:membtwogroups1} (in particular,  $[\matr{G}]_1$ and $[\matr{H}]_2$ are independent of $(\matr{\Lambda}, \matr{\Xi})$).}
\end{itemize}
\end{proof}
