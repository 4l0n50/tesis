Many protocols use proofs that a commitment opens to a bit-string as a 
  building block. 
Since our commitments are still of size $\Theta(n)$,
  our results may not apply to some of these protocols. 
Yet, there are several applications where 
  bits need to be used independently and our results provide 
  significant improvements.
Table \ref{table:app} summarizes them.
 
\subsubsection{Signatures} Some application examples are the signature schemes of  \cite{PKC:BFPV11,SCN:BlaPoiVer12,RSA:Camacho13,AFRICACRYPT:EscHerMor11}. For example, in the revocable attribute-based signature scheme of Escala \textit{et. al} \cite{AFRICACRYPT:EscHerMor11}, every signature includes a proof that a set of GS commitments, whose size is the number of attributes, opens to a bit-string. 

Further, the set membership proof discussed below can also be used to reduce the size of the Ring Signature scheme of  \cite{ICALP:ChaGroSah07}, which is the most efficient ring signature in the standard model. Indeed, to sign a message $m$, among other things, the signer picks a one-time signature key and certifies the one-time verification key by signing it with a Boneh-Boyen signature under $vk_\alpha$. Then, the signer commits to $vk_\alpha$ and shows that $vk_\alpha$ belongs to the set of Boneh-Boyen verification keys $(vk_1,\ldots,vk_n)$ of the parties in the ring $R$. 


\begin{table}[h]
\begin{center}
\begin{minipage}{\textwidth}
\begin{center}
\begin{small}
             %   proof  % CRS   %
% group type %  Gr % Hr % Gr Hr % #Pairings
\begin{tabular}{|l||l|l|}
\hline
Proof System & Author                                                    & Proof Size 
\\ \hline\hline
\multirow{3}{*}{Threshold GS} & R\`afols \cite{TCC:Rafols15} (1) & $(m_x + 3(n-t) + 2\bar{n},0)$     \\
\cline{2-3}                   & R\`afols \cite{TCC:Rafols15} (2) & $2(n-t+1,n)$         \\
\cline{2-3}                   & This work                         & $(2n+12,10)$             \\
\hline
\multirow{3}{*}{\minitbl{Set-Membership proof}{(Ring Signature)}} & Chandran et al. \cite{ICALP:ChaGroSah07}  
&  $(16\sqrt{n}+4,16\sqrt{n}+4)$    \\
\cline{2-3}
                                         & R\`afols \cite{TCC:Rafols15}              &  $(8\sqrt{n}+6,12\sqrt{n})$    \\
\cline{2-3} 
                                         & This work                                 & $(4 \sqrt{n}+14,8\sqrt{n}+14)$ \\ \hline
\multirow{2}{*}{\minitbl{Set-Membership proof}{(fixed set)}}  & This work (first scheme)                  & $(4\sqrt{n}+16,2\sqrt{n}+22)$\\
\cline{2-3}
                                         & This work (second scheme)                 & {$(6 \sqrt[3]{n}+36,6\sqrt[3]{n}+60)$}\\ \hline
% using proof of memb in a list again \minitbl{$(8 \sqrt[3]{n}+6 \sqrt[6]{n}+28)\sG+$}{$(2\sqrt[3]{n}+6\sqrt[6]{n}+38)\sH$}
\end{tabular}
\end{small}
\end{center}
\caption{Comparison of the application of our techniques and results from the literature. Notation $(a,b)$ means $a$ elements of $\GG_1$ and $b$ elements of $\GG_2$. In rows labeled  as ``Threshold GS'' 
we give the size of the proof of satisfiability of $t$-out-of-$n$ sets $\mathcal{S}_i$, where $m_x$ is the sum of the number of variables in $\Gr$ in each set $\mathcal{S}_i$, and $\bar{n}$ is the total number of two-sided and quadratic equations in $\bigcup_{i\in[n]}\mathcal{S}_i$. For all rows, we must add to the proof size the cost of a GS proof of each equation in one of the sets $\mathcal{S}_i$. In the other rows $n$ is the size of the set.\label{table:app}}
\end{minipage}
\end{center}
\end{table}


\subsubsection{Threshold GS Proofs for PPEs} There are two approaches to construct threshold GS proofs for PPEs, i.e. proofs of satisfiability of $t$-out-of-$n$ equations. One is due to \cite{AC:Groth06} and consists of compiling the $n$ equations into a single equation which is satisfied only if $t$ of the original equations are satisfied. For the case of PPEs, this method adds new variables and proves that each of them opens to a bit.  Our result reduces the cost of this approach, but we omit any further discussion as it is quite inefficient because the number of additional variables is $\Theta(m_{var}+n)$, where $m_{var}$ is the total number of variables in the original $n$ equations.

The second approach is due to R\`afols \cite{TCC:Rafols15}. The basic idea behind \cite{TCC:Rafols15}, which extends \cite{C:GroOstSah06}, follows from the observation that for each GS equation type 
$\mathsf{tp}$,  the CRS space $\mathcal{K}$ is partitioned into a perfectly sound CRS space $\mathcal{K}^b_\mathsf{tp}$ and a perfectly witness indistinguishable CRS space $\mathcal{K}^{h}_\mathsf{tp}$.

In particular, to prove satisfiability of $t$-out-of-$n$ sets of equations from $\{\mathcal{S}_i:i\in[n]\}$ of type $\mathsf{tp}$, it suffices to construct an algorithm $\mathsf{K}_{\mathsf{corr}}$ which on input
$\mathsf{crs}_{\GS}$ and some set of indexes $A \subset [n]$, $|A|=t$, generates $n$ GS common reference strings 
$\{\mathsf{crs}_i, i \in [n]\}$ and simulation trapdoors $\tau_{i,sim}$, $i \in A^c$, in a such a way that\footnote{More technically, this is the notion of \textit{Simulatable Verifiable Correlated Key Generation} in \cite{TCC:Rafols15}, which extends the definition of Verifiable Correlated Key Generation of \cite{C:GroOstSah06}.}:
\begin{itemize}
\item[a)] it can be publicly verified the set of perfectly sound keys, 
$\{\mathsf{crs}_i : \mathsf{crs}_i \in \mathcal{K}_{\mathsf{tp}}^{b}\}$ is of size at least $t$,
\item [b)] there exists a simulator $\mathsf{S}_{\mathsf{corr}}$ who outputs $(\mathsf{crs}_i,\tau_{i,sim})$ for all $i \in [n]$, and the distribution of $\{\mathsf{crs}_i : i \in [n]\}$ is the same as the one of the keys  output by $\mathsf{K}_{\mathsf{corr}}$ when $\mathsf{crs}_{\GS}$ is the perfectly witness-indistinguishable CRS.
\end{itemize}
The prover of $t$-out-of-$n$ satisfiability can run $\mathsf{K}_{\mathsf{corr}}$ and, for all $i \in [n]$, compute a real (resp. simulated) proof for 
$\mathcal{S}_i$ with respect to $\mathsf{crs}_i$
when $i \in A$ (resp. when $i \in A^c$).

R\`afols gives two constructions for PPEs, the first one can be found in \cite{TCC:Rafols15}, App. C and the other follows from \cite[Sect.~7]{TCC:Rafols15}\footnote{The construction in \cite[Sect.~7]{TCC:Rafols15} is for other equation types but can be used to prove that $t$-out-of-$n$ of $\mathsf{crs}_1,\ldots,\mathsf{crs}_n$ are perfectly binding for PPEs.
}. 
Our algorithm $\mathsf{K}_\mathsf{corr}$ for PPEs\footnote{Properly speaking the construction is for PPEs which are left-simulatable in the terminology of \cite{TCC:Rafols15}.} goes as follows:

\begin{itemize}
\item Define $(b_1,\ldots,b_n)$ as $b_i=1$
if $i \in A$ and $b_i=0$ if $i \in A^c$. For all $i \in [n]$, let $[\vecb{z}_i]_1:=\mathsf{GS.Comm}_{ck}(b_i)=b_i [\vecb{u}_1]_1+ r_i [\vecb{u}_2]_1$, $r_i \in \Z_q$, and define $\tau_{sim,i}=r_i$, for all $i \in A^c$. Define $\mathsf{crs}_i:=(\Gamma,[\vecb{z}_i]_1,[\vecb{u}_2]_1,[\vecb{v}_1]_2,[\vecb{v}_2]_2)$. 
\item Prove that $([\vecb{c}_1]_1,\ldots,[\vecb{c}_n]_1)$ opens to $\vecb{b} \in \{0,1\}^{n}$ and that $\sum_{i\in[n]} b_i=t$.
\end{itemize}
The simulator just defines $\vecb{b}=\vecb{0}$. The reason why this works is that when $b_i=1$, $([\vecb{z}_i]_1-[\vecb{u}_1]_1) \in \mathsf{Span}( [\vecb{u}_2]_1)$, therefore $\mathsf{crs}_i \in \mathcal{K}^{b}_{PPE}$ and when
$b_i=0$,  $([\vecb{z}_i]_1-[\vecb{u}_1]_1) \notin \mathsf{Span}( [\vecb{u}_2]_1)$ so $\mathsf{crs}_i \in \mathcal{K}^{h}_{PPE}$.

\subsubsection{More Efficient Set-Membership Proofs} Chandran \textit{et al.} construct a ring signature of size $\Theta(\sqrt{n})$ \cite{ICALP:ChaGroSah07}, which is the most efficient ring signature in the standard model. Their construction uses as a subroutine a non-interactive proof of membership in some set $S=([s_1]_1,\ldots,[s_n]_1)$ which is of size $\Theta(\sqrt{n})$.  The trick of Chandran \textit{et al.} to achieve this asymptotic complexity is to view $S$ as a matrix $[\matr{S}]_1\in\Gr^{m\times m}$, for $m=\sqrt{n}$, where the $i,j$ th element of $[\matr{S}]_1$ is $[s_{i,j}]_1 := [s_{(i,j)}]_1$ and $(i,j):=(i-1)m+j$. Given a commitment $[\vecb{c}]_1$ to some element $[s_{\alpha}]_1$, where $\alpha = (i_\alpha,j_\alpha)$, their construction in asymmetric bilinear groups works as follows :
\begin{enumerate}
\item Compute GS commitments in $\Hr$ to $b_1\ldots,b_m$ and $b'_1,\ldots,b'_m$,
      where $b_i = 1$ if $i=i_\alpha$ and $0$ otherwise, and $b'_{j}=1$ if $j=j_\alpha$, and $0$ otherwise.
\item Compute a GS proof that $b_i \in \{0,1\}$ and $b'_j \in \{0,1\}$ for all $i,j\in[m]$, and that $\sum_{i\in[m]}b_i=1$, and $\sum_{j\in[m]}b'_j=1$.
\item Compute GS commitments to $[x_1]_1:=[s_{(i_\alpha,1)}]_1,\ldots,[x_m]_1:=[s_{(i_\alpha,m)}]_1$.
\item Compute a GS proof that $[x_j]_1 = \sum_{i\in[m]}b_i[s_{(i,j)}]_1$, for all $j\in[m]$, is satisfied. 
\item Compute a GS proof that $[s_{\alpha}]_1 = \sum_{j\in[m]}b'_j[x_j]_1$ is satisfied.
\end{enumerate}
With respect to the naive use of GS proofs, Step $2$ was improved by R\`afols \cite{TCC:Rafols15}.  Using our proofs for bit-strings of weight $1$ from Sect. \ref{sec:bits-extensions}, we can further reduce the size of the proof in step 2, 
see table.

We note that although in step 4 the equations are all two-sided linear equations, proofs can only be aggregated if the set comes from a witness samplable distribution and the CRS is set to depend on that specific set. 
This is not useful for the application 
to ring signatures, since the CRS should be independent of the ring $R$ (which defines the set). If aggregation is possible then the size of the proof in step 4 is reduced from $(2\sG+4\sH)\sqrt{n}$ to $4\sG+8\sH$.

Next we describe the construction generalized to vectors (which will be useful in the next construction) and then we show that,
when the CRS depends on the set and the set is witness samplable, the proof can be further reduced to $\Theta(\sqrt[3]{n})$.

\subsubsection{More Efficient Set-Membership proof for Set of Vectors}  
Here we describe a more efficient version of Chandran's \textit{et al.} Set-Membership proof,  
which is also extended to vectors --i.e. to the case where $S=([\vecb{s}_1]_1,\ldots,[\vecb{s}_n]_1)$ is a set of vectors of length $\ell$. In such a proof, we show that some commitment $[\vecb{c}]_1$ opens to a vector $[\vecb{s}_{\alpha}]_1$, where $\alpha=(i_\alpha,j_\alpha)$ (recall that $(i,j)=\sqrt{n}(i-1)+j$).

\begin{enumerate}
\item Compute GS commitments in $\Hr$ to $b_1\ldots,b_m$ and $b'_1,\ldots,b'_m$,
      where $b_i = 1$ if $i=i_\alpha$ and $0$ otherwise, and $b'_{j}=1$ if $j=j_\alpha$, and $0$ otherwise.
\item Compute a proof that $b_i \in \{0,1\}$ and $b'_j \in \{0,1\}$ for all $i,j\in[m]$, using the proof system of Sect.
~ \ref{sec:bits-scheme}.
\item Compute GS proofs that $\sum_{i\in[m]}b_i=1$ and $\sum_{j\in[m]}b'_j=1$. 
\item Compute GS commitments to each coordinate of $[\vecb{x}_1]_1:=[\vecb{s}_{(i_\alpha,1)}]_1,\ldots,[\vecb{x}_m]_1:=[\vecb{s}_{(i_\alpha,m)}]_1$.
\item Compute an aggregated GS proof that the equations $[\vecb{x}_j]_1 = \sum_{i\in[m]}b_i[\vecb{s}_{(i,j)}]_1$, for all $j\in[m]$, are satisfied, as detailed in Sect.~\ref{sec:aggcommit}.
\item Compute a GS proof that $[\vecb{s}_{\alpha}]_1 = \sum_{j\in[m]}b'_j[\vecb{x}_j]_1$ is satisfied.
\end{enumerate}

We emphasize that the CRS depends on the set $S$. This is necessary to aggregate the proofs as in Step 4. More  specifically, to aggregate the proof of the equations  $[\vecb{x}_j]_1 = \sum_{i\in[m]}b_i [\vecb{s}_{(i,j)}]_1$ , $j\in[m]$ (that is, a total of $\ell k$ equations), we need to include in the CRS some information which depends on the coordinates of $[\vecb{s}_{(i,j)}]_1$.

\begin{theorem}\label{theo:listsquarevect}
If $S$ is witness samplable, the above protocol is a perfectly complete, computationally sound, and computationally zero-knowledge proof system for the language of commitments to elements from the set $S$.
\end{theorem}

\begin{proof} Completeness follows directly from the completeness of the 
building blocks. Soundness follows directly from the perfect soundness of GS proofs together with the computational soundness of aggregation of GS proofs. For computational zero-knowledge, if  
$\crs_\GS:=(\Gamma,[\vecb{u}_1]_1,[\vecb{u}_2]_1,[\vecb{v}_1]_2,[\vecb{v}_2]_2)$ is the GS common reference string in the soundness setting as defined in Sect.~ \ref{sec:gs-proofs}, switch to a game where $[\vecb{v}_1]_2= \epsilon [\vecb{v}_2]_2$. Under the DDH Assumption in $\Hr$, the new CRS is computationally indistinguishable from the original CRS. In a simulated proof, commit to $b_i=0$, $b'_j=0$ for all $i,j \in [m]$. In step 2, simply compute a real proof. In step 3, use the GS simulation algorithm  (with trapdoor $\epsilon$) to simulate the proof. In Step 4, set $[\vecb{x}_j]_1=[\vecb{0}]_1$. Finally, in step 6, simulate a proof using $\epsilon$. It is not hard to see that such a proof can be simulated even without knowledge of an opening of $[\vecb{c}]_1$. 
\end{proof}

\subsubsection{A $\Theta(\sqrt[3]{n})$ Set-Membership proof for Witness Samplable non-Adaptive Sets} We give Set-Membership proof with improved asymptotic proof size when the set is drawn from a witness samplable distribution and the CRS depends on the set (two different sets require two different CRS).

The main idea is to combine the previous Set-Membership proof with a Split Kernel Assumption. Specifically, the CRS includes a matrix $[\matr{A}]_2$, $\matr{A} \gets\dist_{m,2}$, whose rows are denoted $[\grkb{a}_1]_2,\ldots,[\grkb{a}_m]_2$ and a set 
$$S':=\left(\sum_{i\in[m]}\grkb{a}_i[s_{(i,1,1)}]_1,\sum_{i\in[m]}\grkb{a}_i[s_{(i,1,2)}]_1,\ldots,\sum_{i\in[m]}\grkb{a}_i[s_{(i,m,m)}]_1\right)\in\Gr^{2\times m^2},$$
where $m:=\sqrt[3]{n}$, $(i,j,k)=m^2(i-1)+m(j-1)+k\in[n]$. As before, the goal to prove is that a commitment $[\vecb{c}]_1$ opens to some $[s_{\alpha}]_1 \in S=\{[s_1]_1,\ldots,[s_n]_1\}$ and  $(i_{\alpha},j_{\alpha},k_{\alpha})$ are such that $\alpha=(i_{\alpha},j_{\alpha},k_{\alpha})$.

\begin{enumerate}
\item Commit to $[\vecb{y}]_1:=\sum_{i\in[m]}\grkb{a}_i[s_{(i,j_\alpha,k_\alpha)}]_1$ such that $\alpha=(i_\alpha,j_\alpha,k_\alpha)$.
\item Using the Set-Membership proof for vectors to show that $[\vecb{y}]_1$ is an element of $S'$.
\item Compute commitments to $[z_i]_1:=[s_{(i,j_\alpha,k_\alpha)}]_1$, for each $i\in[m]$.
\item Compute a GS proof for the equations $[\vecb{y}]_1[h]_2=\sum_{i\in[n]}[\grkb{a}_i]_2 [z_i]_1$. 
\item Compute GS commitments in $\Hr$ to $b_1\ldots,b_m\in\bits$,
      where $b_i = 1$ if $i=i_\alpha$ and $0$ otherwise.
\item Using our proof system from Sect. \ref{sec:bits-scheme} prove that $b_i\in\bits$ for all $i\in[m]$.
\item Compute GS proofs for the satisfiability of equations $\sum_{i\in[m]}b_i=1$ and $[s_{\alpha}]_1=\sum_{i\in[m]}b_i[z_i]_1$.
\end{enumerate}

The first step is to commit to $[\vecb{y}]_1:= \sum_{i\in[m]} \grkb{a}_i [s_{(i,j_\alpha,k_\alpha)}]_1$ and use the previous proof system to prove $[\vecb{y}]_1 \in S'$. The next step is to commit to $[z_i]_1:=[s_{(i,j_\alpha,k_\alpha)}]_1$ and prove that $\sum_{i\in[m]}[\grkb{a}_i]_2[z_i]_1 = [h]_2[\vecb{y}]_1$ holds. Finally, steps 5 and 6 prove that $[s_{\alpha}]_1$ is an element of the set $([z_1]_1,\ldots,[z_m]_1)$. For the last statement, compute GS commitments to $b_i$, $i \in [m]$, 
and prove that $\sum_{i\in[m]}b_i[z_i]_1=[s_{\alpha}]_1$, $\sum_{i\in[m]}b_i=1$ and $b_i \in \{0,1\}$.\footnote{Such statement can also be proven using again the Set-Membership proof, and the proof will be of size $\Theta(\sqrt[6]{n})$. Note this is not exactly a Set-Membership proof, since only the commitments to the elements in the set are public. However, it is not hard to construct a proof system for that statement using the same ideas as Chandran \textit{et al.} }

\begin{theorem}
If $S$ is witness samplable, the above protocol is a perfectly complete, computationally sound, and computationally zero-knowledge proof system for the language of commitments to elements from the set $S$.
\end{theorem}

\begin{proof} Completeness follows directly from the completeness of the 
building blocks.  Soundness can be argued as follows. If the set is witness samplable, the CRS can be generated given an instance of the $\dist_{m,2}\mbox{-}\skermdh$ Assumption, $([\matr{A}]_1, [\matr{A}]_2)$. By the soundness of the extension to vectors of the proof of Chandran \textit{et al.}, it holds that $[\vecb{y}]_1 = \sum_{i\in[m]}\grkb{a}_i[s_{(i,j,k)}]_1$ for some $j,k\in[m]$.
Because of the perfect soundness of GS proofs it must hold that $\sum_{i\in[m]}[\grkb{a}_i]_2[z_i]_1=[\vecb{y}]_1[h]_2=\sum_{i\in[m]}[\grkb{a}_i]_2[s_{(i,j,k)}]_1$. It must also be the case that $[z_1]_1=[s_{(1,j,k)}]_1,\allowbreak\ldots,[z_m]_1=[s_{(m,j,k)}]_1$, because otherwise the pair $([\grkb{\rho}]_1,[\grkb{0}]_2)$, where 
$[\grkb{\rho}]_1:=([z_1]_1-[s_{(1,j,k)}]_1,\allowbreak\ldots,[z_m]_1-[s_{(m,j,k)}]_1)$ is a solution to the $\dist_{m,2}\mbox{-}\skermdh$ challenge, as $[\grkb{\rho}]_1[\matr{A}]_2=[\grkb{0}]_2[\matr{A}]_1$. Soundness of the last step implies that $b_i\in\bits$, for all $i\in[m]$, and that $\sum_{i\in[m]} b_i=1$. Therefore, there exists a unique $i\in[m]$ such that $b_i=1$. Finally, $[s_{\alpha}]_1=\sum_{i\in[m]}b_i[z_i]_1$ implies that $[\vecb{c}]_1$ opens to $[s_{\alpha}]_1=[z_i]_1=[s_{(i,j,k)}]_1$.
Zero-knowledge follows from the same argument as in the proof of Theorem \ref{theo:listsquarevect}.
\end{proof} 

