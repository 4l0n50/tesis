\label{sec:instantiations}
In this chapter we construct a constant-size QA-NIZK for proving that a commitment of a vector of integers opens to a 
an element of $\{0,1\}^n$ for some fixed $n\in\mathbb{N}$. We distinguish two cases: perfectly binding commitments, were all the information of the opening is preserved within the commitment, and length-reducing commitments, where some information is definitively lost.

First, we consider length-reducing commitments that maps elements in $\mathbb{Z}_q^n$ to elements of $\GG_1^2$ and the information of one and only one coordinate is preserved. With this tool we show how to construct QA-NIZK proofs for perfectly binding case in Sect. \ref{sec:bits-perfbinding} and other applications in Sect. \ref{sec:bits-applications}. Then in Chapt. \ref{chap:listas} we show how to construct \emph{Aggregated Set-Membership} NIZK proofs, and Shuffle and Range NIZK proofs.

We construct a QA-NIZK argument of membership in the language
$$
\Lang_{ck,\sfbits} := \{[\vecb{c}]_1\in\GG_1^{k+1} : \exists \vecb{b}\in\bits^m,\vecb{r}\in\Z_q^k \text{ s.t. } [\vecb{c}]_1 = \MP.\Com_{ck}(\vecb{b};\vecb{r})\},
$$
where $ck:=[\matr{G}]_1$ and $\matr{G}$ is a matrix sampled from 
some distribution $\distink$, as defined in Sect. \ref{sec:mddh}. For simplicity, in the exposition we restrict ourselves to the case $\dist_k=\distlin_{1}$ so  $\matr{G}$ is sampled from $\distlininone$, for some $0 \leq i \leq m$.

To prove that a commitment in $\Gr$ opens to a vector of bits $\vecb{b}$, the usual strategy is to compute another commitment $[\vecb{d}]_2\in\Hr^{\bar{n}}$ to a vector $\bar{\vecb{b}}\in\Z_q^n$ and prove 
  (1) $b_i(\overline{b}_i-1)=0$, for all $i \in [n]$, and 
  (2) $b_i-\overline{b}_i=0$, for all $i \in [n]$. 
For statement  (2), since $[\matr{G}]_1$ is witness samplable, we can use our most efficient QA-NIZK from Sect.\ \ref{sec:aggcomms} for equal opening in different groups.  Under the $\SSDP$ Assumption, which is the $\skermdh$ Assumption of minimal size conjectured to hold in asymmetric groups, the proof is of size $2\s$. Thus, the challenge is to aggregate $n$ equations of the form $b_i(\overline{b}_i-1)=0$. We note that this is a particular case of the problem of aggregating proofs of quadratic equations, which was left open in \cite{C:JutRoy14}.

%We finally remark that the proof must include $\cvecb{d}$ and thus it may be not of size independent of $n$. However, it turns out that $\cvecb{d}$ needs not be perfectly binding, in fact $\bar{n}=2$ suffices.



