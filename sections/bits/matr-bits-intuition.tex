We would like to prove that some tuple
$([\vecb{c}_1]_1,\ldots, [\vecb{c}_n]_1) \in \Lang_{ck,\sfbits}^n$, where $[\vecb{c}_j]_1=\MP.\Com_{ck}(\vecb{b}_j;\vecb{r}_j)$, $j \in [n]$, $ck:=[\matr{G}]_1$, and $\matr{G} \gets \distink$, for some $0 \leq i \leq n$. 
Denote $\vecb{c}=\vecb{c}_1 \dsum \ldots \dsum \vecb{c}_n$, $\vecb{b}=\vecb{b}_1 \dsum \ldots \dsum \vecb{b}_n$ and $\vecb{r}=\vecb{r}_1 \dsum \ldots \dsum \vecb{r}_n$ (concatenation of column vectors as defined in Section~\ref{sec:notation}). The proof system works as follows:
\begin{enumerate}
\item It defines  
 $\overline{ck}:=[\overline{\matr{G}}]_1\gets\MP.\algK(gk,mn,\Lang_{1}^{mn,0})$ for computing multi-Pedersen commitments to vectors of size $mn$

\item it computes $[\ovecb{c}]_1\gets\MP.\Com_{\overline{ck}}(\vecb{b};\vecb{s})$ for some 
randomness $\vecb{s}$, and proves that $[\ovecb{c}]_1\in\Lang_{\overline{ck},\sfbits}$ with the proof system $\Pi_\sfbits$,
\item it  proves that there exists an equal opening of $[\vecb{c}]_1$ and $[\ovecb{c}]_1$ with  $\Pi_\sfcom$.
\end{enumerate}
First, note that in step 3, we can use the proof system $\Pi_\sfcom$, as both $[\vecb{c}]_1$ and $[\ovecb{c}]_1$ are commitments of the required form, as if $\matr{G}_2$ denotes the last $k$ columns 
of $\matr{G}$ and $\matr{G}_1$ the rest,
$\vecb{c}=\pdsum{\matr{G}_1}{n}\vecb{b}+\pdsum{\matr{G}_2}{n}\vecb{r}.
$
We give some intuition on why is the above scheme sound. For the case $\matr{G} \gets \distzeronk$ it suffices to note that the proof that  $[\vecb{c}]_1$ and $[\ovecb{c}]_1$ share an opening implies in particular that they are both valid commitments. But if 
$[\ovecb{c}]_1$ is a valid commitment then $([\vecb{c}_1]_1,\ldots, [\vecb{c}_n]_1) \in \Lang_{ck,\sfbits}^n$ because for this distribution the commitments are perfectly hiding. 

For the case where $\matr{G} \gets \distink$, $i>0$, 
recall that the MP commitment with this key is 
perfectly binding at coordinate $i$. In particular, this implies that if some $[\vecb{c}_j]_1 \notin \Lang_{ck,\sfbits}$, the $ith$ coordinate of $\vecb{b}_j$, denoted $b_{i,j}$, satisfies that $b_{i,j} \notin \{0,1\}$. Therefore, given some $j^* \gets [n]$, if the adversary breaks soundness, then, with probability at least $1/n$, $b_{i,j^*} \notin \{0,1\}$. In the soundness proof, we switch to a game where the distribution of $\overline{\matr{G}}$ is changed so that now 
$\MP.\Com_{\overline{ck}}$ is perfectly binding for $b_{i,j^*}$. Now it is easy to prove that if $b_{i,j^*} \notin \{0,1\}$, the soundness of $\Pi_\sfbits$ or of $\Pi_\sfcom$ is broken, because this is incompatible with $[\vecb{c}]_1$ and $[\ovecb{c}]_1$ sharing an opening and $[\ovecb{c}]_1 \in\Lang_{\overline{ck},\sfbits}$.
