In this Section we introduce a new commitment scheme which is a generalization of Multi-Pedersen commitments and which was implicitly used in Sect. \ref{sec:bits-scheme}. 

Given a vector $\vecb{m}\in\Z_q^m$, the Multi-Pedersen commitment in $\GG_{\gamma}$ is a single group element $[c]_\gamma:=\sum_{i\in [m]} m_i[g_i]_\gamma+r[g_{n+1}]_\gamma \in\GG_\gamma$, where $[g_i]_{\gamma}\in\GG_\gamma$, $i\in[m+1]$, and $r\gets\Z_q$. \footnote{Written in the usual multiplicative notation $c=\prod_{i\in[m]}g_i^{m_i} \cdot g_{m+1}^r$.}  The $(k+1)$-dimensional Multi-Pedersen commitments 
differs only in that the keys and the resulting commitments are in 
$\GG_{\gamma}^{k+1}$, for $k\geq 1$. 


While the original MP commitments are perfectly hiding, the interest of the new commitments is that, if the keys come from the distribution $\distink$ defined in Sect. \ref{sec:mddh}, they are perfectly binding at coordinate $i$. Intuitively, the new commitment is defined in a larger space so that not all the information about the witness is destroyed (in an information-theoretic sense). 

\begin{definition} The $(k+1)$-dimensional Multi-Pedersen commitment scheme in the group $\GG_\gamma$ 
%is parameterized by a matrix distribution $\dist_{2,n+1}$ and 
is specified by the following three algorithms 
	$\mathsf{MP}=(\mathsf{MP}.\algK,\mathsf{MP}.\Com,$ $ \mathsf{MP}.\algVrfy)$:
	\begin{itemize} 
		\item  $\mathsf{MP}.\algK$ is a randomized algorithm, which on input the group key $gk$, a natural number $n \in \N$, and the description of some matrix distribution $\dist_{k+1,n+k}$, 
		outputs a commitment key $ck:=[\matr{G}]_\gamma$, where $\matr{G} \gets \dist_{k+1,n+k}$.
		\item $\mathsf{MP}.\Com$ is a randomized algorithm which, on input a commitment key $ck=[\matr{G}]_\gamma$, and a message 
		$\vecb{m}$ in the message space $\mathcal{M}_{ck}=\Z_q^{m}$, samples $\vecb{r} \gets \Z_q^k$ and outputs a commitment $\bvecb{c}_\gamma:=\bmatr{G}_\gamma\smallpmatrix{\vecb{m} \\ \vecb{r}}$ in the commitment space $\mathcal{C}_{ck}=\GG_\gamma^{k+1}$ and an opening $Op=\vecb{r}$, 
		\item $\mathsf{MP}.\algVrfy$ is a deterministic algorithm which, on input the commitment key $ck=\bmatr{G}_\gamma$, a commitment $\bvecb{c}_\gamma$,  a message 
		$\vecb{m} \in \Z_q^{m}$ and an opening $Op=\vecb{r}\in\Z_q^k$, outputs $1$ if $\bvecb{c}_\gamma=\bmatr{G}_\gamma\smallpmatrix{\vecb{m} \\ \vecb{r}}$
		and $0$ otherwise. 
	\end{itemize}
\end{definition}

\begin{theorem} \label{theo:mp} The $\MP$ scheme is computationally binding if  the discrete logarithm assumption holds in $\GG_\gamma$. Further, if 
$\dist_{k+1,n+k}=\distink$, it holds that: 
\begin{itemize}
\item If $i=0$,  then $\MP$ is perfectly hiding,
\item If $i \in [n]$, then $\MP$ is statistically binding at coordinate $i$, which means that for each $[\vecb{c}]_\gamma \in \GG_{\gamma}^{k+1}$,
there exists a unique $\tilde{m}_i \in\Z_q$ such that for all $\vecb{m} \in\Z_q^m, \vecb{r} \in\Z_q^{k}$ such that  $\bvecb{c}_{\gamma}=\bmatr{G}_{\gamma}\smallpmatrix{\vecb{m}\\\vecb{r}}$, $m_i=\tilde{m}_i$. Further, the scheme is perfectly hiding at the rest of coordinates. 
\end{itemize}
\end{theorem}

\begin{proof}
(Computationally binding.) (This follows a proof due to Villar). Let $[a]_\gamma\in\GG_{\gamma}$ be the discrete logarithm challenge. To sample the commitment key according to $\distink$, choose $\matr{G}_{2} \gets \dist_k$, and define the last $k$ columns of $[\matr{G}]_\gamma$ as $[\matr{G}_{2}]_{\gamma}$. For the rest of the columns of $[\matr{G}]_\gamma$, independently  for each $j \in [n]$, $i \neq j$, sample a pair $\vecb{\alpha}_j,\vecb{\beta}_j$ and define $[\vecb{g}_{j}]_{\gamma}=[\matr{G}_{2} (a \vecb{\alpha}_j+\beta_j)]_\gamma$,
which can be computed as $[a]_\gamma \matr{G}_{2}\vecb{\alpha}_j+[ \matr{G}_{2} \beta_j]_\gamma$. If $i \neq 0$, set $\vecb{g}_{i} \gets \Z_q^{k+1}$. In this case, with overwhelming probability,  $\vecb{g}_{i}$ is linearly independent of the rest of the columns and we will assume so in the following. 
The commitment key is then given to the adversary against the binding property of the scheme, and it outputs a commitment $[\vecb{c}]_{\gamma}$, together with two valid openings 
$(\vecb{m},\vecb{r}), (\vecb{m}',\vecb{r}')$ such that $\vecb{m}\neq \vecb{m}'$. It follows that $[\vecb{c}]_{\gamma}=[\matr{G}]_{\gamma} \smallpmatrix{\vecb{m} \\ \vecb{r}} = \bmatr{G}_{\gamma}\smallpmatrix{\vecb{m}' \\ \vecb{r}'}$, which implies that $[\vecb{0}]_{\gamma}=[\matr{G}]_{\gamma} \smallpmatrix{\vecb{m}-\vecb{m}' \\ \vecb{r}-\vecb{r}'}$. Further, because $\vecb{g}_i$ is linearly independent of the rest of the columns, it holds that:
\begin{equation}\label{recovera}
a \left(\matr{G}_{2} (\sum_{j \neq i} (m'_j-m_j)  \boldsymbol{\alpha}_j) \right) = \left(  \matr{G}_{2}  (\vecb{r}-\vecb{r}' + \sum_{j \neq i} \boldsymbol{\beta}_j  (m_j-m_j'))\right).
\end{equation}
W.l.o.g we can assume that $\matr{G}_{2}$ has full rank (it can be shown that if $\dist_k$-$\mddh$ is a generically hard assumptions in $k$-linear groups, then matrices sampled from $\dist_k$ have full rank with overwhelming probability).  Then, we can recover $a \in \Z_q$ from equation \ref{recovera} except if $\sum_{j \neq i} (m'_j-m_j)  \boldsymbol{\alpha}_j = \vecb{0}$. But since, for all $j$, $\boldsymbol{\alpha}_j$ is information theoretically hidden from the adversary, the probability of this event is at most 
$1/q^k$. 

(Perfectly binding at coordinate $i$.) With overwhelming probability, $\vecb{g}_i$ is linearly independent of the rest of the columns of $\matr{G}$. Therefore, given any $[\vecb{c}]_{\gamma} \in \GG_{\gamma}^{k+1}$, if $\vecb{m}\in\Z_q^n, \vecb{r} \in\Z_q^k$ are such that $\vecb{c}=\matr{G}\smallpmatrix{\vecb{m}\\ \vecb{r}}$, there exists a unique $\tilde{m}_i \in \Z_q$ such 
that $m_i=\tilde{m}_i$. 

(Perfectly hiding at coordinate $j$, $j \neq i$.) This follows immediately from the fact that $\vecb{g}_j$ is in the image 
of $\matr{G}_{2}$. 
\end{proof}


