\begin{description} \label{bits-proof-system}

%\item[$\algK_0(1^\lambda)$:]  Return $gk := (q,\Gr,\Hr,\GG_T,e,\mathcal{P}_1,\mathcal{P}_2) \leftarrow \ggen_a(1^{\lambda})$.

%\item[$\dist_{gk}$:] The distribution $\dist_{gk}$ over $\Gr^{(n+1) \times (n+1)}$ is some witness samplable distribution which 
%defines the relation $\R_{gk} = \{\R_{[\matr{U}]_1}\} 
%\subseteq \Gr^{n+1}\times(\bits^{n}\times\Z_q)$,
%where $[\matr{U}]_1\gets\dist_{gk}$,
%such that $([\vecb{c}]_1,\langle\vecb{b}, w\rangle)\in\R_{[\matr{U}]_1}$ iff
%$[\vecb{c}]_1=[\matr{U}]_1\binom{\vecb{b}}{w}$ and $\vecb{b}\in\bits^n$. The relation $\R_{par}$ consists of pairs $([\matr{U}]_1,\matr{U})$ where $[\matr{U}]_1 \gets \dist_{{gk}}$.
\item[$\algK({gk}, {[\matr{U}]_1})$:]
Let $\lrck_{n+1}\gets \Z_q^2$
and for all $i \in [n]$, $\lrck_{i}:=\epsilon_{i}\lrck_{n+1}$, where
$\epsilon_{i} \leftarrow \Z_q$. Define
$\Rck := ([\vecb{h}_1]_2\cat \ldots \cat[\vecb{h}_{n+1}]_2)$.
Choose 
$\matr{\Delta} \leftarrow \Z_q^{2 \times (n+1)}$,
define $\Lck := \matr{\Delta}[\matr{U}]_1$
and $[\llck_{i}]_1:=\matr{\Delta} [\vecb{u}_i]_1 \in \Gr^2$, for all $i \in [n+1]$. 
Let $\vecb{a} \leftarrow \distlin_{1}$ and define $[\vecb{a}_{\Delta}]_2:=\matr{\Delta}^\top[\vecb{a}]_2 \in \Hr^{n+1}$. 
For any pair $(i,j) \in \indexSet{n}{1}$ (as defined in Section~\ref{sec:notation}), let 
$\matr{T}_{i,j}\gets\Z_q^{2\times2}$ and set:
$$[\matr{C}_{i,j}]_1:=[\vecb{g}_i]_1 \lrck_j^{\top} - [\matr{T}_{i,j}]_1  \in \Gr^{2 \times 2},
\qquad \qquad 
[\matr{D}_{i,j}]_2:=[\matr{T}_{i,j}]_2 \in \Hr^{2 \times 2}.$$ 
Note that $[\matr{C}_{i,j}]_1$ can be efficiently computed 
as $\lrck_j \in \Z_q^{2}$ is the vector of discrete logarithms of $[\vecb{h}_j]_1$.

Let ${\Psi_\sfsum}$ be the proof system for sum in subspace 
(Section~\ref{sec:sum}) and ${\Psi_\sfcom}$
be an instance of the proof system for equal opening (Section~\ref{sec:aggcommit}).

Let
$\crs_{\Psi_\sfsum} \gets \algK_1({gk}, \{[\matr{C}_{i,j}]_1,[\matr{D}_{i,j}]_2:(i,j)\in\indexSet{n}{1}\})$ and   
 $\crs_{\Psi_\sfcom} \gets \algK_1({gk}, \Lck,\Rck,n)$. The common reference string is given by:
\begin{eqnarray*}
\mathsf{crs}_P&:=&\left( [\matr{U}]_1,  \Lck,
    [\Lrck]_1, \{[\matr{C}_{i,j}]_1,[\matr{D}_{i,j}]_2 :(i,j) \in \indexSet{n}{1}\},\crs_{\Psi_\sfsum},\crs_{\Psi_\sfcom} \right), \\
\mathsf{crs}_V&:=&\left([\vecb{a}]_2, [\vecb{a}_\Delta]_2, \crs_{\Psi_\sfsum},\crs_{\Psi_\sfcom} \right). 
 \end{eqnarray*}
\item[{$\algP(\mathsf{crs}_P, [\vecb{c}]_1, \langle \vecb{b}, w_g \rangle)$:}]
Pick $w_h \gets \Z_q$,  $\matr{R} \gets \Z_q^{2\times 2}$ and then: 
\begin{enumerate}
\item Define 
$$[\vecb{c}_{\Delta}]_1 := \Lck \begin{pmatrix} \vecb{b} \\ w_g \end{pmatrix},
\qquad [\vecb{d}]_2 := \Rck \begin{pmatrix} \vecb{b} \\ w_h \end{pmatrix}.$$ 
\item Compute 
 $([\matr{\Theta}]_1, [\matr{\Pi}]_2) :=$
\begin{eqnarray} \label{eq:ThetaPi}
& &
    \sum_{i \in [n]}\left(
        b_i w_h ([\matr{C}_{i,n+1}]_1,[\matr{D}_{i,n+1}]_2)+
        w_g(b_i-1) ([\matr{C}_{n+1,i}]_1, [\matr{D}_{n+1,i}]_2)\right)
        \nonumber\\ & &           +
       \sum_{i \in [n]}  \sum_{\substack{j \in [n]\\ j\neq i}} b_i (b_j-1) ([\matr{C}_{i,j}]_1, [\matr{D}_{i,j}]_2)\nonumber\\
       & &
     +
    w_gw_h ([\matr{C}_{n+1,n+1}]_1, [\matr{D}_{n+1,n+1}]_2) +  ([\matr{R}]_1,-[\matr{R}]_2).
 \end{eqnarray}

\item Compute a proof $\pi_\sfsum$
that $\matr{\Theta}+\matr{\Pi}$
belongs to the space spanned by $\{\matr{C}_{i,j}+\matr{D}_{i,j}:(i,j)\in\indexSet{n}{1}\}$,
 and a proof 
$\pi_\sfcom$
that
$([\vecb{c}_\Delta]_1,[\vecb{d}]_2)$ open to the same value,
using $\vecb{b},w_g$, and $w_h$. 
\end{enumerate}

\item[{$\algV(
    \mathsf{crs}_V,
    [\vecb{c}]_1,
    [\vecb{c}_{\Delta}]_1, [\vecb{d}]_2,
    ([\matr{\Theta}]_1, [\matr{\Pi}]_2), 
    \pi_\sfsum,\pi_\sfcom)$:}] ~
\begin{enumerate}
\item  Check if $[\vecb{c}]_1^\top[\vecb{a}_\Delta]_2 = [\vecb{c}_\Delta]_1^\top[\vecb{a}]_2$. 
\item Check if 
\begin{equation}\label{eq:ver1}[\vecb{c}_{\Delta}]_1
\left(
    [\vecb{d}]_2-
    \sum_{j \in [n]} [\vecb{h}_{j}]_2
\right)^{\top} =
    [\matr{\Theta}]_1[\matr{I}_{2 \times 2}]_2 +
    [\matr{I}_{2 \times 2}]_1[\matr{\Pi}]_2.
    \end{equation}  
  \item Verify that $\pi_\sfsum,\pi_\sfcom$ are valid proofs for 
  $([\matr{\Theta}]_1,[\matr{\Pi}]_2)$
        and $([\vecb{c}_\Delta]_1,[\vecb{d}]_2)$ using $\Psi_\sfsum$ and $\Psi_\sfcom$ respectively.
\end{enumerate}
If any of these checks fails, the verifier outputs $0$, else it outputs $1$.
\item[{$\mathsf{S}_1({gk},[\matr{U}]_1)$:}] The simulator receives as input a description of an asymmetric bilinear group ${gk}$ and a matrix $[\matr{U}]_1 \in \Gr^{(n+1) \times (n+1)}$ sampled according to distribution $\dist_{gk}$. It generates and outputs the CRS in the same way as $\algK_1$, but additionally it also  outputs the simulation trapdoor 
$$\tau=\left(\Lrck, \matr{\Delta}, \tau_{\Psi_\sfsum}, \tau_{\Psi_\sfcom}\right),$$
where $\tau_{\Psi_\sfsum}$ and $\tau_{\Psi_\sfcom}$ are, respectively, ${\Psi_\sfsum}$'s and ${\Psi_\sfcom}$'s simulation trapdoors.
\item[{$\mathsf{S}_2(\crs_P,[\vecb{c}]_1,\tau)$:}] Compute $[\vecb{c}_{\Delta}]_1:=\matr{\Delta}[\vecb{c}]_1$.
      Then pick random $\overline{w}_h \gets \Z_q$, $\matr{R} \gets \Z_q^{2 \times 2}$ and define 
 $\vecb{d}:= \overline{w}_{h} \lrck_{n+1}.$
 Then set:
\begin{align*} 
[\matr{\Theta}]_1 & :=  [\vecb{c}_\Delta]_1 \left(\vecb{d}-\sum_{i \in [n]} \lrck_i\right)^\top + [\matr{R}]_1,
    &
    [\matr{\Pi}]_2 & := - [\matr{R}]_2.
\end{align*}
Finally, simulate proofs $\pi_\sfsum,\pi_\sfcom$ using $\tau_{\Psi_\sfsum}$ and $\tau_{\Psi_\sfcom}$.
\end{description}

