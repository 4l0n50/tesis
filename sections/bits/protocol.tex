\begin{description}

\item[$\algK_0(1^\lambda)$:]  Return $\Gamma := (q,\Gr,\Hr,\T,e,\hat{g},\check{h}) \leftarrow \ggen_a(1^{\lambda})$.

\item[$\dist_\Gamma$:] The distribution $\dist_\Gamma$ over $\Gr^{(n+1) \times (n+1)}$ is some witness samplable distribution which 
defines the relation $\R_\Gamma = \{\R_{\hmatr{U}}\} 
\subseteq \Gr^{n+1}\times(\bits^{n}\times\Z_q)$,
where $\hmatr{U}\gets\dist_\Gamma$,
such that $(\hvecb{c},\langle\vecb{b}, w\rangle)\in\R_{\hmatr{U}}$ iff
$\hvecb{c}=\hmatr{U}\binom{\vecb{b}}{w}$. The relation $\R_{par}$ consists of pairs $(\hmatr{U},\matr{U})$ where $\hmatr{U} \gets \dist_{\Gamma}$.
\item[$\algK_1(\Gamma, \hmatr{U})$:]
Let $\lrck_{n+1}\gets \Z_q^2$
and for all $i \in [n]$, $\lrck_{i}:=\epsilon_{i}\lrck_{n+1}$, where
$\epsilon_{i} \leftarrow \Z_q$. Define
$\Rck := ([vecb{h}_1]_2|| \ldots ||[vecb{h}_{n+1}]_2)$.
Choose 
$\matr{\Delta} \leftarrow \Z_q^{2 \times (n+1)}$,
define $\Lck := \matr{\Delta}\hmatr{U}$
and $\hat{\llck}_{i}:=\matr{\Delta} \hat{\vecb{u}}_i \in \Gr^2$, for all $i \in [n+1]$. 
Let $\vecb{a} \leftarrow \distlin_{1}$ and define $\cvecb{a}_{\Delta}:=\matr{\Delta}^\top\cvecb{a} \in \Hr^{n+1}$. 
For any pair $(i,j) \in \indexSet{n}{1}$, let 
$\matr{T}_{i,j}\gets\Z_q^{2\times2}$ and set:
$$\hmatr{C}_{i,j}:=[vecb{g}_i]_1 \lrck_j^{\top} - \hmatr{T}_{i,j}  \in \Gr^{2 \times 2},
\qquad \qquad 
\cmatr{D}_{i,j}:=\cmatr{T}_{i,j} \in \Hr^{2 \times 2}.$$ 
Note that $\hmatr{C}_{i,j}$ can be efficiently computed 
as $\lrck_j \in \Z_q^{2}$ is the vector of discrete logarithms of $\check{\lrck}_j$.

Let $\QANIZKsum$ be the proof system for Sum in Subspace 
(Sect. \ref{sec:QANIZKsum}) and $\QANIZKcomms$
%=(\algK_0,\algK_1,\algP,\algV,\algS_1,\algS_2)$
be an instance of our proof system for Equal Opening (Sect. \ref{sec:aggcomms}).

Let
$\crs_\QANIZKsum \gets \algK_1(\Gamma, \{\hmatr{C}_{i,j},\cmatr{D}_{i,j}\}_{(i,j)\in\indexSet{n}{1}})$ and \footnote{We identify
matrices in $\Gr^{2 \times 2}$ (resp. in $\Hr^{2 \times 2}$) with vectors in $\Gr^{4}$ (resp. in $\Hr^{4}$).}  
 $\crs_\QANIZKcomms \gets \algK_1(\Gamma, \Lck,\Rck,n)$. The common reference string is given by:
\begin{eqnarray*}
\mathsf{crs}_P&:=&\left( \hat{\matr{U}},  \Lck,
    \check{\Lrck}, \{\hmatr{C}_{i,j},\cmatr{D}_{i,j} \}_{(i,j) \in \indexSet{n}{1}},\crs_\QANIZKsum,\crs_\QANIZKcomms \right), \\
\mathsf{crs}_V&:=&\left(\cvecb{a}, \cvecb{a}_\Delta, \crs_\QANIZKsum,\crs_\QANIZKcomms \right). 
 \end{eqnarray*}
\item[$\algP(\mathsf{crs}_P, \hvecb{c}, \langle \vecb{b}, w_g \rangle)$:]
Pick $w_h \gets \Z_q$,  $\matr{R} \gets \Z_q^{2\times 2}$ and then: 
\begin{enumerate}
\item Define 
$$\hvecb{c}_{\Delta} := \Lck \begin{pmatrix} \vecb{b} \\ w_g \end{pmatrix},
\qquad \cvecb{d} := \Rck \begin{pmatrix} \vecb{b} \\ w_h \end{pmatrix}.$$ 
\item Compute 
 $(\hmatr{\Theta}_{b(\overline{b}-1)}, \cmatr{\Pi}_{b(\overline{b}-1)})\, :=$
\begin{eqnarray} \label{eq:ThetaPi}
%(\hmatr{\Theta}_{b(\overline{b}-1)}, \cmatr{\Pi}_{b(\overline{b}-1)})
%& := &
%    \sum_{i \in [n]}\left(
%        b_i w_h (\hmatr{C}_{i,n+1},\cmatr{D}_{i,n+1})+
%        w_g(b_i-1) (\hmatr{C}_{n+1,i}, \cmatr{D}_{n+1,i})\right)
%        \nonumber\\ & &           +
%       \sum_{i \in [n]}  \sum_{\substack{j \in [n]\\ j\neq i}} b_i (b_j-1) (\hmatr{C}_{i,j}, \cmatr{D}_{i,j})\nonumber\\
%       & &
%     +
%    w_gw_h (\hmatr{C}_{n+1,n+1}, \cmatr{D}_{n+1,n+1}) +  (\hmatr{R},-\cmatr{R}).
& &
    \sum_{i \in [n]}\left(
        b_i w_h (\hmatr{C}_{i,n+1},\cmatr{D}_{i,n+1})+
        w_g(b_i-1) (\hmatr{C}_{n+1,i}, \cmatr{D}_{n+1,i})\right)
        \nonumber\\ & &           +
       \sum_{i \in [n]}  \sum_{\substack{j \in [n]\\ j\neq i}} b_i (b_j-1) (\hmatr{C}_{i,j}, \cmatr{D}_{i,j})\nonumber\\
       & &
     +
    w_gw_h (\hmatr{C}_{n+1,n+1}, \cmatr{D}_{n+1,n+1}) +  (\hmatr{R},-\cmatr{R}).
%\\
%(\hmatr{\Theta}_{b-\overline{b}}, 
%\cmatr{\Pi}_{b-\overline{b}})
%  & := &
%    \sum_{i\in[n]}\left(
%        (b_i-w_h)(\hmatr{C}_{i,n+1},\cmatr{D}_{i,n+1})
%         +
%        (w_g-b_i)(\hmatr{C}_{n+1,i}, \cmatr{D}_{n+1,i})\right)\\
%     & &  + \sum_{i\in[n]}\sum_{\substack{j\in[n]\\j\neq i}} (b_i-b_j)(\hmatr{C}_{i,j}, \cmatr{D}_{i,j}) 
%\\ & &+
%    (w_g-w_h)(\hmatr{C}_{n+1,n+1},\cmatr{D}_{n+1,n+1}) + (\hmatr{S},-\cmatr{S}).
 \end{eqnarray}

\item Compute a proof $(\hat{\boldsymbol \rho}_{b(\overline{b}-1)},\check{\boldsymbol \sigma}_{b(\overline{b}-1)})$
that $\matr{\Theta}_{b(\overline{b}-1)}+\cmatr{\Pi}_{b(\overline{b}-1)}$
belongs to the space spanned by $\{\matr{C}_{i,j}+\matr{D}_{i,j}\}_{(i,j)\in\indexSet{n}{1}}$,
 and a proof 
$(\hgrkb{\rho}_{b-\overline{b}}, \cgrkb{\sigma}_{b-\overline{b}})$
that
$(\hvecb{c}_\Delta,\cvecb{d})$ open to the same value,
using $\vecb{b},w_g$, and $w_h$. 
\end{enumerate}

\item[$\algV(
    \mathsf{crs}_V,
    \hvecb{c},
    \langle
        \hvecb{c}_{\Delta}, \cvecb{d},
        (\hmatr{\Theta}_{b(\overline{b}-1)}, \cmatr{\Pi}_{b(\overline{b}-1)}), 
        \{(\hat{\boldsymbol \rho}_{X}, \check{\boldsymbol \sigma}_{X})\}_{X \in \{b(\overline{b}-1), b-\overline{b}\}} \rangle)$:] ~
%   
\begin{enumerate}
\item  Check if $\hvecb{c}^\top\cvecb{a}_\Delta = \hvecb{c}_\Delta^\top\cvecb{a}$. 
\item Check if 
\begin{equation}\label{eq:ver1}\hvecb{c}_{\Delta}
\left(
    \check{\vecb{d}}-
    \sum_{j \in [n]} [vecb{h}_{j}]_2
\right)^{\top} =
    \hmatr{\Theta}_{b(\overline{b}-1)} \cmatr{I}_{2 \times 2} +
    \hmatr{I}_{2 \times 2}\cmatr{\Pi}_{b(\overline{b}-1)}.
    \end{equation}  
  \item Verify that $(\hgrkb{\rho}_{b(\overline{b}-1)}, \cgrkb{\sigma}_{b(\overline{b}-1)}),(\hgrkb{\rho}_{b-\overline{b}},\cgrkb{\sigma}_{b-\overline{b}})$ are valid proofs for %\linebreak 
  $(\hmatr{\Theta}_{b(\overline{b}-1)},$ $\cmatr{\Pi}_{b(\overline{b}-1)})$
        and $(\hvecb{c}_\Delta,\cvecb{d})$ using $\crs_\QANIZKsum$ and $\crs_\QANIZKcomms$ respectively.
\end{enumerate}
If any of these checks fails, the verifier outputs $0$, else it outputs $1$.
%\item[$\mathsf{S}_1(\Gamma,\hat{\matr{U}})$:] The simulator receives as input a description of an asymmetric bilinear group $\Gamma$ and a matrix $\hat{\matr{U}} \in \Gr^{(n+1) \times (n+1)}$ sampled according to distribution $\dist_{\Gamma}$. It generates and outputs the CRS in the same way as $\algK_1$, but additionally it also  outputs the simulation trapdoor 
%$$\tau=\left(\Lrck, \matr{\Delta}, \tau_\QANIZKsum, \tau_\QANIZKcomms\right),$$
%where $\tau_\QANIZKsum$ and $\tau_\QANIZKcomms$ are, respectively, $\QANIZKsum$'s and $\QANIZKcomms$'s simulation trapdoors.
%\item[$\mathsf{S}_2(\crs_P,\vc,\tau)$:] Compute $\vc_{\Delta}:=\matr{\Delta} \vc$.
%      Then pick random $\overline{w}_h \gets \Z_q$, $\matr{R} \gets \Z_q^{2 \times 2}$ and define 
% $\vecb{d}:= \overline{w}_{h} \lrck_{n+1}.$
% Then set:
%\begin{align*} 
%\hmatr{\Theta}_{b(\overline{b}-1)} & :=  \hvecb{c}_\Delta \left(\vecb{d}-\sum_{i \in [n]} \lrck_i\right)^\top + \hmatr{R},
%    &
%    \cmatr{\Pi}_{b(\overline{b}-1)} & := - \cmatr{R}.
%%\\
%%\hmatr{\Theta}_{b-\overline{b}} & :=  \hvecb{c}_\Delta \left( \sum_{i\in[n]}\lrck_{i}^\top \right) -
%%                          \left( \sum_{i\in[n]}\hvecb{u}_{i} \right)\vecb{d}^\top + \hmatr{S},
%%    &
%%    \cmatr{\Pi}_{b-\overline{b}}& := - \cmatr{S}.
%\end{align*}
%Finally, simulate proofs $(\hat{\boldsymbol \rho}_{X},
%  \check{\boldsymbol \sigma}_{X})$
%%for $(\hmatr{\Theta}_{X}, \cmatr{\Pi}_{X})$
%for $X \in \{b(\overline{b}-1),  b-\overline{b}
%\}$  using $\tau_\QANIZKsum$ and $\tau_\QANIZKcomms$.
%
%\end{description} 
%The simulators $\algS_1$ and $\algS_2$ are defined as follows.
%\begin{description}
\item[$\mathsf{S}_1(\Gamma,\hat{\matr{U}})$:] The simulator receives as input a description of an asymmetric bilinear group $\Gamma$ and a matrix $\hat{\matr{U}} \in \Gr^{(n+1) \times (n+1)}$ sampled according to distribution $\dist_{\Gamma}$. It generates and outputs the CRS in the same way as $\algK_1$, but additionally it also  outputs the simulation trapdoor 
$$\tau=\left(\Lrck, \matr{\Delta}, \tau_\QANIZKsum, \tau_\QANIZKcomms\right),$$
where $\tau_\QANIZKsum$ and $\tau_\QANIZKcomms$ are, respectively, $\QANIZKsum$'s and $\QANIZKcomms$'s simulation trapdoors.
\item[$\mathsf{S}_2(\crs_P,\vc,\tau)$:] Compute $\vc_{\Delta}:=\matr{\Delta} \vc$.
      Then pick random $\overline{w}_h \gets \Z_q$, $\matr{R} \gets \Z_q^{2 \times 2}$ and define 
 $\vecb{d}:= \overline{w}_{h} \lrck_{n+1}.$
 Then set:
\begin{align*} 
\hmatr{\Theta}_{b(\overline{b}-1)} & :=  \hvecb{c}_\Delta \left(\vecb{d}-\sum_{i \in [n]} \lrck_i\right)^\top + \hmatr{R},
    &
    \cmatr{\Pi}_{b(\overline{b}-1)} & := - \cmatr{R}.
%\\
%\hmatr{\Theta}_{b-\overline{b}} & :=  \hvecb{c}_\Delta \left( \sum_{i\in[n]}\lrck_{i}^\top \right) -
%                          \left( \sum_{i\in[n]}\hvecb{u}_{i} \right)\vecb{d}^\top + \hmatr{S},
%    &
%    \cmatr{\Pi}_{b-\overline{b}}& := - \cmatr{S}.
\end{align*}
Finally, simulate proofs $(\hat{\boldsymbol \rho}_{X},
  \check{\boldsymbol \sigma}_{X})$
%for $(\hmatr{\Theta}_{X}, \cmatr{\Pi}_{X})$
for $X \in \{b(\overline{b}-1),  b-\overline{b}
\}$  using $\tau_\QANIZKsum$ and $\tau_\QANIZKcomms$.
\end{description}

