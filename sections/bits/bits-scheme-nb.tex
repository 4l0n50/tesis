We construct a QA-NIZK argument of membership in the language
$$
\Lang_{ck,\sfbits} := \{[\vecb{c}]_1\in\GG_1^{k+1} : \exists \vecb{b}\in\bits^m,\vecb{r}\in\Z_q^k \text{ s.t. } [\vecb{c}]_1 = \MP.\Com_{ck}(\vecb{b};\vecb{r})\},
$$
where $ck:=[\matr{G}]_1$ and $\matr{G}$ is a matrix sampled from 
some distribution $\distink$ (as defined on Sect. \ref{sec:mddh}). For simplicity, in the exposition we restrict ourselves to the case $\dist_k=\distlin_{1}$ so  $\matr{G}$ is sampled from $\distlininone$, for some $0 \leq i \leq m$.

It is important to note that, as an extended MP commitment is at best only binding at one coordinate, a priori showing that it opens to $\vecb{b} \in \{0,1\}^m$ is not very meaningful, as it does open to other values as well. However, when combined with external protocols that unequivocally define $\vecb{b}$, it becomes a key building block to obtain the the results of chapter \ref{sec:shuf-rp}.

The argument is implicit in Sect. \ref{sec:bits-binding}, where we construct a QA-NIZK argument for proving that a perfectly binding commitment opens to a bit-string. More technically, to prove that a  perfectly binding commitment $[\vecb{c}']_1$ opens to a bit-string $\vecb{b}$, the argument in Sect. \ref{sec:bits-binding} takes the following steps:
\begin{enumerate}
\item Construct two MP commitments $[\vecb{c}]_1$, 
$[\vecb{d}]_2$ to $\vecb{b}$. 
\item Prove that $[\vecb{c}]_1$ and $[\vecb{c}']_1$ open to the same value. 
\item Prove that the two MP commitments $[\vecb{c}]_1$ and $[\vecb{d}]_2$ open to the same value.
\item Prove that $\vecb{c}(\vecb{d}-\sum_{j \in [m]}
\vecb{h}_j)^\top\in\Span(\{\vecb{g}_i\vecb{h}_j^\top:i,j\in[m+1]\}\setminus\{\vecb{g}_i\vecb{h}_i^\top:i\in[m]\})$, where $ck:=[(\vecb{g}_1,\ldots,\vecb{g}_{m+1})]_1$ and $ck':=[(\vecb{h}_1,\ldots,\vecb{h}_{m+1})]_2$.
\end{enumerate}
%The last step guarantees that 
%$b_i(b_i-1)=0$ for all $i \in [m]$. Indeed, 
%$\vecb{c}(\vecb{d}-\sum_{j \in [m]}
%\vecb{h}_j)^\top$ can be written as a linear combination of the vectors $\{\vecb{g}_i\vecb{h}_j^\top\}$ where the coefficient of $\vecb{g}_i\vecb{h}_i^{\top}$ is $b_i(b_i-1)$. Intuitively, an adversary will be able to prove that $\vecb{c}(\vecb{d}-\sum_{j \in [m]}
%\vecb{h}_j)^\top$ is in the span of the vectors  $\{\vecb{g}_i\vecb{h}_j^\top\}$ without those pairs where $i=j$ only if $b_i(b_i-1)=0$ for all $i \in [m]$. 


The argument we need for our results eliminates the perfectly binding commitment, which of course also means that step 2 disappears. Additionally, in the original scheme from Sect. \ref{sec:bits-binding}, the distribution of $ck=[\matr{G}]_1$ is uniform over $\GG_1^{2\times(n+1)}$, while in our argument of membership in 
$\Lang_{ck,\sfbits}$, $\matr{G}$ can follow any distribution $\distlininone$ for some $0 \leq i \leq m$. 
However, it is not hard to adapt the original proof to these distributions (in fact, in the soundness proof of 
Lemma \ref{lemma:bits1}, there is a game where the distribution of $\matr{G}$ is changed to $\distlininone$, for some $i \gets [m]$). The proof that $\Lang_{ck,\sfbits}$ admits a constant-size QA-NIZK argument essentially reuses parts of the proof of Theorem \ref{teo:bitstr-soundness}. For completeness, we give a full description of the scheme for the non-binding case below. 

\subsubsection{Detailed Description}
\begin{description}
\item[{$\algK(\gk, [\matr{G}]_1)$}:] 
Pick $\matr{H}\gets\distlinizeroone$, and denote by $\vecb{h}_{j}$ the $j$ th column of $\matr{H}$. Pick $\matr{T}\gets\Z_q^{2\times 2}$ and for each $(i,j)\in\indexSet{m}{1}$ define matrices
$$([\matr{C}_{i,j}]_1,[\matr{D}_{i,j}]_2):=([\vecb{g}_i\vecb{h}_j^{\top}+\matr{T}]_1,[-\matr{T}]_2).$$

Let $\Pi_\sfsum$ be the proof system for Sum in Subspace 
(Sect. \ref{sec:sum}) and $\Pi_\sfcom$
be an instance of the proof system for Equal Commitment Opening (Sect. \ref{sec:aggcommit}).
Let
$\crs_\sfsum \gets \Pi_\sfsum.\algK(\gk, \{([\matr{C}_{i,j}]_1,[\matr{D}_{i,j}]_2):(i,j)\in\indexSet{m}{1}\})$.\footnote{We identify
matrices in $\GG_1^{2 \times 2}$ (respectively in $\GG_2^{2 \times 2}$) with vectors in $\GG_1^{4}$ (resp. in $\GG_2^{4}$).} and let $\crs_\sfcom \gets \Pi_\sfcom.\algK(\gk, \bmatr{G}_1,\bmatr{H}_2,m)$. 

The common reference string is given by:
\begin{eqnarray*}
\mathsf{crs}&:=&\left( gk, \bmatr{G}_1,
    [\matr{H}]_2, \{([\matr{C}_{i,j}]_1, [\matr{D}_{i,j}]_2): (i,j) \in \indexSet{m}{1}\},\crs_\sfsum,\crs_\sfcom \right).
 \end{eqnarray*}
\item[$\algP(\mathsf{crs}_, \bvecb{c}_1, \langle \vecb{b}, r \rangle)$:] The proof $([\vecb{d}]_1,([\matr{\Theta}]_1,[\matr{\Pi}]_2),\pi_\sfcom,\pi_\sfsum)$ is computed as follows:
\begin{enumerate}
\item $[\vecb{d}]_2:= \mathsf{MP}.\Com_{[\matr{H}]_2} (\vecb{b};s)$, $s \gets \Z_q$. 
\item Pick $\matr{R} \gets \Z_q^{2\times 2}$ and compute:
\begin{eqnarray*}
([\matr{\Theta}]_1, [\matr{\Pi}]_2) & := & ([\matr{R}]_1,[-\matr{R}]_2)+
\sum_{i\in[m]}\sum_{j\in[m]}b_i(b_j-1)([\matr{C}_{i,j}]_1,[\matr{D}_{i,j}]_2)+\\
& &     rs([\matr{C}_{m+1,m+1}]_1,[\matr{D}_{m+1,m+1}]_2)+\\
& &
  \sum_{i \in [m]}
 \left(b_i s  ([\matr{C}_{i,m+1}]_1,[\matr{D}_{i,m+1}]_2)\right. +\\
& & \left.r(b_i-1) ([\matr{C}_{m+1,i}]_1,[\matr{D}_{m+1,i}]_2)\right).
\end{eqnarray*}

\item Compute a proof $\pi_\sfsum$
that $\matr{\Theta}+\matr{\Pi}$ is in the span of 
$\{\matr{C}_{i,j}+\matr{D}_{i,j}\}_{(i,j)\in\indexSet{m}{1}}$
and a proof $\pi_\sfcom$
that
$([\vecb{c}]_1,[\vecb{d}]_2)$ open to the same value,
using $\vecb{b},r$, and $s$. \\
\end{enumerate}

\item[{$\algV(
    \mathsf{crs},
    [\vecb{c}]_1,
    [\vecb{d}]_2,
        ([\matr{\Theta}]_1, [\matr{\Pi}]_2), 
        \pi_\sfcom,\pi_\sfsum )$}:] If any of the following checks fails, the verifier outputs $0$, else it outputs $1$:
%   
\begin{enumerate}
\item 
$[\vecb{c}]_1 \left([\vecb{d}]_2^\top-
    \sum_{j \in [m]} [\vecb{h}_j]_2^\top\right) =
    [\matr{\Theta}]_1 [\matr{I}]_2 +  [\matr{I}]_1 [\matr{\Pi}]_2.$
  \item $\Pi_\sfsum.\algV(\crs_\sfsum,([\matr{\Theta}]_1,[\matr{\Pi}]_2),\pi_\sfsum))=1$ and $\Pi_\sfcom.\algV(\crs_\sfcom,([\vecb{c}]_1,[\vecb{d}]_2),\pi_\sfcom)=1$.
\end{enumerate}

\item[{$\mathsf{S}_1(\gk,[\matr{G}]_1)$}:] It generates and outputs the CRS in the same way as $\algK$ and additionally outputs the simulation trapdoor 
$\tau=\left(\matr{H},\tau_\sfsum, \tau_\sfcom\right)$,
where $\tau_\sfsum$ and $\tau_\sfcom$ are, respectively, $\Pi_\sfsum$'s and $\Pi_\sfcom$'s simulation trapdoors.
\item[{$\mathsf{S}_2(\crs,[\vecb{c}]_1,\left(\matr{H},\tau_\sfsum, \tau_\sfcom\right))$}:] The proof $([\vecb{d}]_1,([\matr{\Theta}]_1,[\matr{\Pi}]_2),\pi_\sfcom,\pi_\sfsum)$ is simulated as follows:
\begin{enumerate}
\item $\vecb{d}:= \MP.\Com_{[\matr{H}]_2}(\vecb{0}_{n\times1},\overline{w}_h)$, $\overline{w}_h \gets \Z_q$.
\item Pick $\matr{R} \gets \Z_q^{2 \times 2}$ and define:
\begin{align*} 
[\matr{\Theta}]_1 & :=  [\vecb{c}]_1\left(\vecb{d}-\sum_{i \in [m]} \lrck_i\right)^\top + [\matr{R}]_1,
    &
[\matr{\Pi}]_2 & := - [\matr{R}]_2.
\end{align*}
\item $\pi_\sfsum\gets\Pi_\sfsum.\algS_2(\crs_\sfsum,([\matr{\Theta}]_1,[\matr{\Pi}]_2),\tau_\sfsum)$ and $\pi_\sfcom\gets\Pi_\sfcom.\algS_2(\crs_\sfcom,([\vecb{c}]_1,[\vecb{d}]_2),\tau_\sfcom)$.
\end{enumerate}
\end{description}




\subsubsection{Security Proof}
\begin{theorem} \label{theo:bits}
The proof system described above is a QA-NIZK proof system with Perfect Completeness, Computational Soundness, and Perfect Zero-Knowledge.
\end{theorem}	
\begin{proof}
We remark that proof of Completeness and Zero-Knowledge is the same for any distribution $\distlininone$.
\begin{description}
\item[Perfect Completeness:]
Note that,
by definition of $\matr{C}_{i,j}$ and $\matr{D}_{i,j}$, 
$[\matr{C}_{i,j}]_1[\matr{I}]_2+[\matr{I}]_1[\matr{D}_{i,j}]_2=$
$[\vecb{g}_{i}]_1[\vecb{h}_j]_2^\top$.  Since $b_i(b_i-1) = 0$ for each $i\in[m]$,
\begin{eqnarray*}
\lefteqn{
[\vecb{c}]_1\left( [\vecb{d}]_2 - \sum_{i\in[m]} [\vecb{h}_{i}]_2\right)^\top}\\
& = & 
    \sum_{i \in [m]}\left(
        b_i s[\vecb{g}_{i}]_1[\vecb{h}_{m+1}]_2^{\top}+
        r(b_i-1)[\vecb{g}_{m+1}]_1[\vecb{h}_i]_2^{\top}+
        \sum_{j \in [m]} b_i (b_j-1)[\vecb{g}_{i}]_1[\vecb{h}_{j}]_2^{\top}
    \right)
\\ & & \mbox{ }
    + rs[\vecb{g}_{m+1}]_1[\vecb{h}_{m+1}]_2^{\top}\\
& = & 
    \left(\sum_{i\in[m]}b_is[\vecb{g}_{i}]_1\vecb{h}_{m+1}^{\top}+r(b_i-1)[\vecb{g}_{m+1}]_1\vecb{h}_i^{\top}+
        \sum_{\substack{j \in [m]\\j\neq i}}b_i(b_j-1)[\vecb{g}_{i}]_1\vecb{h}_{j}^{\top}\right)[\matr{I}]_2\\
& & \mbox{ }
    +rs[\vecb{g}_{m+1}]_1\vecb{h}_{m+1}^{\top}[\matr{I}]_2
    +[\matr{R}]_1[\matr{I}]_2 + [\matr{I}]_1[-\matr{R}]_2
\\ & = &
    [\matr{\Theta}]_1[\matr{I}]_2+
    [\matr{I}]_1[\matr{\Pi}]_2.
\end{eqnarray*}
Finally, the rest of the proof follows from completeness of $\Pi_\sfsum$ and $\Pi_\sfcom$. 

\item[Soundness:] When $\matr{G}$ is sampled from $\distlinizeroone$ it suffices to prove that the commitment $[\vecb{c}]_1$ output by the adversary is in $\Span([\matr{G}]_1)$ since, by the perfect hiding property, $[\vecb{c}]_1$ can be opened to any $\vecb{b}\in\bits^m$ thus $[\vecb{c}]_1\in\Lang_{ck,\sfbits}$. If $[\vecb{c}]_1\notin\Span([\matr{G}]_1)$, then we can break the (strong) soundness of the proof that $[\vecb{c}]_1$ and $[\vecb{d}]_2$ open to the same value, since that proof implies that there exist $\vecb{x},r,s$ such that $[\vecb{c}]_1=[\matr{G}]_1\smallpmatrix{\vecb{x}\\r}$ and $[\vecb{d}]_2=[\matr{H}]_2\smallpmatrix{\vecb{x}\\s}$. Therefore, we construct an adversary $\advB$ against the strong soundness of $\Pi_\sfcom$ that simulates $\advA$ until it halts and outputs $([\vecb{c}]_1,[\vecb{d}]_2,\pi_\sfcom)$. Note that, in order to simulate the CRS $\advB$ requires $\matr{H}$, but this is not a problem since is part of the input in the strong soundness game.
 
When $\matr{G}$ is sampled from $\distlinisnone$, ${i^*}>0$, the proof follows from the indistinguishability of the following three games:
\begin{itemize}
\item[$\mathsf{Real}$:] This is the real Soundness game. The output is 1 if the adversary submits some $[\vecb{c}]_1\notin\Lang_{ck,\sfbits}$ and the corresponding proof which is accepted by the verifier.
\item[$\sfGame_0$:] This identical to $\mathsf{Real}$, except that $\algK$ does not receive $[\matr{G}]_1$ as a input but
it samples $\matr{G}$ itself according to $\distlinisnone$.
\item[$\sfGame_1$:] This game is identical to $\sfGame_0$ except that now $\matr{H}\gets\distlinisnone$.
\end{itemize}

It is obvious that the first two games are indistinguishable. The rest of the argument goes as follows.

\begin{lemma} There exists a\ $\distlin_1$-$\mddh_{\GG_2}$ adversary $\advD$ such that
$|\Pr\left[\mathsf{Game}_{0}(\advA)=1\right]$ $-\Pr\left[\mathsf{Game}_{1}(\advA)=1\right]|$ $\leq \mathsf{Adv}_{\distlin_1,\ggen_a}(\advD).$
\end{lemma}
\begin{proof}
We construct an adversary $\advD$ that receives 
a challenge $([\matr{A}]_2,[\vecb{u}]_2)$ of the 
$\distlin_1$-$\mddh_{\GG_2}$ Assumption. From this challenge, $\advD$ just defines the matrix  $[\matr{H}]_2\in\GG_2^{2\times(m+1)}$ as the matrix whose last column consists of $[\matr{A}]_2$, the ith column consists of $[\vecb{u}]_2$ and the rest of the columns are random vectors in the image of $[\matrA]_2$. 
Obviously, when $[\vecb{u}]_2$ is sampled from 
the image of $[\matr{A}]_2,$ $\matr{H}$ follows the distribution $\distlinizeroone$, while if $[\vecb{u}]_2$ is a uniform element of $\GG^2$, $\matr{H}$ follows the distribution $\distlinisnone$. 
 
Adversary $\advD$ samples
$\matr{G} \gets \distlinisnone$. Given that $\advD$ does not know the discrete logarithms of $[\matr{H}]_2$, it cannot compute the pairs $(\matr{C}_{i,j},\matr{D}_{i,j})$ exactly as in $\sfGame_0$. Nevertheless, for each $(i,j)\in\indexSet{m}{1}$ it can compute identically distributed pairs by picking $\matr{T}\gets\Z_q^{2\times 2}$ and defining
$$
([\matr{C}_{i,j}]_1,[\matr{D}_{i,j}]_2):=([\matr{T}]_1,\vecb{g}_i[\vecb{h}_j]_2^\top-[\matr{T}]_2).
$$
The rest of the elements of the CRS, namely $\crs_\sfcom$ and $\crs_\sfsum$, are honestly computed. When $\matr{H}\gets\distlinizeroone$, $\advD$ perfectly simulates $\sfGame_0$, and when $\matr{H}\gets\distlinisnone$, $\advD$ perfectly simulates $\sfGame_1$, which concludes the proof. 
\end{proof}

\begin{lemma}
There exist adversaries $\advB_1$, against the strong soundness of $\Pi_\sfcom$, and $\advB_2$, against the soundness of $\Pi_\sfsum$, such that $\Pr[\sfGame_1(\advA)=1]\leq 4/q+ \adv_{\Pi_\sfcom}(\advB_1)+\adv_{\Pi_\sfsum}(\advB_2)$.
\end{lemma}
\begin{proof}
With probability $1-4/q$, $\{\vecb{g}_{i^*},\vecb{g}_{m+1}\}$ and $\{\vecb{h}_{i^*},\vecb{h}_{m+1}\}$ are both bases of $\Z_q^2$,
we can define $b_{i^*},\overline{w}_g,\overline{w}_h,\overline{b}_{i^*}$ as the unique coefficients in $\Z_q$ such that $\vecb{c}=b_{i^*}\vecb{g}_{i^*} + \overline{w}_g \vecb{g}_{m+1}$ and $\vecb{d}= \bb_{i^*} \vecb{h}_{i^*} + \overline{w}_h \vecb{h}_{m+1}$.

In particular, if $\advA$ breaks soundness, this implies that $b_{i^*} \notin \{0,1\}$ (since for $i\neq i^*$, 
$\vecb{c}$ can always be opened to 
  choose $b_i=0$). Further, the verifier accepts the proof proof:
$ (
        [\vecb{d}]_2,
        ([\matr{\Theta}]_1, [\matr{\Pi}]_2), 
        \pi_\sfcom,\pi_\sfsum )$
  produced by $\advA$.
We distinguish two cases:
\begin{description}
\item[$b_{i^*} \neq \overline{b}_{i^*}$:] Given that $[\vecb{c}]_1$ and $[\vecb{d}]_2$ are perfectly binding at coordinate $i^*$, if $b_{i^*}\neq\bb_{i^*}$ it is not possible that $[\vecb{c}]_1$ and $[\vecb{d}]_2$ open to the same value. We construct an adversary $\advB_1$ against the strong soundness 
of $\Pi_\sfcom$ that simulates game $\sfGame_1$ with $\advA$ (using $\matr{H}$ to simulate the CRS) until it halts and outputs $([\vecb{c}]_1,[\vecb{d}]_2,\pi_\sfcom)$. If $b_{i^*}\neq\bb_{i^*}$, $\pi_\sfcom$ is a fake proof for $([\vecb{c}]_1,[\vecb{d}]_2)$ opening to the same value and then $\advB_1$ breaks the strong soundness of $\Pi_\sfcom$.
\item[$b_{i^*} = \overline{b}_{i^*}$, 
$b_{i^*}(\overline{b}_{i^*} -1) \neq 0$:]
If we express $\matr{\Theta}+\matr{\Pi}$
as a linear combination of $\{\vecb{g}_{i}\vecb{h}_{j}^{\top}:i,j\in[n+1]\}$, the coordinate of
$\vecb{g}_{i^*}\vecb{h}_{i^*}^\top$ is $b_{i^*}(\bb_{i^*}-1)\neq 0$ and thus $\matr{\Theta}+\matr{\Pi}\notin\Span(\{\matr{C}_{i,j}+\matr{D}_{i,j}:(i,j)\in\indexSet{m}{1}\})$. We construct an adversary $\advB_2$ against the soundness of $\Pi_\sfsum$ that simulates game $\sfGame_1$ with $\advA$ until it halts and outputs $([\matr{\Theta}]_1,[\matr{\Pi}]_2,\pi_\sfsum)$. If $b_{i^*} = \overline{b}_{i^*}$ but $b_{i^*}(\overline{b}_{i^*} -1) \neq 0$, $\pi_\sfsum$ is a fake proof for $([\matr{\Theta}]_1,[\matr{\Pi}]_2)$ and then $\advB_2$ breaks the soundness of $\Pi_\sfsum$.
\end{description}
\end{proof}

\item[Perfect Zero-Knowledge:] First, note that the vector $[\vecb{d}]_2 \in \GG_2^2$ output by the prover and the vector output by $\algS_2$ follow exactly the same distribution. This is because $\matr{H}\gets\distlinizeroone$ defines perfectly hiding commitments. In particular, although the simulator $\algS_2$ does not know $\vecb{b} \in \{0,1\}^{m}$ such that $[\vecb{c}]_1=[\matr{G}]_1\smallpmatrix{\vecb{b}\\r}$, for some $r\in\Z_q$, 
there exists $s \in \Z_q$ such that $[\vecb{d}]_2=[\matr{H}]_2\smallpmatrix{\vecb{b}\\ s}$. 

Since $\matr{R}$ is chosen uniformly at random in $\Z_q^{2 \times 2}$, the proof $([\matr{\Theta}]_1, [\matr{\Pi}]_2)$ is uniformly distributed conditioned on satisfying check 1) of algorithm $\algV$.
 Finally, the rest of the proof follows from Zero-Knowledge of $\Pi_\sfsum$ and $\Pi_\sfcom$.
\end{description}
\end{proof}
