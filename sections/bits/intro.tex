In this chapter construct a \emph{constant-size proof} that a set of $n$ commitments to elements in some field $\Z_q$ open to 0 or 1. Equivalently, we construct a constant size proof for the satisfiability of the equations $b_1(b_1-1)=0,\ldots,b_n(b_n-1)=0$.
Although solutions for this problem can be easily derived from general results of constant-size NIZK for any NP language \cite{EC:GGPR13,AC:DFGK14,EC:Groth16}, they would rely on strong and controversial assumptions, namely non-falsifiable assumptions. Therefore, it is an open question how to build constant-size proofs for this statement using only standard falsifiable assumptions. 

A set of $n$ commitments $\vecb{c}_1,\ldots,\vecb{c}_n$ to elements of $\Z_q$, each commitment of size $s$, defines a single commitment $\vecb{c}=(\vecb{c}_1,\ldots,\vecb{c}_n)$ to an element of $\Z_q^n$, of size $n\cdot s$. Alternatively, one can define the commitment $\vecb{c}$ so that its size may be $<n\cdot s$ and, depending on the size of $\vecb{c}$, there may or not be a unique opening. Thereby, we distinguish two different cases:

\begin{description}
\item[Perfectly Binding Commitment:] The commitment defines a unique vector $\vecb{x}\in\Z_q^n$. It must hold that $|\vecb{c}|\geq n\log q = \Omega(n)$.
\item[Non-binding Commitment:] In this case $\vecb{c}$ can be opened to many values and it is possible that $|\vecb{c}|= o(n)$.
\end{description}

In the second case it is not clear what a proof that the openings are in $\bits$ means. For example, if we use \emph{Multi-Pedersen} commitment, where a commitment to $\vecb{x}\in\Z_q^n$ is $[c]_1=\sum_{i\in[n]}x_i[g_i]_1+r[g_{n+1}]_1\in\GG_1$, each $[c]_1$ can be opened to any value, in particular to 0 or 1, and thus the proof is trivial. Although it does makes sense to do a \emph{Proof of Knowledge}, where one can extract a witness, we do not know how construct such proof system using only falsifiable assumptions.

For this reason, in \cite{AC:GonHevRaf15} we first concentrated in the perfectly binding case, specifically Groth-Sahai commitments and other perfectly-binding scheme with reuse of randomness. We find two interesting applications of this proof system: more efficient signature schemes, with emphasis on the case of \emph{Ring Signatures}, and more efficient \emph{Threshold Groth-Sahai proofs}.
 
Later, in \cite{ACNS:GonRaf16}, we tackle the non-binding case for a commitment scheme which is an ``hybrid'' between Multi-Pedersen commitments and Groth-Sahai commitments. We call this commitment scheme \emph{Extended Multi-Pedersen commitments} and, unlike Multi-Pedersen commitments, they can not be always opened to any value. 
While we realized that the techniques from the first work implicitly build a proof system for non-binding commitments, we did not know how to prove soundness (and we still do not know) for the specific distribution of the commitment keys used there. In turn, in the latter work we proved soundness for a specific set of distributions and showed that it imply previous results and can be also used to construct \emph{Aggregated Set-Membership Proofs} and more efficient \emph{Range Proofs} and \emph{Proofs of Correctness of a Shuffle}.

In Sect. \ref{sec:bits-binding} we describe our results for the perfectly binding case and the applications, and in Sect. \ref{sec:bits-non-binding} we describe our results for the non-binding case.
