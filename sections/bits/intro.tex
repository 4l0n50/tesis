In this chapter we solve the problem of showing that a set of commitments open to elements of the set $\bits$. If we think of these set of commitments as a single commitment $[\vecb{c}]_1$ to an element of $\bits^n$, for some $n\in\mathbb{N}$, we can distinguish two different cases:

\begin{description}
\item[Perfectly Binding Commitment:] The commitment defines a unique vector $\vecb{x}\in\Z_q^n$. It must hold that $|\vecb{c}|\geq n$.
\item[Non-binding Commitments:] In this case $[\vecb{c}]$ can be opened to many values and $|\vecb{c}|\leq n$.
\end{description}

Note that in the second case it is not clear what a proof of membership in $\Lang_{ck,\sfbits}^n$ means. For example, if we use \emph{Multi-Pedersen} commitment, where a commitment to $\vecb{x}\in\Z_q^n$ is $[c]_1=\sum_{i\in[n]}x_i[g_i]_1+r[g_{n+1}]_1\in\GG_1$, each $[c]_1$ can be opened to any value and thus $\Lang_{ck,\sfbits}^n$ is trivial. Although it do makes sense to do a \emph{Proof of Knowledge} that $[c]\in\Lang_{gk,\sfbits}^n$, where one can even extract a witness, we do not know how construct such proof system using only falsifiable assumptions.

For this reason, in \cite{AC:GonHevRaf15} we firstly concentrated in the perfectly binding case, Groth-Sahai commitments and other perfectly-binding scheme with reuse of randomeness. We find two interesting applications of this proof system: more efficient signature schemes, with emphasis on the case of \emph{Ring Signatures}, and more efficient \emph{Threshold Groth-Sahai proofs}.
 
Later, in \cite{ACNS:GonRaf16}, we tackle the non-binding case for a commitment scheme which is an ``hybrid'' between Multi-Pedersen commitments and Groth-Sahai commitments. We call this commitment scheme \emph{Extended Multi-Pedersen commitments}. 
While we realized that the techniques from the first work implicitly build a proof system for non-binding commitments, the extended Multi-Pedersen commitment scheme was not explicit and, for the specific distribution of the commitment keys used, we did not know how to prove soundness (and we still do not know). In turn, in the latter work we proved soundness for a specific set of distributions and showed that it imply previous results and also have more interesting applications: \emph{Aggregated Set-Membership Proofs}, \emph{Range Proofs}, and \emph{Proofs of Correctness of a Shuffle}.

In Sect. \ref{sec:bits-binding} we describe our results for the perfectly binding case and the applications, and in Sect. \ref{sec:bits-non-binding} we describe our results for the non-binding case.
