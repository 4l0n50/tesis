In this thesis we show that \textbf{all linear equations} and \textbf{all quadratic equations over $\Z_q$} admit an aggregated proof of size $\Theta(1)$. We also show that \textbf{all set-membership proofs} over $\Z_q$, with set size $\ell$, admit aggregated proofs of size $\Theta(\log \ell)$. We show that this results can be extended to linear equations over $\GG_1$ and/or $\GG_2$ and to set-membership proofs over $\GG_1$ or $\GG_2$ meeting some restrictions. We use this results to improve the efficiency of several protocols. 

%This thesis is divided in seven chapters. Chapter \ref{sec:intro} contains the introduction and this section.

%Chapter \ref{sec:prelim} contains the basic definitions which will be used in the other parts of this thesis. The first section contains the basic notation. The second section contains a brief description of public-key cryptography in the discrete logarithm setting. The third section introduces bilinear groups and the associated cryptographic assumptions, and the fourth section introduces Matrix Diffie-Hellman assumptions. The fifth section introduces Non-Interactive Zero-Knowledge proofs, the sixth section introduces Groth-Sahai proofs, and the seventh section describes Quasi-Adaptive Non-Interactive Zero-Knowledge proofs of membership in linear subspaces.

In Chapter \ref{sec:agg-asym} we develop new techniques to aggregate linear equations, recovering all the aggregation results of \cite{C:JutRoy14} (in particular, two-sided linear equations) in asymmetric bilinear groups. The latter (Type III bilinear groups, according to the classification of \cite{DAM:GalPatSma08}) are the most 
attractive 
from the perspective of a performance and security trade off, specially since the recent attacks on discrete logarithms in finite fields by Joux \cite{SAC:Joux13} and subsequent improvements. As applications we construct constant size proofs that many commitments -- even in different groups -- can be opened to the same values; and one-time linear homomorphic structure preserving signatures for messages splitted in two groups.

Chapter \ref{sec:bits} is devoted to obtain efficient proofs for quadratic equations over the integers. We construct constant size proofs for the satisfiability of many equations of the form $b(b-1)=0$. While this is just a particular type of quadratic equation, is the most representative type of quadratic equation and efficient proofs for other equations can be build using the same techniques. We distinguish two cases depending on the commitment scheme used to commit to the solutions: \emph{perfectly binding} or \emph{length-reducing}. 

In the perfectly binding case we consider \emph{Groth-Sahai commitments} to $b_1,\ldots,\allowbreak b_n\allowbreak\in\Z_q$ and show that $b_1(b_1-1)=0,\ldots,b_n(b_n-1)=0$ with a constant size proof (we also consider another perfectly binding commitment scheme and obtain similar results). We show how to apply this results to build more efficient signature schemes, more efficient proofs that \emph{1 out of many} equations are satisfied, and more efficient set-membership proofs.

Although in the case of length-reducing commitments a proof that $b(b-1)=0$ is in general useless -- since in the extreme case of \emph{perfectly hiding commitments} there is always an opening that satisfies the equation -- we introduce a new length-reducing commitment scheme that overcomes this problem. We call this commitment scheme \emph{Extended Multi-Pedersen commitments} which is an hybrid between Groth-Sahai and Multi-Pedersen commitments. We construct a constant size proof that the opening $(b_1,\ldots,b_n)^\top\in\Z_q^n$ of an Extended Multi-Pedersen commitment satisfies equations $b_1(b_1-1)=0,\ldots,b_n(b_n-1)=0$.
%What is interesting about Extended Multi-Pedersen commitments is that they can be perfectly hiding but they can also be perfectly binding at one (and only one) coordinate, depending on the setup (the commitment key). In this way, our NIZK proof for Extended Multi-Pedersen commitments implies that  $b_i\in\bits$).
Our proof for the length-reducing case is a key ingredient for the results of the last part of this thesis.
 
Chapter \ref{sec:shuf-rp} focuses on Set-Membership proofs, that is show that a variable $x$ is in some set $S$. We show that many Set-Membership proofs -- i.e.~$x_1,\ldots,x_n\in S$ -- can be proven with a single proof of size $\Theta(|S|)$, when $S$ is a set of group elements, and $\Theta(\log |S|)$, when $S$ is a range of integers. We call this primitive \emph{Aggregated Zero-Knowledge Set-Membership proof}, because the proof size is independent of the number of variables. We show that aggregated Zero-Knowledge Set-membership proofs allow efficiency improvements for two non-interactive arguments, namely, range proofs and proofs of correctness of a shuffle. Apart from efficiency improvements, we obtain a unified modular construction for the two aforementioned problems, while state of the art solutions are build from diverse techniques and assumptions.

Chapter \ref{sec:extras} contains two additional results which remain unpublished up to the publication of this thesis. The first result is the first $\Theta(\sqrt[3]{n})$ ring signature without random oracles. The second result is an improved aggregated set-membership proof of size $\Theta(\log |S|)$ for $S$ a set of integers. Further, our improved proof works for non-fixed sets -- i.e.~while each instance of the proof system from Chapter \ref{sec:shuf-rp} works for a fixed set, a single instance of the new proof system works for any set. Then, we show how to obtain proofs of size $\Theta(\log|S|)$ when $S\subset\GG_s$, $s\in\{1,2\}$, for fixed sets. As an application we construct theoretical -- i.e.~with large constants -- $\Theta(\log n)$ ring-signatures without random oracles.
