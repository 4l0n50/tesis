In this thesis we show that \textbf{all linear equations} and \textbf{all quadratic equations over $\Z_q$} admit an aggregated proof of size $\Theta(1)$. We also show that \textbf{all set-membership proofs} over $\Z_q$, with set size $\ell$, admit aggregated proofs of size $\Theta(\log \ell)$. We show that this results can be extended to linear equations over $\GG_1$ and/or $\GG_2$ and to set-membership proofs over $\GG_1$ or $\GG_2$ meeting some restrictions, and that all set-membership proofs over $\GG_1$ or $\GG_2$ admit aggregated proofs of size $\Theta(\ell)$ without any restriction. We use this results to improve the efficiency of several protocols. 

In the first part of this thesis we develop new techniques to aggregate
other types of linear equations, recovering all the aggregation results of \cite{C:JutRoy14} (in particular, two-sided linear equations) in asymmetric bilinear groups. The latter (Type III bilinear groups, according to the classification of \cite{DAM:GalPatSma08}) are the most 
attractive 
from the perspective of a performance and security trade off, specially since the recent attacks on discrete logarithms in finite fields by Joux \cite{SAC:Joux13} and subsequent improvements. Considerable research effort 
(e.g. \cite{C:AGOT14a,EC:Freeman10})
has been put into translating pairing-based cryptosystems from a setting with more structure in which design is simpler (e.g. composite-order or symmetric bilinear groups) to a more efficient setting (e.g. prime order or asymmetric bilinear groups). In this line, we aim not only at obtaining new results in the asymmetric setting but also to translate known results and develop new tools specifically designed for it which might be of independent interest.

The second part of this thesis is devoted to obtain efficient proofs for quadratic equations over the integers. We construct constant size proofs for the satisfiability of many equations of the form $b(b-1)=0$. While this is just a particular type of quadratic equation, it is the most representative type of quadratic equation and efficient proofs for other equations can be build using the same techniques. We distinguish two cases depending on the commitment scheme used to commit to the solutions: \emph{perfectly binding} or \emph{length-reducing}.

In the perfectly binding case we consider \emph{Groth-Sahai commitments} to $b_1,\ldots,\allowbreak b_n\allowbreak\in\Z_q$ and show that $b_1(b_1-1)=0,\ldots,b_n(b_n-1)=0$ with a constant size proof (we also consider another perfectly binding commitment scheme and obtain similar results). We show how to apply this results to build more efficient signature schemes and more efficient proofs that \emph{1 out of many} equations are satisfied. In the case of length-reducing commitments, in general, it is not clear what a proof that $b(b-1)=0$ means, since in the extreme case of \emph{perfectly hiding commitments} there is always an opening that satisfy the equation (in fact in this case it is more usual to give \emph{Proof of Knowledge}). We introduce a new commitment scheme, \emph{Extended Multi-Pedersen commitments}, which is an hybrid between Groth-Sahai and Multi-Pedersen commitments and construct a constant size proof that the opening $(b_1,\ldots,b_n)^\top\in\Z_q^n$ of an Extended Multi-Pedersen commitment satisfies equations $b_1(b_1-1)=0,\ldots,b_n(b_n-1)=0$. What is interesting about Extended Multi-Pedersen commitments is that they can be perfectly hiding but they can also be perfectly binding at one (and only one) coordinate, depending on the setup (the commitment key). Furthermore, the different setups are computationally indistinguishable and, thereby, one can randomly choose an index which remains hidden to the adversary such that $b_i$ is uniquely defined (and thus our NIZK proof for Extended Multi-Pedersen commitments implies that $b_i\in\bits$). Our proof for the length-reducing case is a key ingredient for the results of the last part of this thesis.
 
The last part of this thesis focuses on Set-Membership proofs, that is show that $x$ is in some set $S$. We show that many Set-Membership proofs -- i.e.~$x_1,\ldots,x_n\in S$ -- can be proven with a single proof of size $\Theta(|S|)$, when $S$ is a set of group elements, and $\Theta(\log |S|)$, when $S$ is a set of integers. We call this primitive \emph{Aggregated Zero-Knowledge Set-Membership proof}, because the proof size is independent of the number of variables. We show that aggregated Zero-Knowledge Set-membership proofs allow efficiency improvements for two non-interactive arguments, namely, range proofs and proofs of correctness of a shuffle. Apart from efficiency improvements, we obtain a unified modular construction for the to problems mentioned above, while the state of the art solutions are build from different techniques and assumptions.
