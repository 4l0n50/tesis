As announced in the previous section, our construction proves two different statements. For the second statement, conditions 2.a), 3.b) are written as a single system of equations with a single matrix $\matr{\Xi}$ and use $\Pi_\sflin$ to prove that certain vector of $\GG_1$ is in the span of $\matr{\Xi}$.

This matrix is defined as:
$$\matr{\Xi}(\matr{M},\matr{N},\matr{\Lambda},\matr{G}):=
\left(\begin{array}{ccc|ccc}
\matr{\Sigma}             & \cdots & \matr{0}                  & \matr{0} \\
\vdots                    & \ddots & \vdots                    & \vdots   \\
\matr{0}                  & \cdots & \matr{\Sigma}             & \matr{0} \\
\hline
\overline{\vecb{g}}_1^\lb & \cdots & \overline{\vecb{g}}_n^\lb & \vecb{g}_{n+1}^\lb
\end{array}\right),
$$ where $\matr{\Sigma}:=\begin{pmatrix}
    \matr{M}       & \matr{N}\\
    \matr{\Lambda} & \matr{0}
\end{pmatrix}$ and
 $\overline{\vecb{g}}_i^\lb:=\vecb{g}_i^\lb||\vecb{0}$, $i\in[n]$. 
 
Define $\vecb{y}:=\vecb{c}_1\oplus \grkb{\alpha}
\oplus \ldots \oplus \vecb{c}_n\oplus \grkb{\alpha} 
\oplus \vecb{d}_1 \oplus  \ldots \oplus  \vecb{d}_m$
and 
 $\vecb{v}:=\vecb{b}_1\oplus \vecb{w}_1 \oplus \ldots \oplus \vecb{b}_n\oplus \vecb{w}_n \oplus r_1 \oplus \ldots \oplus r_n.$ The statement we want to prove is that $[\vecb{y}]_1 \in \mathbf{Im}([\matr{\Xi}]_1)$, and the witness is $\vecb{v}$. The upper left block of the matrix guarantees condition 2.a), while the two lower blocks guarantee 
condition 2.b).

The full description of the proof system is in Fig. \ref{fig:bin-leng-nizk} and security follows from Theorem \ref{theo:aggset}.
\begin{figure}
\begin{\algSize}
$$
\begin{array}{l}
\begin{array}{l}
\algK(\gk,[\matr{M}]_1,[\matr{N}]_1,n)
\quad (\mathsf{S}_1(\gk,[\matr{M}]_1,[\matr{N}]_1,n))
\\
\hline
[\matr{G}]_1 \gets \MP.\algK(1^\lambda,n)\\
{[\matr{\Xi}]_1 := [\matr{\Xi}(\matr{M},\matr{N},\matr{\Lambda},\matr{G})]_1}\\
\crs_\sflin\gets\Pi_\sflin.\algK(\gk,[\matr{\Xi}]_1)\\
\crs_\sfbits\gets\Pi_\sfbits.\algK(\gk,[\matr{G}]_1,\lb)\\
\text{Return } \ \crs:=(\crs_\sflin,\crs_\sfbits).\\
(\tau_\sflin\gets\Pi_\sflin.\algS_1(\gk,[\matr{\Xi}]_1)\\
\tau_\sfbits\gets\Pi_\sfbits.\algS_1(\gk,[\matr{G}]_1,\lb).\\
\tau := (\tau_\sflin,\tau_\sfbits)).\\
\\
\end{array}
\\
\begin{array}{l}
{\algP(\mathsf{crs}, \{[\vecb{c}_j]_1,\langle \vecb{b}_j,\vecb{w}_j\rangle:j\in[n]\})}\\
\hline
{[\vecb{d}_i]_1} := \MP.\Com_{[\matr{G}]_1}(\vecb{b}_i^*;r_i),\\
r_i \gets\Z_q, \forall i\in[\lb]\\
\pi_\sflin \gets 
    \Pi_\sflin.\algP
    (
        \crs_\sflin,
            [\vecb{y}]_1,
            \vecb{v}
    )\\
\pi_\sfbits \gets
    \Pi_\sfbits.\algP
    (
        \crs_\sfbits,
        \{[\matr{d}_i]_1, \langle\matr{b}^*_i,r_i\rangle:i \in[\lb]\}
    )\\
\text{Return } \  ([\vecb{d}]_1,\pi_\sflin,\pi_\sfbits). \\
\\
\end{array}
\\
\begin{array}{l}
{\algV(\mathsf{crs},\{[\vecb{c}_j]_1:j\in[n]\},([\vecb{d}]_1,\pi_\sflin,\pi_\sfbits))}\\
\hline
\mathsf{ans}_1 \gets
    \Pi_\sflin.\algV
    (
        \crs_\sflin,
            [\vecb{y}]_1,
        \pi_\sflin
    )\\
\mathsf{ans}_2 \gets \Pi_\sfbits.\algV(\crs_\sfbits,\{[\vecb{d}_i]_1:i\in[\lb]\},\pi_\sfbits)\\
\text{Return } \ \mathsf{ans}_1\wedge\mathsf{ans}_2.
\\
\\
\end{array}
\\
\begin{array}{l}
{\mathsf{S}_2(\crs,[\vecb{c}]_1,\tau)}\\
\hline
{[\vecb{d}_i]_1} := \MP.\Com_{[\matr{G}]_1}(\matr{0}_{n\times 1};\tilde{{r}}_i)\\
\tilde{{r}}_i\gets\Z_q, \forall i\in[\lb]\\
\pi_\sflin \gets 
    \Pi_\sflin.\algS
    (
        \crs_\sflin,
            [\vecb{y}]_1,
       \tau_\sflin
    )\\
\pi_\sfbits \gets
    \Pi_\sfbits.\algS
    (
        \crs_\sfbits,\{[\vecb{d}_i]_1:i\in[\lb]\},
        \tau_\sfbits
    )\\
\text{Return } \  ([\vecb{d}]_1,\pi_\sflin,\pi_\sfbits). \\
\end{array}
\end{array}$$
\end{\algSize}
\caption{Proof system for the language $\Lang_{\matr{M},\matr{N},\matr{\Lambda},\grkb{\alpha}}^n$, where $\Pi_\sfbits$ is the proof system for $\Lang_{ck,\sfbits}^m$ from Section~\ref{sec:matr-bits}, $\vecb{d}:=\vecb{d}_1\oplus\ldots\oplus\vecb{d}_\lb$, and $\vecb{c}:=\vecb{c}_1\oplus\ldots\oplus\vecb{c}_n$. The proof size is $(2\lb+11)|\GG_1|+10|\GG_2|$.\label{fig:bin-leng-nizk}}
\end{figure}



\begin{theorem} \label{theo:aggset} There exists a QA-NIZK argument $\Pi_\sfset$ for membership in the language $\Lang_{[\matr{M}]_1,[\matr{N}]_1,\matr{\Lambda},\grkb{\alpha}}^n$ with proof size  $(2\lb+11)|\GG_1|+10|\GG_2|$, perfect completeness, perfect-zero knowledge and computational soundness. 
\end{theorem}

\paragraph{Completeness.} If $([\vecb{c}_1]_1,\ldots,
[\matr{c}_n]_1) \in\Lang_{[\matr{M}]_1,[\matr{N}]_1,\matr{\Lambda},\grkb{\alpha}}^n$, then for every $j\in[\lb]$ there exists $\matr{b}_j\in\bits^{\lb},\matr{w}_j \in\Z_q^{\lc}$ such that 
$$
\pmatri
{
\matr{c}_j\\
\grkb{\alpha}
}
=
\begin{pmatrix}
\matr{M}       & \matr{N}\\
\matr{\Lambda} & \matr{0}_{\ld\times \lc}
\end{pmatrix}
\pmatri
{
    \matr{b}_j\\
    \matr{w}_j
}.
$$
Given that $[\matr{d}_i]_1 = \MP.\Com_{[\matr{G}]_1}(\vecb{b}_i^*;r_i)=\sum_{j\in[n]}b_{i,j}[\vecb{g}_i]_1+r_i[\vecb{g}_{n+1}]_1$,
\begin{eqnarray*}
 \vecb{d}_1\oplus\ldots\oplus\vecb{d}_\lb
          &=&  {\sum_{j\in[n]}\sdmatrix{\vecb{g}_i}\smallpmatrix{b_{1,j}\\\vdots\\b_{\lb,j}}+\sdmatrix{\vecb{g}_{n+1}}\smallpmatrix{r_1\\\vdots\\r_\lb}}\\
         &= & (\overline{\vecb{g}}_1^\lb||\cdots||\overline{\vecb{g}}_{n}^\lb||\vecb{g}_{n+1}^\lb)(\vecb{b}_1\oplus\vecb{w}_1\oplus\ldots\oplus\vecb{b}_n\oplus\vecb{w}_n\oplus\vecb{r}).
\end{eqnarray*}
Combining these two facts we conclude that
$[\vecb{y}]_1 = [\matr{\Xi}]_1 \vecb{v}$, which implies that $\mathsf{ans}_1=1$. On the other hand, $\vecb{b}_1^*,\ldots,\vecb{b}_\lb^*\in\bits^n$ implies that $\mathsf{ans}_2=1$.
  
\paragraph{Soundness.}

\begin{theorem}
Let $\adv_{\mathcal{PS}}(\advA)$ be the advantage of an adversary $\advA$ against the soundness of the proof system described in Fig. \ref{fig:bin-leng-nizk}. There exist PPT adversaries $\advD$, against $\distlin_1\mbox{-}\mddh$ in $\GG_1$, $\advB_1$ against the soundness of $\Pi_\sflin$, and $\advB_2$, against the soundness of $\Pi_\sfbits$, such that
$$
\adv_{\mathcal{PS}}(\advA)\leq n\left(2/q + \adv_{\distlin_1,\GG_1}(\advD)+\adv_{\Pi_{\sflin}}(\advB_1)+\adv_{\Pi_\sfbits}(\advB_2)\right).
$$
\end{theorem}

The proof follows from the indistinguishability of the following games:
\begin{itemize}
\item[$\mathsf{Real}$] This is the real soundness game. The output is $1$ if the adversary breaks soundness, that is, if the adversary submits $([\vecb{c}_1]_1,\ldots,[\vecb{c}_n]_1)\notin\Lang_{[\matr{M}]_1,[\matr{N}]_1,\matr{\Lambda},\grkb{\alpha}}^n$ and the corresponding proof which is accepted by the verifier.
\item[$\sfGame_0$] This is identical as $\mathsf{Real}$ except that algorithm $\algK_1$ does not receive $[\matr{N}]_1$ as input but it samples $\matr{N}$ itself.
\item[$\sfGame_1$] This game is identical to $\sfGame_0$ except that the simulator picks a random $j^*\in[n]$, and 
aborts if $\algF([\matr{M}]_1,\matr{N},[\vecb{c}_{j^*}]_1)=1$ (that is to say if $[\vecb{c}_{j^*}]_1 \in \Lang_{[\matr{M}]_1,[\matr{N}]_1,\matr{\Lambda},\grkb{\alpha}}$.)
\item[$\sfGame_2$] This game is identical to $\sfGame_1$ but now $\matr{G} \gets \distlinjsnzero$.
\end{itemize}

It is obvious that the first two games are indistinguishable. 
The rest of the argument goes as follows. 

\begin{lemma} $\Pr\left[ \mathsf{Game}_1(\advA)=1\right]\geq\dfrac{1}{n}\Pr\left[\mathsf{Game}_0(\advA)=1\right].$
\end{lemma}

\begin{proof}  The probability that
 $\mathsf{Game}_1(\advA)=1$ is the probability that  a) $\mathsf{Game}_0(\advA)=1$ and
b)  $[\vecb{c}_{j^*}]_1 \notin \Lang_{\matr{M},\matr{N},\matr{\Lambda},\grkb{\alpha}}$. The view of adversary $\advA$ is independent of $j^*$, while, if $\mathsf{Game_0}(\advA)=1$, then there is at least one index $j \in [n]$ such that $[\vecb{c}_{j}]_1 \notin \Lang_{[\matr{M}]_1,[\matr{N}]_1,\matr{\Lambda},\grkb{\alpha}}$. Thus, 
the probability that the event described in b) occurs conditioned on $\mathsf{Game_0}(\advA)=1$, is greater than or equal to $1/n$ and the lemma follows.
\end{proof}

\begin{lemma} There exists a\ $\distlin_1$-$\mddh_{\GG_1}$ adversary $\advD'$ such that
$|\Pr\left[\mathsf{Game}_{1}(\advA)=1\right]$ $-\Pr\left[\mathsf{Game}_{2}(\advA)=1\right]|$ $\leq
    \mathsf{Adv}_{\distlin_1,\ggen_a}(\advD').$
\end{lemma}

The proof is obvious in the light of lemma \ref{lemma:dist-i}. 
%\begin{proof}
%We constr:uct an adversary $\algD$ (as the one in Lemma \ref{lemma:dist-i}) that receives a matrix $[\matr{G}]_1\in\GG_1^{2\times(n+1)}$, where $\matr{G}$ is sampled either from $\dist_{2,n+1}^0$ or $\dist_{2,n+1}^{j^*}$.
%Adversary $\advD$ honestly simulates the rest of the CRS replacing $[\matr{G}]_1$ by the one received as input.
%
%It is immediate to see that adversary $\advD$ perfectly simulates $\sfGame_1$ when $\matr{G}\gets\dist_{2,n+1}^0$ and $\sfGame_2$ when $\matr{G}\gets\dist_{2,n+1}^{j^*}$. The rest of the proof follows from Lemma \ref{lemma:dist-i}.  
%\end{proof}

\begin{lemma}
There exist adversaries $\advB_1,\advB_2$ such that $\Pr[\sfGame_2(\advA)=1]\leq\adv_\sflin(\advB_1)+\adv_{\sfbits}(\advB_2)$.
\end{lemma}

\begin{proof}
Let $E$ the event where $\vecb{y}\notin\mathbf{Im}(\matr{\Xi})$, and let $\advB_1$ the adversary against $\Pi_\sflin$ that outputs $[\tvecb{x}]_1\oplus[\vecb{d}]_1$ and $\pi_\sflin$. Obviously, $\Pr[\sfGame_2(\advA)=1]\leq\adv_{\Pi_\sflin}(\advB_1)+\Pr[\sfGame_2(\advA)=1|\neg E]$, because 
if $E$ occurs the adversary breaks the soundness of
 $\Pi_\sflin$. To prove the lemma, there is only left to bound this last probability.

If $\sfGame(\advA)=1$, then $[\vecb{c}_{j^*}]_1\notin \Lang_{[\matr{M}]_1,[\matr{N}]_1,\matr{\Lambda},\grkb{\alpha}}$ while all the verification equations are accepted. $\neg E$ implies that there exists $\vecb{v}$, 
which uniquely defines some $(\widetilde{\vecb{b}}_{j^*},\vecb{w}_{j^*})$ and some $\widetilde{\vecb{b}}_i^*$ such that:
$$
\vecb{y}=\matr{\Xi} \vecb{v}
\Longrightarrow
\pmatri{\vecb{c}_{j^*}\\ \grkb{\alpha}}=
\begin{pmatrix}
    \matr{M}       & \matr{N}\\
    \matr{\Lambda} & \matr{0}
\end{pmatrix}
\pmatri
{
    \widetilde{\vecb{b}}_{j^*}\\
    \vecb{w}_{j^*}
}
$$
 and $[\matr{d}_i]_1=\MP.\Com_{[\matr{G}]_1}(\widetilde{\vecb{b}}_i^*;r), \forall i\in[\lb].$
Since in this game $[\vecb{c}_{j^*}]_1 \notin \Lang_{\matr{M},\matr{N},\matr{\Lambda},\grkb{\alpha}}$ (otherwise the game aborts), then $\widetilde{\vecb{b}}_{j^*}\notin\bits^\lb$. 

Since the MP commitment is perfectly binding at coordinate 
$j^*$, this implies that  $([\vecb{d}_1]_1\ldots,[\vecb{d}_\lb]_1)\notin \Lang_{[\matr{G}]_1,\sfbits}^m$. 
Therefore, an adversary $\advB_2$ that simulates $\advA$ and outputs $(([\vecb{d}_1]_1\ldots,[\vecb{d}_\lb]_1),\pi_\sfbits)$ violates soundness of $\Pi_\sfbits$ with probability at least $\Pr[\sfGame_2(\advA)=1|\neg E]$.
\end{proof}

\paragraph{Zero-Knowledge.}

\begin{theorem}
The proof system is perfect quasi-adaptive zero-knowledge.
\end{theorem}

\begin{proof}
Recall that $([\vecb{c}_1]_1,\ldots,[\vecb{c}_n]_1)\in\Lang^n_{[\matr{M}]_1,[\matr{N}]_1,\matr{\Lambda},\grkb{\alpha}}$ implies that
$
\tvecb{x}
=\matr{\Sigma}^n\tvecb{b}
$
and, given that $\distlin_1^{n,0}$ defines perfectly hiding commitments, for all $i\in[\lb]$ there is some $r_i\in\Z_q$ such that
$[\vecb{d}_i]_1=\MP.\Com(\matr{0}_{n\times1};\tilde{{r}}_i)=\MP.\Com(\vecb{b}_i^*;r_i)$. Then $\vecb{y}\in\mathbf{Im}(\matr{\Xi})$ and thus perfect zero-knowledge of $\Pi_\sflin$ and $\Pi_\sfbits$ implies that $\pi_\sflin$ and $\pi_\sfbits$ are correctly distributed.
\end{proof}
