The main result of this Section is a proof, of roughly the same size as in the last Section ($2\lb|\GG_1|+\Theta(1)$), that  $([\vecb{c}_1]_1,\ldots,[\vecb{c}_n]_1)$ is in $\Lang_{[\matr{M}]_1,[\matr{N}]_1,\matr{\Lambda},\grkb{\alpha}}^n$.

For all $j \in [n]$, let $(\vecb{b}_j,\vecb{w}_j) \in \{0,1\}^{\lb} \times \Z_q^{\lc}$ be the witness of $\vecb{c}_j \in \Lang_{[\matr{M}]_1,[\matr{N}]_1,\matr{\Lambda},\grkb{\alpha}}.$ Let $\matr{B}=(\vecb{b}_1|| \ldots ||\vecb{b}_n)$ and let $\vecb{b}^*_{i}$, $i \in [m]$ the ith row of $\matr{B}$. To get a proof of size independent of $n$ we commit to $\matr{B}$ ``compressing the rows'', that is, the proof includes MP commitments $[\vecb{d}_i]_1$, $i \in [n]$ to $\vecb{b}_i^*$.\footnote{To get a constant-size proof, it would be tempting to compress the commitments to all of $\matr{B}$, but we do not know how to prove soundness in this case.} Further, as announced in Section~\ref{sec:techniques}, the proof consists of two independent statements:
\begin{itemize}
\item $\exists \vecb{r} \in \Z_q^\lb, \matr{B} \in \Z_q^{\lb \times n}$ such that  
$\text{1'')} \matr{B} \in \{0,1\}^{\lb \times n}$ and $\text{3'')} \forall i \in [\lb]: \vecb{d}_i=\matr{G}\smallpmatrix{\vecb{b}_i^*  \\ r_i}$,
\item $\exists \widetilde{\vecb{r}} \in \Z_q^\lb, \vecb{w}_1,\ldots,\vecb{w}_n \in\Z_q^\lc, \widetilde{\matr{B}} \in \Z_q^{\lb \times n}$ such that  
   $\text{2'')}\forall j \in [n], \smallpmatrix
{
    \vecb{c}_j\\
    \grkb{\alpha}
}
=
\smallpmatrix
{
    \matr{M}       & \matr{N}\\
    \matr{\Lambda} & \matr{0}
}
\smallpmatrix
{
    \widetilde{\vecb{b}}_j\\
    \vecb{w}_j
}$ and $3'') \forall i \in [\lb], \vecb{d}_i=\matr{G}   \smallpmatrix{\widetilde{\vecb{b}}^*_i  \\ \widetilde{r}_i}  $.
\end{itemize} 
For the first statement we use the constant-size argument for $\Lang_{ck,\sfbits}^m$ of Section~\ref{sec:bits}. For the second statement, we write conditions 2''), 3'') as a single system of equations and use $\Pi_\sflin$ to prove that it can be satisfied. 

The soundness argument follows from the arguments exposed in Section~\ref{sec:techniques}. The full description of the argument is in Fig. \ref{fig:bin-leng-nizk} and the proof of the following theorem can be found in Appendix~\ref{app:bin-lang}.
\begin{figure}
\begin{\algSize}
$$
\begin{array}{ll}
\begin{array}{l}
\algK_1(\gk,[\matr{M}]_1,[\matr{N}]_1,n)
\quad (\mathsf{S}_1(\gk,[\matr{M}]_1,[\matr{N}]_1,n))
\\
\hline
[\matr{G}]_1 \gets \MP.\algK(1^\lambda,n)\\
{[\matr{\Xi}]_1 := [\matr{\Xi}(\matr{M},\matr{N},\matr{\Lambda},\matr{G})]_1}\\
\crs_\sflin\gets\Pi_\sflin.\algK_1(\gk,[\matr{\Xi}]_1)\\
\crs_\sfbits\gets\Pi_\sfbits.\algK_1(\gk,[\matr{G}]_1,\lb)\\
\text{Return } \ \crs:=(\crs_\sflin,\crs_\sfbits).\\
(\tau_\sflin\gets\Pi_\sflin.\algS_1(\gk,[\matr{\Xi}]_1)\\
\tau_\sfbits\gets\Pi_\sfbits.\algS_1(\gk,[\matr{G}]_1,\lb).\\
\tau := (\tau_\sflin,\tau_\sfbits)).\\
\\
\end{array}
&
\begin{array}{l}
{\algP(\mathsf{crs}, \{[\vecb{c}_j]_1,\langle \vecb{b}_j,\vecb{w}_j\rangle:j\in[n]\})}\\
\hline
{[\vecb{d}_i]_1} := \MP.\Com_{[\matr{G}]_1}(\vecb{b}_i^*;r_i),\\
r_i \gets\Z_q, \forall i\in[\lb]\\
\pi_\sflin \gets 
    \Pi_\sflin.\algP
    (
        \crs_\sflin,
            [\vecb{y}]_1,
            \vecb{v}
    )\\
\pi_\sfbits \gets
    \Pi_\sfbits.\algP
    (
        \crs_\sfbits,
        \{[\matr{d}_i]_1,\\
\qquad
        \langle\matr{b}^*_i,r_i\rangle:i \in[\lb]\}
    )\\
\text{Return } \  ([\vecb{d}]_1,\pi_\sflin,\pi_\sfbits). \\
\\
\\
\\
\end{array}\\
\begin{array}{l}
{\algV(\mathsf{crs},\{[\vecb{c}_j]_1:j\in[n]\},([\vecb{d}]_1,\pi_\sflin,\pi_\sfbits))}\\
\hline
\mathsf{ans}_1 \gets
    \Pi_\sflin.\algV
    (
        \crs_\sflin,
            [\vecb{y}]_1,
        \pi_\sflin
    )\\
\mathsf{ans}_2 \gets \Pi_\sfbits.\algV(\crs_\sfbits,\{[\vecb{d}_i]_1:i\in[\lb]\},\pi_\sfbits)\\
\text{Return } \ \mathsf{ans}_1\wedge\mathsf{ans}_2.
\\
\\
\\
\\
\end{array}
&
\begin{array}{l}
{\mathsf{S}_2(\crs,[\vecb{c}]_1,\tau)}\\
\hline
{[\vecb{d}_i]_1} := \MP.\Com_{[\matr{G}]_1}(\matr{0}_{n\times 1};\tilde{{r}}_i)\\
\tilde{{r}}_i\gets\Z_q, \forall i\in[\lb]\\
\pi_\sflin \gets 
    \Pi_\sflin.\algS
    (
        \crs_\sflin,
            [\vecb{y}]_1,
       \tau_\sflin
    )\\
\pi_\sfbits \gets
    \Pi_\sfbits.\algS
    (
        \crs_\sfbits,\\
\qquad\quad  \{[\vecb{d}_i]_1:i\in[\lb]\},
        \tau_\sfbits
    )\\
\text{Return } \  ([\vecb{d}]_1,\pi_\sflin,\pi_\sfbits). \\
\end{array}
\end{array}$$
\end{\algSize}
\caption{Proof system for the language $\Lang_{\matr{M},\matr{N},\matr{\Lambda},\grkb{\alpha}}^n$, where $\Pi_\sfbits$ is the proof system for $\Lang_{ck,\sfbits}^m$ from Sect. \ref{sec:matr-bits}, $\vecb{d}:=\vecb{d}_1\oplus\ldots\oplus\vecb{d}_\lb$, and $\vecb{c}:=\vecb{c}_1\oplus\ldots\oplus\vecb{c}_n$. The proof size is $(2\lb+11)|\GG_1|+10|\GG_2|$.\label{fig:bin-leng-nizk}}
\end{figure}



\begin{theorem} \label{theo:aggset} There exists a QA-NIZK argument $\Pi_\sfset$ for membership in the language $\Lang_{[\matr{M}]_1,[\matr{N}]_1,\matr{\Lambda},\grkb{\alpha}}^n$ with proof size  $(2\lb+11)|\GG_1|+10|\GG_2|$, perfect completeness, perfect-zero knowledge and computational soundness. 
\end{theorem}

