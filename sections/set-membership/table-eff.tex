
\begin{table}[h]
\begin{center}
\begin{minipage}{\textwidth}
\begin{center}
\begin{scriptsize}
\begin{tabular}{|l|ll|ll|}
\hline
                                                   & \multicolumn{2}{c|}{Shuffle Argument} & \multicolumn{2}{c|}{Range Argument} \\
                                                   & \cite{AC:GroLu07}          
%& \cite{EPRINT:FauLip15}
 & $\Pi_\mathsf{shuffle}$
                                                   & \cite{PAIRING:RiaKohPre09} & $\Pi_{\mathsf{range}\mbox{-}\mathsf{proof}}$ 
\\ \hline\hline
\rule{0pt}{2.5ex}CRS size                          & $2n + 8$                   
%& $8n + 17$
              & $(n^2+24n+36,23n+37)$                
                                                   & $\Theta(\frac{n}{\log n-\log\log n})$ & $(6n^2,6n^2)$ \\
% (6n^2+13n+n+\frac{n}{klogn}+2klogn,6n^2+13n+n+\frac{n}{klogn}+34)
\rule{0pt}{2.5ex}Proof size                        & $15n + 120$                
%& $(5n+2,2n)$
            & $(4n+17,14)$
                                                   & $\Theta(\frac{n}{\log n-\log\log n})$ & $(\frac{2n}{k\log n},10)$ \\
%$(\frac{2n}{k\log n}+2k\log n+11,10)$
\rule{0pt}{2.5ex}$\algP$'s comp.                   & $51n + 246$               
%& $22n + 11$
             & $11n+17$
                                                   & $\Theta(\frac{n}{\log n-\log\log n})$ & $2n$ \\
%2n+\frac{3n}{klogn}+3k\log n+2
\rule{0pt}{2.5ex}$\algV$'s comp.                   & $75n + 282$               
%& $18n + 6$
              & $13n+55$
                                                   & $\Theta(\frac{n}{\log n-\log\log n})$ & $\frac{4n}{k\log n}$ \\
%\frac{n}{k\log n}+6k\log n+62
\rule{0pt}{2.5ex}Assumption                        & PP                        
%& KE
                    & SXDH+SSDP
                                                   & $q$-HSDH                   & SXDH+SSDP \\\hline 
\end{tabular}
\end{scriptsize}
\end{center}
\caption{Comparison of our Shuffle, $\Pi_\mathsf{shuffle}$, and Range, $\Pi_{\mathsf{range}\mbox{-}\mathsf{proof}}$, arguments with the literature. To increase readability, for $\Pi_{\mathsf{range}\mbox{-}\mathsf{proof}}$ we include only the leading part of the sizes, that is, we write $f(n)$ and we mean $f(n)+o(f(n))$. Notation $(x,y)$ means $x$ elements of $\GG_1$ and $y$ elements of $\GG_2$. ``PP'' stands for the Permutation Pairing assumption.
%and ``KE'' for Knowledge of Exponent assumption ({\color{red} ver cual}).
The prover's computation is measured by the number of exponentiations (i.e. $z[x]_i$) and the verifier's computation is measured by the number of pairings.\label{table:eff}  } 
\end{minipage}
\vspace{-0.54cm}

\end{center}
\end{table}
