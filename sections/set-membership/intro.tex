This chapter focuses on obtaining efficiency improvements for non-interactive arguments: \emph{Range proofs} and \emph{Proof of Correctness of a Shuffle}, based only on falsifiable assumptions.  To derive efficiency improvements these arguments we develop a new cryptographic primitive which we call \emph{Aggregated Zero-Knowledge Set-Membership Proof} (aZKSMP). An aZKSMP allows to show that the openings of many commitments belong to a public set, and we say that the proof is \emph{aggregated} because its size does not depend on the number of commitments.

Our resulting proofs are more efficient in terms of proof size and are based on more standard assumptions, but they have a rather large common reference string. They build on the recent arguments for membership in linear spaces of \cite{EC:LPJY14,C:JutRoy14,EC:KilWee15} and the argument for proving that some commitment to a vector of integers in $\Z_q^{n}$ opens to $\{0,1\}^n$ due to \cite{AC:GonHevRaf15}. 
 

