This chapter focuses on obtaining efficiency improvements for two non-interactive arguments, namely \emph{range proofs} and \emph{proof of correctness of a shuffle}, based only on falsifiable assumptions.  To derive efficiency improvements for these arguments we develop a new cryptographic primitive which we call \emph{aggregated zero-knowledge set-membership proof} (aZKSMP). A zero-knowledge set-membership proof allows to show, in zero-knowledge, that the openings of many commitments belong to a public set. We say that the proof is {aggregated} when the size of the proof does not depend on the number of commitments.

Our resulting proofs are more efficient in terms of proof size and are based on more standard assumptions, but they have a rather large common reference string. They build on the recent arguments for membership in linear spaces \cite{EC:LPJY14,C:JutRoy14,EC:KilWee15}, arguments of membership in linear spaces of Chapter \ref{sec:agg-asym}, and the argument for proving that some commitment to a vector of integers in $\Z_q^{n}$ opens to $\{0,1\}^n$ from Section~\ref{sec:bits-non-binding}.
 
