\subsubsection{Zero Knowledge Set Membership Arguments.}
Camenisch et al. constructed $\Theta(1)$ interactive zero-knowledge set membership arguments using Boneh-Boyen Signatures, and they prove them secure under the $q$-SDH assumption \cite{AC:CamChaShe08}. Bayer and Groth constructed $\Theta(\log |S|)$ interactive zero-knowledge arguments for polynomial evaluation, which can be used to construct set membership arguments, relying only on the discrete logarithm assumption \cite{EC:BayGro13}.
However, none of the previous constructions has addressed the problem of aggregating many proofs, and a direct use of them will end up with a proof of size $\Omega(n)$.

\subsubsection{NIZK Shuffle and Range Arguments.}
The most efficient NIZK shuffle argument under falsifiable assumptions is the one from Groth and Lu \cite{AC:GroLu07}, which works for BBS ciphertexts. The proof size is linear in the number of ciphertexts, specifically $15n + 120$ group elements in type I groups. The security of their construction relies on two assumptions: the \emph{pairing product assumption} and the \emph{permutation pairing assumption}. The first assumption is a $\dist_{n,2}\mbox{-}\kermdh$ assumption, when $\matr{M}\gets\dist_{n,2}$ is of the form $\matr{M}^\top:=\pmatri{x_1,\ldots,x_n\\x_1^2,\ldots,x_n^2}$ for $x_i\gets\Z_q$, $i\in[n]$. The second assumption is proven generically secure by Groth and Lu, but it seems to be unrelated with any other assumption.

Using non-falsifiable assumptions (i.e. knowledge of exponent type of assumptions), Lipmaa and Zhang \cite{SCN:LipZha12} constructed a shuffle argument with communication $6n\sG+11\sH$, and recently Fauzi and Lipmaa  \cite{EPRINT:FauLip15} constructed a shuffle argument with communication $(5n+2)\sG+2n\sH$.

Rial, Kohlweiss, and Preneel constructed a range argument in $[0,2^n-1]$ with communication $\Theta(\frac{n}{\log n -\log\log n})$ and prove it secure under the $q$-HSDH assumption \cite{PAIRING:RiaKohPre09}. One might get rid of the $q$-HSDH assumption replacing the \emph{P-signature} with any \emph{structure preserving signature}, but, since the proof requires $\frac{n}{\log n-\log \log n}$ Groth-Sahai proofs of satisfiability of the signature's verification equation and the signature's size is at least 6 group elements \cite{EPRINT:JutRoy17}, the resulting protocol is far less efficient.
Using non-falsifiable assumptions, Chaabouni, Lipmaa, and Zhang constructed a range argument with constant communication~\cite{FC:ChaLipZha12}. 

A detailed comparison of our shuffle and range arguments with the most efficient constructions under falsifiable assumptions is depicted in Table \ref{table:eff}.


\begin{table}[h]
\begin{center}
\begin{minipage}{\textwidth}
\begin{center}
\begin{scriptsize}
\begin{tabular}{|l|ll|ll|}
\hline
                                                   & \multicolumn{2}{c|}{Shuffle Argument} & \multicolumn{2}{c|}{Range Argument} \\
                                                   & \cite{AC:GroLu07}          
%& \cite{EPRINT:FauLip15}
 & $\Pi_\mathsf{shuffle}$
                                                   & \cite{PAIRING:RiaKohPre09} & $\Pi_{\mathsf{range}\mbox{-}\mathsf{proof}}$ 
\\ \hline\hline
\rule{0pt}{2.5ex}CRS size                          & $2n + 8$                   
%& $8n + 17$
              & $(n^2+24n+36,23n+37)$                
                                                   & $\Theta(\frac{n}{\log n-\log\log n})$ & $(6n^2,6n^2)$ \\
% (6n^2+13n+n+\frac{n}{klogn}+2klogn,6n^2+13n+n+\frac{n}{klogn}+34)
\rule{0pt}{2.5ex}Proof size                        & $15n + 120$                
%& $(5n+2,2n)$
            & $(4n+17,14)$
                                                   & $\Theta(\frac{n}{\log n-\log\log n})$ & $(\frac{2n}{k\log n},10)$ \\
%$(\frac{2n}{k\log n}+2k\log n+11,10)$
\rule{0pt}{2.5ex}$\algP$'s comp.                   & $51n + 246$               
%& $22n + 11$
             & $11n+17$
                                                   & $\Theta(\frac{n}{\log n-\log\log n})$ & $2n$ \\
%2n+\frac{3n}{klogn}+3k\log n+2
\rule{0pt}{2.5ex}$\algV$'s comp.                   & $75n + 282$               
%& $18n + 6$
              & $13n+55$
                                                   & $\Theta(\frac{n}{\log n-\log\log n})$ & $\frac{4n}{k\log n}$ \\
%\frac{n}{k\log n}+6k\log n+62
\rule{0pt}{2.5ex}Assumption                        & PP                        
%& KE
                    & SXDH+SSDP
                                                   & $q$-HSDH                   & SXDH+SSDP \\\hline 
\end{tabular}
\end{scriptsize}
\end{center}
\caption{Comparison of our shuffle, $\Pi_\mathsf{shuffle}$, and range, $\Pi_{\mathsf{range}\mbox{-}\mathsf{proof}}$, arguments with the literature. To increase readability, for $\Pi_{\mathsf{range}\mbox{-}\mathsf{proof}}$ we include only the leading part of the sizes, that is, we write $f(n)$ and we mean $f(n)+o(f(n))$. Notation $(x,y)$ means $x$ elements of $\GG_1$ and $y$ elements of $\GG_2$. ``PP'' stands for the permutation pairing assumption.
%and ``KE'' for knowledge of exponent assumption ({\color{red} ver cual}).
The prover's computation is measured by the number of exponentiations (i.e. $z[x]_i$) and the verifier's computation is measured by the number of pairings.\label{table:eff}  } 
\end{minipage}
\vspace{-0.54cm}

\end{center}
\end{table}


