This thesis contributed many new and more efficient Non-Interactive Zero-Knowledge proofs. In general it showed that any set of quadratic equations of the type $b(b-1)=0$ and any set of linear equations with variables in $\Z_q$ have a proof whose size is linear in the number of variables and independent of the number of equations. In the case equations where variables are group elements, we showed that any set of linear equations have a proof whose size is linear in the number of variables and independent of the number of equations, with the drawback that the CRS is fixed to the specific set of equations. We then moved to the case of higher degree equations, which can be equivalently seen as Set-Membership proofs --a proof of membership in the set of roots of the equation in the case where the variables are $\Z_q$ elements-- and has the advantage that can also be applied to the case where the variables are group elements (since it is not clear how to define higher degree equations where variables are group elements). We showed that any set of Set-Membership proof have a proof whose size is linear in the size of the set and independent of the number of variables. When the set is a subset of $\Z_q$ or the CRS is fixed to a specific subset of $\GG_s$, we showed that the proof is logarithmic in the size of the set and independent of the number of variables.

In Chapter \ref{sec:agg-assym} we showed that any set of linear equations over $\Z_q$ and any set of linear equations over $\GG_s$ have a proof pf size linear in the number of variables and independent of the number of equations. The case of equations over $\GG_s$ has the additional restriction of the CRS being fixed to a specific set of equations and that the constants must be witness samplable. This results extends the result of Jutla and Roy \cite{C:JutRoy15} from one-sided to two-sided equations. We also showd that the same techniques allows to build QA-NIZK constant size proofs of membership in linear subspaces of $\GG_1\times\GG_s$, which extends the constant size QA-NIZK of membership in linear subspaces of $\GG_s$ from \cite{EC:LPJY14,C:JutRoy15,EC:KilWee15,EC:AbdBenPoi15}. Finally we showed how to extend the Linearly Homomorphic Structure-Preserving Signatures of Libert et al. \cite{EC:LPJY14} from the message space $\GG_s^n$ to $\GG_1^m\times\GG_2^n$.

In Chapter \ref{sec:bits} we constructed a constant-size QA-NIZK proof that a set of commitments open to a bit-string or equivalently, we showed that a set of equations of the type $b(b-1)=0$ have a proof of size linear in the number of variables and independent of the number of equations. In the first part of the chapter we devoted to case where the commitment is perfectly binding and then showed that it allowed more efficient ring signatures and threshold Groth-Sahai proofs. In the second part we devoted to case where the commitment is computationally binding and length-reducing. We introduced an extension of Pedersen commitments and showed that for this commitments we can build a QA-NIZK proof that there exists an opening which is a bit-strings.

In Chapter \ref{sec:shuf-rp} we constructed QA-NIZK proofs that many commitments open to elements in a set, where the proof size is linear in the size of the set and independent of the number of commitments. We the showed how to construct more efficient proofs of correctness of a shuffle and range proof under standard assumptions.

In Chapter \ref{sec:extras} we shoed how to construct a more efficient Ring Signature and also showed that the results from Chapter \ref{sec:shuf-rp} can be improved to logarithmic size proofs.

\section{Open Problems}
One of the main problems that remains open is to extend (or give an impossibility result) the efficiency improvements to any kind of equation, that is, construct proofs for a set of pairing product equations of size linear in the number of variables and independent of the number of equations. One indication that could be impossible is the fact that shrinking group to group commitments do not exists \cite{EC:AbeHarOhk12} and that in our results for quadratic equations over $\Z_q$ we used in a crucial way the Multi-Pedersen commitments (which are shrinking commitments). A related or more simple question is whether linear equations over $\GG_s$ allow more efficient proofs without computing an ad-hoc CRS. We note that this case suffer the same problem, since for example the values 
$$\pmatri{{[a_{1,1}]_1}\\\vdots\\{[a_{1,k}]_1}},\ldots,\pmatri{{[a_{1,1}]_1}\\\vdots\\{[a_{1,k}]_1}}$$
can be thought as commitment keys of a length reducing commitment scheme (much like Multi-Pedersen commitments). The fact that constants discrete logarithms of the equation appears multiplied by this commitment keys are allowing to compute a length reducing commitment of the constants of the equation.

After this thesis the big picture about the efficiency of Groth-Sahai proofs is: for equations where variables are $\Z_q$ we can compute very efficient proofs independent of the number of equations, however when variables are $\GG_s$ elements we are, in general, not able to do so. As we mentioned earlier, perhaps the main reason of this dichotomy is the lack of length reducing group to group homomorphic commitments. Abe showed that such commitment schemes do not exists if the key generation and commitment functions are \emph{algebraic}, that is, the group elements are only ``manipulated'' using the group operations and its actual representation is ignored.

A possible avenue to bypass this impossibility is to exploit the NP-completeness of quadratic equations over $\Z_q$ and translate any set of pairing product equations to a set of quadratic equations over $\Z_q$. Note that in this case we are taking in count the actual representation of group elements since, for example to translate an equation $e([x]_1,[y]_2)=[0]_T$, we are taking in count the actual circuit which computes the function $C_e:\bits^{\ell_1}\times \bits^{\ell_2} to\bits^{\ell_T}$, where $\ell_s$ is the size of the actual representation of an element of $\GG_s$, and expressing the circuit as a set of quadratic equations. This is the approach taken in Sect. \ref{sec:log-ring-signature} to show that at least, theoretically, there exists ring signatures of size $\Theta(\log n)$.

Gentry and Wichs showed general that any sublinear NIZK proof must rely on Non-Falsifiable assumptions if the language is ``hard'' (say an NP-hard language) \cite{STOC:GenWic}. This result seems to rule out any sublinear NIZK proof if the assumption is any ``DDH-like'' assumption. However, many cryptographically interesting languages are seem not to be not NP-hard. For example, the language of vectors in a subspace of $\GG_1$ seems not be NP-hard and also, we have seen that this languages allow constant size proofs. In fact, what we believe that is happening is that the language is in fact ``easy'' given the appropriate trapdoor (in this case a basis of the kernel of the generating matrix) while an NP-hard language should not have such a trapdoor.

In general, many languages the statement is the commitment to some value and we want to prove some assertion about the opening. Given a circuit $C$ and a commitment key $ck$, we define the following language
$$
\Lang_{ck,C}:= \{([\vecb{c}_1]_1,\ldots,[\vecb{c}_n]_1,a)\in\GG_1^2\times\bits^k:= \exists [x_i]_1\in\GG_1,w_i\in\Z_q \text{ s.t. } [\vecb{c}_i]_1=\GS.\Com_{ck}([x_i]_1;w_i)\text{ and }C([x]_1,a)=1\}.
$$
Many interesting problems can be instantiated in this way: in Ring Signatures $[x]_1$ is the verification key, $a$ the description of the ring, and $C$ outputs 1 iff $[x]_1$ is in the ring; b) in a shuffle $[\vecb{c}_1]_1,\ldots,[\vecb{c}_n]_1$, $n=2m$, are cyphertexts and $C([x]_1,\ldots,[x_n]_1)=1$ iff the first $m$ openings are a permutation of the las $m$ openings; c) in Groth-Sahai proofs the openings are the variables and the $C$ outputs 1 iff the variables satisfies the equations. In fact, the only interesting problem which is not of this type is a Range Proofs and is an interesting questions if there are other problems not of this type.

Using techniques from \cite{EC:GroOstSah06} one can construct NZIK proofs for $\Lang_{ck,C}$ of size $\Theta(|C|)=\Theta(n+|a|)$, since $|C|\leq n+|a|$, which boils down to compute linear Shuffles or Ring Signatures. We have seen on Sect. \ref{sec:log-ring-signature} that there exists ring signatures of size $\Theta(\log n)$, combining our results for aZKSMP and the same ``reduction to circuit'' approach. We beleive that this is an indication that more efficient proofs for languages $\Lang_{ck,C}$ exists.

Mostly all of our constructions derived from the proofs that many commitments open to a bit-string suffer from the same drawback: the CRS is quadratic in the size of the bit-string, which of course limits their applicability. In Sect. \ref{sec:bits-extensions} we showed that when the bit-string is of weight 1 the CRS can be made linear, however this is not always the case and the problem is still open.
%Bootle et al. \cite{EC:BCCGP16} construct a proof system which, although in a different setting, would help to shed lights on how to solve this problem. There, at some of the protocol, it is computed a set of elements which can be arranged matrix such that the $i,j$ th is the coefficient of the monomial $x^{i-j}$. Thereby, all elements at positions $i,j$ such that $i-j=k$ are coefficients of the same monomial $x^k$. Despite this very informal description, we can apply this idea as follows: compute commitments keys $[\vecb{g}_i],[\vecb{h}_j]_2$ in such a way that $\vecb{g}_i\vecb{h}_j=\vecb{g}_{k+i} \vecb{h+k}^\top$ .
 
