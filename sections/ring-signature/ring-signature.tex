Ring signatures, introduced by Rivest, Shamir and Tauman, \cite{AC:RivShaTau01}, allow the holder of the secret key $sk$ for a signature scheme to \emph{anonymously} sign a message on behalf of a \emph{ring} of users $P_1,\ldots,P_n$, where $vk_i$ is the verification key of party $P_i$, only if $sk$ is the corresponding secret key of $vk_i$ for some $i\in[n]$. We define $R:=\{vk_1,\ldots,vk_n\}$ the set of public keys associated to the ring $P_1,\ldots,P_n$.

The literature on ring signatures is vast, and, while there exist even constant size solutions \cite{EC:DKNS04}, most of them rely on the \emph{Random Oracle Model} (ROM). The ROM idealises the behavior of hash functions and it has been showed to be flawed, in the sense that there are protocols secure in the ROM but insecure using any real hash functions \cite{STOC:CanGolHal98}. Without random oracles all the constructions have signatures of size linear in the size of the ring, being the the sole exception the $\Theta(\sqrt{n})$ Ring Signature of Chandran et al.~\cite{ICALP:ChaGroSah07} (already discussed, and optimized, in Section~\ref{sec:bits-applications}). We remark that no asymptotic improvements to Chandran et al.'s construction have been made since their introduction (only improvements in the constants by R\`afols \cite{TCC:Rafols15} and the improvements from Section~\ref{sec:bits-applications}). Moreover, in Section \ref{sec:rs-flawed} we show that two schemes which claim to obtain constant and logarithmic signature size, respectively, are flawed.


In this section we present the first Ring Signature whose signature size is asymptotically smaller than Chandran et al.'s. Specifically, our Ring Signature is of size $\Theta(\sqrt[3]{n})$. Interestingly, the security of our construction relies on a security assumption -- the \emph{Permutation Pairing assumption} -- introduced by Groth and Lu \cite{AC:GroLu07} in an unrelated setting: proofs of Correctness of a Shuffle. While the assumption is ``non-standard'', in the sense that is not a ``DDH like'' assumption, it is a falsifiable assumption and it was proven to be generically hard by Groth and Lu. For simplicity, we work on symmetric groups ($\GG_1=\GG_2$), but our techniques can be easily extended to asymmetric groups if a natural translation to asymmetric groups of the underlying assumption is given.

