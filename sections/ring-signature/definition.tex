We follow Chandran et al.'s definitions \cite{ICALP:ChaGroSah07} described below, which extends the original definition of Bender et al. \cite{TCC:BenKatMor06} in order to include a CRS and perfect anonymity.

\begin{definition}[Ring Signature]
A ring signature scheme consists of a quadruple of
PPT algorithms $(\mathsf{CRSGen}, \KG, \mathsf{Sign}, \mathsf{Verify})$ that respectively, generate the common
reference string, generate keys for a user, sign a message, and verify the signature of a
message. More formally:
\begin{itemize}
\item $\mathsf{CRSGen}(gk)$, where $gk$ is the group key, outputs the common reference
string $\rho$.
\item $\KG(\rho)$ is run by the user. It outputs a public verification key $vk$ and a private
signing key $sk$.
\item $\mathsf{Sign}_{\rho,sk}(m, R)$ outputs a signature $\sigma$ on the message $m$ with respect to the ring
$R = \{vk_1,\ldots,vk_n\}$. We require that $(vk, sk)$ is a valid key-pair output by $\KG$
and that $vk \in R$.
\item $\mathsf{Verify}_{\rho,R}(m, \sigma)$ verifies a purported signature $\sigma$ on a message $m$ with respect to
the ring of public keys $R$.
\end{itemize}
The quadruple $(\mathsf{CRSGen}, \KG, \mathsf{Sign}, \mathsf{Verify})$ is a ring signature with perfect
anonymity if it has perfect correctness, computational unforgeability and perfect
anonymity as defined below.
\end{definition}

\begin{definition}[Perfect Correctness]
We require that a user can sign any message on behalf of a ring where she is a member. A ring signature $(\mathsf{CRSGen},\allowbreak \KG, \mathsf{Sign}, \mathsf{Verify})$
has perfect correctness if for all adversaries $\advA$ we have:
$$
\Pr\left[\begin{array}{l}
gk\gets\G(1^\lambda);\rho\gets\mathsf{CRSGen}(gk);(vk,sk)\gets\KG(\rho);\\
(m,R)\gets\advA(\rho,vk,sk);\sigma\gets\mathsf{Sign}_{\rho,sk}(m;R):\\
\mathsf{Verify}_{\rho,R}(m,\sigma)\text{ or }vk\notin R
\end{array}\right]=1
$$
\end{definition}

\begin{definition}[Computational Unforgeability]
A ring signature scheme $(\mathsf{CRSGen}, \KG, \mathsf{Sign}, \mathsf{Verify})$
is unforgeable if it is infeasible to forge a ring
signature on a message without controlling one of the members in the ring. Formally, it
is unforgeable when for any non-uniform polynomial
time adversaries $\advA$ we have that
$$
\Pr\left[\begin{array}{l}
gk\gets\G(1^\lambda);\rho\gets\mathsf{CRSGen}(gk);(m,R,\sigma)\gets\advA^{\mathsf{VKGen},\mathsf{Sign},\mathsf{Corrupt}}(\rho):\\
\mathsf{Verify}_{\rho,R}(m,\sigma)=1
\end{array}\right]
$$
is negligible in th security parameter, where

\begin{itemize}
\item $\mathsf{VKGen}$ on query number $i$ selects randomness $w_i$, computes $(vk_i,sk_i):= \KG(\rho; w_i)$
and returns $vk_i$.
\item $\mathsf{Sign}(i, m, R)$ returns $\sigma \gets \mathsf{Sign}_{\rho,sk_i}(m, R)$, provided $(vk_i, sk_i)$ has been generated
by $\mathsf{VKGen}$ and $vk_i\in R$.
\item $\mathsf{Corrupt}(i)$ returns $w_i$ (from which $sk_i$ can be computed) provided $(vk_i, sk_i)$ has
been generated by $\mathsf{VKGen}$.
\item $\advA$ outputs $(m, R, \sigma)$ such that $\mathsf{Sign}$ has not been queried with $(*, m, R)$ and $R$
only contains keys $vk_i$ generated by $\mathsf{VKGen}$ where $i$ has not been corrupted.
\end{itemize}
\end{definition}

\begin{definition}[Perfect Anonymity]
A ring signature scheme
$(\mathsf{CRSGen},\allowbreak \KG,\allowbreak \mathsf{Sign}, \mathsf{Verify})$ has perfect anonymity, if a signature on a message
$m$ under a ring $R$ and key $vk_{i_0}$
looks exactly the same as a signature on the
message $m$ under the ring $R$ and key $vk_{i_1}$, where $vk_{i_0},vk_{i_1}\in R$. This means that the signer's key is hidden
among all the honestly generated keys in the ring. Formally, we require that for any unbounded
adversary $\advA$:
\begin{align*}
&\Pr\left[\begin{array}{l}
gk\gets\G(1^\lambda);\rho\gets\mathsf{CRSGen}(gk);\\
(m,i_0,i_1,R)\gets\advA^{\KG(\rho)}(\rho);\sigma\gets\mathsf{Sign}_{\rho,sk_{i_0}}(m,R):\\
\advA(\sigma)=1
\end{array}\right]
=\\
&\Pr\left[\begin{array}{l}
gk\gets\G(1^\lambda);\rho\gets\mathsf{CRSGen}(gk);\\
(m,i_0,i_1,R)\gets\advA^{\KG(\rho)}(\rho);\sigma\gets\mathsf{Sign}_{\rho,sk_{i_1}}(m,R):\\
\advA(\sigma)=1
\end{array}\right]
\end{align*}
where $\advA$ chooses $i_0, i_1$ such that $(vk_{i_0}, sk_{i_0}),(vk_{i_1}, sk_{i_1})$ have been generated by the
oracle $\KG(\rho)$.
\end{definition}

