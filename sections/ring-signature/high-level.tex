Our scheme follows the ring signature of Chandran et al.~Given a \emph{Boneh-Boyen signature scheme} (Section \ref{sec:bbs}), where the secret/verification keys are of the form $(sk,[vk])$ and $sk=vk\in\Z_q$, and given a \emph{One-time signature scheme} (Section \ref{sec:ots}), the signature of the message $m$ for a ring $R=\{[vk_1],\ldots,[vk_n]\}$ is computed as follows:
\begin{itemize}
\item[a)] pick a one-time signature key $(sk_\mathsf{ot},vk_\mathsf{ot})$, sign $m$ with the one-time signature, and sign the one-time verification key with $sk$,
\item[b)] commit to the signature of the one-time verification key and to $[vk]$ and show that it is a valid signature key using GS proofs,
\item[c)] show that $[vk]\in R$.
\end{itemize}
The most costly part is c) and our contribution is a proof of size $\Theta(\sqrt[3]{n})$ of c).

Our construction is similar to the proof system for set membership with proof size $\Theta(\sqrt[3]{n})$ from Section~\ref{sec:bits-applications}. However, note that the proof system from Section~\ref{sec:bits-applications} does not suffice for constructing a ring signature because the CRS is fixed to a specific set and thus, the resulting ring signature will be fixed to a specific ring. 

In our scheme the secret/verification keys of party $P$ are $(sk,\vecb{vk})$, where $\vecb{vk}=([vk],[\vecb{a}],\vecb{a}[vk])$, $(sk,[vk])$ are secret/verification keys of the Boneh-Boyen signature scheme, and $\vecb{a}\in\Z_q^2$ is chosen independently for each key from some distribution $\mathcal{Q}$ to be specified later. Suppose that $\vecb{vk}$ is the $\alpha$ th element in the ring $R=\{\vecb{vk}_{(1,1,1)},\ldots,\vecb{vk}_{(m,m,m)}\}$, where $\alpha=(i_\alpha,j_\alpha,k_\alpha):=(i_\alpha-1)m^2+(j_\alpha-1)m+k_\alpha$ for $i_\alpha,j_\alpha,k_\alpha\in[m],m:=\sqrt[3]{n}$. The prover commits to $[\vecb{x}]=[\vecb{s}_\mu]$ and $[\vecb{y}]=[\vecb{s}'_{\mu'}]$, for $\mu=\mu'=(j_\alpha-1)m+k_\alpha$, and shows, using the set-membership proof from Section \ref{sec:bits-applications} (which is the proof from Chandran et al.~adapted to work with vectors), that $[\vecb{x}]\in S$ and that $[\vecb{y}]\in S'$, 
where
\begin{align*}
&S:=\{[\vecb{s}_1],\ldots,[\vecb{s}_{n^{2/3}}]\}:=\left\{\sum_{i\in[m]}[\vecb{a}_{(i,1,1)}],\ldots,\sum_{i\in[m]}[\vecb{a}_{(i,m,m)}]\right\}\text{ and }\\
&S':=\{[\vecb{s}'_1],\ldots,[\vecb{s}'_{n^{2/3}}]\}:=\left\{\sum_{i\in[m]}\vecb{a}_{(i,1,1)}[vk_{(i,1,1)}],\ldots,\sum_{i\in[m]}\vecb{a}_{(i,m,m)}[vk_{(i,m,m)}]\right\}.
\end{align*}

The prover also needs to assure that $\mu=\mu'$, which can be done reutilizing the commitment to $\mu$ (in fact to its $m$-ary representation ) used in the proof that $[\vecb{x}]\in S$ in the proof that $[\vecb{y}]\in S'$. Since both sets are of size $n^{2/3}$, the two set membership proofs are of size $\Theta(\sqrt[3]{n})$.
 
Now that the prover has commited to elements $[\vecb{x}]=\sum_{i\in[m]}[\vecb{a}_{(i,j_\alpha,k_\alpha)}]$ and $[\vecb{y}]=\sum_{i\in[m]}\vecb{a}_{(i,j_\alpha,k_\alpha)}[vk_{(i,j_\alpha,k_\alpha)}]$, it additionally commits to $[\kappa_1]:=[vk_{(1,j_\alpha,k_\alpha)}],\allowbreak\ldots,\allowbreak[\kappa_m]:=[vk_{(m,j_\alpha,k_\alpha)}]$ and $[\vecb{z}_1]:=[\vecb{a}_{(1,j_\alpha,k_\alpha)}],\ldots,[\vecb{z}_m]:=[\vecb{a}_{(m,j_\alpha,k_\alpha)}]$. The prover now gives a proof that
\begin{equation}
\sum_{i\in[m]}[\vecb{z}_i][\kappa_i]=[\vecb{y}][1]. \label{eq:verif1}
\end{equation}

Provided that $\vecb{z}_1,\ldots,\vecb{z}_m$ is a permutation of $\vecb{a}_{(1,j_\alpha,k_\alpha)},\ldots,\vecb{a}_{(1,j_\alpha,k_\alpha)}$,
we can show that if $[\kappa_1],\ldots,[\kappa_m]$ is not a permutation of $[vk_{(1,j_\alpha,k_\alpha)}],\allowbreak\ldots,\allowbreak[vk_{(m,j_\alpha,k_\alpha)}]$, then we can extract an element from the kernel of the matrix $([\vecb{a}_{1,j_\alpha,k_\alpha}]\cdots\allowbreak [\vecb{a}_{m,j_\alpha,k_\alpha}])$. Thereby, provided that the corresponding kernel assumption is hard, the prover can simply select the $i_\alpha$ th element from $[\kappa_1],\ldots,\allowbreak [\kappa_m]$ which is guaranteed to be an element from the ring.

Therefore, it is only left the to show that $\vecb{z}_1,\ldots,\vecb{z}_m$ is a permutation of $\vecb{a}_{(1,j_\alpha,k_\alpha)},\ldots,\vecb{a}_{(1,j_\alpha,k_\alpha)}$. To do so we will use the following assumption introduced by Groth and Lu \cite{AC:GroLu07}.
\begin{definition}[Permutation Pairing Assumption]
Let $\mathcal{Q}_{m}=\underbrace{\mathcal{Q}||\ldots||\mathcal{Q}}_{m\text{ times}}$, where concatenation of matrix distributions is defined in the natural way and 
$$\mathcal{Q}: \vecb{a}=\pmatri{x\\x^2},\quad x\gets\Z_q.$$
We say that the $m$-Permutation Pairing Assumption holds relative to $\G_s$ if for any adversary $\advA$
$$
\Pr\left[
\begin{array}{l}
gk\gets\G_s(1^k);\matr{A}\gets\mathcal{Q}_{m};[\matr{Z}]\gets\advA(gk,[\matr{A}]):\\
\mathrm{(i)} \sum_{i\in[m]}[\vecb{z}_i]=\sum_{i\in[m]}[\vecb{a}_i], \mathrm{(ii)}\ \forall i\in[m]\ [z_{2,i}][1]=[z_{1,i}][z_{1,i}],\\
\text{ and }\matr{Z}\text{ is not a permutation of the columns of }\matr{A}
\end{array}
\right],
$$
where $[\matr{Z}]=[(\vecb{z}_1,\ldots,\vecb{z}_m)],[\vecb{A}]=[(\vecb{a}_1,\ldots,\vecb{a}_m)]\in\GG^{2\times m}$,
is negligible in $k$.
\end{definition}

If the prover additionally proves that equations (i) and (ii) are satisfied for $\matr{A}:=(\vecb{a}_{(1,j_\alpha,k_\alpha)},\ldots,\vecb{a}_{(m,j_\alpha,k_\alpha)})$, which can be done with $\Theta(m)$ group elements using Groth-Sahai proofs, the assumption is guaranteeing that the columns of $\matr{Z}$ are a permutation of the columns of $\matr{A}$, for some permutation $\pi\in S_m$. Therefore, equation (\ref{eq:verif1}) implies that
\begin{align*}
\sum_{i\in[m]}[\vecb{z}_i][\kappa_i]&=\sum_{i\in[m]}[\vecb{a}_{(\pi(i),j_\alpha,k_\alpha)}][\kappa_i]=\sum_{i\in[m]}[\vecb{a}_{(i,j_\alpha,k_\alpha)}][\kappa_{\pi^{-1}(i)}]\\
&=\sum_{i\in[m]}[\vecb{a}_{(i,j_\alpha,k_\alpha)}][vk_{(i,j_\alpha,k_\alpha)}].
\end{align*}
Then $\kappa_1,\ldots,\kappa_m$ is a permutation of $\vecb{a}_{(1,j_\alpha,k_\alpha)},\ldots,\vecb{a}_{(m,j_\alpha,k_\alpha)}$ (the same defined by $\vecb{z}_1,\ldots,\vecb{z}_m$), unless $(\kappa_{\pi^{-1}(1)}-{vk_{(1,j_\alpha,k_\alpha)}),\ldots,\kappa_{\pi^{-1}(m)}-vk_{(m,j_\alpha,k_\alpha)})})^\top$ is in the kernel of $\matr{A}$. Groth and Lu showed the hardness of finding an element from $\ker(\matr{A})$, when $\matr{A}$ is sampled from $\mathcal{Q}_m$, in the generic group model. They called this assumption the \emph{Simultaneous Pairing Assumption} and it corresponds to the $\mathcal{Q}_m^\top\mbox{-}\kermdh$ assumption.


\paragraph{Remark.}
A natural question is if this technique can be applied once again. That is, to compute a $\Theta(\sqrt[4]{n})$  proof, compute commitments to an element from $S=\{\sum_{i\in[m]}\vecb{a}_{(i,1,1,1)}[vk_{(i,1,1,1)}],\ldots,\sum_{i\in[m]}\vecb{a}_{(i,m,m,m)}[vk_{(i,m,m,m)}]\}$ and $S'=\allowbreak\{\sum_{i\in[m]}[\vecb{a}_{(i,1,1,1)}],\ldots,\sum_{i\in[m]}[\vecb{a}_{(i,m,m,m)}]\}$, and then prove that they belong to the respective sets with the proof of size $\Theta(\sqrt[3]{n})$. Since $|S|=|S'|=n^{3/4}$, proof will be of size $\Theta(\sqrt[3]{n^{3/4}})=\Theta(\sqrt[4]{n})$. However, this is not possible since the $\Theta(\sqrt[3]{n})$ proof is not a set membership proof for arbitrary sets, but only for sets where each element is of the form $([vk],\vecb{a}[vk],[\vecb{a}])$.

