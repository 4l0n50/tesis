\label{def:splh}
Linearly-homomorphic structure preserving signatures  \cite{C:AFGHO10,PKC:BFKW09} enable to sign group elements 
in $G$, where $G$ is a group and to publicly derive signatures of new elements which are a linear combination of other signed messages. We take the definition from \cite{C:LPJY13}, except that we do not identify the elements of $G$ with vectors in $\Gr^n$, for some group $\Gr$. The reason is that $G$ might be some space of the form $\Gr^m \times \Hr^n$. 

\begin{definition}[SPLHS scheme]
A linearly homomorphic structure-preserving signature scheme over the group $G$  consists
of a tuple of efficient algorithms $\SPLHSinst$$=$$(\SG,$ $\SN,$ $\SD,$ $\SV)$ for which the message space
is $\mathcal{M} := G$, with the following specifications.

\begin{description}

\item[$\SG(gk, n):$] is a randomized algorithm that takes as input a group key $gk$ and an integer $n$ and  outputs a key pair $(\pk, \sk)$. The public key $pk$ specifies a $\Z_q$ vector space $G$ of dimension $n$. 

\item[$\SN(\sk,\vecb{m})$:] is a possibly probabilistic algorithm that takes as input a private key $\sk$
 and $\vecb{m}\in G$. It outputs a signature $\grkb{\sigma}\in G$.

\item[$\SD(\pk,\{\omega_i,\grkb{\sigma}_i,\vecb{m}_i \}_{i\in[\ell]})$:] is a possibly probabilistic signature derivation algorithm. It
takes as input a public key $\pk$ as well as $\ell$ pairs $(\omega_i,\grkb{\sigma}_i)$, each of which
consists of a weight $\omega_i\in\Z_q$ and a signature $\grkb{\sigma_i}\in G$. The output is a signature
$\grkb{\sigma}\in G$ on the vector $\vecb{m} = \sum_{i\in[\ell]} \omega_i\vecb{m}_i $.

\item[$\SV(\pk, \vecb{m},  \grkb{\sigma})$:] is a deterministic algorithm that takes in a public key $\pk$,
a signature $\grkb{\sigma}$, and a vector $\vecb{m}$. It outputs 1 if $\grkb{\sigma}$ is deemed valid and 0 otherwise.
\end{description}
\end{definition}

Correctness is expressed by imposing that, for all security parameters $\lambda\in\N$, all integers $n\in\poly(\lambda)$
and all pairs $(\pk,\sk) \gets \SG(gk, n)$, the following holds:
\begin{enumerate}
\item For all $\vecb{m} \in G$, if $\grkb{\sigma} = \SN(\sk, \vecb{m})$, then we have $\SV(\pk, \vecb{m}, \grkb{\sigma}) = 1$.
\item For any $\ell > 0$ and any set of triples $\{(\omega_i, \grkb{\sigma}_i, \vecb{m}_i)\}_{i\in[\ell]}$,
if $\SV(\pk, \vecb{m}_i, \grkb{\sigma}_i) = 1$ for each $i \in [\ell]$, then
$\SV(
    \pk,
    \sum_{i\in[\ell]} \omega_i\vecb{m}_i,
    \SD(
        \pk,
        \{(\omega_i, \grkb{\sigma}_i)\}
        )
    )=1
$
\end{enumerate}

\label{splhs-unforgeability}

In order to get a uniform definition for different types of forgery, we will say that a pair
$(\vecb{m}^*,\grkb{\sigma}^*)$ is a forgery if $P(\vecb{m}^*, Q)=1$, where
$P$ is a predicate on $(\vecb{m}^*, Q)$ and $Q$ is the set of reveal queries
made by the adversary. We stress that the predicate $P$ is not always efficiently computable. For instance, for the scheme of Libert \textit{et al.} (\cite{C:LPJY13}), this predicate is $1$ iff $\vecb{m}^*$ is in the linear span of 
previous queries, and this is, in general, hard to decide in the group $G$ (although it might be easy for some set $Q$).  

\begin{definition} A SPLHS scheme $\SPLHSinst = (\SG, \SN, \SV, \SD)$ is secure against Type P adversaries if no PPT
adversary has non-negligible advantage in the game below:
\begin{enumerate}
\item The adversary $\advA$ chooses an integer $n \in \N$ and sends it to the challenger who runs
$\SG(gk, n)$
and obtains $(\pk,\sk)$ before sending $\pk$ to $\advA$.
\item On polynomially-many occasions, $\advA$ can interleave the following kinds of queries.
\begin{description}

\item[Signing queries:] $\advA$ chooses a vector $\vecb{m}\in G^n$.
The challenger picks a handle
$h$ and computes $\grkb{\sigma}\gets\SN(\sk, \vecb{m})$. It stores $(h,\vecb{m}, \grkb{\sigma})$
in a table $T$ and returns $h$.

\item[Derivation queries:] $\advA$ chooses a vector of handles $\vecb{h} = (h_1, \ldots , h_\ell)$ and a set of coefficients
$\{\omega_i\}_{i\in[\ell]}$. The challenger retrieves the tuples $\{(h_i,\vecb{m}_i, \grkb{\sigma}_i)\}_{i\in[\ell]}$
from $T$ and returns $\perp$ if one
of these does not exist. Otherwise, it computes
$\vecb{m} = \sum_{i\in[\ell]} \omega_i\vecb{m}_i$ and runs
$\grkb{\sigma}\gets\SD(\pk,\{(\omega_i,\grkb{\sigma}_i)\}_{i\in[\ell]})$.
It also chooses a handle $h$, stores $(h,\vecb{m}, \grkb{\sigma})$ in $T$ and returns $h$
to $\advA$.

\item[Reveal queries:] $\advA$ chooses a handle $h$. If no tuple of the form $(h,\vecb{m},\grkb{\sigma})$
exists in $T$, the challenger returns $\perp$. Otherwise, it returns $\grkb{\sigma}$
to $\advA$ and adds $(\vecb{m}, \grkb{\sigma})$ to the set $Q$.
\end{description}

\item $\advA$ outputs a signature $\grkb{\sigma}^*$ and a vector $\vecb{m}^*$.
The adversary $\advA$ wins if $P(\vecb{m}^*,Q)=1$.
\end{enumerate}
$\advA$’s advantage is its probability of success taken over all coin tosses.
\end{definition}

Libert \textit{et al.} also used a set $\mathcal{T}$ of tags in order to add up many instances of their signature scheme
in only one. For simplicity, we omit this parameter.


\subsection{One-Time LHSPS Signatures in Different Groups} \label{sec:newhomtwogroups}

The one-time linearly homomorphic signature of Libert, Peters and Yung \cite{EC:LPJY14}  implies a QA-NIZK Argument for linear spaces. Similarly, our constructions of QA-NIZK proofs for membership in concatenated subspace and for sum in subspace (in the case where the space is not from a witness samplable distribution) is implied by a one-time structure preserving signature scheme with different security properties. 

In particular, for subspace concatenation, ``one-time" means that the adversary is unable to sign vectors which are not in the span of previously signed vectors, namely,
the adversary cannot output a signature for a pair
$([\vecb{x}]_1^*,[\vecb{y}]_2^*) \in \Gr^{m} \times \Hr^{n}$ if $((\vecb{x}^*)^{\top}||(\vecb{y}^*)^{\top})$ is
linearly independent from the 
vectors $(\vecb{x}_i^{\top}||\vecb{y}_i^{\top})$, $i \in [q_s]$, (the concatenation of two vectors), where $([\vecb{x}_i]_1,[\vecb{y}_i]_2)$ are the signing queries of the adversary. The discussion for the scheme which results from our 
Sum-in-Subspace QA-NIZK proof, results in a different notion of ``one-time" --- this is captured in the security definition by a different predicate $P$ ---, see discussion below. 

In either case, the size of the resulting signatures is $(k+1)\s$ under the $\skermdh$ Assumption, but if security against random message attacks is sufficient (meaning that the signatures in the set $Q$ which are seen by the adversary are sampled uniformly at random), the signature size can be reduced to $k\s$ (essentially, in this case one can sample $\matrA$ from $\overline{\dist}_k$). This is inspired by the one-time constructions of structure-preserving signatures of \cite{C:KilPanWee15} secure against random message attacks. 
 We omit any further discussion of this case, as it is a straightforward generalization of our QA-NIZK proofs in the witness samplable setting using the ideas of \cite{C:KilPanWee15}. 


% For subspace concatenation, ``one-time" means that the adversary is unable to sign vectors which are not in the span of previously signed vectors. For the one-time signature scheme derived from the sum in subspace proof, ``one-time" refers to the fact that the adversary cannot obtain signatures for vectors which are  half of the components in $\Gr$ and other half in $\Hr$ and such that the sum of the vecto


Our construction is based on the $\skermdh$ Assumption introduced in Section~\ref{sec:mddh}. Following the syntactic definition of Section~\ref{sec:mddh},
our scheme assumes $\Group = \Gr^m\times\Hr^n$ and the length of the messages is $n+m$.
 
\begin{itemize}
\item $\SG(gk,m,n)$:
Choose $\matr{A} \leftarrow \dist_{k}$,
$\matr{\Lambda},\matr{\Xi} \leftarrow \Z_q^{(k+1) \times m}$, $\vecb{A}_{\Lambda}:=\matr{\Lambda}^\top \matr{A}$, 
$\vecb{A}_{\Xi}:=\matr{\Xi}^\top \matr{A}$
The secret key is $\mathsf{sk}=(\matr{\Lambda},\matr{\Xi})$, while the public key is defined to be
$$\mathsf{pk}=([\vecb{A}]_1, [\vecb{A}_{\Xi}]_1,[\vecb{A}]_2, [\vecb{A}_{\Lambda}]_2) \in  \Gr^{(k+1) \times k} \times \Gr^{m \times k} \times \Hr^{(k+1) \times k} \times \Hr^{m \times k}.$$
\item $\SN(\mathsf{sk},([\vecb{x}]_1,[\vecb{y}]_2))$: To sign a vector $([\vecb{x}]_1,[\vecb{y}]_2) \in \Gr^{m} \times \Hr^{m}$, pick 
$\vecb{z} \leftarrow \Z_q^{(k+1)}$ and output the pair 
$([\grkb{\rho}]_1, [\grkb{\sigma}]_2) \in \Gr^{(k+1)} \times \Hr^{(k+1)}$, defined as:
$$[\grkb{\rho}]_1:=   \matr{\Lambda}[\vecb{x}]_1 + [\vecb{z}]_1,
                            \qquad \qquad 
[\grkb{\sigma}]_2 := \matr{\Xi}[\vecb{y}]_2 - [\vecb{z}]_2.$$   
 \item  $\SD(\mathsf{pk},\{(\omega_i,[\grkb{\rho}_i]_1, [\grkb{\sigma}_i]_2)\}_{i=1}^{\ell})$: given the public key $pk$, and $\ell$ tuples 
 $(\omega_i,[\grkb{\rho}_i]_1,$ $[\grkb{\sigma}_i]_2)$, output the pair 
 $(\sum_{i=1}^{\ell} \omega_i [\grkb{\rho}_i]_1, \sum_{i=1}^{\ell} \omega_i  [\grkb{\sigma}_i]_2) \in \Gr^{(k+1)} \times \Hr^{(k+1)}$. 
 \item $\SV(\mathsf{pk},([\vecb{x}]_1,[\vecb{y}]_2), ([\grkb{\rho}]_1, [\grkb{\sigma}]_2))$  is a deterministic algorithm, that takes as input a public key $\mathsf{pk}$,  a signature $([\grkb{\rho}]_1, [\grkb{\sigma}]_2)$ and returns $1$ if and only if 
$([\grkb{\rho}]_1, [\grkb{\sigma}]_2)$ satisfies
$$ [\grkb{\rho}^{\top}]_1 [\vecb{A}]_2+ [\grkb{\sigma}^{\top}]_2 [\vecb{A}]_1= [\vecb{x}^{\top}]_1 [\vecb{A}_{\Lambda}]_2+[\vecb{y}^{\top}]_2[\vecb{A}_{\Xi}]_1. $$
\end{itemize}

\paragraph{Correctness.} If a signature is correctly generated then 
$$[\grkb{\rho}^{\top}]_1 [\matr{A}]_2-[\vecb{x}^{\top}]_1 [\matr{A}_{\Lambda}]_2= [\vecb{z}^{\top}]_1 [\matr{A}]_2 \qquad \qquad  [\grkb{\sigma}^{\top}]_2[\matr{A}]_1 -[\vecb{y}^{\top}]_2  [\matr{A}_{\Xi}]_1= -[\vecb{z}^{\top}]_2 [\matr{A}]_1.$$
Therefore the verification algorithm outputs $1$ on a correctly generated signature. The proof of correctness  of the signature derivation algorithm follows a similar argument.  

Let $Q=\{ ([\vecb{x}_i]_1,[\vecb{y}_i]_2) \}_{i \in [q_s]}$ be some set of elements of $\Gr^{m} \times \Hr^n$. We define the predicate $P$ as $P(([\vecb{x}]_1,[\vecb{y}]_2),Q)=1$ iff $(\vecb{x}^{\top}||\vecb{y}^{\top}) \in \Z_q^{2m}$ is not in the space spanned by $\{(\vecb{x}_i^{\top}||\vecb{y}_i^{\top}) :i \in [q_s]\}$. 

\begin{theorem} The signature scheme is Type P unforgeable if the $\skermdh$ Assumption holds in $\Gr,\Hr$.
\end{theorem}
%\textcolor{red}{seguramente sacar si ya es evidente por lo otro}
The argument is almost identical to  \cite{C:LPJY13}.
\begin{proof} We show how to construct an algorithm $\advB$ which takes as input an instance $([\matr{A}]_1,[\matr{A}]_2)$ of the $\skermdh$ Assumption and outputs a pair of vectors $([\vecb{r}]_1,[\vecb{s}]_2) \in \Gr^{3} \times \Hr^{3}$, $\vecb{r} \neq \vecb{s}$, such that 
$[\vecb{r}^{\top}]_1 [\matr{A}]_2=[\vecb{s}^{\top}]_2 [\matr{A}]_1$ given oracle access to a forger $\Forger$ against the signature scheme
(see Section~\ref{splhs-unforgeability}). 

Algorithm $\advB$ starts by honestly running the key generation algorithm 
using a randomly chosen $\mathsf{sk}=(\matr{\Lambda},\matr{\Xi})$. Any signature query of $\Forger$ on a vector
$([\vecb{x}]_1,[\vecb{y}]_2)$ is honestly answered by $\advB$, by running the signing algorithm.
The game ends with $\Forger$ outputting a vector $([\vecb{x}]_1^*,[\vecb{y}]_2^*)$ 
with a valid signature $([\grkb{\rho}^{*}]_1, [\grkb{\sigma}^{*}]_2)$. At this point, $\advB$ computes its own signature $([\grkb{\rho}^{\dagger}]_1, [\grkb{\sigma}^{\dagger}]_2)$ using the secret key $\mathsf{sk}:=(\matr{\Lambda},\matr{\Xi})$. The adversary $\advB$ will output as a response to the $\skermdh$ challenge the pair $([\grkb{\rho}^{*}]_1-[\grkb{\rho}^{\dagger}]_1,[ \grkb{\sigma}^{\dagger}]_2-[\grkb{\sigma}^{*}]_2)$.

We now see that, with overwhelming probability, this is a valid answer to the $\skermdh$ challenge.
Indeed, since both signatures satisfy the verification equation, we can subtract the verification equation of each pair, obtaining:
\begin{equation*}
([\grkb{\rho}^{*}]_1-[\grkb{\rho}^{\dagger}]_1)^{\top} [\matr{A}]_2=  ([\grkb{\sigma}^{\dagger}]_2-[\grkb{\sigma}^{*}]_2)^{\top} [\matr{A}]_1
\end{equation*}
Therefore, all we need to argue is that $\grkb{\rho}^{*}-\grkb{\rho}^{\dagger} \neq \grkb{\sigma}^{\dagger}-\grkb{\sigma}^{*}$
with overwhelming probability. This is equivalent to show that  the probability that $\grkb{\rho}^{*}+\grkb{\sigma}^{*} = 
\grkb{\rho}^{\dagger}+ \grkb{\sigma}^{\dagger}$ is negligible.  
The key point of the argument is that 
 \begin{equation*}
 \grkb{\rho}^{\dagger}+ \grkb{\sigma}^{\dagger}=\matr{\Lambda}\vecb{x}^*+\matr{\Xi}\vecb{y}^*= 
  \begin{pmatrix}  
 {\matr{\Lambda}} & {\matr{\Xi}}
 \end{pmatrix}
 \begin{pmatrix}\vecb{x}^*\\ \vecb{y}^* \end{pmatrix}
  \end{equation*}
is information theoretically hidden to $\Forger$. 

The rest of the argument is identical to  \cite{C:LPJY13}. The argument goes as follows: since, by assumption, $\begin{pmatrix}\vecb{x}^*\\ \vecb{y}^* \end{pmatrix}$ is independent of all  previous queries, then  there is some information about $\begin{pmatrix}  
 {\matr{\Lambda}} & {\matr{\Xi}}
 \end{pmatrix}$ 
which is information theoretically hidden. Thus,  $\grkb{\rho}^{\dagger}+ \grkb{\sigma}^{\dagger}$ is information theoretically hidden and from the adversary's point of view it is equally likely that it has any out of $q$ potential values.  
\end{proof}

\paragraph{Signing the Sum of Two Linear Spaces.} \label{sec:newhom} When $m=n$, we can adapt the previous construction to a different forgery condition namely, we 
can prove security against a different type of adversary. Namely, in  \cite{C:LPJY13} the scheme is secure against an adversary whose goal is to output a forgery for a message which is linearly independent from all of its signing queries. In our case, we require that the adversary cannot output a signature for a pair
$([\vecb{x}]_1^*,[\vecb{y}]_2^*) \in \Gr^{m} \times \Hr^{m}$ if $\vecb{x}^*+\vecb{y}^*$ is
linearly independent from the 
vectors $\vecb{x}_i+\vecb{y}_i$, $i \in [q_s]$, where $([\vecb{x}_i]_1,[\vecb{y}_i]_2)$ are the signing queries of the adversary. 

Our construction is like the previous one taking $\matr{\Xi}=\matr{\Lambda}$. Indeed, in this case the adversary only learns $\matr{\Lambda} \vecb{x}^*+ \matr{\Xi} \vecb{y}^*=\matr{\Lambda} (\vecb{x}^*+\vecb{y}^*)$, and identically the same argument follows. 
