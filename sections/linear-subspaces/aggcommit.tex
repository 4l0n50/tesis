Given $\sps$ or $\spsws$, it is direct to construct constant-size NIZK Arguments of membership in:
$$\mathcal{L}_{\mathsf{com},[\matr{U}]_1,[\matr{V}]_2,\nu}:=\left\{([\vecb{c}]_1,[\vecb{d}]_2) \in \GG_1^{m} \times \GG_2^{n}:
\begin{array}{l}
    \exists (\vecb{w},\vecb{r},\vecb{s}) \text{ s.t. }\\
    {[\vecb{c}]_1= [\matr{U}]_1 \begin{pmatrix} \vecb{w}\\ \vecb{r} \end{pmatrix} \text{ and }}\\
    {[\vecb{d}]_2=[\matr{V}]_2 \begin{pmatrix} \vecb{w} \\ \vecb{s} \end{pmatrix}}
\end{array}
\right\},$$
where $[\matr{U}]_1\in \GG_1^{m \times \tilde{m}}$, 
$[\matr{V}]_2 \in \GG_2^{n \times \tilde{n}}$
and $\vecb{w} \in \Z_q^{\nu}$. The witness is 
 $(\vecb{w},\vecb{r},\vecb{s}) \in \Z_q^{\nu} \times \Z_q^{\tilde{m}-\nu} \times \Z_q^{\tilde{n}-\nu}$. This language is interesting because it can express the fact that 
$([\vecb{c}]_1,[\vecb{d}]_2)$ are commitments to the same vector 
$\vecb{w} \in \Z_q^{\nu}$ in different groups.
 
The construction is an immediate consequence of the observation 
that $\mathcal{L}_{\mathsf{com},[\matr{U}]_1,[\matr{V}]_2,\nu}$  can be rewritten as some concatenation language $\mathcal{L}_{[\matr{M}]_1,[\matr{N}]_2}$.
Denote by $\matr{U}_1$ the first $\nu$ columns of $\matr{U}$  and $\matr{U}_2$ the remaining ones, and $\matr{V}_1$ the first $\nu$ columns of $\matr{V}$ and $\matr{V}_2$ the remaining ones. If we define: 
\begin{equation*}
 \matr{M} := (\matr{U}_1||\matr{U}_2||\matr{0}_{m\times (\tilde{n}-\nu)}) \qquad
\matr{N} := (\matr{V}_1||\matr{0}_{n\times (\tilde{m}- \nu)}||\matr{V}_2).
\end{equation*}
then it is immediate to verify that $\mathcal{L}_{\mathsf{com},[\matr{U}]_1,[\matr{V}]_2,\nu}=\mathcal{L}_{[\matr{M}]_1,[\matr{N}]_2}$.

In the rest of the paper, we denote as $\spswscomm$ the proof 
system for $\mathcal{L}_{\mathsf{com},\hmatr{U},\cmatr{V},\nu}$ which corresponds to $\spsws$ for $\mathcal{L}_{[\matr{M}]_1,[\matr{N}]_2}$, where $[\matr{M}]_1,[\matr{N}]_2$ are the matrices defined above. Note that for commitment schemes we can generally assume $[\matr{U}]_1,[\matr{V}]_2$ to be drawn from some witness samplable distribution. Therefore, it follows from Theorem~\ref{theo:membtwogroups2} that $\spswscomm$ satisfies the notion of strong soundness.  
