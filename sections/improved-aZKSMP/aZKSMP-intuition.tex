For simplicity, we will restrict to the case \(S=\{s_1,\ldots,s_\setsize \}\subset\Z_q\) without aggregation, that is, there is a single commitment \([\vecb{c}]_1=\GS.\Com_{ck_\GS}(x;r)\) and we want to show that \(x=s_\alpha\), for some \(\alpha\in[\setsize ]\). In Section \ref{sec:log-set-memb-Z} we will show how to aggregate many proofs.

The (non-aggregated) proof from Section~\ref{sec:aZKSMP} essentially codifies the position \(\alpha\) as a weight 1 binary vector \(\vecb{b}\) of size \(\setsize \) such that \(x=\sum_{i\in[\setsize ]}b_is_i\) and
\(b_i=1\) if \(i=\alpha\) and \(0\) if not.\footnote{The case \(S\subset\Z_q\) is not really discussed in Section~\ref{sec:aZKSMP}, but it is straightforward that the same techniques from the case \(S\subset\GG_s\) apply.} A step further in efficiency was given by Chandran et al.~\cite{ICALP:ChaGroSah07} (already discussed in Section~\ref{sec:bits-applications}). There the position \(\alpha\) is codified as two weight 1 binary vectors \(\vecb{b}\) and \(\vecb{b}'\) of size \(\sqrt{\setsize }\) such that
\[
\begin{pmatrix}
x_1\\\vdots\\x_{\sqrt{\setsize }}
\end{pmatrix}
=\sum_{i=1}^{\sqrt{\setsize }}b_i
\begin{pmatrix}
s_{(i-1)\sqrt{\setsize }+1}\\\vdots\\s_{(i-1)\sqrt{\setsize }+\sqrt{\setsize }}
\end{pmatrix},
\text{ } x=\sum_{i=1}^{\sqrt{\setsize }}b'_ix_i,\]
and \(b_i=1\) iff \(i=i_\alpha\) and \(b'_j=1\) iff \(j=j_\alpha\), where \(\alpha=(i_\alpha-1)\sqrt{\setsize }+j_\alpha\). Since \(\sqrt{\setsize }\) new variables are added (variables \(x_1,\ldots,x_{\sqrt{\setsize }}\)), the proof must contain \(\sqrt{\setsize }\) new commitments to these variables. However, this does not not affect the asymptotic size of the proof, which is \(\Theta(\sqrt{\setsize })\) anyway.


Let $m:=\log \setsize$.\footnote{ W.l.o.g.~we assume that \(\log \setsize \in\mathbb{N}\), because we can always prove membership in the (multi-)set \(S'=S\biguplus_{i=1}^{2^{\lceil \log \setsize  \rceil}-\setsize}\{s_{\setsize} \}\) and it holds that $|S'|=2^{\lceil \log t \rceil}$ and that $x\in S \Longleftrightarrow x\in S'$.} The natural next step is to codify \(\alpha\) as \(m\) weight 1 binary vectors of size 2 (note that a weight 1 binary vector of size 2 can be always written as \((1-b,b)\), \(b\in\bits\) ) such that
\begin{align}
&\begin{pmatrix}
x_{\ell,1}\\\vdots\\x_{\ell,{2^{\ell-1}}}
\end{pmatrix}
=
(1-b_\ell)
\begin{pmatrix}
x_{\ell+1,1}\\\vdots\\x_{\ell+1,{2^{\ell-1}}}
\end{pmatrix}
+
b_\ell
\begin{pmatrix}
x_{\ell+1, 2^{\ell-1}+1}\\\vdots\\x_{\ell+1,2^{\ell}}
\end{pmatrix}
\text{ if } \ell \in[m],\label{eq-log-2}\\
&
x= x_{1,1}\label{eq-log-3},
\end{align}
where \(x_{m+1,i}:=s_i\), \(i\in[t]\), and \(\alpha=\sum_{i=1}^{m}b_i2^{i-1}+1\). Note that we have added the additional variables \(x_{\ell,i}\), \(\ell\in[m]\) and \(i\in[2^\ell]\).

Consider the binary tree whose leaves are $x_{m+1,1}=s_1,\ldots,x_{m+1,t}=s_{t}$, where the leftmost leaf is $s_1$ and the rightmost leaf is $s_{2^m}=s_t$. Intuitively, equation (\ref{eq-log-2}) for $\ell=m$ says that variables $x_{m,1},\ldots, x_{m,2^{m-1}}$ are the leaves of the subtree under the path $(b_m)$. For example, if $b_m=1$, the variables $x_{m,1},\ldots,x_{m,2^{m-1}}$ are equal to $s_{2^{m-1}+1},\ldots,s_{t}$, which are the leaves of the subtree under the path (1) as depicted below.
\begin{center}
\begin{tikzpicture}
[n/.style={draw=none},
 every node/.append style={inner ysep=+0pt,outer ysep=+0pt,minimum size=+0pt}]
%\hspace{-3.9cm}
\Tree 
    [.{}
        \edge node[auto=right] {\small 0};
        \qroof{\parbox{\widthof{$s_{2^{m-1}+1}\cdots s_{t}$}}{\vspace*{.1cm}$s_1$\hfill$\cdots$\hfill$s_{2^{m-1}}$}}.{}
        \edge node[auto=left] {\small 1};
        \qroof{\parbox{\widthof{$s_{2^{m-1}+1}\cdots s_{t}$}}{\vspace*{.1cm}$s_{2^{m-1}+1}\cdots s_{t}$}}.{}
    ]
\end{tikzpicture}
\end{center}
Similarly, equation (\ref{eq-log-2}) for $\ell=m-1$ says that variables $x_{m-1,1},\ldots,x_{m-1,2^{m-2}}$ are the leaves of the subtree under the path $(b_m,b_{m-1})$. For example, if $(b_m,b_{m-1})=(1,0)$, the variables $x_{m-1,1},\ldots,x_{m-1,2^{m-2}}$ are equal to $x_{m,1}=s_{2^{m-1}+1},\ldots,\allowbreak x_{m,2^{m-2}}=s_{2^{m-1}+2^{m-2}}$, which are the leaves of the subtree under the path (1,0) as depicted below.
\begin{center}
\begin{tiny}\begin{tikzpicture}
[n/.style={draw=none},
 every node/.append style={inner ysep=+0pt,outer ysep=+0pt,minimum size=+0pt}]
\Tree 
    [.{}
        \edge node[auto=right] {\small 0};
        [.{}
            \edge node[auto=right] {\small 0};
            \qroof{\parbox{\widthof{$s_{2^{m-1}+2^{m-2}+1}\cdots s_{2^{m}}$}}{\vspace*{.1cm}$s_1$\hfill$\cdots$\hfill$s_{2^{m-2}}$}}.{}
            \edge node[auto=left] {\small 1};
            \qroof{\parbox{\widthof{$s_{2^{m-1}+2^{m-2}+1}\cdots s_{2^{m}}$}}{\vspace*{.1cm}$s_{2^{m-2}+1}$\hfill$\cdots$\hfill$s_{2^{m-1}}$}}.{}
        ]
        \edge node[auto=left] {\small 1};
        [.{}
            \edge node[auto=right] {\small 0};
            \qroof{\parbox{\widthof{$s_{2^{m-1}+2^{m-2}+1}\cdots s_{2^{m}}$}}{\vspace*{.1cm}$s_{2^{m-1}+1}\cdots s_{2^{m-1}+2^{m-2}}$}}.{}
            \edge node[auto=left] {\small 1}; 
            \qroof{\parbox{\widthof{$s_{2^{m-1}+2^{m-2}+1}\cdots s_{2^{m}}$}}{\vspace*{.1cm}$s_{2^{m-1}+2^{m-2}+1}\cdots s_{2^{m}}$}}.{}
        ]
    ]
\end{tikzpicture}\end{tiny}
\end{center}

In general, the variables $x_{\ell,1},\ldots,x_{\ell,2^{\ell-1}}$ are equal to the leaves $s_{\sfleft},\ldots,\allowbreak s_{\sfright}$ under the path $(b_m,\ldots, b_\ell)$, where $\sfleft=\sum_{i=\ell}^m b_i 2^{i-1}+1$ and $\sfright=\sfleft+2^{\ell-1}-1$. 
Therefore, for $\ell=1$ equation (\ref{eq-log-2}) says that the variable $x_{1,1}$ is equal to the leaf $s_{\sfleft}=s_{\sfright}=s_\alpha$, since $\sfleft=\sfright=\sum_{i=1}^m b_i2^{i-1}+1=\alpha$, which is the unique leaf (and the unique node) in the subtree under the path $(b_m,\ldots, b_1)$.

Similarly as in Chandran et al.'s proof, for each new variable a new commitment must be added to the proof. But, in contrast with Chandran et al.'s proof, in this case the additional commitments do increase the asymptotic size of the proof. Indeed, the total number of new variables is \(2^{m-1}+2^{m-2}+\ldots+1=2^m-1=t-1\), and thus $t-1$ new commitments must be added.

One can reduce the total size of the commitments using the length reducing Multi-Pedersen commitments from Section~\ref{sec:ext-mp}. However, this must be done carefully in order to be able to express equation (\ref{eq-log-2}) with a Groth-Sahai proof of an equation that involves the MP commitments and the variables $b_1,\ldots,b_m$. For example, if one computes a single commitment to all variables $\MP.\Com_{ck}((x_{1,1},\ldots,x_{m,2^{m-1}})^\top;r)$ it is not clear how to use it to express equation (\ref{eq-log-2}), because not all the variables appear at once in this equation (but all the variables appear in the previous commitment). Our solution is to compute a single MP commitment to each vector that appears in equation (\ref{eq-log-2}) in order to show with Groth-Sahai proofs that
\begin{small}
\begin{align*}
\MP.\Com_{ck_\ell}\left(
                    \pmatri{x_{\ell,1}\\\vdots\\x_{\ell,2^{\ell-1}}};r_\ell\right)
=\ &(1-b_\ell)\MP.\Com_{ck_\ell}\left(\pmatri{x_{\ell+1,1}\\\vdots\\x_{\ell+1,2^{\ell-1}}};r_{\ell,1}\right)+\\
& b_\ell\MP.\Com_{ck_\ell}\left(\pmatri{x_{\ell+1,2^{\ell-1}+1}\\\vdots\\x_{\ell+1,2^\ell}};r_{\ell,2}\right)+\\
& \MP.\Com_{ck_\ell}(\vecb{0};y_\ell),
\end{align*}
\begin{align*}
&\Longleftrightarrow\\
&\MP.\Com_{ck_\ell}\left(\begin{array}{c}
    \pmatri{x_{\ell,1}\\\vdots\\x_{\ell,2^{\ell-1}}}
    -(1-b_\ell)\pmatri{x_{\ell+1,1}\\\vdots\\x_{\ell+1,2^{\ell-1}}}
    -b_\ell\pmatri{x_{\ell+1,2^{\ell-1}+1}\\\vdots\\x_{\ell+1,2^\ell}};\\
    r_\ell-(1-b_\ell)r_{\ell,1}-b_\ell r_{\ell,2}
\end{array}\right)\\
&=
\MP.\Com_{ck_\ell}(\vecb{0};y_\ell)
\end{align*}\end{small}
for each $\ell\in[m]$ and some $y_\ell\in\Z_q$. In this way, we only need $3m=3\log t$ additional commitments. The reason for using different commitment keys for each $\ell\in[m]$ will be clear when we explain soundness.

Concretely, the prover computes
\begin{align*}
&[\vecb{c}_\ell]_1=\MP.\Com_{ck_\ell}((x_{\ell,1},\ldots,x_{1,2^{\ell-1}})^\top;r_\ell),\\
\end{align*}
for random $r_\ell\in\Z_q$ and $\ell\in[m]$, and 
\begin{align*}
&[\vecb{c}_{\ell,1}]_1=\MP.\Com_{ck_{\ell}}((x_{\ell+1,1},\ldots,x_{\ell+1,2^{\ell-1}})^\top;r_{\ell,1}),\\
&[\vecb{c}_{\ell,2}]_1=\MP.\Com_{ck_{\ell}}((x_{\ell+1,2^{m-1}+1},\ldots,x_{\ell+1,2^\ell})^\top;r_{\ell,2}),\\
\end{align*}
for random $r_{\ell,1},r_{\ell,2}\in\Z_q$ and $\ell\in[m-1]$. Note that the prover does not need to compute commitments to $(x_{m+1,1},\ldots,x_{m+1,t})^\top$ since $x_{m+1,i}=s_i$, $i\in[t]$, and thus they can be computed by the verifier.

Then, the prover shows that equation (\ref{eq-log-2}) holds with a GS proof of
the satisfiability of
\begin{align}
&[\vecb{c}_\ell]_1-(1-b_\ell)[\vecb{c}_{\ell,1}]_1-b_\ell[\vecb{c}_{\ell,2}]_1 = y_\ell[\vecb{g}_{\ell,2^{\ell-1}+1}]_1, \text{ for } \ell \in[m],  \label{eq-log-5}
\end{align}
where $[\vecb{c}_{m,1}]:=\MP.\Com_{ck_m}((s_1,\ldots,s_{2^{m-1}})^\top;0)$ and $[\vecb{c}_{m,2}]:=\allowbreak \MP.\Com_{ck_m}(\allowbreak (s_{2^{m-1}+1},\allowbreak\ldots,s_{t})^\top;0)$ can be directly computed by the verifier, and $y_\ell:=r_\ell-(1-b_\ell)r_{\ell,1}-b_\ell r_{\ell,2}$. It also computes Groth-Sahai proofs that
\begin{equation}
b_\ell(b_\ell-1)=0 \label{eq-bit-gs}
\end{equation}
for each $\ell\in[m]$ (or equivalently a proof that $b_\ell\in\bits$).

The prover also shows that equation (\ref{eq-log-3}) is satisfied with a QA-NIZK proof that
\begin{equation}
[\vecb{c}]_1\text{ and }[\vecb{c}_1]_1\text{ open to the same value}, \label{eq-log-6}
\end{equation}
using the proof system from Section \ref{sec:aggcommit}.

Note that variables $x_{\ell+1,1},\ldots,x_{\ell+1,2^{\ell-1}}$ appear in both $[\vecb{c}_{\ell,1}]_1$ and $[\vecb{c}_{\ell+1}]_1$, as well as $x_{\ell+1,2^{\ell-1}+1},\ldots,x_{\ell+1,2^\ell}$ appear in both $[\vecb{c}_{\ell,2}]_1$ and $[\vecb{c}_{\ell+1}]_1$. To get a sound proof, the prover needs to show that this redundancy is consistent. That is, the prover needs to show that $[\vecb{c}_{\ell,1}]_1$ and $[\vecb{c}_{\ell,2}]_1$ are commitments to the first and last halves of the opening of $[\vecb{c}_{\ell+1}]_1$.

For \(\ell\in[m]\), let \(ck_\ell:=([\matr{G}_\ell]_1,\allowbreak[\vecb{g}_{\ell,2^{\ell-1}+1}]_1)\in\GG_1^{2\times{2^{\ell-1}+1}}\) the commitment key of a MP commitment scheme and let
\begin{align*}
&\matr{G}_{\ell,1}:=
\begin{pmatrix}
    \vecb{g}_{\ell,1}&\cdots&\vecb{g}_{\ell,2^{\ell-2}}
\end{pmatrix},
&\matr{G}_{\ell,2}:=
\begin{pmatrix}
    \vecb{g}_{\ell,2^{\ell-2}+1}&\cdots&\vecb{g}_{\ell,2^{\ell-1}}
\end{pmatrix}\\
&\matr{G}_\ell:=\matr{G}_{\ell,1}||\matr{G}_{\ell,2}
\end{align*}
To prove consistency the prover will show that, for each $\ell\in [m-1]$, the following linear system is satisfied
{\begin{align}
&\begin{pmatrix}
\vecb{c}_{\ell+1}\\
\vecb{c}_{\ell,1}\\
\vecb{c}_{\ell,2}
\end{pmatrix}
=
&\left(\begin{array}{cc|ccc}
\matr{G}_{\ell+1,1}           & \matr{G}_{\ell+1,2}            & \vecb{g}_{\ell+1,2^\ell+1} & \vecb{0}                     & \vecb{0}\\
\matr{G}_{\ell}               & \matr{0}_{2\times2^{\ell-1}}   & \vecb{0}                   & \vecb{g}_{\ell,2^{\ell-1}+1} & \vecb{0} \\
\matr{0}_{2\times 2^{\ell-1}} & \matr{G}_{\ell}                & \vecb{0}                   & \vecb{0}                     & \vecb{g}_{\ell,2^{\ell-1}+1}
\end{array}\right)
\vecb{w},\label{eq-log-split}
\end{align}}%
for some \(\vecb{w}\in\Z_q^{2^\ell+3}\), which can be proven using the proof system from Section \ref{sec:concat}.

Intuitively, \(\vecb{w}\) should be equal to \((x_{\ell+1,1},\ldots,x_{\ell+1,2^{\ell}},\allowbreak r_{\ell+1},r_{\ell,1},r_{\ell,2})\) and thus 
\begin{align*}
&[\vecb{c}_{\ell,1}]_1=\MP.\Com_{ck_\ell}(\allowbreak(x_{\ell+1,1},\ldots,x_{\ell+1,2^{\ell-1}})^\top;r_{\ell,1})\text{ and }\\
&[\vecb{c}_{\ell,2}]_1=\MP.\Com_{ck_\ell}((x_{\ell+1,2^{\ell-1}+1},\ldots, x_{\ell+1,2^{\ell}})^\top;r_{\ell,2}).
\end{align*}
However, since Multi-Pedersen commitments have multiple openings it might be the case that the satisfying witness of the proof is different from \((x_{\ell+1,1},\ldots,\allowbreak x_{\ell+1,2^\ell},r_{\ell+1},r_{\ell,1},r_{\ell,2})\) and thus the intuitive reasoning is invalid.

Despite this flawed reasoning, we will show that the proof system is still sound.

\subsubsection{Soundness Intuition}
Suppose that an adversary against soundness outputs GS commitments to \(b_1,\ldots,b_{m}\in\Z_q\), outputs commitments \([\vecb{c}_\ell]_1\), $\ell\in[m]$, and \([\vecb{c}_{\ell,1}]_1,[\vecb{c}_{\ell,2}]\), $\ell\in[m-1]$, a GS proofs of the satisfiability of equation (\ref{eq-log-5}), and QA-NIZK proofs of (\ref{eq-log-6}) and (\ref{eq-log-split}) for each \(\ell\in[m-1]\).
Note that perfect soundness of Groth-Sahai proofs for equation (\ref{eq-bit-gs}) imply that \(b_1,\ldots,b_m\in\bits\).

For $\ell\in[m]$, define $\alpha_\ell$ as the position of $s_\alpha$ respective to the leaves under the path $(b_m,\ldots, b_\ell)$, that is $\alpha_\ell:=\alpha-\sfleft+1$. Note that $\alpha_\ell \in[1,2^{\ell-1}]$ since
$$
1\leq\alpha_\ell = \sum_{i=1}^m b_i2^{i-1}-\sum_{i=\ell}^mb_i2^{i-1}+1 = \sum_{i=1}^{\ell-1}b_i2^{i-1}+1\leq 2^{\ell-1}.
$$

The key observation is that, for a fixed $\alpha\in[m]$, even if in equation (\ref{eq-log-2}) $x_{\ell,j}$ is not correctly computed for $j\neq\alpha_\ell$, it holds that $x_{m,1}=s_\alpha$ anyway. We will take advantage of this observation and the fact that the adversary commits to a fixed $\alpha=\sum_{i=1}^n b_i2^{i-1}+1$ to guarantee perfect soundness of equation (\ref{eq-log-2}) at least for coordinate $\alpha_\ell$ for each $\ell\in[m]$. We do so by picking the commitment key $ck_\ell$ in such a way that its $\alpha_\ell$ th column is linearly independent from the other columns. Although we will not be able to guarantee that $x_{\ell,j}$ is correctly computed if $j\neq\alpha_\ell$, at least we will be able to do so for $x_{\ell,\alpha_\ell}$.

In the reduction we will guess the (sub-)path $(b_{m-1},\ldots, b_1)$ (it will be not necessary to guess first the edge of the path) chosen by the adversary, and for each $\ell\in[m]$, we will chose $\vecb{g}_{\ell,\alpha_\ell}$ linearly independent from the other $2^{\ell-1}$ vectors in $ck_\ell$. This can be done choosing  random $b'_{m-1},\ldots, b'_1\in\bits$ and aborting if $(b'_{m-1},\ldots, b'_{1})\neq(b_{m-1},\cdots, b_1)$. Therefore, our security reduction will have a security loss factor of $1/2^{m-1}=2/\setsize$. We sample $ck_\ell\gets\distlin_1^{2^{\ell-1},\alpha_\ell}$, as defined on Section \ref{sec:mddh}, which implies that for every $\ell\in[m]$ there exists unique $\tilde{x}_{\ell},\tilde{r}_\ell\in\Z_q$ such that $\vecb{c}_\ell:=\tilde{x}_\ell\vecb{g}_{\ell,\alpha_\ell}+\tilde{r}_\ell\vecb{g}_{\ell,2^{\ell-1}+1}$.

We prove by induction on \(\ell\) that \(\vecb{c}_\ell= s_\alpha\vecb{g}_{\ell,\alpha_\ell}+\tilde{r}_{\ell}\vecb{g}_{\ell,2^{\ell-1}+1}\), for some $\tilde{r}_\ell\in\Z_q$. If this is the case $\vecb{c}_1=s_\alpha\vecb{g}_{1,1}+\tilde{r}_{1}\vecb{g}_{1,2}$. Soundness of proof for equation (\ref{eq-log-6}) together with the fact that \(ck_1\) is perfectly binding implies that \(x=\tilde{x}_{1}=s_{\alpha}\in S\) which proves soundness.

First, it will be useful to prove the next lemma about $\alpha_\ell$.
\begin{lemma} Let $b_m,\ldots,b_1\in\bits$. For all $\ell\in[m-1]$, $\alpha_{\ell+1}=\alpha_{\ell}+b_\ell2^{\ell-1}$.
\label{lemma:alpha}
\end{lemma}
\begin{proof}
To avoid confusion, define here $\sfleft_\ell:=\sum_{i=\ell}^m b_i2^{i-1}+1$ (previously simply defined as $\sfleft$, the index of the leftmost leaf under the path $(b_m,\cdots, b_\ell)$). It holds that
\begin{align*}
\alpha_{\ell+1} &= \alpha - \sfleft_{\ell+1} + 1\\
&= \sum_{i=1}^{m} b_i2^{i-1} - \sum_{i=\ell+1}^{m}b_i2^{i-1} +1\\
&= \sum_{i=1}^{m} b_i2^{i-1} - \sum_{i=\ell}^{m}b_i2^{i-1} +1 +b_\ell2^{\ell-1}\\
&= \alpha - \sfleft_\ell +1 -b_\ell2^{\ell-1}\\
&=\alpha_\ell+b_\ell2^{\ell-1}
\end{align*}
\end{proof}

Now we prove that, for all $\ell\in[m]$, \(\vecb{c}_\ell= s_\alpha\vecb{g}_{\ell,\alpha_\ell}+\tilde{r}_{\ell}\vecb{g}_{\ell,2^{\ell-1}+1}\).
In the base case ($\ell=m$) the fact that \(\vecb{g}_{m,i}\in\Span(\vecb{g}_{m,2^{m-1}+1})\) if \(i\neq \alpha_m\) together with Lemma \ref{lemma:alpha} implies that 
\begin{align*}
\vecb{c}_m &= (1-b_m)\sum_{i=1}^{2^{m-1}}s_i \vecb{g}_{m,i}+b_m\sum_{i=1}^{2^{m-1}}s_{i+2^{m-1}}\vecb{g}_{m,i}\\
&= (1-b_m)s_{\alpha_m}\vecb{g}_{m,\alpha_m} +b_ms_{\alpha_m+2^{m-1}}\vecb{g}_{m,\alpha_m}+\tilde{r}_1\vecb{g}_{m,2^{m-1}+1}\\
&= (1-b_m)s_{\alpha-\sfleft+1}\vecb{g}_{m,\alpha_m} +b_ms_{\alpha-\sfleft+1+2^{m-1}}\vecb{g}_{m,\alpha_m}+\tilde{r}_1\vecb{g}_{m,2^{m-1}+1}\\
&=\begin{cases}
    s_{\alpha-1+1}\vecb{g}_{m,\alpha_m}+\tilde{r}_1\vecb{g}_{m,t/2} & \text{ if } b_m=0 \ (\text{and thus }\sfleft=1) \\
    s_{\alpha-(2^{m-1}+1)+1+2^{m-1}}\vecb{g}_{m,\alpha_m}+\tilde{r}_1\vecb{g}_{m,t/2} & \text{ if } b_m=1 \ (\text{and thus }\sfleft=2^{m-1}+1) 
\end{cases}
\end{align*}
for some \(\tilde{r}_1\in\Z_q\). In both cases $\vecb{c}_1=s_\alpha\vecb{g}_{1,\alpha_m}+\tilde{r}_1\vecb{g}_{m,t/2}$.

In the inductive case we assume that \(\vecb{c}_{\ell+1}=s_{\alpha}\vecb{g}_{\ell+1,\alpha_{\ell+1}}+\tilde{r}_{\ell+1}\vecb{g}_{\ell+1,2^\ell+1}\) and we want to show that $\vecb{c}_\ell = s_\alpha\vecb{g}_{\ell,\alpha_\ell}+\tilde{r}_\ell\vecb{g}_{\ell,2^{\ell-1}+1}$. Since \(\vecb{g}_{\ell+1,\alpha_{\ell+1}}\) is linearly independent from the rest of vectors in \(ck_{\ell+1}\), any solution to equation (\ref{eq-log-split}) is equal to \(s_{\alpha}\) at position \(\alpha_{\ell+1}=\alpha_\ell+b_\ell2^{\ell-1}\) as depicted below.
\begin{align*}
\pmatri{\vecb{c}_{\ell+1}\\\vecb{c}_{\ell,1}\\\vecb{c}_{\ell,2}}=
\pmatri{
\cdots & \vecb{g}_{\ell+1,\alpha_\ell} & \cdots  & \vecb{g}_{\ell+1,\alpha_\ell+2^{\ell-1}} & \cdots\\
\cdots & \vecb{g}_{\ell,\alpha_\ell}     & \cdots  & \vecb{0}                           & \cdots\\
\cdots & \vecb{0}                        & \cdots  & \vecb{g}_{\ell,\alpha_\ell}        & \cdots
}
\pmatri{\vdots\\s_\alpha\\\vdots}
\end{align*}
If $b_{\ell}=0$, by Lemma \ref{lemma:alpha}, $\alpha_{\ell+1}=\alpha_\ell$. Therefore, any solution to equation (\ref{eq-log-split}) is equal to $s_\alpha$ at position $\alpha_\ell$ and thus $\vecb{c}_{\ell,1} = s_\alpha\vecb{g}_{\ell,\alpha_\ell}+\tilde{r}_{\ell,1}\vecb{g}_{\ell,2^{\ell-1}+1}$.
Equation (\ref{eq-log-5}) implies that
\begin{align*}
\vecb{c}_{\ell}=&(1-b_\ell)(s_\alpha\vecb{g}_{\ell,\alpha_\ell}+\tilde{r}_{\ell,1}\vecb{g}_{\ell,2^{\ell-1}+1})+b_\ell\vecb{c}_{\ell,2}+y_\ell\vecb{g}_{\ell,2^{\ell-1}+1}\\
               =& s_\alpha\vecb{g}_{\ell,\alpha_\ell}+(\tilde{r}_{\ell,1}+y_\ell)\vecb{g}_{\ell,2^{\ell-1}+1}.
\end{align*}
If $b_{\ell}=1$, then $\alpha_{\ell+1}=\alpha_\ell+2^{\ell-1}$ and similarly, $\vecb{c}_{\ell}=s_\alpha\vecb{g}_{\ell,\alpha_\ell}+(\tilde{r}_{\ell,2}+y_\ell)\vecb{g}_{\ell,2^{\ell-1}+1}$.


