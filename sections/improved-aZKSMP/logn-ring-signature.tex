In this section we will describe how the improved aZKSMP can be used to obtain a {theoretical} ring signature with signature size \(\Theta(\log n)\), where \(n\) is the size of the ring, in the standard model (i.e. without {random oracles} nor {non-falsifiable  assumptions}). In other words we show that there exist set-membership proofs of size \(\Theta(\log \setsize )\) even when \(S\subset \GG_1\) is not fixed (that is, prove membership in $\Lang_{ck,\mathsf{set}}^n\subset\GG_s^n$),\footnote{It seems that in the aggregated case this is also true. However, we leave this as an open problem.} and from this proof is direct to construct a ring signature using the techniques of Chandran et al.~\cite{ICALP:ChaGroSah07}.

We say that our construction is theoretical because, although the asymptotic signature size is \(\Theta(\log n)\), we use general results of NP-completeness and thus the ``real'' signature size is \(\Theta(\log n)+\mathsf{poly}(\lambda)\), where \(\lambda\) is the security parameter. Nonetheless, our result should be interpreted as a feasibility result and it poses the interesting question of whether the \(\mathsf{poly}(\lambda)\) part can be reduced to a practical value.

Lets see in a little more detail what is hidden in the asymptotics. When working on bilinear groups, the size of the proof is usually measured in number of group elements, and thus, if the proof is of size $f(n)$ groups elements its size in bits is $|\GG_s|f(n)$. Since $|\GG_s|$ is a linear function of the security parameter (and in practice of the order of one kilobit for $\lambda=128$) it is always ignored. In our case our proof additionally adds a constant number of group elements, with respect to $n$, but for which we only know that is upper bounded by some polynomial of $\lambda$. Therefore, although this part of the proof is independent of $n$, it is misleading to ignore its contribution to the total proof size.
 
We define an encoding function \(\mathcal{E}:\GG_1\cup\GG_2\to\bits^\ell\), where \(\ell=\mathsf{poly}(\lambda)\), which translates group elements to their natural bit encoding. Given a ring \(R=\{[vk_1]_1,\ldots,[vk_n]_1\}\) and \([vk]\in R\), we commit to \(\vecb{x}:=\mathcal{E}([vk]_1)\) using GS commitments and show that \(\mathcal{E}([vk]_1)\in\mathcal{E}(R)\), bit-by-bit, using the improved aZKSMP for \(S\subset \Z_q\). Let \([\vecb{c}]_1:=\GS.\Com_{ck}(\vecb{x}^\top;\matr{R}^\top)=([\vecb{c}_1]\cat\cdots\cat[\vecb{c}_\ell])\in\GG_1^{2\times \ell}\) and compute \([\vecb{d}]_1:=\GS.\Com_{ck}([vk]_1;(w_1,w_2)^\top)\), we would like to show that \(\mathcal{E}([vk]_1)=\vecb{x}\).

Given \(\vecb{x}\in\bits^\ell\), there exists a circuit \(C(\vecb{x},\vecb{w}_1,\vecb{w}_2,\vecb{y})\) that interprets \(\vecb{x},\vecb{w}_1,\vecb{w}_2\) as  elements \([x]_1,w_1,w_2\) of \(\GG_1,\Z_q,\Z_q\), respectively, and \(\vecb{y}\) as an element \([\vecb{d}]_1\) of \(\GG_1^2\), computes the bit-string
\begin{align*}
\vecb{s}:=
    &C_{\GG_1,-}(\\
        &\quad C_{\GG_1,-}(\\
            &\qquad C_{\GG_1,-}\left(
                \vecb{y},
                \mathcal{E}\left(
                        ({[x]_1},{[0]_1})^\top\right)\right),\\
            &\qquad C_{\GG_1,\cdot}(
                        \vecb{w}_1,
                        \mathcal{E}([\vecb{u}_1]_1)),\\
        &\quad C_{\GG_1,\cdot}(
                    \vecb{w}_2,
                    \mathcal{E}([\vecb{u}_2]_1)),
\end{align*}
where the circuit $C_{\GG_1,-}:\bits^{2\ell}\to\bits^\ell$ computes the subtraction (inverse and addition computation) of two elements of $\GG_1$, and the circuit $C_{\GG_1,\cdot}:\bits^{\log q+\ell}\to\bits^\ell$ computes the multiplication of an element from $\GG_1$ and an integer in $\Z_q$, and where \(([\vecb{u}_1]\cat[\vecb{u}_2]_1)=ck\). That is, $C$ is computing an encoding of $[\vecb{d}]_1-([x]_1,[0]_1)^\top-w_1[\vecb{u}]_1-w_2[\vecb{u}_2]$. Finally, $C$ returns 1 iff \(\vecb{s}=\mathcal{E}([0]_1)\). Using the NIZK proof system for circuit satisfiability of Groth et al.~\cite{EC:GroOstSah06} we can prove this statement with a proof of size \(\Theta(|C|)=\mathsf{poly}(\lambda)\), because \(|C|\) only depends on \(q\in\poly(\lambda)\) and \(\ell\in\poly(\lambda)\).

We conclude that we can prove that \([vk]_1\in R\) with a proof of size \(\Theta(\log n +\poly(\lambda))=\Theta(\log n)\), and thus we can construct a ring signature with signature size \(\Theta(\log n)\)
