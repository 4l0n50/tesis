
\subsection{Definition}
We follow the definitions from \cite{ICALP:ChaGroSah07} described below.

\begin{definition}[Ring Signature]
A ring signature scheme consists of a quadruple of
PPT algorithms $(\mathsf{CRSGen}, \mathsf{Gen}, \mathsf{Sign}, \mathsf{Verify})$ that respectively, generate the common
reference string, generate keys for a user, sign a message, and verify the signature of a
message.
\begin{itemize}
\item $\mathsf{CRSGen}(1^k)$, where $k$ is a security parameter, outputs the common reference
string $\rho$.
\item $\mathsf{Gen}(\rho)$ is run by the user. It outputs a public verification key $vk$ and a private
signing key $sk$.
\item $\mathsf{Sign}_{\rho,sk}(m, R)$ outputs a signature $\sigma$ on the message $m$ with respect to the ring
$R = \{vk_1,\ldots,vk_n\}$. We require that $(vk, sk)$ is a valid key-pair output by $\mathsf{Gen}$
and that $vk \in R$.
\item $\mathsf{Verify}_{\rho,R}(m, \sigma)$ verifies a purported signature $\sigma$ on a message $m$ with respect to
the ring of public keys $R$.
\end{itemize}
The quadruple $(\mathsf{CRSGen}, \mathsf{Gen}, \mathsf{Sign}, \mathsf{Verify})$ is a ring signature with perfect
anonymity if it has perfect correctness, computational unforgeability and perfect
anonymity as defined below.
\end{definition}

\begin{definition}[Perfect Correctness]
We require that a user can sign any message on behalf of a ring where she is a member. A ring signature $(\mathsf{CRSGen}, \mathsf{Gen}, \mathsf{Sign}, \mathsf{Verify})$
has perfect correctness if for all adversaries $\advA$ we have:
$$
\Pr\left[\begin{array}{l}
\rho\gets\mathsf{CRSGen}(1^k);(vk,sk)\gets\mathsf{Gen}(\rho);(m,R)\gets\advA(\rho,vk,sk);\\
\sigma\gets\mathsf{Sign}_{\rho,sk}(m;R):\mathsf{Verify}_{\rho,R}(m,\sigma)\text{ or }vk\notin R
\end{array}\right]=1
$$
\end{definition}

\begin{definition}[Computational Unforgeability]
A ring signature scheme $(\mathsf{CRSGen}, \mathsf{Gen}, \mathsf{Sign}, \mathsf{Verify})$
is unforgeable (with respect to insider corruption) if it is infeasible to forge a ring
signature on a message without controlling one of the members in the ring. Formally, it
is unforgeable when for any non-uniform polynomial
time adversaries $\advA$ we have that
$$
\Pr\left[\begin{array}{l}
\rho\gets\mathsf{CRSGen}(1^k);(m,R,\sigma)\gets\advA^{\mathsf{VKGen},\mathsf{Sign},\mathsf{Corrupt}}(\rho):\\
\mathsf{Verify}_{\rho,R}(m,\sigma)=1
\end{array}\right]
$$
is negligible in $k$, where

\begin{itemize}
\item $\mathsf{VKGen}$ on query number $i$ selects a randomizer $w_i$, runs $(vk_i,sk_i) \gets \mathsf{Gen}(\rho; w_i)$
and returns $vk_i$.
\item $\mathsf{Sign}(\alpha, m, R)$ returns $\sigma \gets \mathsf{Sign}_{\rho,sk_\alpha}(m, R)$, provided $(vk_\alpha, sk_\alpha)$ has been generated
by $\mathsf{VKGen}$ and $vk_\alpha\in R$.
\item $\mathsf{Corrupt}(i)$ returns $w_i$ (from which $sk_i$ can be computed) provided $(vk_i, sk_i)$ has
been generated by $\mathsf{VKGen}$.
\item $\advA$ outputs $(m, R, \sigma)$ such that $\mathsf{Sign}$ has not been queried with $(*, m, R)$ and $R$
only contains keys $vk_i$ generated by $\mathsf{VKGen}$ where $i$ has not been corrupted.
\end{itemize}
\end{definition}

\begin{definition}[Perfect Anonymity]
A ring signature scheme
$(\mathsf{CRSGen}, \mathsf{Gen}, \mathsf{Sign}, \mathsf{Verify})$ has perfect anonymity, if a signature on a message
$m$ under a ring $R$ and key $vk_{i_0}$
looks exactly the same as a signature on the
message $m$ under the ring $R$ and key $vk_{i_1}$. This means that the signer's key is hidden
among all the honestly generated keys in the ring. Formally, we require that for any
adversary $\advA$:
\begin{align*}
&\Pr\left[\begin{array}{l}
\rho\gets\mathsf{CRSGen}(1^k);(m,i_0,i_1,R)\gets\advA^{\mathsf{Gen}(\rho)}(\rho):\\
\sigma\gets\mathsf{Sign}_{\rho,sk_{i_0}}(m,R):\advA(\sigma)=1
\end{array}\right]
=\\
&\Pr\left[\begin{array}{l}
\rho\gets\mathsf{CRSGen}(1^k);(m,i_0,i_1,R)\gets\advA^{\mathsf{Gen}(\rho)}(\rho):\\
\sigma\gets\mathsf{Sign}_{\rho,sk_{i_1}}(m,R):\advA(\sigma)=1
\end{array}\right]
\end{align*}
where $\advA$ chooses $i_0, i_1$ such that $(vk_{i_0}, sk_{i_0}),(vk_{i_1}, sk_{i_1})$ have been generated by the
oracle $\mathsf{Gen(\rho)}$.
\end{definition}

\subsection{High Level Description}
The starting point of this construction is the proof system for set membership with proof size $\Theta(\sqrt[3]{n})$ from Sect. \ref{sec:bits-applications}. Such construction does not suffice to construct a ring signature because the CRS is fixed to a specific set and thus, the resulting ring signature will be fixed to a specific ring. We will show how to overcome this problem, but first it will be useful to see how the ``flawed'' ring signature looks like.

We will define the ring signature CRS as the part which is independent of the ring, that is the matrix $[\matr{A}]_2$, and the other part will be the public keys. Given the ring $R=\{P_1,\ldots,P_n\}$, the public key of party $P=P_\alpha$, where $i=m^2(i_\alpha-1)+m(j_\alpha-1)+k_\alpha$, is $[vk_\alpha]_1$ and we also need to add $\vecb{a}_{i_\alpha}[vk_\alpha]_1$ to carry on the proof of membership in $R$. 

Given that we are assuming that the ring is $R=\{\ldots,P_\alpha,\ldots\}$ we are not prepared for another ring $R'$ where $P_\alpha$ appears in other position. One possible solution is to enlarge the public key including also $\vecb{a}_{1}[vk_\alpha]_1,\ldots,\vecb{a}_{m}[vk_\alpha]_1$, however the maximum ring size will be fixed and the public key size will be $\Theta(\sqrt[3]{n})$ and not constant. We would like to build a scheme without this disadvantages.

In our scheme the public key of party $P$ will be $[vk]_1$, $[\vecb{a}]_2$, and $\vecb{a}[vk]_1$, where $\vecb{a}\gets\dist_{k,1}$ is chosen independently for each key. Now, using the $\Theta(\sqrt{n})$ set membership proof from Sect. \ref{sec:bits-applications} (the one where the set is not fixed) and given the ring $R=\{[vk_{(1,1,1)}]_1,\ldots,[vk_{(m,m,m)}]_1\}$ (recall notation $(i,j,k):=(i-1)m^2+(j-1)m+k$, where $m=\sqrt[3]{n}$), the prover selects an element $[\vecb{x}]_2$ from the set $S_1=\{\sum_{i\in[m]}[\vecb{a}_{(i,1,1)}]_2,\ldots,\sum_{i\in[m]}[\vecb{a}_{(i,m,m)}]_2\}$ and an element $[\vecb{y}]_1$ from the set $S_2=\{\sum_{i\in[m]}\vecb{a}_{(i,1,1)}[vk_{(i,1,1)}]_1,\ldots,\sum_{i\in[m]}\vecb{a}_{(i,m,m)}[vk_{(i,m,m)}]_1\}$. Both sets are of size $n^{2/3}$ and thus the set membership proof is of size $\Theta(\sqrt[3]{n})$.

Now that the prover has selected elements $[\vecb{x}]_2=\sum_{i\in[m]}[\vecb{a}_{(i,j_\alpha,k_\alpha)}]_2$ and $[\vecb{y}]_1=\sum_{i\in[m]}\vecb{a}_{(i,j_\alpha,k_\alpha)}[vk_{(i,j_\alpha,k_\alpha)}]_1$, we would like to extract from them  $[vk_\alpha]_1=[vk_{(i_\alpha,j_\alpha,k_\alpha)}]_1$. To do so we will use the following assumption introduced by Groth and Lu \cite{AC:GroLu07}.

\begin{definition}[Permutation Pairing Assumption]
Let $\dist_{k,m}=\underbrace{\dist_{k,1}||\ldots||\dist_{k,1}}_{m\text{ times}}$, where concatenation of matrix distributions is defined in the natural way. We say that the $m$-Permutation Pairing Assumption holds relative to $\G$ if for any adversary $\advA$
$$
\Pr\left[
\begin{array}{l}
gk\gets\G(1^k);\vecb{a}\gets\dist_{k,m};[\vecb{z}]_2\gets\advA(gk,[\vecb{a}]_2):\\
\sum_{i\in[m]}[\vecb{z}_i]_2=\sum_{i\in[m]}[\vecb{a}_i]_2\text{ and }
\vecb{z}\text{ is not a permutation of }\vecb{a}
\end{array}
\right]
$$
is negligible in $k$.
\end{definition}

The prover additionally commits to $[\kappa_1]_1:=[vk_{(1,j_\alpha,k_\alpha)}]_1,\ldots,[\kappa_m]_1:=[vk_{(m,j_\alpha,k_\alpha)}]_1$ and $[\vecb{z}_1]_2:=[\vecb{a}_{(1,j_\alpha,k_\alpha)}]_2,\ldots,[\vecb{z}_m]_2:=[\vecb{a}_{(m,j_\alpha,k_\alpha)}]_2$. A proof that $\sum_{i\in[m]}[\vecb{z}_i]_2=[\vecb{x}]_2$ guarantees that $[\vecb{z}_1]_2,\ldots,[\vecb{z}_m]_2$ is a permutation of $[\vecb{a}_{(1,j_\alpha,k_\alpha)}]_2,\ldots,[\vecb{a}_{(m,j_\alpha,k_\alpha)}]_2$, unless one can break the $m$-Permutation Pairing assumption, and a proof that $\sum_{i\in[m]}[\kappa]_1[\vecb{z}_i]_2=[\vecb{y}]_1[1]_2$ guarantees that $[\kappa_1]_1,\ldots,[\kappa_m]_1$ is a permutation of $[vk_{(1,j_\alpha,k_\alpha)}]_1,\ldots,[vk_{(m,j_\alpha,k_\alpha)}]_1$, unless one can break the $\dist_{k,m}\mbox{-}\kermdh$ assumption.

At this point the prover can simply select an element from $[\kappa_1]_1,\ldots,[\kappa_m]_1$, which is guaranteed to be an element from the ring.

\subsection{Detailed Description}
