%% (Master) Thesis template
% Template version used: v1.4
%
% Largely adapted from Adrian Nievergelt's template for the ADPS
% (lecture notes) project.


%% We use the memoir class because it offers a many easy to use features.
\documentclass[11pt,a4paper,titlepage]{memoir}

%% Packages
%% ========

%% LaTeX Font encoding -- DO NOT CHANGE
\usepackage[OT1]{fontenc}

%% Babel provides support for languages.  'english' uses British
%% English hyphenation and text snippets like "Figure" and
%% "Theorem". Use the option 'ngerman' if your document is in German.
%% Use 'american' for American English.  Note that if you change this,
%% the next LaTeX run may show spurious errors.  Simply run it again.
%% If they persist, remove the .aux file and try again.
\usepackage[english]{babel}

%% Input encoding 'utf8'. In some cases you might need 'utf8x' for
%% extra symbols. Not all editors, especially on Windows, are UTF-8
%% capable, so you may want to use 'latin1' instead.
\usepackage[utf8]{inputenc}

%% This changes default fonts for both text and math mode to use Herman Zapfs
%% excellent Palatino font.  Do not change this.
%\usepackage[sc]{mathpazo}

%% The AMS-LaTeX extensions for mathematical typesetting.  Do not
%% remove.
\usepackage{amsmath,amssymb,amsfonts,mathrsfs}

%% NTheorem is a reimplementation of the AMS Theorem package. This
%% will allow us to typeset theorems like examples, proofs and
%% similar.  Do not remove.
%% NOTE: Must be loaded AFTER amsmath, or the \qed placement will
%% break
\usepackage[amsmath,thmmarks]{ntheorem}

%% LaTeX' own graphics handling
\usepackage{graphicx}

%% We unfortunately need this for the Rules chapter.  Remove it
%% afterwards; or at least NEVER use its underlining features.
\usepackage{soul}

%% This allows you to add .pdf files. It is used to add the
%% declaration of originality.
\usepackage{pdfpages}

%% Some more packages that you may want to use.  Have a look at the
%% file, and consult the package docs for each.
%% See the TeXed file for more explanations

%% [OPT] Multi-rowed cells in tabulars
%\usepackage{multirow}

%% [REC] Intelligent cross reference package. This allows for nice
%% combined references that include the reference and a hint to where
%% to look for it.
\usepackage{varioref}

%% [OPT] Easily changeable quotes with \enquote{Text}
%\usepackage[german=swiss]{csquotes}

%% [REC] Format dates and time depending on locale
\usepackage{datetime}

%% [OPT] Provides a \cancel{} command to stroke through mathematics.
%\usepackage{cancel}

%% [NEED] This allows for additional typesetting tools in mathmode.
%% See its excellent documentation.
\usepackage{mathtools}

%% [ADV] Conditional commands
%\usepackage{ifthen}

%% [OPT] Manual large braces or other delimiters.
%\usepackage{bigdelim, bigstrut}

%% [REC] Alternate vector arrows. Use the command \vv{} to get scaled
%% vector arrows.
\usepackage[h]{esvect}

%% [NEED] Some extensions to tabulars and array environments.
\usepackage{array}

%% [OPT] Postscript support via pstricks graphics package. Very
%% diverse applications.
%\usepackage{pstricks,pst-all}

%% [?] This seems to allow us to define some additional counters.
%\usepackage{etex}

%% [ADV] XY-Pic to typeset some matrix-style graphics
%\usepackage[all]{xy}

%% [OPT] This is needed to generate an index at the end of the
%% document.
%\usepackage{makeidx}

%% [OPT] Fancy package for source code listings.  The template text
%% needs it for some LaTeX snippets; remove/adapt the \lstset when you
%% remove the template content.
\usepackage{listings}
\lstset{language=TeX,basicstyle={\normalfont\ttfamily}}

%% [REC] Fancy character protrusion.  Must be loaded after all fonts.
\usepackage[activate]{pdfcprot}

%% [REC] Nicer tables.  Read the excellent documentation.
\usepackage{booktabs}


% Mios de mi
\usepackage{enumitem}
\usepackage{currfile}


%% Our layout configuration.  DO NOT CHANGE.
%% Memoir layout setup

%% NOTE: You are strongly advised not to change any of them unless you
%% know what you are doing.  These settings strongly interact in the
%% final look of the document.

% Dependencies
\usepackage{ETHlogo}

% Turn extra space before chapter headings off.
\setlength{\beforechapskip}{0pt}

\nonzeroparskip
\parindent=0pt
\defaultlists

% Chapter style redefinition
\makeatletter

\if@twoside
  \pagestyle{Ruled}
  \copypagestyle{chapter}{Ruled}
\else
  \pagestyle{ruled}
  \copypagestyle{chapter}{ruled}
\fi
\makeoddhead{chapter}{}{}{}
\makeevenhead{chapter}{}{}{}
\makeheadrule{chapter}{\textwidth}{0pt}
\copypagestyle{abstract}{empty}

\makechapterstyle{bianchimod}{%
  \chapterstyle{default}
  \renewcommand*{\chapnamefont}{\normalfont\Large\sffamily}
  \renewcommand*{\chapnumfont}{\normalfont\Large\sffamily}
  \renewcommand*{\printchaptername}{%
    \chapnamefont\centering\@chapapp}
  \renewcommand*{\printchapternum}{\chapnumfont {\thechapter}}
  \renewcommand*{\chaptitlefont}{\normalfont\huge\sffamily}
  \renewcommand*{\printchaptertitle}[1]{%
    \hrule\vskip\onelineskip \centering \chaptitlefont\textbf{\vphantom{gyM}##1}\par}
  \renewcommand*{\afterchaptertitle}{\vskip\onelineskip \hrule\vskip
    \afterchapskip}
  \renewcommand*{\printchapternonum}{%
    \vphantom{\chapnumfont {9}}\afterchapternum}}

% Use the newly defined style
\chapterstyle{bianchimod}

\setsecheadstyle{\Large\bfseries\sffamily}
\setsubsecheadstyle{\large\bfseries\sffamily}
\setsubsubsecheadstyle{\bfseries\sffamily}
\setparaheadstyle{\normalsize\bfseries\sffamily}
\setsubparaheadstyle{\normalsize\itshape\sffamily}
\setsubparaindent{0pt}

% Set captions to a more separated style for clearness
\captionnamefont{\sffamily\bfseries\footnotesize}
\captiontitlefont{\sffamily\footnotesize}
\setlength{\intextsep}{16pt}
\setlength{\belowcaptionskip}{1pt}

% Set section and TOC numbering depth to subsection
\setsecnumdepth{subsection}
\settocdepth{subsection}

%% Titlepage adjustments
\pretitle{\vspace{0pt plus 0.7fill}\begin{center}\HUGE\sffamily\bfseries}
\posttitle{\end{center}\par}
\preauthor{\par\begin{center}\let\and\\\Large\sffamily}
\postauthor{\end{center}}
\predate{\par\begin{center}\Large\sffamily}
\postdate{\end{center}}

\def\@advisors{}
\newcommand{\advisors}[1]{\def\@advisors{#1}}
\def\@department{}
\newcommand{\department}[1]{\def\@department{#1}}
\def\@thesistype{}
\newcommand{\thesistype}[1]{\def\@thesistype{#1}}

\renewcommand{\maketitlehooka}{\noindent\ETHlogo[2in]}

\renewcommand{\maketitlehookb}{\vspace{1in}%
  \par\begin{center}\Large\sffamily\@thesistype\end{center}}

\renewcommand{\maketitlehookd}{%
  \vfill\par
  \begin{flushright}
    \sffamily
    \@advisors\par
    \@department, ETH Z\"urich
  \end{flushright}
}

\checkandfixthelayout

\setlength{\droptitle}{-48pt}

\makeatother

% This defines how theorems should look. Best leave as is.
\theoremstyle{plain}
\setlength\theorempostskipamount{0pt}

%%% Local Variables:
%%% mode: latex
%%% TeX-master: "thesis"
%%% End:


%% Theorem environments.  You will have to adapt this for a German
%% thesis.
%% Theorem-like environments

%% This can be changed according to language. You can comment out the ones you
%% don't need.

\numberwithin{equation}{chapter}

%% German theorems
%\newtheorem{satz}{Satz}[chapter]
%\newtheorem{beispiel}[satz]{Beispiel}
%\newtheorem{bemerkung}[satz]{Bemerkung}
%\newtheorem{korrolar}[satz]{Korrolar}
%\newtheorem{definition}[satz]{Definition}
%\newtheorem{lemma}[satz]{Lemma}
%\newtheorem{proposition}[satz]{Proposition}

%% English variants
\newtheorem{theorem}{Theorem}[chapter]
\newtheorem{example}[theorem]{Example}
\newtheorem{remark}[theorem]{Remark}
\newtheorem{corollary}[theorem]{Corollary}
\newtheorem{definition}[theorem]{Definition}
\newtheorem{lemma}[theorem]{Lemma}
\newtheorem{proposition}[theorem]{Proposition}

%% Proof environment with a small square as a "qed" symbol
\theoremstyle{nonumberplain}
\theorembodyfont{\normalfont}
\theoremsymbol{\ensuremath{\square}}
\newtheorem{proof}{Proof}
%\newtheorem{beweis}{Beweis}


%% Helpful macros.
%% Custom commands
%% ===============

%% Fixed/scaling delimiter examples (see mathtools documentation)
\DeclarePairedDelimiter\abs{\lvert}{\rvert}
\DeclarePairedDelimiter\norm{\lVert}{\rVert}

\newcommand{\sblt}{\stackrel{s}{\bullet}}

\newcommand{\algSize}{normalsize} 
%\newcommand{\algSize}{footnotesize}

\newcommand{\ps}{\Psi({\dist_k})}
\newcommand{\psws}{{\Psi(\overline{\dist}_k)}}
\newcommand{\sps}{{\Psi_{\mathsf{spl}}(\dist_k)}}
\newcommand{\Sps}{\Psi_{\mathsf{spl}}}
\newcommand{\spsws}{{\Psi_\mathsf{spl}(\overline{\dist}_k)}}
\newcommand{\Spsws}{\Psi_\mathsf{spl}}
\newcommand{\spsmas}{\Psi_{\sfsum}(\dist_k)}
\newcommand{\Spsmas}{\Psi_{\sfsum}}
\newcommand{\spswsmas}{\Psi_{\sfsum}(\overline{\dist}_k)}
\newcommand{\Spswsmas}{\Psi_{\sfsum}}
\newcommand{\spswscomm}{{\Psi_{\mathsf{com}}(\overline{\dist}_k)}}
\newcommand{\Spswscomm}{\Psi_{\mathsf{com}}}
\newcommand{\bbb}{\bar{b}}
\newcommand{\capprox}{\overset{c}{\approx}}

\newcommand{\latexDeMierdaEstupido}{]}

\newcommand{\Comm}{\mathsf{Comm}}
\newcommand{\Com}{\mathsf{Com}}
\newcommand{\vect}{\mathbf{vec}}

\newcommand{\lef}{{\mathtt{l}}}
\newcommand{\rig}{{\mathtt{r}}}
\newcommand{\stmnt}{\mathsf{stm}}
%Commitment keys

%Log Left Commitment Key
\newcommand{\llck}{\vecb{g}}
%Log Right Commitment Key
\newcommand{\lrck}{\vecb{h}}
%Left Commitment Key
\newcommand{\lck}{[{\llck}]_1}
%Right Commitment Key
\newcommand{\rck}{[{\lrck}]_2}
\newcommand{\rcks}{[{\lrck}]_1}

%Commitement keys matrices

%Log Left Commitment Keys
\newcommand{\Llck}{\matr{G}}
%Log Right Commitment Keys
\newcommand{\Lrck}{\matr{H}}
%Left Commitment Keys
\newcommand{\Lck}{[\Llck]_1}
%Right Commitment Keys
\newcommand{\Rck}{[{\Lrck}]_2}
\newcommand{\Rcks}{[{\Lrck}]_1}

%For quadratic info \lck\rck^\top
\newcommand{\Lqmatr}{\matr{C}}
\newcommand{\Qmatr}{{[\Lqmatr]_1}}
\newcommand{\Qspace}{\mathcal{C}}

% c_\Delta
\newcommand{\lccom}{\vecb{c}_\Delta}
\newcommand{\ccom}{\hvecb{c}_\Delta}

\newcommand{\pke}{\mathsf{PKE}}
\newcommand{\kem}{\mathsf{KEM}}
\newcommand{\prf}{\mathsf{PRF}}
\newcommand{\ev}{\mathsf{F}}
\newcommand{\KEM}{\mathsf{KEM}}
\newcommand{\gen}{\mathsf{Gen}}
\newcommand{\enc}{\mathsf{Enc}}
\newcommand{\Enc}{\mathsf{Enc}}
\newcommand{\dec}{\mathsf{Dec}}
\newcommand{\Dec}{\mathsf{Dec}}
\newcommand{\pk}{\mathit{pk}}
\newcommand{\sk}{\mathit{sk}}
\newcommand{\cdh}{\ensuremath{\mathsf{CDH}}}
\newcommand{\ddh}{\ensuremath{\mathsf{DDH}}}
\newcommand{\sxdh}{\ensuremath{\mathsf{SXDH}}}
\newcommand{\mddh}{\ensuremath{\mathsf{MDDH}}}
\newcommand{\mcdh}{\ensuremath{\mathsf{MCDH}}}
\newcommand{\fmdh}{\ensuremath{\mathsf{KerMDH}}}
\newcommand{\bddh}{\ensuremath{\mathsf{BDDH}}}
\newcommand{\mat}[1]{\ensuremath{#1\mbox{-}\mathsf{Mat}}}
\newcommand{\pddh}[1]{\ensuremath{#1\mbox{-}\mathsf{PDDH}}}
\newcommand{\mlddh}[1]{\ensuremath{#1\mbox{-}\mathsf{MLDDH}}}
\newcommand{\eddh}[1]{\ensuremath{#1\mbox{-}\mathsf{EDDH}}}
\newcommand{\casc}[1]{\ensuremath{#1\mbox{-}\mathsf{Casc}}}
\newcommand{\scasc}[1]{\ensuremath{#1\mbox{-}\mathsf{SCasc}}}
\newcommand{\lin}[1]{\ensuremath{#1\mbox{-}\mathsf{Lin}}}
\newcommand{\rlin}[1]{\ensuremath{#1\mbox{-}\mathsf{RLin}}}
\newcommand{\re}{\mathsf{RE}_\G}
\newcommand{\kcirc}[1]{\ensuremath{#1\mbox{-}\mathsf{Circ}}}
\newcommand{\escQE}{\gamma}
\newcommand{\EscQE}{\Gamma}


% M \in Z^{\la \times \lb}, N\in Z^{\la \times \lc}, \Lambda\in Z^{\ld\times \lb}
% x \in \Z_q^\la, b\in \Z_q^\lb, w\in\Z_q^\lc, \alpha\in \Z_q^\ld
\newcommand{\la}{{\ell_1}}
\newcommand{\lb}{{m}}
\newcommand{\lc}{{\ell_2}}
\newcommand{\ld}{{\ell_3}}


\newcommand{\LangMN}{{\Lang_{[\matr{M}]_1,[\matr{N}]_1,\matr{\Lambda},\grkb{\alpha}}}}



\newcommand{\skermdh}{\ensuremath{\mathsf{SKerMDH}}}
\newcommand{\kermdh}{\ensuremath{\mathsf{KerMDH}}}

\newcommand{\KG}{\mathsf{KeyGen}}
\newcommand{\GS}{{\mathsf{GS}}}

%HPS definitions
\newcommand{\univo}{universal$_1$\xspace}
\newcommand{\univt}{universal$_2$\xspace}
\newcommand{\distance}[2]{\Delta\left[#1 \,,\, #2\right]}
\newcommand{\entropic}{entropic\xspace}
\newcommand{\ciphertext}{{c}}
\def\params{\mathit{params}}
\newcommand{\structure}{\mathcal{S}}
\newcommand{\ciphersp}{\mathcal{C}}
\newcommand{\conssp}{\mathcal{V}}
\newcommand{\primeorder}{p}
\newcommand{\keysp}{\mathcal{K}}
\def\hps{\varfont{hps}}
\newcommand{\advBhps}{\calB}
\newcommand{\advBhpso}{\calB_{1}}
\newcommand{\advBhpst}{\calB_{2}}
\newcommand{\PK}{\mathcal{PK}}
\newcommand{\SK}{\mathcal{SK}}
\newcommand{\hash}{\mu}
\newcommand{\bigiota}{\mathcal{I}}
\newcommand{\var}[1]{{\mathsf{#1}}}
\newcommand{\vvar}[1]{{\mathbf{\var{#1}}}}
\newcommand{\varb}{\mathsf{b}}
\newcommand{\varx}{\mathsf{x}}
\newcommand{\vvarx}{\textbf{\textsf{x}}}
\newcommand{\vary}{\mathsf{y}}
\newcommand{\vvary}{\textbf{\textsf{y}}}
\newcommand{\varz}{\mathsf{z}}
\newcommand{\varX}{\mathsf{X}}
\newcommand{\varY}{\mathsf{Y}}
\newcommand{\varZ}{\mathsf{Z}}
\newcommand{\varvecx}{\vec{\varx}}
\newcommand{\varvecy}{\vec{\vary}}
\newcommand{\varvecz}{\vec{\varx}}
\newcommand{\hcx}{(\hat{x}_1,\ldots,\hat{x}_m)}
\newcommand{\matrB}{\matr{B}} 
\newcommand{\vecL}{(\hat{l}_1,\ldots,\hat{l}_n)}



% Las instancias de SPLHS se van a llamr \Phi
\newcommand{\SPLHSinst}{\Phi}
\newcommand{\SG}{\mathsf{SignGen}}
\newcommand{\SN}{\mathsf{Sign}}
\newcommand{\SD}{\mathsf{SignDerive}}
\newcommand{\SV}{\mathsf{Verify}}
\newcommand{\SP}{\ensuremath{\mathsf{SP}}}
\newcommand{\poly}{\mathsf{poly}}

%Comicmens a los eltos de la listas
\newcommand{\lcom}{\vecb{f}}
\newcommand{\Lcom}{\matr{F}}

%Definition
\newcommand{\MP}{\mathsf{MP}}
\newcommand{\GScom}{\mathsf{GS.Com_{\hvecb{U}}}}
\newcommand{\MPcomg}{\mathsf{MP.Com}_{\hmatr{G}}}
\newcommand{\MPcomh}{\mathsf{MP.Com}_{\hmatr{H}}}




% QA-NIZK for linear spaces
\newcommand{\ZKLinInst}{\mathsf{ZKLin}}
\newcommand{\ZKLinK}{\mathsf{ZKLin.K}_0}
\newcommand{\ZKLinKK}{\mathsf{ZKLin.K}_1}
\newcommand{\ZKLinCRS}{\mathsf{ZKLin.crs}}

\newcommand{\QANIZKsum}{{\spswsmas}}
\newcommand{\QANIZKcomms}{{\spswscomm}}
\newcommand{\QANIZKsym}{{\Psi_{\mathsf{sym}}}}

\newcommand{\nb}{{\overline{n}}}

\newcommand{\bb}{\overline{b}}
\newcommand{\bub}{{b(\overline{b}-1)}}
\newcommand{\tm}{\tilde{m}}

\newcommand{\rank}{\mathbf{rank}}
\newcommand{\HPSscheme}{\mathsf{HPS}}
%\newcommand{\THPSscheme}{\schemefont{HPS_t}}
% \newcommand{\THPSscheme}{\mathsf{HPS^{td}}}
% \newcommand{\HHPSscheme}{\mathsf{HPS}_2}
\newcommand{\HPSsys}{\mathsf{Param}}
\newcommand{\HPSpub}{\mathsf{Pub}}
\newcommand{\HPSpriv}{\mathsf{Priv}}
\newcommand{\HPSdec}{\mathsf{Decide}}
\newcommand{\eval}{\Lambda}
%\newcommand{\hpscu}{{\notionfont{cu}_2}}
%\newcommand{\ExpHPScu}[2]{\Exp^{\hpscu}_{#1,#2}}
%\newcommand{\AdvHPScu}[2]{\Adv^{\hpscu}_{#1,#2}}
%\newcommand{\hpscub}{{\notionfont{\hpscu\mbox{-}b}}}
%\newcommand{\hpscuz}{{\notionfont{\hpscu\mbox{-}0}}}
%\newcommand{\hpscuo}{{\notionfont{\hpscu\mbox{-}1}}}
%\newcommand{\ExpHPScub}[2]{\Exp^{\hpscub}_{#1,#2}}
%\newcommand{\ExpHPScuo}[2]{\Exp^{\hpscuo}_{#1,#2}}
%\newcommand{\ExpHPScuz}[2]{\Exp^{\hpscuz}_{#1,#2}}
% \newcommand{\pcol}{\delta}
\newcommand{\prcol}{\delta}
\newcommand{\trapdoor}{\omega}
\newcommand{\witness}{r}
\newcommand{\algD}{\mathsf{D}}
\newcommand{\algK}{\mathsf{K}}
\newcommand{\algG}{\mathsf{G}}
\newcommand{\algP}{\mathsf{P}}
\newcommand{\algV}{\mathsf{V}}
\newcommand{\algS}{\mathsf{S}}
\newcommand{\algF}{\mathsf{F}}
\newcommand{\algVrfy}{\mathsf{Vrfy}}

\newcommand{\R}{\mathcal{R}}
\newcommand{\M}{\mathcal{M}}
\newcommand{\dist}{\mathcal{D}}
\newcommand{\distw}{\mathcal{W}}
\newcommand{\distk}{\mathcal{K}}
\newcommand{\distlin}{\mathcal{L}}
\newcommand{\distrlin}{\mathcal{RL}}
\newcommand{\distc}{\mathcal{C}}
\newcommand{\distsc}{\mathcal{SC}}
\newcommand{\distcirc}{\mathcal{CI}}
\newcommand{\distu}{\mathcal{U}}
\newcommand{\distp}{\mathcal{P}}
\newcommand{\distink}{\dist_k^{m,i}}
\newcommand{\distjnk}{\dist_1^{m,i}}

\newcommand{\distinmk}{\dist_k^{mn,i}}
\newcommand{\distzeronmk}{\dist_k^{mn,0}}
\newcommand{\distzeronk}{\dist_k^{m,0}}
\newcommand{\distlininone}{\distlin_1^{m,i}}
\newcommand{\distlinisnone}{\distlin_1^{m,i^*}}
\newcommand{\distlinizeroone}{\distlin_1^{m,0}}
\newcommand{\distlinjsnzero}{\distlin_1^{n,j^*}}
\newcommand{\block}[1]{
  \underbrace{\begin{matrix}1 & \cdots & 1\end{matrix}}_{#1}
}
%\newcommand{\gets}{\leftarrow}
\newcommand{\Z}{\mathbb{Z}}
\newcommand{\N}{\mathbb{N}}
\newcommand{\G}{\mathsf{Gen}}
\newcommand{\advD}{\mathsf{D}}
\newcommand{\advA}{\mathsf{A}}
\newcommand{\advB}{\mathsf{B}}
\newcommand{\adv}{\mathbf{Adv}}
\newcommand{\group}{{gk}}
\newcommand{\gk}{\group}
\newcommand{\pgroup}{\mathcal{PG}}
\newcommand{\mgroup}[1]{\mathcal{MG}_{#1}}
\newcommand{\ggen}{\mathsf{Gen}}
\newcommand{\pggen}{\mathsf{PGen}}
\newcommand{\mggen}[1]{\mathsf{MGen}_{#1}}
\newcommand{\heading}[1]{\smallskip\noindent{\sc{#1}}}
\newcommand{\vecb}[1]{\mathbf{#1}}
\newcommand{\vecbt}[1]{\vec{#1}^{\ \top}}
\newcommand{\uvecb}[1]{{\vect({\vecb{#1}})}}
\newcommand{\tvecb}[1]{{\tilde{\vecb{#1}}}}
\newcommand{\tgrkb}[1]{\tilde{\grkb{#1}}}
\newcommand{\ovecb}[1]{{\overline{\vecb{#1}}}}
\newcommand{\Pt}{\mathcal{P}}
\newcommand{\Opt}{\mathcal{O}}
\newcommand{\pt}[1]{\mathcal{#1}}
\newcommand{\stbl}{\ \tilde \bullet \ }
\newcommand{\bilgroup}{\mathcal{PG}}
\newcommand{\matr}[1]{\mathbf{{#1}}}
\newcommand{\vecw}{\vecb{w}}
\newcommand{\vecr}{\vecb{r}}
\newcommand{\vecz}{\vecb{z}}
\newcommand{\vecy}{\vecb{y}}
\newcommand{\vecx}{\vecb{x}}
\newcommand{\veca}{\vecb{a}}
\newcommand{\matrA}{\matr{A}}
\newcommand{\hmatrA}{\hmatr{A}}
\newcommand{\cmatrA}{\cmatr{A}}
\newcommand{\smallpmatrix}[1]{\left(\begin{smallmatrix}#1\end{smallmatrix}\right)}
\newcommand{\pmatri}[1]{\left(\begin{matrix}#1\end{matrix}\right)}
\newcommand{\bmatri}[1]{\left[\begin{matrix}#1\end{matrix}\right]}
\newcommand{\matri}[1]{{\begin{matrix}#1\end{matrix}}}
\newcommand{\smatri}[1]{{\begin{smallmatrix}#1\end{smallmatrix}}}
\newcommand{\sfsplit}{\mathsf{spl}}
\newcommand{\bulletsp}{ \bullet }
\newcommand{\newf}{\widehat{f}}
\newcommand{\newF}{\widehat{F}}
\newcommand{\com}{\mathsf{com}}
\newcommand{\eq}{\mathsf{eq}}
\newcommand{\eqd}{\equiv}
\newcommand{\negl}{\mathsf{negl}}
\newcommand{\A}{\mathcal{A}}
\newcommand{\GG}{\mathbb{G}}
\newcommand{\Gr}{\ensuremath{\mathbb{G}_1}}
\newcommand{\Hr}{\ensuremath{\mathbb{G}_2}}
\newcommand{\T}{\ensuremath{\mathbb{T}}}
\newcommand{\SSDP}{\ensuremath{\mathsf{SSDP}}}
\newcommand{\PermP}{\ensuremath{\mathsf{PermP}}}
\newcommand{\PP}{\ensuremath{\mathsf{PP^*}}}
\newcommand{\bmatr}[1]{\left[\matr{#1}\right]}
\newcommand{\hmatr}[1]{{\hat{\matr{#1}}}}
\newcommand{\cmatr}[1]{\check{\matr{#1}}}
\newcommand{\bvecb}[1]{\left[\vecb{#1}\right]}
\newcommand{\hvecb}[1]{{\hat{\vecb{#1}}}}
\newcommand{\cvecb}[1]{\check{\vecb{#1}}}
\newcommand{\bits}{\{0,1\}}
\newcommand{\rmIm}{\mathbf{Im}}
\newcommand{\sfGame}{\mathsf{Game}}
\newcommand{\sfReal}{\mathsf{Real}}
\newcommand{\grkb}[1]{{\boldsymbol #1}}
\newcommand{\ugrkb}[1]{{\underline{\grkb{#1}}}}
\newcommand{\hgrkb}[1]{\hat{\grkb{#1}}}
\newcommand{\cgrkb}[1]{\check{\grkb{#1}}}
\newcommand{\SDP}{\ensuremath{\mathsf{SDP}}}
\newcommand{\Span}{\mathbf{Span}}
\newcommand{\Group}{G}
\newcommand{\Forger}{\mathsf{F}}
\newcommand{\advSound}{\mathsf{P}^*}
\newcommand{\Lang}{\mathcal{L}}
\newcommand{\crs}{\mathsf{crs}}
\newcommand{\sfproof}{\mathsf{proof}}
\newcommand{\sfbits}{\mathsf{bits}}
\newcommand{\sfbitsn}{{\mathsf{bits},n}}
\newcommand{\sflin}{\mathsf{lin}}
\newcommand{\sfcom}{\mathsf{com}}
\newcommand{\sfbin}{\mathsf{bin}}
\newcommand{\sfset}{\mathsf{set}}
\newcommand{\sfsum}{\mathsf{sum}}
\newcommand{\rp}{{\mathsf{range}\mbox{-}\mathsf{proof}}}
\newcommand{\ovG}{\overline{\matr{G}}}
\newcommand{\ovc}{\overline{\vecb{c}}}
\newcommand{\ovb}{\overline{\vecb{b}}}

\newcommand{\dmatrix}[1]{\begin{pamtrix}#1 & \cdots & \vecb{0}\\\vdots & \ddots & \vdots\\\vecb{0}& \ldots & #1\end{pmatrix}}
\newcommand{\sdmatrix}[1]{\smallpmatrix{#1 & \cdots & \vecb{0}\\\vdots & \ddots & \vdots\\\vecb{0}& \ldots & #1}}


%weas q hay que hacer pa q no webee el latex
\newsavebox{\smlmat}% Box to store smallmatrix content
\newsavebox{\smat}
\savebox{\smlmat}{$\left(\begin{smallmatrix}
\matr{G}_1 & \ldots & \vecb{0}   & \vecb{g}_{n+1} & \ldots & \vecb{0}\\
\vdots     & \ddots & \vdots     & \vdots         & \ddots & \vdots\\
\vecb{0}   & \ldots & \matr{G}_1 & \vecb{0}       & \ldots & \vecb{g}_{n+1}
\end{smallmatrix}\right)$}

\savebox{\smat}{$\left(\begin{matrix}
s_1 & \ldots & s_n\\
0   & \ldots & 0
\end{matrix}\right)$}






\newcommand{\sG}{|\GG_1|}
\newcommand{\sH}{|\GG_2|}
\newcommand{\s}{(\sG+\sH)}

\newcommand{\vu}{\hat{\vecb{u}}}
%\newcommand{\vv}{\check{\vecb{v}}}
\newcommand{\vc}{\hat{\vecb{c}}}
\newcommand{\vd}{\check{\vecb{d}}}
\newcommand{\zip}{\mathbf{zip}}

\newcommand{\indexSet}[2]{\mathcal{I}_{#1,#2}}
\newcommand{\SignaturesSet}{\mathcal{S}}

\newcommand{\bit}{\mathsf{bit}}
\newcommand{\sfts}{\mathsf{ts}}

%El-Gamal keys
\newcommand{\egpk}{\hat{x}}
\newcommand{\egsk}{x}
\newcommand{\egvpk}{\hvecb{k}}
\newcommand{\egvsk}{\vecb{k}}

%The set of permutation matrices
\newcommand{\matrPerms}{\mathcal{S}}
%The set of permutations
\newcommand{\Perms}{S}

\newcommand{\duda}[1]{{\iffalse\color{red}#1\fi}}

\newcommand{\cambio}[2]{{\iffalse\color{blue}Ahora: \fi#1}{\iffalse\color{red}(Antes: #2)\fi}}

\newenvironment{code}
   {\begin{tabbing}
   \hspace{4mm} \= \hspace{4mm} \= \hspace{4mm} \= \hspace{4mm} \= \hspace{4mm} \= \kill \\
   }
   {\end{tabbing}}

%\newcommand{\authnote}[2]{\medskip \noindent {\bf #1 says:} #2}
\newcommand{\authnote}[2]{\medskip \noindent {\bf #1 says:} {\textcolor{blue}{#2}}}

%\newtheorem{corollary}{Corollary}

\newtheorem{fact}{Fact}
\newtheorem{observation}{Observation}

\newcommand{\Am}{A}
\newcommand{\vX}{\ensuremath{\hat{\vecb{x}}}}
\newcommand{\VX}{\ensuremath{\hat{\vecb{v}}}}
\newcommand{\WX}{\ensuremath{\hat{\vecb{w}}}}
\newcommand{\VY}{\ensuremath{\check{\vecb{v}}}}
\newcommand{\WY}{\ensuremath{\check{\vecb{w}}}}
\newcommand{\vy}{\ensuremath{\vecb{y}}}
\newcommand{\vY}{\ensuremath{\check{\vecb{y}}}}
\newcommand{\vx}{\ensuremath{\vecb{x}}}


\newcommand{\U}{\ensuremath{\vecb{u}}}
\newcommand{\V}{\ensuremath{\vecb{v}}}
\newcommand{\vr}{\ensuremath{\vecb{r}}}
\newcommand{\vs}{\ensuremath{\vecb{s}}}
\newcommand{\vt}{\ensuremath{\vecb{t}}}


\newcommand{\ux}{\ensuremath{\hat{\vecb{u}}}}
\newcommand{\uy}{\ensuremath{\check{\vecb{u}}}}

\newcommand{\minitbl}[2]{\begin{tabular}{l}{#1}\\{#2}\end{tabular}}

\newcommand{\ef}{\iffalse}

\makeatletter
\newcommand*{\inlineequation}[2][]{%
  \begingroup
    % Put \refstepcounter at the beginning, because
    % package `hyperref' sets the anchor here.
    \refstepcounter{equation}%
    \ifx\\#1\\%
    \else
      \label{#1}%
    \fi
    % prevent line breaks inside equation
    \relpenalty=10000 %
    \binoppenalty=10000 %
    \ensuremath{%
      % \displaystyle % larger fractions, ...
      #2%
    }%
    ~\@eqnnum
  \endgroup
}
\makeatother

\makeatletter
\renewcommand*{\@opargbegintheorem}[3]{\trivlist
  \item[\hskip \labelsep{\bfseries #1\ #2}] \textbf{(#3)}\ \itshape}
\makeatother



%% Make document internal hyperlinks wherever possible. (TOC, references)
%% This MUST be loaded after varioref, which is loaded in 'extrapackages'
%% above.  We just load it last to be safe.
\usepackage[linkcolor=black,colorlinks=true,citecolor=black,filecolor=black]{hyperref}


%% Document information
%% ====================

\title{Efficient Zero-Knowledge Proofs}
\author{Alonso Gonz\'alez Ulloa}
\thesistype{PhD. Thesis}
\advisors{Advisors: Alejandro Hevia Angulo, Carla R\`afols Salvador}
\department{Departmento de Ciencias de la Computaci\'on}
\date{January 19, 5038}

\begin{document}

\frontmatter

%% Title page is autogenerated from document information above.  DO
%% NOT CHANGE.
\begin{titlingpage}
  \calccentering{\unitlength}
  \begin{adjustwidth*}{\unitlength-24pt}{-\unitlength-24pt}
    \maketitle
  \end{adjustwidth*}
\end{titlingpage}

%% The abstract of your thesis.  Edit the file as needed.
\begin{abstract}

Non-Interactive Zero-Knowledge (NIZK) proofs, are proofs that yield nothing beyond their validity. As opposed to the interactive variant, NIZK proofs consist of only one message and are more suited for high-latency scenarios and for building inherently non-interactive schemes, like signatures or encryption. 

With the advent of pairing-based cryptography many cryptosystems have been built using bilinear groups, that is, three abelian groups $\GG_1$, $\GG_2$, $\GG_T$ of order $q$ together with a bilinear function $e : \GG_1 \times \GG_2 \to \GG_T$. Statements related to pairing-based cryptographic schemes are naturally expressed as the satisfiability of equations over these groups and $\Z_q$.

The Groth-Sahai proof system, introduced by Groth and Sahai at Eurocrypt 2008, provides NIZK proofs for the satisfiability of equations over bilinear groups and over the integers modulo a prime $q$. Although Groth-Sahai proofs are quite efficient, they easily get expensive unless the statement is very simple. Specifically, proving satisfiability of $m$ equations in $n$ variables requires 
sending as commitments to the solutions $\Theta(n)$ elements of a bilinear group, and a proof that they satisfy the equations, which we simply call the proof, requiring additional $\Theta(m)$ group elements. 

In this thesis we study how to construct aggregated proofs -- i.e.~ proofs of size independent of the number of equations  -- for different types of equations and how to use them to build more efficient cryptographic schemes.

We show that linear equations admit aggregated proofs of size $\Theta(1)$. We then study the case of quadratic integer equations, more concretely the equation $b(b-1)=0$ which is the most useful type of quadratic integer equation, and construct an aggregated proof of size $\Theta(1)$. We use these results to build more efficient threshold Groth-Sahai proofs and more efficient ring signatures.

We also study a natural generalization of quadratic equations which we call set-membership proofs -- i.e.~show that a variable belongs to some set. We first construct an aggregated proof of size $\Theta(t)$, where $t$ is the set size, and of size $\Theta(\log t)$ if the set is of the form $[0,t-1]\subset\Z_q$.
Then, we further improve the size of our set-membership proofs and construct aggregated proofs of size $\Theta(\log t)$.
We note that some cryptographic schemes can be naturally constructed as set-membership proofs, specifically we study the case of proofs of correctness of a shuffle and range proofs. Starting from set-membership proofs as a common building block, we build the shortest proofs for both proof systems.
 
\end{abstract}


%% TOC with the proper setup, do not change.
\cleartorecto
\tableofcontents
\mainmatter

%% Your real content!
%% Some commands used in this file
\newcommand{\package}{\emph}

\chapter{Introduction}

This is version \verb-v1.4- of the template.

We assume that you found this template on our institute's website, so
we do not repeat everything stated there.  Consult the website again
for pointers to further reading about \LaTeX{}.  This chapter only
gives a brief overview of the files you are looking at.

\section{Features}
\label{sec:features}

The rest of this document shows off a few features of the template
files.  Look at the source code to see which macros we used!

The template is divided into \TeX{} files as follows:
\begin{enumerate}
\item \texttt{thesis.tex} is the main file.
\item \texttt{extrapackages.tex} holds extra package includes.
\item \texttt{layoutsetup.tex} defines the style used in this document.
\item \texttt{theoremsetup.tex} declares the theorem-like environments.
\item \texttt{macrosetup.tex} defines extra macros that you may find
  useful.
\item \texttt{introduction.tex} contains this text.
\item \texttt{sections.tex} is a quick demo of each sectioning level
  available.
\item \texttt{refs.bib} is an example bibliography file.  You can use
  Bib\TeX{} to quote references.  For example, read
  \cite{bringhurst1996ets} if you can get a hold of it.
\end{enumerate}


\subsection{Extra package includes}

The file \texttt{extrapackages.tex} lists some packages that usually
come in handy.  Simply have a look at the source code.  We have
added the following comments based on our experiences:
\begin{description}
\item[REC] This package is recommended.
\item[OPT] This package is optional.  It usually solves a specific
  problem in a clever way.
\item[ADV] This package is for the advanced user, but solves a problem
  frequent enough that we mention it. Consult the package's
  documentation.
\end{description}

As a small example, here is a reference to the Section \emph{Features}
typeset with the recommended \package{varioref} package:
\begin{quote}
  See Section~\vref{sec:features}.
\end{quote}


\subsection{Layout setup}

This defines the overall look of the document -- for example, it
changes the chapter and section heading appearance.  We consider this
a `do not touch' area.  Take a look at the excellent \emph{Memoir}
documentation before changing it.

In fact, take a look at the excellent \emph{Memoir} documentation,
full stop.


\subsection{Theorem setup}

This file defines a bunch of theorem-like environments.

\begin{theorem}
  An example theorem.
\end{theorem}

\begin{proof}
  Proof text goes here.
\end{proof}

Note that the q.e.d.\ symbol moves to the correct place automatically
if you end the proof with an \texttt{enumerate} or
\texttt{displaymath}.  You do not need to use \verb-\qedhere- as with
\package{amsthm}.

\begin{theorem}[Some Famous Guy]
  Another example theorem.
\end{theorem}

\begin{proof}
  This proof
  \begin{enumerate}
  \item ends in an enumerate.
  \end{enumerate}
\end{proof}

\begin{proposition}
  Note that all theorem-like environments are by default numbered on
  the same counter.
\end{proposition}

\begin{proof}
  This proof ends in a display like so:
  \begin{displaymath}
    f(x) = x^2.
  \end{displaymath}
\end{proof}


\subsection{Macro setup}

For now the macro setup only shows how to define some basic macros,
and how to use a neat feature of the \package{mathtools} package:
\begin{displaymath}
  \abs{a}, \quad \abs*{\frac{a}{b}}, \quad \abs[\big]{\frac{a}{b}}.
\end{displaymath}

%\chapter{Writing scientific texts in English}

This chapter was originally a separate document written by Reto
Spöhel.  It is reprinted here so that the template can serve as a
quick guide to thesis writing, and to provide some more example
material to give you a feeling for good typesetting.

% We're going to need an extra theorem-like environment for this
% chapter
\theoremstyle{plain}
\theoremsymbol{}
\newtheorem{Rule}[theorem]{Rule}

\section{Basic writing rules}

The following rules need little further explanation; they are best
understood by looking at the example in the booklet by Knuth et al.,
§2--§3.

\begin{Rule}
  Write texts, not chains of formulas.
\end{Rule}

More specifically, write full sentences that are logically
interconnected by phrases like `Therefore', `However', `On the other
hand', etc.\ where appropriate.

\begin{Rule}
  Displayed formulas should be embedded in your text and punctuated
  with it.
\end{Rule}

In other words, your writing should not be divided into `text parts'
and `formula parts'; instead the formulas should be tied together by
your prose such that there is a natural flow to your writing.

\section{Being nice to the reader}

Try to write your text in such a way that a reader enjoys reading
it. That's of course a lofty goal, but nevertheless one you should
aspire to!

\begin{Rule}
  Be nice to the reader.
\end{Rule}

Give some intuition or easy example for definitions and theorems which
might be hard to digest. Remind the reader of notations you introduced
many pages ago -- chances are he has forgotten them. Illustrate your
writing with diagrams and pictures where this helps the reader. Etc.

\begin{Rule}
  Organize your writing.
\end{Rule}

Think carefully about how you subdivide your thesis into chapters,
sections, and possibly subsections.  Give overviews at the beginning
of your thesis and of each chapter, so the reader knows what to
expect. In proofs, outline the main ideas before going into technical
details. Give the reader the opportunity to `catch up with you' by
summing up your findings periodically.

\emph{Useful phrases:} `So far we have shown that \ldots', `It remains
to show that \ldots', `Recall that we want to prove inequality (7), as
this will allow us to deduce that \ldots', `Thus we can conclude that
\ldots. Next, we would like to find out whether \ldots', etc.

\begin{Rule}
  Don't say the same thing twice without telling the reader that you
  are saying it twice.
\end{Rule}

Repetition of key ideas is important and helpful. However, if you
present the same idea, definition or observation twice (in the same or
different words) without telling the reader, he will be looking for
something new where there is nothing new.

\emph{Useful phrases:} `Recall that [we have seen in Chapter 5 that]
\ldots', `As argued before / in the proof of Lemma 3, \ldots', `As
mentioned in the introduction, \ldots', `In other words, \ldots', etc.

\begin{Rule}
  Don't make statements that you will justify later without telling
  the reader that you will justify them later.
\end{Rule}

This rule also applies when the justification is coming right in the
next sentence!  The reasoning should be clear: if you violate it, the
reader will lose valuable time trying to figure out on his own what
you were going to explain to him anyway.

\emph{Useful phrases:} `Next we argue that \ldots', `As we shall see,
\ldots', `We will see in the next section that \ldots, etc.


\section{A few important grammar rules}

\begin{Rule}
  \label{rule:no-comma-before-that}
  There is (almost) \emph{never} a comma before `that'.
\end{Rule}

It's really that simple. Examples:
\begin{quote}
  We assume that \ldots\\
  \emph{Wir nehmen an, dass \ldots}

  It follows that \ldots\\
  \emph{Daraus folgt, dass \ldots}

  `thrice' is a word that is seldom used.\\
  \emph{`thrice' ist ein Wort, das selten verwendet wird.}
\end{quote}
Exceptions to this rule are rare and usually pretty obvious. For
example, you may end up with a comma before `that' because `i.e.' is
spelled out as `that is':
\begin{quote}
  For \(p(n)=\log n/n\) we have \ldots{} However, if we choose \(p\) a
  little bit higher, that is \(p(n)=(1+\varepsilon)\log n/n\) for some
  \(\varepsilon>0\), we obtain that\ldots
\end{quote}
Or you may get a comma before `that' because there is some additional
information inserted in the middle of your sentence:
\begin{quote}
  Thus we found a number, namely \(n_0\), that satisfies equation (13).
\end{quote}
If the additional information is left out, the sentence has no comma:
\begin{quote}
  Thus we found a number that satisfies equation (13).
\end{quote}
(For `that' as a relative pronoun, see also
Rules~\ref{rule:non-defining-has-comma}
and~\ref{rule:defining-without-comma} below.)

\begin{Rule}
  There is usually no comma before `if'.
\end{Rule}

Example:
\begin{quote}
  A graph is not \(3\)-colorable if it contains a \(4\)-clique.\\
  \emph{Ein Graph ist nicht \(3\)-färbbar, wenn er eine \(4\)-Clique
    enthält.}
\end{quote}
However, if the `if' clause comes first, it is usually separated from
the main clause by a comma:
\begin{quote}
  If a graph contains a \(4\)-clique, it is not \(3\)-colorable .\\
  \emph{Wenn ein Graph eine \(4\)-Clique enthält, ist er nicht
    \(3\)-färbbar.}
\end{quote}

There are more exceptions to these rules than to
Rule~\ref{rule:no-comma-before-that}, which is why we are not
discussing them here. Just keep in mind: don't put a comma before `if'
without good reason.

\begin{Rule}
  \label{rule:non-defining-has-comma}
  Non-defining relative clauses have commas.
\end{Rule}
\begin{Rule}
  \label{rule:defining-without-comma}
  Defining relative clauses have no commas.
\end{Rule}

In English, it is very important to distinguish between two types of
relative clauses: defining and non-defining ones. This is a
distinction you absolutely need to understand to write scientific
texts, because mistakes in this area actually distort the meaning of
your text!

It's probably easier to explain first what a \emph{non-defining}
relative clause is. A non-defining relative clauses simply gives
additional information \emph{that could also be left out} (or given in
a separate sentence). For example, the sentence
\begin{quote}
  The \textsc{WeirdSort} algorithm, which was found by the famous
  mathematician John Doe, is theoretically best possible but difficult
  to implement in practice.
\end{quote}
would be fully understandable if the relative clause were left out
completely. It could also be rephrased as two separate sentences:
\begin{quote}
  The \textsc{WeirdSort} algorithm is theoretically best possible but
  difficult to implement in practice. [By the way,] \textsc{WeirdSort}
  was found by the famous mathematician John Doe.
\end{quote}
This is what a non-defining relative clause is. \emph{Non-defining
  relative clauses are always written with commas.} As a corollary we
obtain that you cannot use `that' in non-defining relative clauses
(see Rule~\ref{rule:no-comma-before-that}!). It would be wrong to
write
\begin{quote}
  \st{The \textsc{WeirdSort} algorithm, that was found by the famous
    mathematician John Doe, is theoretically best possible but
    difficult to implement in practice.}
\end{quote}
A special case that warrants its own example is when `which' is
referring to the entire preceding sentence:
\begin{quote}
  Thus inequality (7) is true, which implies that the Riemann
  hypothesis holds.
\end{quote}
As before, this is a non-defining relative sentence (it could be left
out) and therefore needs a comma.

So let's discuss \emph{defining} relative clauses next. A defining
relative clause tells the reader \emph{which specific item the main
  clause is talking about}. Leaving it out either changes the meaning
of the sentence or renders it incomprehensible altogether.  Consider
the following example:

\begin{quote}
  The \textsc{WeirdSort} algorithm is difficult to implement in
  practice. In contrast, the algorithm that we suggest is very simple.
\end{quote}

Here the relative clause `that we suggest' cannot be left out -- the
remaining sentence would make no sense since the reader would not know
which algorithm it is talking about. This is what a defining relative
clause is. \textit{Defining relative clauses are never written with
  commas.} Usually, you can use both `that' and `which' in defining
relative clauses, although in many cases `that' sounds better.

As a final example, consider the following sentence:
\begin{quote}
  For the elements in \(\mathcal{B}\) which satisfy property (A), we
  know that equation (37) holds.
\end{quote}
This sentence does not make a statement about all elements in
\(\mathcal{B}\), only about those satisfying property (A). The relative
clause is \emph{defining}. (Thus we could also use `that' in place of
`which'.)

In contrast, if we add a comma the sentence reads
\begin{quote}
  For the elements in \(\mathcal{B}\), which satisfy property (A), we
  know that equation (37) holds.
\end{quote}

Now the relative clause is \emph{non-defining} -- it just mentions in
passing that all elements in \(\mathcal{B}\) satisfy property (A). The
main clause states that equation (37) holds for \emph{all} elements in
\(\mathcal{B}\). See the difference?


\section[Things you (usually) don't say in English]%
{Things you (usually) don't say in English -- and what to say
  instead}
\label{sec:list}

Table~\ref{tab:things-you-dont-say} lists some common mistakes and
alternatives.  The entries should not be taken as gospel -- they don't
necessarily mean that a given word or formulation is wrong under all
circumstances (obviously, this depends a lot on the context). However,
in nine out of ten instances the suggested alternative is the better
word to use.

\begin{table}
  \centering
  \caption{Things you (usually) don't say}
  \label{tab:things-you-dont-say}
  \begin{tabular}{lll}
    \toprule
    \st{It holds (that) \dots} & We have \dots & \emph{Es gilt \dots}\\
    \multicolumn{3}{l}{\quad\footnotesize(`Equation (5) holds.' is fine, though.)}\\
    \st{$x$ fulfills property $\mathcal{P}$.}& \(x\) satisfies property \(\mathcal{P}\). & \emph{\(x\) erfüllt Eigenschaft \(\mathcal{P}\).} \\
    \st{in average} & on average & \emph{im Durchschnitt}\\
    \st{estimation} & estimate   & \emph{Abschätzung}\\
    \st{composed number} & composite number & \emph{zusammengesetzte Zahl}\\
    \st{with the help of} & using & \emph{mit Hilfe von}\\
    \st{surely} & clearly & \emph{sicher, bestimmt}\\
    \st{monotonously increasing} & monotonically incr. & \emph{monoton steigend}\\
    \multicolumn{3}{l}{\quad\footnotesize(Actually, in most cases `increasing' is just fine.)}\\
    \bottomrule
  \end{tabular}
\end{table}

%%% Local Variables:
%%% mode: latex
%%% TeX-master: "thesis"
%%% End:

%\chapter{Typography}


\section{Punctuation}

\begin{Rule}
  Use opening (`) and closing (') quotation marks correctly.
\end{Rule}

In \LaTeX, the closing quotation mark is typed like a normal
apostrophe, while the opening quotation mark is typed using the French
\emph{accent grave} on your keyboard (the \emph{accent grave} is the
one going down, as in \emph{frère}).

Note that any punctuation that \emph{semantically} follows quoted
speech goes inside the quotes in American English, but outside in
Britain.  Also, Americans use double quotes first.  Oppose
\begin{quote}
  ``Using `lasers,' we punch a hole in \ldots\ the Ozone Layer,''
  Dr.\ Evil said.
\end{quote}
to
\begin{quote}
  `Using ``lasers'', we punch a hole in \ldots\ the Ozone Layer',
  Dr.\ Evil said.
\end{quote}

\begin{Rule}
  Use hyphens (-), en-dashes (--) and em-dashes (---) correctly.
\end{Rule}

A hyphen is only used in words like `well-known', `$3$-colorable'
etc., or to separate words that continue in the next line (which is
known as hyphenation).  It is entered as a single ASCII hyphen
character (\texttt{-}).

To denote ranges of numbers, chapters, etc., use an en-dash (entered
as two ASCII hyphens \texttt{--}) with no spaces on either side.  For
example, using Equations (1)--(3), we see\ldots

As the equivalent of the German \emph{Gedankenstrich}, use an en-dash
with spaces on both sides -- in the title of Section \ref{sec:list},
it would be wrong to use a hyphen instead of the dash. (Some English
authors use the even longer emdash (---) instead, which is typed as
three subsequent hyphens in \LaTeX. This emdash is used without spaces
around it---like so.)


\section{Spacing}

\begin{Rule}
  \label{rule:no-manual-spacing}
  Do not add spacing manually.
\end{Rule}

You should never use the commands \lstinline-\\- (except within
tabulars and arrays), \lstinline[showspaces=true]-\ - (except to
prevent a sentence-ending space after Dr.\ and such),
\lstinline-\vspace-, \lstinline-\hspace-, etc.  The choices programmed
into \LaTeX{} and this style should cover almost all cases.  Doing it
manually quickly leads to inconsistent spacing, which looks terrible.
Note that this list of commands is by no means conclusive.

\begin{Rule}
  Judiciously insert spacing in maths where it helps.
\end{Rule}

This directly contradicts Rule~\ref{rule:no-manual-spacing}, but in
some cases \TeX{} fails to correctly decide how much spacing is
required.  For example, consider
\begin{displaymath}
  f(a,b) = f(a+b, a-b).
\end{displaymath}
In such cases, inserting a thin math space \lstinline-\,- greatly
increases readability:
\begin{displaymath}
  f(a,b) = f(a+b,\, a-b).
\end{displaymath}

Along similar lines, there are variations of some symbols with
different spacing.  For example, Lagrange's Theorem states that
\(\abs{G}=[G:H]\abs{H}\), but the proof uses a bijection \(f\colon
aH\to bH\).  (Note how the first colon is symmetrically spaced, but
the second is not.)

\begin{Rule}
  Learn when to use \lstinline[showspaces=true]-\ - and
  \lstinline-\@-.
\end{Rule}

Unless you use `french spacing', the space at the end of a sentence is
slightly larger than the normal interword space.

The rule used by \TeX{} is that any space following a period,
exclamation mark or question mark is sentence-ending, except for
periods preceded by an upper-case letter.  Inserting \lstinline-\-
before a space turns it into an interword space, and inserting
\lstinline-\@- before a period makes it sentence-ending.  This means
you should write
\begin{lstlisting}
Prof.\ Dr.\ A. Steger is a member of CADMO\@.
If you want to write a thesis with her, you
should use this template.
\end{lstlisting}
which turns into
\begin{quote}
  Prof.\ Dr.\ A. Steger is a member of CADMO\@.  If you want to write
  a thesis with her, you should use this template.
\end{quote}
The effect becomes more dramatic in lines that are stretched slightly
during justification:
\begin{quote}
  \parbox{\linewidth}{\hbox to \linewidth{%
      Prof.\ Dr.\ A. Steger is a member of CADMO\@.  If you}}
\end{quote}

\begin{Rule}
  Place a non-breaking space (\lstinline-~-) right before references.
\end{Rule}

This is actually a slight simplification of the real rule, which
should invoke common sense.  Place non-breaking spaces where a line
break would look `funny' because it occurs right in the middle of a
construction, especially between a reference type (Chapter) and its
number.


\section{Choice of `fonts'}

Professional typography distinguishes many font attributes, such as
family, size, shape, and weight.  The choice for sectional divisions
and layout elements has been made, but you will still occasionally
want to switch to something else to get the reader's attention.  The
most important rule is very simple.

\begin{Rule}
  When emphasising a short bit of text, use \lstinline-\emph-.
\end{Rule}

In particular, \emph{never} use bold text (\lstinline-\textbf-).
Italics (or Roman type if used within italics) avoids distracting the
eye with the huge blobs of ink in the middle of the text that bold
text so quickly introduces.

Occasionally you will need more notation, for example, a consistent
typeface used to identify algorithms.

\begin{Rule}
  Vary one attribute at a time.
\end{Rule}

For example, for \textsc{WeirdSort} we only changed the shape to small
caps.  Changing two attributes, say, to bold small caps would be
excessive (\LaTeX{} does not even have this particular variation).
The same holds for mathematical notation: the reader can easily
distinguish \(g_n\), \(G(x)\), \(\mathcal{G}\) and \(\mathsf{G}\).

\begin{Rule}
  Never underline or uppercase.
\end{Rule}

No exceptions to this one, unless you are writing your thesis on a
typewriter.  Manually.  Uphill both ways.  In a blizzard.


\section{Displayed equations}

\begin{Rule}
  Insert paragraph breaks \emph{after} displays only where they
  belong.  Never insert paragraph breaks \emph{before} displays.
\end{Rule}

\LaTeX{} translates sequences of more than one linebreak (i.e., what
looks like an empty line in the source code) into a paragraph break in
almost all contexts.  This also happens before and after displays,
where extra spacing is inserted to give a visual indication of the
structure.  Adding a blank line in these places may look nice in the
sources, but compare the resulting display

\begin{displaymath}
  a = b
\end{displaymath}

to the following:
\begin{displaymath}
  a = b
\end{displaymath}
The first display is surrounded by blank lines, but the second is not.
It is bad style to start a paragraph with a display (you should always
tell the reader what the display means first), so the rule follows.

\begin{Rule}
  Never use \lstinline-eqnarray-.
\end{Rule}

It is at the root of most ill-spaced multiline displays.  The
\package{amsmath} package provides better alternatives, such as the
\lstinline-align- family
\begin{align*}
  f(x) &= \sin x, \\
  g(x) &= \cos x,
\end{align*}
and \lstinline-multline- which copes with excessively long equations:
\begin{multline*}
  \def\P{\mathrm P}
  \P\bigl[X_{t_0} \in (z_0, z_0+dz_0],\ldots, X_{t_n}\in(z_n,z_n+dz_n]\bigr]
  \\= \nu(dz_0) K_{t_1}(z_0,dz_1) K_{t_2-t_1}(z_1,dz_2)\cdots
  K_{t_n-t_{n-1}}(z_{n-1},dz_n).
\end{multline*}


\section{Floats}

By default this style provides floating environments for tables and
figures.  The general structure should be as follows:
\begin{lstlisting}
\begin{figure}
  \centering
  % content goes here
  \caption{A short caption}
  \label{some-short-label}
\end{figure}
\end{lstlisting}
Note that the label must follow the caption, otherwise the label will
refer to the surrounding section instead.  Also note that figures
should be captioned at the bottom, and tables at the top.

The whole point of floats is that they, well, \emph{float} to a place
where they fit without interrupting the text body.  This is a frequent
source of confusion and changes; please leave it as is.

\begin{Rule}
  Do not restrict float movement to only `here'
  \textnormal{(\lstinline-h-)}.
\end{Rule}

If you are still tempted, you should avoid the float altogether and
just show the figure or table inline, similar to a displayed equation.

%%% Local Variables:
%%% mode: latex
%%% TeX-master: "thesis"
%%% End:

\chapter{Introduction} \label{sec:intro}

    With the growth and ubiquity of Internet more and more of our life has been moving from the ``physical world'' to the ``digital world''. Along with these changes new problems have raised: we moved from a world where communication was mostly without any intermediary to a world where communication goes through an uncontrollable set of servers from which no privacy or secrecy guarantee can be obtained. Modern Cryptography has raised as an answer to these and many other related problems, providing \emph{provable} methods for securing information.  

Among the vast variety of cryptographic constructions, this thesis is concerned with the study of \emph{non-interactive  zero-knowledge proofs}.
A zero-knowledge proof, introduced by Goldwasser, Micali, and Rackoff \cite{GolMicRac89}, is a protocol between two parties, the \emph{prover} and the \emph{verifier}, where the prover wants to convince the verifier that some statement is true. At the onset of the protocol the verifier is completely convinced that the statement is true without learning any extra information. \emph{non-interactive  zero-knowledge proofs}, introduced by Blum, Feldman, and Micali \cite{STOC:BluFelMic88}, restrict the proof to consist of a single message (in opposition of an interactive protocol) making the protocol more suited for constructing inherenlty non-interactive primtives, such as encryption and signatures,  high-latency scenarios.
The importance of non-interactive zero-knowledge proofs in cryptography was recognized early \cite{STOC:NaoYun90,STOC:DolDwoNao91,CCS:BelRog93}, but for many years the existing constructions were either completely impractical or could only be realized under very strong assumptions like the random oracle model, via the Fiat-Shamir heuristic \cite{C:FiaSha86}. 

Ideally, a NIZK proof system should be both expressive and efficient, meaning that it should allow to prove
statements which are general enough to be useful in practice using a small amount of resources.
Furthermore, it should be constructed under
mild security assumptions.
As it is usually the case for most cryptographic primitives, there is a trade off between these three design goals.
For instance,
to prove very general statements, one can use the NIZK proof 
system for circuit satisfiability of Groth, Ostrovsky, and Sahai 
\cite{EC:GroOstSah06}, which is based on standard assumptions but 
whose proof size depends on the number of gates. 
Alternatively,
there exist constant-size proofs for any language in NP
  (e.g. \cite{EC:GGPR13}) but based on very strong and controversial assumptions, 
  namely knowledge-of-exponent type of assumptions 
  (which are non-falsifiable, according to Naor's classification 
  \cite{C:Naor03}) or the random oracle model. 
\footnote{There is evidence that the use of knowledge-of-exponent type of  assumptions 
may be unavoidable for constant-size NIZK proofs for NP-complete languages \cite{STOC:GenWic11}.}

Despite the use of non-falsifiable assumptions, the generality of NIZK proofs for NP-complete languages hides a subtlety. In order to prove the validity of a statement, it is necessary to express it as the satisfiability of a circuit. Apart from the cost of expressing the statement as a circuit (a Cook reduction), many cryptographic statements are more naturally expressed by other means.
In fact, with the advent of \emph{pairing-based cryptography} many cryptosystems have been built using \emph{bilinear groups}, that is, three abelian groups $\GG_1,\GG_2,\GG_T$ of order $q$ together with a bilinear function $e:\GG_1\times\GG_2\to\GG_T$. As consequence,
statements related to pairing-based cryptographic schemes are more naturally expressed as the satisfiability of equations over these groups and $\Z_q$.

The Groth-Sahai proof system (GS proofs) \cite{SIAMJC:GroSah12} 
  provides a proof system for satisfiability of this type of equations: \emph{pairing product equations}.
  This language suffices to capture almost all of the 
  statements which appear 
  in practice when designing cryptographic schemes over bilinear groups.  
Although GS proofs are quite efficient, proving satisfiability of $m$ equations in $n$ variables requires 
sending the solutions, using an appropriated encryption or commitment scheme, requiring $\Theta(n)$ group elements, and a proof that the encrypted values are indeed solutions, requiring $\Theta(m)$ group elements. Although linear in both $m$ and $n$, the constants are on the order of $\sim 10$. Consequently, a rough approximation of the average proof size would be $(m+n)$10*64 bytes = $(m+n)$640 bytes, which limits $m+n$ to be less than $1600$ whenever we want the proof to be less that $1$ Megabyte.\footnote{Using Barreto-Nahering curves with security parameter $\lambda=128$ the base group elements are of size $32$ and $64$ bytes.}  For this reason, several recent works 
 have focused on further improving the proof efficiency 
 (e.g.\ \cite{PKC:EscGro14,C:EHKRV13,TCC:Rafols15})

%The more general kind of equations are \emph{pairing Product equations} --equations where variables might appear ``multiplied'' between them using the bilinear function-- but \emph{quadratic equations over $\Z_q$} --similar to pairing product equations but variables are integers modulo $q$-- are also useful. A natural generalization of quadratic equations are \emph{high-Degree equations} --equations of the kind $p(x)=0$, where $p$ is some polynomial-- or equivalently \emph{set-Membership proofs}, that is a proof that some variable belongs to some set (the set of roots of the polynomial for example).\footnote{Note that set-membership proofs are still meaningful when the variable is a group element, while high-degree equations not (when the natural translation of the polynomial $xy$ over $\Z_q$ is $e(x,y)$).} 

A recent line of work 
  \cite{AC:JutRoy13,C:JutRoy14,EC:KilWee15,EC:LPJY14} 
has succeeded in constructing constant-size  
  arguments for very specific statements, namely, for membership in subspaces of $\Gr^{m}$, 
  where $\Gr$ is some group equipped with a bilinear map where the discrete logarithm is hard. 
The soundness of the schemes is based on standard, falsifiable assumptions 
  and the proof size is independent of both $m$ and the witness size.  These improvements are in a  \textit{quasi-adaptive} 
  model (QA-NIZK, \cite{AC:JutRoy13}).  This means that the common reference string of these proof systems is 
  specialized to the linear space where one wants to prove membership.
  
Interestingly, Jutla and Roy  \cite{C:JutRoy14} also showed that their techniques to construct 
  constant-size NIZK in linear spaces can be used to aggregate the GS proofs of $m$ equations in $n$ variables, that is, the proof --without considering the $n$ commitments to variables-- is of size $\Theta(1)$. However, aggregation is only possible if the equations are linear and the equation type is more limited when working with more efficient \emph{asymmetric} bilinear groups. 

The main objective of this thesis is to explore more efficient proofs for linear and other equations with special focus on asymmetric groups.
We put emphasis on constructing \emph{aggregated proofs}, that is, a single proof for many statements whose size is independent from the number of staments.
Specifically we consider:
\begin{itemize}
\item Linear and quadratic equations over $\Z_q$, where the variables are restricted to be integers modulo $q$.
\item Linear equations over $\GG_1$ and/or $\GG_2$ with the additional restriction that the constants are fixed -- one proof system for each set of equations -- and that they 
can be sampled together with their discrete logarithms.
\item Set membership proofs, where one shows that a variable is an element from some set $S$. This is a natural generalization of \emph{high-degree equations} of the form $p(x)=0$, where $p$ is a polynomial, that also allows the ``roots'' to be group elements.
\end{itemize}
The second objective is to use these more efficient proofs to develop new and more efficient cryptographic protocols.


    \section{Our Results}

        In this thesis we show that \textbf{all linear equations} and \textbf{all quadratic equations over $\Z_q$} admit an aggregated proof of size $\Theta(1)$. We also show that \textbf{all set-membership proofs} over $\Z_q$, with set size $\ell$, admit aggregated proofs of size $\Theta(\log \ell)$. We show that this results can be extended to linear equations over $\GG_1$ and/or $\GG_2$ and to set-membership proofs over $\GG_1$ or $\GG_2$ meeting some restrictions, and that all set-membership proofs over $\GG_1$ or $\GG_2$ admit aggregated proofs of size $\Theta(\ell)$ without any restriction. We use this results to improve the efficiency of several protocols. 

In the first part of this thesis we develop new techniques to aggregate
other types of linear equations, recovering all the aggregation results of \cite{C:JutRoy14} (in particular, two-sided linear equations) in asymmetric bilinear groups. The latter (Type III bilinear groups, according to the classification of \cite{DAM:GalPatSma08}) are the most 
attractive 
from the perspective of a performance and security trade off, specially since the recent attacks on discrete logarithms in finite fields by Joux \cite{SAC:Joux13} and subsequent improvements. Considerable research effort 
(e.g. \cite{C:AGOT14a,EC:Freeman10})
has been put into translating pairing-based cryptosystems from a setting with more structure in which design is simpler (e.g. composite-order or symmetric bilinear groups) to a more efficient setting (e.g. prime order or asymmetric bilinear groups). In this line, we aim not only at obtaining new results in the asymmetric setting but also to translate known results and develop new tools specifically designed for it which might be of independent interest.

The second part of this thesis is devoted to obtain efficient proofs for quadratic equations over the integers. We construct constant size proofs for the satisfiability of many equations of the form $b(b-1)=0$. While this is just a particular type of quadratic equation, it is the most representative type of quadratic equation and efficient proofs for other equations can be build using the same techniques. We then show how to apply our results to build more efficient signature schemes and more efficient proofs that \emph{1 out of many} equations are satisfied.

The last part of this thesis focuses on Set-Membership proofs: show that $x$ is in the set $S$. We show that many Set-Membership proofs can be proven with a proof of size $\Theta(|S|)$, when $S$ is a set of group elements, and $\Theta(\log |S|)$, when $S$ is a set of integers. We call this primitive \emph{Aggregated Zero-Knowledge Set-Membership proof}, because the proof size is independent of the number of variables. We show that aggregated Zero-Knowledge Set-membership proofs allows efficiency improvements for two non-interactive arguments, namely, range proofs and proofs of correctness of a shuffle. Apart from efficiency improvements, we obtain a unified modular construction for the to problems mentioned above, while the state of the art solutions are build from different techniques and assumptions.


    %\section{Organization}

     %   \inpute{sections/intro/organization}

\chapter{Preliminaries} \label{sec:prelim}

    \section{Notation} \label{sec:notation}

        Let $\advA$ a Polynomial Time Probabilistic Turing Machine (PPT). We denote by $x:=\advA(y)$ the assignment of $x$ to the output of $\advA$ when run on input $y$. Given a distribution $\dist$ we write $x\gets\dist$ when $x$ is sampled following distribution $\dist$, and given a set $S$ we denote $x\gets S$ when $x$ is sampled uniformly from the set $S$. We denote by $x\gets\advA(y)$ the assignment of $x$ to the output of $\advA$ when run on input $y$ and random coins $r\gets\bits^\ell$, for $\ell$ long enough, which can be equivalently written as $x:=\advA(y;r)$. 

We say that a function $f:\mathbb{N}\to\mathbb{R}$ is negligible if for any $c\in\mathbb{N}$ there exists an integer $n_c\in\mathbb{N}$ such that for any $n> n_c$, $f(n)<1/n^c$. We write $f(n)=\negl(n)$ as shorthand for ``$f$ is negligible'' and we write $f(n)\approx g(n)$ when $|f(n)-g(n)|=\negl(n)$. We say that a function $f:\mathbb{N}\to\mathbb{R}$ is polynomial if there exists $c\in\mathbb{N},n_c\in\mathbb{N}$ such that for any $n\geq n_c$, $f(n)\leq n^c$. We write $f(n)=\poly(n)$ as a shorthand for ``$f$ is polynomial''.
We say that two distributions $\dist_1,\dist_2$ are computationally indistinguishable if for any PPT adversary $\advA$, $|\Pr[x\gets\dist_1: \advA(x)=1]-\Pr[x\gets\dist_2: \advA(x)=1]|\approx 0$ 

Vectors are denoted in boldface and lower case, usually elements of $\Z_q^n$, and matrices in boldface and upper case, usually elements of $\Z_q^{m\times n}$. We denote by $\vecb{e}_i^n$ the $i$ th canonical vector of $\Z_q^n$ and by $\matr{I}_{n}$ the identity matrix of size $n\times n$. We $n$ can be understood from the context, we simply write $\vecb{e}_i$ and $\matr{I}$. Given some matrices $\matr{A}\in\Z_q^{m\times t},\matr{A}_1\in\Z_q^{m_1\times t},\ldots,\matr{A}_n\in\Z_q^{m_n\times n}$, we define the operations
 $$\vecb{A}_1 \oplus \ldots \oplus \vecb{A}_n:=\smallpmatrix{ \vecb{A}_1 \\ \vdots \\  \vecb{A}_n} \qquad 
\matr{A}^n:=\smallpmatrix{ \matr{A} &  & \matr{0} \\   & \ddots &   \\ \matr{0} &  & \matr{A}
}.$$
In Chapter \ref{sec:bits} we make extensive use of the set $[n+k]\times[n+k]\setminus\{(i,i):i\in[n]\}$ and for brevity we denote it by $\indexSet{n}{k}$.

Cryptographic schemes are constituted by many algorithms and among them there is usually a key generation algorithm, which receives the security parameter and returns a set of keys. In all the schemes used in this work the security parameter is used to choose a (bilinear) group of size polynomially related to the security parameter, and then the security parameter is never used again. For this reason the key generation algorithms used in this work will receive the group description instead of the security parameter.


    \section{Public-Key Cryptography}

        Perhaps the most groundbreaking achievement in modern Cryptography was the work of Diffie and Hellman \cite{DifHel76} where they introduce \emph{Public Key Cryptography}. Diffie and Hellman conceived systems where each entity publish a \emph{public key} $\mathcal{X}$ while keeping her \emph{secret key} $x$.  It should be computationally infeasible to compute the secret key from the public, which was supposed to be true when $x$ is the \emph{discrete logarithm} of $\mathcal{X}$ in an \emph{Prime Order Cyclic Group}.\footnote{A couple years after Diffie and Hellman seminal paper,  Rivest, Shamir, and Adlelman came up with the first realization of a public key cryptosystem, the RSA cryptosystem. The RSA cryptosystem was build on top of the hardness of factorizing big numbers, and thus it rely on \emph{composite order} groups with unknown factorization. Although such groups have many interesting applications, this work only focus on prime order groups and we will not refer to composite order groups anymore.{\color{red} MmMmMmMm ni idea si esto esta del todo bien}}
\begin{definition}[Abelian Group]\footnote{Historically, the discrete logarithm problem was defined in multiplicative groups. However in this work we will be using additive notation to avoid tangled expressions in the exponent.} 
An abelian group is a set $\GG$ together with a map $+:\GG\times\GG\to\GG$ (written in infix notation) and  it holds that
\begin{description}
\item[Associativity]
For all $\mathcal{X}, \mathcal{Y}$, and $\mathcal{Z}$ in $\GG$, $(\mathcal{X} + \mathcal{Y}) + \mathcal{Z}= \mathcal{X} + (\mathcal{Y} + \mathcal{Z})$ holds.
\item[Identity element]
There exists an element $0$ in $\GG$, such that for all element $\mathcal{X}$ in $\GG$, $0 + \mathcal{X} = \mathcal{X} + 0 = \mathcal{X}$ holds.
\item[Inverse element]
For each $\mathcal{X}$ in $\GG$, there exists an element $-\mathcal{X}$ in $\GG$ such that $\mathcal{X}+ (-\mathcal{X}) = (-\mathcal{X}) +\mathcal{X} = 0$.
\item[Commutativity]
For all $\mathcal{X}, \mathcal{Y}$ in $\GG$, $\mathcal{X} + \mathcal{Y}=\mathcal{Y}+\mathcal{X}$.
\end{description}
We say that $\GG$ is cyclic if there exists an element $\mathcal{P}\in\GG$, a generator of $\GG$, such that $\GG=\{\mathcal{P}, 2\mathcal{P}, 3\mathcal{P},\ldots\}$. We say that $|\GG|$ is the order of $\GG$.

We denote by $\G(1^\lambda)$ a randomized algorithm which on input the security parameter $\lambda$ outputs $gk:=(\GG,\mathcal{P},q)$, $q=|\GG|$, the description of a cyclic group of order $q$.
\end{definition}

\begin{definition}[Discrete Logarithm Assumption (DL)]
We say that the discrete logarithm assumption holds relative to $\G$ if for any adversary $\advA$
$$
\Pr[gk\gets\G(1^\lambda); x\gets\Z_q; \mathcal{X}:=x\mathcal{P}:\advA(gk,\mathcal{X})=x]\approx 0
$$
\end{definition}
 
Diffie and Hellman also introduced a novel \emph{Key-Exchange} protocol, which was later known as the \emph{Diffie-Hellman Key-Exchange}, based on the following assumption

\begin{definition}[Computational Diffie-Hellman Assumption (CDH)]
We say that the computational Diffie-Hellman assumption holds relative to $\G$ if for any adversary $\advA$
$$
\Pr[gk\gets\G(1^\lambda); x,y\gets\Z_q; \mathcal{X}:=x\mathcal{P};\mathcal{Y}:=y\mathcal{P}:\advA(gk,\mathcal{X},\mathcal{Y})=xy\mathcal{P}]\approx 0
$$
\end{definition}

The Diffie-Hellman key-exchange allows two parties $A$, in possession of random secret $x\in\Z_q$ and $\mathcal{Y}=y\mathcal{P}$, and $B$, in possession of random secret $y\in\Z_q$ and $\mathcal{X}=x\mathcal{P}$, to compute the shared secret key $\mathcal{Z}=x\mathcal{Y}=y\mathcal{X}=xy\mathcal{P}$. The computational Diffie-Hellman assumption says that the only way to compute the shared secret key is to know one of the secrets. The \emph{Decisional Diffie-Hellman} assumptions goes a step beyond and says that the shared key ``looks random'' to any other than $A$ and $B$

\begin{definition}[Decisional Diffie-Hellman Assumption (DDH)]
We say that the decisional Diffie-Hellman assumption holds relative to $\G$ if for any adversary $\advA$
$$
\Pr\left[\begin{array}{l}
gk\gets\G(1^\lambda); x,y,z\gets\Z_q,b\gets\bits;\mathcal{X}:=x\mathcal{P};\mathcal{Y}:=y\mathcal{P};\\
\mathcal{Z}:=(bxy+(1-b)z)\mathcal{P}:\advA(gk,\mathcal{X},\mathcal{Y},\mathcal{Z})=b
\end{array}\right]\approx 1/2
$$
\end{definition}

Later, ElGamal introduced the first \emph{Semantically Secure Encryption Scheme} based on the DDH assumption \cite{ElGamal85}. The idea was simple and clean: encrypt a message $\mathcal{M}$ under public key $pk:=\mathcal{X}$ picking $r\gets\Z_q$ and computing the cyphertext $\Enc_{pk}(\mathcal{M};r):=(\mathcal{C}_1,\mathcal{C}_2)=(r\mathcal{P},\mathcal{M}+r\mathcal{X})$; and decrypt a cyphertext using the secret key $x$ and computing $\Dec_{sk}(\mathcal{C}_1,\mathcal{C}_2)=\mathcal{C}_2-x\mathcal{C}_1$. It follows that $(\mathcal{C}_1,\mathcal{C}_2)$ hides $\mathcal{M}$ since $r\mathcal{X}$, by the DDH assumption, looks like fresh random value, independent of $\mathcal{C}_1$, making $\mathcal{M}+r\mathcal{X}$ independent of $\mathcal{M}$. Formally, it can be shown that ElGamal cryptosystem is \emph{Indistinguishable under Chosen Plaintext Attacks} (also called semantically secure).

\begin{definition}[Indistinguishability under Chosen Plaintext Attacks (IND-CPA) \cite{GolMic84}]
We say that $(\KG,\Enc,\Dec)$ is IND-CPA secure if for any $\advA_1,\advA_2$
$$
\Pr\left[\begin{array}{l}
gk\gets\G;(pk,sk)\gets\KG(gk);b\gets\bits;\\
(m_0,m_1)\gets\mathcal{A}_1^{\mathsf{LoR}_b(\cdot,\cdot)}(pk);
c_b\gets\Enc_{pk}(m_b):\advA_2(c_b)\text{ if }m_0,m_1\notin Q
\end{array}\right]\approx 1/2,
$$
where the oracle $\mathsf{LoR}_b(m_0,m_1)$ returns $\Enc_{pk}(m_b)$ and stores $m_0,m_1$ in the set $Q$.
\end{definition}
 


    \section{Bilinear Groups} \label{sec:bil-groups}

        Cryptographic Abelian groups allows to compute the addition of two group elements and, if used together with decisional assumptions, allows to compute \emph{linear functions} of a hidden value. This is the case of El-Gamal encryption, where for any integer $a$ and any group element $b$, given $c=\Enc_{pk}(\mathcal{X};r)$ one can compute $\Enc_{pk}(a\mathcal{X}+b;r')$ using the homomorphic properties of the encryption scheme.
%This simple property allows to construct an interesting NIZK proof that two hidden values $\mathcal{X}$ and $\mathcal{Y}$ are equal (without revealing their actual values): if $c=\Enc_{pk}(x;r),d=\Enc_{pk}(\mathcal{Y};s)$ then $\pi=r-s$ is a proof that $\mathcal{X}=\mathcal{Y}$, check $\pi:=\Enc(r';s)$ $c-\Enc_{pk}(0;r')$.
While this property have many interesting applications, there are still many other functions of the secret value which are not linear functions.

The ideal solution would be to be able to compute the output of any \emph{circuit} evaluated on the encrypted value, which is known as \emph{Fully Homomorphic Encryption}. Note that to do so it is only necessary to compute quadratic functions of the encrypted value, since any circuit can be equivalently represented as a set of quadratic equations. Although fully homomorphic encryption schemes are known to exist, they are far from being practical and thus it is worthwhile to explore other solutions. 

A intermediate solution is to provide a \emph{restricted} number of quadratic operations. Say, given $\mathcal{X}$ and $\mathcal{Y}$ we want to be able to compute a third value $\mathcal{Z}$ which is essentially the multiplication of $\mathcal{X}$ and $\mathcal{Y}$. Note that this rules out case when $\mathcal{X},\mathcal{Y}$ and $\mathcal{Z}$ lie in the same group, because nothing prevents to multiply $\mathcal{Z}$ again. Therefore, we need to consider three abelian groups:\footnote{One can also want to have many groups $\GG_1,\ldots,\GG_k$ and compute expressions of higher but bounded degree. Such groups are known as \emph{Multilinear Groups} which, although have many interesting applications, are out of the scope of this work.} $\GG_1,\GG_2$, and $\GG_T$ the \emph{target group}. Multiplication is provided by an efficiently computable function $e:\GG_1\times\GG_2\to\GG_T$. More formally, we define \emph{bilinear groups} as follows.

\begin{definition}[Bilinear Groups] \footnote{While the usual notation for the target group has been multiplicative, we write it in additive notation. The reason is just to elude cumbersome expressions in the exponent.}
Let $\GG_1$ and $\GG_2$ be cyclic groups of prime order $q$ and $\mathcal{P}_s$ the generator of $\GG_s$, $s\in\{1,2\}$, and $\GG_T$ be another cyclic group of order $q$. We say that $\GG_1,\GG_2,\GG_T$ form a bilinear group if there exists a map $e:\GG_1\times\GG_2\to\GG_T$ such that:
\begin{description}
\item[Bilinearity:] For all $a,b\in\Z_q$, for all $\mathcal{X}\in\GG_1$, and all $\mathcal{Y}\in\GG_2$
$$
e(a\mathcal{X},b\mathcal{Y})=ab\cdot e(\mathcal{X},\mathcal{Y}),
$$
\item[Non-degeneracy:] If $\mathcal{X},\mathcal{Y}\neq0$, then $e(\mathcal{X},\mathcal{Y})\neq 0$,
\item[Computability:] $e(\mathcal{X},\mathcal{Y})$ is efficiently computable.
\end{description} 
\end{definition} 

The first property says that $e$ allows to ``homomorphically'' compute degree 2 expressions in the field $(\Z_q,+,\cdot)$. Since any $\mathcal{X}$ can be written as $x\mathcal{P}_1$, where $x$ is some element of $\Z_q$ (and the same can be done in $\GG_2$), $e(\mathcal{X},\mathcal{Y})=xy\cdot e(\mathcal{P}_1,\mathcal{P}_2)$. Non-degeneracy says that $e$ is does not map everything to 0. While the third property says that computing $e$ is practical, the reality is that it is still an expensive operation and is one of the critical performance measures of cryptographic constructions.

Galbraith et al.~classify bilinear groups in three types \cite{DAM:GalPatSma08}:

\begin{description}
\item[Type I:] $\GG_1=\GG_2$ and also known as \emph{symmetric} groups. 
\item[Type II:] $\GG_1\neq\GG_2$ but there is an efficiently computable homomorphism $\phi:\GG_1\to\GG_2$.
\item[Type III:] $\GG_1\neq\GG_2$ and no efficiently computable homomorphism is known. Also known as \emph{asymmetric} groups.
\end{description}

Type III bilinear groups are the most attractive from the perspective
of a performance and security trade off, specially since the recent attacks on discrete logarithms in
finite fields by Joux \cite{SAC:Joux13} and subsequent improvements.

We denote by $\ggen_a$ the probabilistic polynomial time algorithm which on input $1^{\lambda}$, where $\lambda$ is the security parameter, returns the \emph{group key} which is the description of an asymmetric bilinear group $gk:=(q,\GG_1,\GG_2,\GG_T,e,\mathcal{P}_1,\mathcal{P}_2)$, where $\GG_1,\GG_2$
and $\GG_T$ are groups of prime order $q$, the elements $\mathcal{P}_1, \mathcal{P}_2$ are generators of 
$\GG_1,\GG_2$ respectively, and $e:\GG_1\times\GG_2\to\GG_T$ is an efficiently
computable, non-degenerate bilinear map. 

\subsubsection{Instantiations}
While we have presented bilinear groups as abstract structures, they are instantiated as concrete algebraic constructions. Such constructions involve fairly complex mathematics and, given that this work can be read using only our abstract presentation, we omit any further detail of the instantiation of bilinear groups.

%Koblitz and Miller independently suggested the use of \emph{Elliptic Curves} as new groups where the discrete logarithm and DDH are hard problems. However, in some Elliptic Curves the DDH problem might be hard, for example in those where a bilinear pairing can be computed $xy\mathcal{P}_1$.

\subsubsection{Implicit Representation of Group Elements}

Elements in $\GG_s$, are denoted implicitly as $[a]_s:=a \Pt_s$, where $s \in \{1,2,T\}$ and $\Pt_T:=e(\Pt_1,\Pt_2)$. 
The pairing operation will be written as a product $\cdot$, that is $[a]_1 \cdot [b]_2=[a]_1 [b]_2=e([a]_1,[b]_2)=[ab]_T$. Given a matrix $\matr{T}=(t_{i,j})$, $[\matr{T}]_s$ is
the natural embedding of $\matr{T}$ in $\GG_s$, that is, the matrix whose $(i,j)$th entry
is $t_{i,j}\mathcal{P}_s$. We denote by $|\GG_s|$ the bit-size of the elements of $\GG_s$.



    \section{Matrix Diffie-Hellman  Assumptions} \label{sec:mddh}

        In this section we review \emph{Matrix Diffie-Hellman Assumptions} (MDDH) of Escala et al. \cite{C:EHKRV13} which are abstractions and generalizations of the $\lin{k}$ family of assumptions. Then, we review \emph{Kernel Matrix Diffie-Hellman Assumptions} (KMDH) of Morillo et al. \cite{EPRINT:MorRafVil15}, which are the natural computational counterpart of Matrix Diffie-Hellman assumptions.

We also put forward a new Kernel assumption which is specific to asymmetric groups, and we prove its security in the \emph{generic group model}.

\subsection{Decisional Matrix Diffie-Hellman Assumptions}
\begin{definition}   \label{def:matrixdef}
Let $\ell,k \in \N$.
We call $\dist_{\ell,k}$ a matrix distribution if it outputs (in poly time, with overwhelming probability) matrices in $\Z_q^{\ell \times k}$. We define $\dist_k := \dist_{k+1,k}$ and $\overline{\dist}_{k}$ the distribution of the first $k$ rows of $\matr{A}$ when $\matr{A}\gets\dist_{k}$. 
\end{definition}

For the following decisional assumption to hold, it is a necessary condition that $\ell>k$. However, in other contexts, we might need $\dist_{\ell,k}$ distributions where 
$\ell \geq k$. 

\begin{definition}[MDDH Assumption in $\GG_{\gamma}$, $\gamma \in \{1,2\}$ \cite{C:EHKRV13}]\label{def:mdh}
Let $\dist_{\ell,k}$ be a matrix distribution  and $\gk\gets \ggen_a(1^\lambda)$. We say that the $\dist_{\ell,k}$-Matrix Diffie-Hellman ($\dist_{\ell,k}$-$\mddh_{\GG_\gamma}$)
Assumption holds relative to $\ggen_a$ if for all PPT adversaries $\advD$,
\begin{eqnarray*}
\adv_{\dist_{\ell,k},\ggen_a}(\advD) & := &
    \left|
        \Pr[\advD(\group,[\matr{A}]_\gamma,[\matr{A}\vecb{w}]_\gamma)=1]-
        \Pr[\advD(\group,[\matr{A}]_\gamma, [\vecb{z}]_\gamma) =1]
    \right|
\end{eqnarray*}
is negligible in $k$,
where the probability is taken over $\gk \gets \ggen_a(1^\lambda)$, $\matr{A} \gets \dist_{\ell,k}, \vecb{w} \gets \Z_q^k, [\vecb{z}]_\gamma  \gets \GG_\gamma^{\ell}$ and the coin tosses of adversary $\advD$.
\end{definition}
 
In this work we will refer to the following matrix distributions: 
\[
\distlin_{k}:\matrA = \left( \begin{smallmatrix}
    a_1 & 0 &  \ldots & 0 \\
    0 &  a_2 &  \ldots & 0\\
    \tiny{\vdots} &  \tiny{\vdots}  &  \tiny{\ddots} & \tiny{\vdots} \\
    0 & 0 &  \ldots  & a_{k}\\
    1 & 1 & \ldots & 1
\end{smallmatrix} \right),
\ 
\mathcal{U}_{\ell,k}: \matrA = \left( \begin{smallmatrix}
    a_{1,1} &  \ldots & a_{1,k}  \\
    \tiny{\vdots} & \tiny{\ddots}  & \tiny{\vdots} \\
    a_{\ell,1} &  \ldots &  a_{\ell,k} 
\end{smallmatrix} \right),
\]
where $a_i,a_{i,j}\leftarrow \Z_q$.  The $\distlin_{k}$-$\mddh$ assumption is the $k$-linear family of Decisional Assumptions
and corresponds to 
 the Decisional Diffie-Hellman (DDH)
Assumption in $\GG_\gamma$ when $k=1$. The SXDH Assumption states that DDH holds in $\GG_\gamma$ for all $\gamma \in \{1,2\}$. The $\mathcal{U}_{\ell,k}$ assumption is the \textit{uniform} assumption and is the weakest of all assumptions of size $\ell \times k$. 

Further, given any matrix distribution $\dist_{k}$, $m \in \mathbb{N}$ and any $i \in [m]$, we introduce the distribution $\distink$, which is defined as follows: 
\[ \distzeronk: \matrA = \left(\begin{smallmatrix} \matr{B}\vecb{w}_1 & \ldots &  
  \matr{B}\vecb{w}_{m} & \matr{B}  \end{smallmatrix} \right)  \qquad
\distink:\matrA = \left(\begin{smallmatrix} \matr{B}\vecb{w}_1 & \ldots & \matr{B}\vecb{w}_{i-1} 
& \vecb{z} &  \matr{B}\vecb{w}_{i+1} & \ldots &  
  \matr{B}\vecb{w}_{m} & \matr{B}  \end{smallmatrix} \right) 
\]
where $\matr{B} \leftarrow \dist_{k}$, $\vecb{w}_i \leftarrow \Z_q^k$ and $\vecb{z} \leftarrow \Z_q^{k+1}$. The following are two trivial properties of the $\distink$ distribution. 

\begin{lemma} Under the $\dist_{k}$-$\mddh$ assumption in $\GG_\gamma$, for any $0 < i \leq n$, the distribution of  $[\matrA]_\gamma$ when $\matrA \leftarrow \distzeronk$ and when $\matrA \leftarrow \distink$ are computationally indistinguishable. Further, if $\ell>k$, for any $i>0$, if $\matr{A} \leftarrow \distink$, then with overwhelming probability its ith column is linearly independent of the rest. \label{lemma:dist-i}
\end{lemma}

\subsection{Computational Matrix Diffie-Hellman Assumptions}
Additionally, we will be using the following family  computational assumptions:
\begin{definition}[Kernel Diffie-Hellman Assumption in $\GG_{\gamma}$ \cite{EPRINT:MorRafVil15}]Let  $\gk 
\hspace*{-1pt}
\gets
\hspace*{-1pt}
\ggen_a(1^\lambda)$.
The Kernel Diffie-Hellman Assumption in $\GG_\gamma$  ($\dist_{\ell,k}\mbox{-}\kermdh_{\GG_\gamma}$) says that every PPT Algorithm has negligible advantage in the following  game: given $[\matr{A}]_\gamma$, where $\matrA \gets \dist_{\ell,k}$, find $[\vecb{x}]_{3-\gamma} \in \GG_{3-\gamma}^{\ell}$, $\vecb{x} \neq \vecb{0}$, such that 
$[\vecb{x}]_{3-\gamma}^{\top}[\matr{A}]_{\gamma}=[\vecb{0}]_T$. 
\end{definition}
The  Simultaneous Pairing Assumption in $\GG_\gamma$  (\SP$_{\GG_{\gamma}}$) is the $ \mathcal{U}_1\mbox{-}\kermdh_{\GG_{\gamma}}$ assumption. The Kernel Diffie-Hellman assumption is a generalization and abstraction of this assumption to other matrix distributions. 
The $\dist_{\ell,k}\mbox{-}\kermdh_{\GG_{\gamma}}$ assumption is weaker than the $\dist_{\ell,k}\mbox{-}\mddh_{\GG_{\gamma}}$ assumption, since a solution to the former allows to decide membership in $\rmIm([\matr{A}]_{\gamma})$.

\subsection{A new Computational Matrix Diffie-Hellman Assumption in Type III Groups}

In asymmetric bilinear groups, we introduce a natural variant of the $\dist_{\ell,k}\mbox{-}\kermdh$ assumption \cite{AC:GonHevRaf15}.  
\begin{definition}[Split Kernel Diffie-Hellman Assumption]
Let  $\gk \hspace*{-3pt} \gets
\hspace*{-3pt}
\ggen_a(1^\lambda)$.
The Split Kernel Diffie-Hellman Assumption in $\GG_1,\GG_2$  ($\dist_{\ell,k}\mbox{-}\skermdh$) says that every PPT Algorithm has negligible advantage in the following  game: given $([\matr{A}]_1,[\matr{A}]_2)$, $\matr{A} \leftarrow \dist_{\ell,k}$, find a pair of vectors $([\vecb{r}]_1,[\vecb{s}]_2) \in \GG_1^{\ell} \times \GG_2^{\ell}$, $\vecb{r} \neq \vecb{s}$, such that 
$[\vecb{r}]_1^{\top}[\matr{A}]_2=[\vecb{s}]_2^{\top}[\matr{A}]_1$. 
\end{definition}

While the Kernel Diffie-Hellman Assumption says one cannot find a non-zero vector in one of the groups which is in the co-kernel of $\matr{A}$, the split assumption says one cannot find a pair of vectors in $\GG_1^{\ell} \times \GG_2^{\ell}$ such that the difference of the vector of their discrete logarithms is in the co-kernel of $\matr{A}$. 
As a particular case we consider the \emph{Split Simultaneous Double Pairing Assumption in} $\GG_1,\GG_2$ ($\SSDP$) which is the $\distrlin_{2}\mbox{-}\skermdh$ assumption, where 
$\distrlin_{2}$ is the distribution which results of sampling a matrix from $\distlin_{2}$ and replacing the last row by random elements. 

To gain confidence in this assumption, we first note that it implies the Kernel MDH Assumption and then we prove that the reciprocal is true in the generic bilinear model. 

\begin{lemma} $\dist_{\ell,k}\mbox{-}\skermdh \Rightarrow \dist_{\ell,k}\mbox{-}\kermdh_{\Hr}$.
\end{lemma}
\begin{proof} Suppose there exists an adversary $\advB$ against the 
$\dist_{\ell,k}\mbox{-}\kermdh_{\Hr}$ assumption. We show how to construct an adversary $\advA$ against the  $\dist_{\ell,k}\mbox{-}\skermdh$ assumption. Adversary $\advA$ receives as a challenge $([\matr{A}]_1,[\matr{A}]_2)$ and forwards $[\matr{A}]_2$ to $\advB$, who outputs with non-negligible probability a vector $[\vecb{r}]_1$ such that $[\vecb{r}]_1^{\top} [\matr{A}]_2=[\vecb{0}]_{T}$. Then $\advA$  simply outputs $([\vecb{r}]_1,[\vecb{0}]_2)$ as a solution to the $\dist_{\ell,k}\mbox{-}\skermdh$ challenge. 
\end{proof}

\subsubsection{Security of the $\dist_{\ell,k}\mbox{-}\skermdh$ in the Generic Group Model}

The generic group model is an idealized model for analysing the security of cryptographic assumptions or cryptographic schemes. A proof of security in the generic group model guarantees that no attacker, that only uses the algebraic structure of the (bilinear) group, is successful in breaking the assumption/scheme. Conversely, for a generically secure assumption/scheme, a successful attack must exploit the structure of the (bilinear) group that is actually used in the protocol (e.g.~a Barreto-Naehring curve in the case of bilinear groups).  

We use the natural generalization of Shoup's generic group model \cite{EC:Shoup97} to the (a)symmetric bilinear setting, as it was used for instance by Boneh et al.~\cite{EC:BonBoyGoh05}. In such a model an adversary can only access elements of $\Gr,\Hr$ or $\GG_T$ via a query to a group oracle, which gives him a randomized  encoding of the queried element. The group oracle must be consistent with the group operations (allowing to query for the encoding of constants in either group, for the encoding of the sum of previously queried elements in the same group and for the encoding of the product of pairs in $\Gr\times \Hr$). %More details, can be found for instance in  \cite{EC:BonBoyGoh05}.

\begin{lemma} If $\dist_{\ell,k}\mbox{-}\kermdh$ holds in generic symmetric bilinear groups, then $\dist_{\ell,k}\mbox{-}\skermdh$ holds in generic asymmetric bilinear groups. 
\end{lemma}

\begin{proof} Suppose there is an adversary $\advA$  in the asymmetric generic bilinear group model against the $\dist_{\ell,k}\mbox{-}\skermdh$ assumption.  We show how to construct an adversary $\advB$ against the  $\dist_{\ell,k}\mbox{-}\kermdh_{\Hr}$ assumption in the symmetric generic group model. 


Adversary $\advB$ has oracle access to the randomized encodings $\sigma: \Z_q \to \{0,1\}^n$, 
and $\sigma_T: \Z_q \to \{0,1\}^n$. It receives as a challenge $\{ \sigma(a_{ij})\}_{1\leq i \leq \ell, 1\leq j \leq k}$.

Adversary $\advB$ simulates the generic hardness game for $\advA$ as follows. It defines encodings  $\xi_1: \Z_q \to \{0,1\}^n$, $\xi_2: \Z_q \to \{0,1\}^n$ and $\xi_T: \Z_q \to \{0,1\}^n$ as $\xi_1=\sigma$, $\xi_T=\sigma_T$ and $\xi_2$ a random encoding function. $\advB$ keeps a list $L_\advA$  with the values that have been queried by $\advA$ to the group oracle. The list is initialized as 
$L_\advA=\{  \{(A_{i,j},\xi_1(a_{ij}),1),(A_{i,j},\xi_2(a_{ij}),2)\}_{1\leq i \leq \ell, 1\leq j \leq k}\}$, where $\xi_2(a_{ij}) \in \{0,1\}^n$ are chosen uniformly at random conditioned on being pairwise distinct.  Adversary $\advB$ also keeps a list $L_\advB$ with the queries 
it makes to its own group oracle. The list $L_\advB$ is initialized as 
$L_\advB=\{  \{(A_{i,j},\sigma(a_{ij}),1)\}_{1\leq i \leq \ell, 1\leq j \leq k}\}$

Each element in the list $L_\advA$ is a tuple $(P_i,s_i,x_i)$, where $P_i \in \Z_q[A_{11}, \ldots,A_{\ell k}]$, $x_i \in \{1,2,T\}$ and $s_i=\xi_{x_i}(P_i(a_{11},\ldots,a_{\ell k}))$. The polynomial $P_i$ is one of the following:  a) $P_i=A_{ij}$, i.e. it is one of the initial values in the query list  
$L_\advA$  or b) a constant polynomial or c) $P_i=P_c+P_d$ for some $(P_c,s_c,x),(P_d,s_d,x) \in L_\advA$ or d) $P_i=P_cP_d$ for some $(P_c,s_c,1),(P_d,s_d,2) \in L_\advA$, $x_i=T$. For $L_\advB$ the same holds except that $x_i \in \{1,T\}$ and except that d) is changed to: d) $P_i=P_cP_d$ for some $(P_c,s_c,1),(P_d,s_d,1) \in L_\advB$ and $x_i=T$. 

Without loss of generality we can identify the queries of $\advA$ with 
pairs $(P_i,x_i)$ meeting the restrictions described above. If $(P_i,x_i)$ was queried before, it replies with the same answer $s_i$.

Else, when $\advB$ receives a (valid) query $(P_i,x_i)$, if $x_i \in \{1,T\}$ it simply forwards  the query to its own group oracle, who replies with $s_i$. Then $(P_i,s_i,x_i)$ is appended to $L_\advB$ and to $L_\advA$. If $x_i =2$, then it forwards the query to its own group oracle as $(P_i,1)$. When it receives the answer $s_i$, 
$\advB$ appends $(P_i,s_i,1)$ to $L_\advB$ and it looks for the set $S$ of all  tuples $(P_j,s_j,1) \in L_\advB$, $P_j \neq P_i$,  such that $s_j=s_i$. 
For every tuple in $S$, $\advB$ checks if there is some $\tilde{s}$ such that $(P_j,\tilde{s},2)$ is in $L_\advA$ (note that, because of the way $L_\advA$ is constructed, if such $\tilde{s}$ exists it is the same for all $P_j$). 

If such $\tilde{s}$ exists, it appends $(P_i, \tilde{s},2)$ in $L_\advA$ and it replies with $\tilde{s}$.  Else it chooses some $\tilde{s}$ uniformly at random conditioned on being distinct from all other values $s$ such that there exist some $P$ such that $(P,s,2)$ is in $L_\advA$. Finally, it appends  $(P_i, \tilde{s},2)$ in $L_\advA$.  


Finally, $\advA$ will output as a solution to the challenge a pair $s_q,s_r$ such that 
$(Q,s_q,1),(R,s_r,2) \in L_\advA$. Because of the way $L_\advA$ and $L_\advB$ were constructed,  there exists some $s'_r$ such that $(Q,s_q,1),(R,s'_r,1) \in L_\advA$. $\advB$ queries its group oracle for $(R-Q,1)$ and obtains as a reply some string $s_{R-Q}$. Finally, it outputs $s_{R-Q}$ as a solution to its challenge. 
It easily follows that $\advA$ and $\advB$ have exactly the same probability of success.
\end{proof} 

Finally, we note that the $\distlin_{2}\mbox{-}\skermdh$ assumption is implied by a decisional assumption introduced by Libert et al.~\cite{EPRINT:LPJY15}. The assumption says that, given $([\matr{A}]_1,[\matr{A}]_2)$, where $\matr{A}\gets\distlin_2$, the vector $([\matr{A}]_1\vecb{w},[\matr{A}]_2\vecb{w})$, $\vecb{w}\gets\Z_q^2$, is computationally indistinguishable from $([\vecb{u}]_1,[\vecb{u}]_2)$, $\vecb{u}\gets\Z_q^3$. The proof is analogous to the proof that $\dist_{\ell,k}\mbox{-}\mddh\Rightarrow\dist_{\ell,k}\mbox{-}\kermdh$. Suppose that $([\vecb{r}]_1,[\vecb{s}]_2)$ is a solution to the $\distlin_2\mbox{-}\skermdh$ assumption, then $[\vecb{r}]_1^\top[\matr{A}]_2\vecb{w}-[\vecb{s}]_2^\top[\matr{A}]_1\vecb{w}=([\vecb{r}]_1^\top[\matr{A}]_2-[\vecb{s}]_2^\top[\matr{A}]_1)\vecb{w}=[0]_T$, while $[\vecb{r}]_1^\top[\vecb{u}]_2-[\vecb{s}]_2^\top[\vecb{u}]_1=[0]_T$ only with negligible probability whenever $\vecb{r}\neq\vecb{s}$. 


    \section{Non-Interactive Zero-Knowledge Proofs} \label{sec:zk}

        Zero-Knowledge Proofs are proofs that reveal nothing beyond their validity. Since their introduction by Goldreich et al.~\cite{GolMicRac89}, Zero-Knowledge proofs have played a central role in cryptography and complexity theory from both the theoretical side --they have been the inspiration of \emph{Probabilistic Checkable Proofs} and the groundbreaking results on \emph{Hardness of Approximation}-- and the practical side --applications ranges from \emph{Multi-Party Computation} to \emph{Electronic voting} and \emph{E-commerce}.

In a Zero-Knowledge proof a \emph{prover} $\algP$ in possession of a secret $w$ wants to convince a \emph{verifier} $\algV$ that some statement $x$ is true, namely that $x$ belongs to some language $\Lang$. To do so the prover and the verifier engage on an \emph{interactive protocol}: the prover starts with a message $a_1$, the verifier answers with $b_1$, and so on. At the end the verifier outputs a bit $b\in\bits$ indicating whether it rejects or accepts the proofs.

There are two basic requirements for a Zero-Knowledge proof: \emph{Completeness}, which says that the prover should be successful when convincing the verifier about a true statement, and \emph{Soundness}, which says that the verifier rejects false statements with \emph{high probability}. There is also a third requirement which  gives the name to Zero-Knowledge proofs and it aims to require that the verifier does not learn nothing beyond the fact that $x$ is true. This is done by requiring the existence of an efficient algorithm $\algS$, the simulator, which, for any true statement, it is able to construct a \emph{transcript} of the interactive execution of $\algP$ and $\algV$ even without knowing any secret $w$.
A weaker variant of a Zero-Knowledge proof is a \emph{Witness-Indistinguishable Proof}, this proofs are not necessarily simulatable and only guarantee to reveal the same information when using two different secrets $w,w'$. 

The focus of this work are \emph{Non-Interactive Zero-Knowledge Proofs} (NIZK), where the prover sends a single message, the \emph{proof}, to the verifier. Since their introduction by Blum et al.~\cite{STOC:BluFelMic88}, it was known that a ``pre-shared'' information, known as the \emph{Common Reference String} (CRS), allows to construct NIZK proof systems (later it was shown a necessary condition if the statement is not trivial \cite{JC:GolOre94}). Thereby, the simulator is allowed to simulate the CRS and thus is able to compute \emph{trapdoors} associated to it. The knowledge of such trapdoors enhance the possibilities of $\algS$ to successfully simulate proofs.

Syntactically, a NIZK proof system consists of three probabilistic polynomial time algorithms: a CRS
generation algorithm $\algK$, a prover $\algP$, and a verifier $\algV$.
The CRS generation algorithm takes a group description $gk$ as input and produces a CRS $\sigma$ (which we assume includes $gk$). The prover takes as input $(\sigma, x, w)$ and produces a proof $\pi$. The verifier takes as input $(\sigma, x, \pi)$ and outputs 1 if the proof is acceptable and 0 if rejecting the proof.

In this work we will consider two particular cases of NIZK: \emph{Composable} NIZK \cite{EC:GroSah08} and  \emph{Quasi-Adaptive} NIZK \cite{AC:JutRoy13}. Both deals with the case of \emph{CRS dependent languages}, say the language language is parameterized by some values which are (randomly) sampled within the CRS. For example, the CRS might contain the group key $gk$ defining a bilinear group and the language might be some set of satisfiable equations over that group. Quasi-Adaptive NIZK goes further and the language might also depend on group constants defined in the CRS.
 
\subsection{Composable Witness-Indistinguishability and Zero-Knowledge}
The following definitions are taken from \cite{SIAMJC:GroSah12}.

\begin{definition}[Group dependent languages] Let $\mathcal{R}$ be an efficiently computable ternary relation.
For triplets $(gk, x, w)\in \mathcal{R}$ we call $gk$ the group key, $x$ the statement, and $w$ the witness.
Given some $gk$, we let $\Lang$ be the language consisting of statements $x$ that have a
witness $w$ so $(gk, x, w) \in \mathcal{R}$. For a relation that ignores $gk$ this is, of course, the
standard definition of an NP-language. We will be more interested in the case where
$gk$ describes a bilinear group, though.
\end{definition}

\begin{definition}[Composable NIZK proof system] We say that a non-interactive proof system $(\algK,\algP,\algV)$ is a composable NIZK proof system with respect to $\G_a$ if
\begin{description}
\item[Perfect Completeness:] For any $x,w$
$$
\Pr\left[
\begin{array}{l}
gk\gets\G_a(1^\lambda);\sigma\gets\algK(gk);\pi\gets\algP(\sigma,x,w):\\
V(\sigma,x,\pi)=1 \text{ if } (gk,x,w)\in\mathcal{R}
\end{array}
\right]=1.
$$
\item[Perfect Soundness:] For any adversary $\advA$
$$
\Pr\left[
\begin{array}{l}
gk\gets\G_a(1^\lambda);\sigma\gets\algK(gk);(x,\pi)\gets\advA(gk,\sigma):\\
V(\sigma,x,\pi)=0 \text{ if } x\notin\Lang
\end{array}
\right]=1.
$$
\item[Composable Zero-Knowledge:] There exist efficient algorithms $\algS_1,\algS_2$ such that for any adversaries $\advA_1,\advA_2$
\begin{align*}
&\Pr\left[
gk\gets\G_a(1^\lambda);\sigma\gets\algK(gk):
\advA_1(gk,\sigma)=1
\right]\approx\\
&\Pr\left[
gk\gets\G_a(1^\lambda);(\sigma,\tau)\gets\algS_1(gk):
\advA_1(gk,\sigma)=1
\right]
\end{align*}
and
\begin{align*}
&\Pr\left[
\begin{array}{l}
gk\gets\G_a(1^\lambda);(\sigma,\tau)\gets\algS_1(gk);(x,w)\gets\advA_2(gk,\sigma,\tau);\\
\pi\gets\algP(\sigma,x,w):\advA_2(\pi)=1\text{ if }(gk,x,w)\in\mathcal{R}
\end{array}
\right]\approx\\
&\Pr\left[
\begin{array}{l}
gk\gets\G_a(1^\lambda);(\sigma,\tau)\gets\algS_1(gk);(x,w)\gets\advA_2(gk,\sigma,\tau);\\
\pi\gets\algS_2(\sigma,\tau,x):\advA_2(\pi)=1\text{ if }(gk,x,w)\in\mathcal{R}
\end{array}
\right].
\end{align*}
\end{description}
\end{definition}

\begin{definition}[Composable NIWI proof system] We say that a non-interactive proof system $(\algK,\algP,\algV)$ is a composable NIWI proof system with respect to $\G_a$ if it have Perfect Completeness and Soundness as defined above and also
\begin{description}
\item[Composable Witness-Indistinguishability:] There exists and efficient algorithm $\algS_1$ such that for any adversaries $\advA_1,\advA_2$
\begin{align*}
&\Pr\left[
gk\gets\G_a(1^\lambda);\sigma\gets\algK(gk):
\advA_1(gk,\sigma)=1
\right]\approx\\
&\Pr\left[
gk\gets\G_a(1^\lambda);(\sigma,\tau)\gets\algS_1(gk):
\advA_1(gk,\sigma)=1
\right]
\end{align*}
and
\begin{align*}
&\Pr\left[
\begin{array}{l}
gk\gets\G_a(1^\lambda);(\sigma,\tau)\gets\algS_1(gk);(x,w,w')\gets\advA_2(gk,\sigma,\tau);\\
\pi\gets\algP(\sigma,x,w):\advA_2(\pi)=1\text{ if }(gk,x,w),(gk,x,w')\in\mathcal{R}
\end{array}
\right]\approx\\
&\Pr\left[
\begin{array}{l}
gk\gets\G_a(1^\lambda);(\sigma,\tau)\gets\algS_1(gk);(x,w,w')\gets\advA_2(gk,\sigma,\tau);\\
\pi\gets\algP(\sigma,x,w'):\advA_2(\pi)=1\text{ if }(gk,x,w),(gk,x,w')\in\mathcal{R}
\end{array}
\right].
\end{align*}
\end{description}
\end{definition}

\subsection{Quasi-Adaptive Non-Interactive Zero-Knowledge Proofs}
A Quasi-Adaptive NIZK proof system \cite{AC:JutRoy13} enables
to prove membership in a language defined by a relation $\R_\rho$, which in turn is completely determined by some parameter
$\rho$ sampled from a distribution $\dist_\gk$.
We say that $\dist_\gk$ is \emph{witness samplable} if there exists an efficient
algorithm that samples $(\rho,\omega)$ from a distribution $\dist_\gk^{\mathsf{par}}$ such that $\rho$ is distributed according to $\dist_\gk$, and membership of $\rho$
in the \emph{parameter language} $\Lang_\mathsf{par}$ can be efficiently verified with $\omega$.
While the Common Reference String can be set based on $\rho$, the zero-knowledge simulator is required to be a single probabilistic polynomial time
algorithm that works for the whole collection of relations $\R_\gk:=\{\R_\rho\}_{\rho \in \mathrm{sup}(\dist_\gk)}$. 

\begin{definition}[Quasi-Adaptive NIZK (QA-NIZK) proof system]
A non-interactive proof system  $(\algK,\algP,\algV)$ is called a QA-NIZK proof system with respect to $\G_a$ for witness-relations
$\R_\gk = \{\R_\rho\}_{\rho \in \mathrm{sup}(\dist_\gk)}$,
with parameters sampled from a distribution $\dist_\gk$ over associated parameter language
$\Lang_\mathsf{par}$, if there exists a probabilistic polynomial time simulator $(\algS_1, \algS_2)$,
such that for all non-uniform PPT adversaries $\advA_1$, $\advA_2$, $\advA_3$ we have:

\begin{description}
\item[Quasi-Adaptive Completeness:]
$$\Pr\begin{bmatrix*}[l]
    \gk \gets \G_a(1^\lambda);
    \rho \gets \dist_\gk;
    \psi \gets \algK(\gk, \rho);
    (x, w) \gets \advA_1(\gk, \psi);\\
    \pi \gets \algP(\psi, x, w) :
        \algV(\psi, x, \pi) = 1 \text{ if } \R_\rho(x, w)
\end{bmatrix*} = 1.$$
\item[Computational Quasi-Adaptive Soundness:]
$$\Pr\begin{bmatrix*}[l]
    \gk \gets \G_a(1^\lambda);
    \rho \gets \dist_\gk;
    \psi \gets \algK(\gk, \rho); \\
    (x, \pi) \gets \advA_2(\gk, \psi) :
        \algV(\psi, x, \pi) = 1 \text{ and } \neg (\exists w : \R_\rho(x, w))
\end{bmatrix*} \approx 0.$$ 
\item[Perfect Quasi-Adaptive Zero-Knowledge:]
\begin{eqnarray*}
\Pr[
    \gk \gets \G_a(1^\lambda);
    \rho \gets \dist_\gk;
    \psi \gets \algK(\gk, \rho):
        \advA_3^{\algP(\psi,\cdot, \cdot)}(\gk,\psi) = 1]
=\;\;\;\\
\Pr[
    \gk \gets \G_a(1^\lambda);
    \rho \gets \dist_\gk;
    (\psi,\tau) \gets \algS_1(\gk, \rho):
        \advA_3^{\algS(\psi,\tau,\cdot,\cdot)}(\gk,\psi)=1]
\end{eqnarray*}
where
\begin{itemize}
\item $\algP(\psi, \cdot, \cdot)$ emulates the actual prover. It takes input $(x,w)$ and outputs a 
proof $\pi$ if $(x,w)\in\R_\rho$. Otherwise, it outputs $\perp$.
\item $\algS(\psi,\tau, \cdot, \cdot)$ is an oracle that takes input $(x,w)$. It outputs a simulated proof
$\algS_2(\psi,\tau, x)$ if $(x, w) \in \R_\rho$ and $\perp$ if $(x, w) \notin \R_\rho$.
\end{itemize}
\end{description}
Note that $\psi$ is the CRS in the above definitions.
We assume that $\psi$ contains an encoding of $\rho$, which is thus available to $\algV$.
\end{definition}

For witness samplable distributions, a stronger notion of soundness, where the adversary has also access to a witness of the parameter $\rho$, is defined in the full version of \cite{AC:GonHevRaf15}.
\begin{description}
\item[Computational Quasi-Adaptive Strong Soundness:]
$$\Pr\begin{bmatrix*}[l]
    \gk \gets \G_a(1^\lambda);
    (\rho,\omega) \gets \dist_\gk^\mathsf{par};
    \psi \gets \algK(\gk, \rho); \\
    (x, \pi) \gets \advA_2(\gk, \omega,\psi) :
        \algV(\psi, x, \pi) = 1 \text{ and } \neg (\exists w : \R_\rho(x, w))
\end{bmatrix*} \approx 0.$$ 

\end{description}


    \section{Groth-Sahai Proofs}\label{sec:gs-proofs}

        The GS proof system allows to prove satisfiability of a set of quadratic equations in a bilinear group. In general, the proof is witness-indistinguishable but for most equations it is also zero-knowledge.

The admissible equation types must be in the following form:
\begin{equation}\label{gseq}
\sum_{j=1}^{m_y} f(\alpha_j, \vary_j)+\sum_{i=1}^{m_x} f(\varx_i, \beta_i)+\sum_{i=1}^{m_x} \sum_{j=1}^{m_y}  f(\varx_i,\escQE_{i,j} \vary_j)=t,
\end{equation}
 where $\Am_1,\Am_2,\Am_T$ are $\Z_q$-vector spaces equipped with some bilinear map $f:\Am_1\times \Am_2 \rightarrow \Am_T$, $\boldsymbol \alpha  \in \Am_1^{m_y}$, $\boldsymbol \beta  \in \Am_2^{m_x}$, $\matr{\EscQE}=(\escQE_{i,j}) \in \Z_q^{m_x\times m_y}$, $t \in \Am_T$. The vector spaces and the map $f$ can be defined in different ways as:
\begin{itemize}
\item[(a)] in pairing-product equations (PPEs), $\Am_1=\Gr$, $\Am_2=\Hr$, $\Am_T=\GG_T$, $f([x]_1,[y]_2)=[x]_1 [y]_2 \in \GG_T$,
\item[(b1)] in multi-scalar multiplication equations in $\Gr$ (MMEs), $\Am_1=\Gr$, $\Am_2=\Z_q$, $\Am_T=\Gr$, $f([x]_1,y)=y [x]_1 \in \Gr$,
\item[(b2)] MMEs in $\Hr$ (MMEs),  $\Am_1=\Z_q$, $\Am_2=\Hr$, $\Am_T=\Hr$, $f(x,[y]_2)=x [y]_2 \in \Hr$, and
\item[(c)] in quadratic equations in $\Z_q$ (QEs), $\Am_1=\Am_2=\Am_T=\Z_q$, $f(x,y)=xy \in \Z_q$.
\end{itemize} 
 An equation is linear if $\matr{\EscQE}=\vecb{0}$, 
 it is \textit{two-sided linear} if both $\boldsymbol \alpha \neq \vecb{0}$ and $\boldsymbol \beta \neq \vecb{0}$, and \textit{one-sided} otherwise.

When $t=f(t_1,1)$ or $t=f(1,t_2)$, for some efficiently computable $t_1\in\Am_1$ or $t_2\in\Am_2$, we say that the equation allows simulation (see \cite[Section~11]{SIAMJC:GroSah12}) and the proof is zero-knowledge (rather than just witness-indistinguishable). Note that this is always the case for equations other than PPEs.

\subsection{Commit-and-Prove Schemes} The GS proof system works as a commit-and-prove scheme: first the prover commits to 
all the variables in an equation with the \emph{Groth-Sahai commitment scheme}, defined in the next section, and then it ``proves'' that the committed values satisfy the equation. The commitment to an element in the vector space $A_i$ lives in another vector space $B_i$ of larger dimension where interesting decisional assumptions exist. To prove that the equation is satisfied the verifier checks some equations in these larger fields.

\subsection{Groth-Sahai Commitments} \label{sec:gs-comms}
 
\begin{definition} The Groth-Sahai commitment scheme in the group $\GG_\gamma$, $\gamma\in\{1,2\}$, is  specified by the following three algorithms 
	$(\mathsf{GS}.\algK,\mathsf{GS}.\Com,\mathsf{GS}.\algVrfy)$ such that:
	\begin{itemize} 
		\item  $\mathsf{GS}.\algK(gk,\dist_{2,2})$ is a randomized algorithm, which on input the group key $gk$ and the description of some matrix distribution $\dist_{2,2}$, outputs a commitment key $ck:=[\matr{U}]_\gamma=[(\vecb{u}_1||\vecb{u}_2)]_\gamma \in\GG_\gamma^{2\times 2}$, where $\matr{U}\gets\dist_{2,2}$.
		\item $\mathsf{GS}.\Com_{ck}(\mathsf{m};r)$ is a randomized algorithm which, on input a commitment key $ck=\bmatr{U}_\gamma$, and a message 
		$\mathsf{m}$ in the message space $\mathcal{M}_{ck}=A_\gamma$, it proceeds as follows. If $\mathsf{m}=m\in\Z_q$, it samples $r \gets \Z_q$ and outputs a commitment $\bvecb{c}_\gamma := m[\vecb{e}_2+\vecb{u}_1]_\gamma+r[\vecb{u}_2]_\gamma$ in the commitment space $\mathcal{C}_{ck}=\GG_\gamma^2$ and an opening $Op=r$. If $\mathsf{m}=[m]_\gamma\in\GG_\gamma$, it samples $\vecb{r} \gets \Z_q^2$ and outputs a commitment $\bvecb{c}_\gamma := [m]_\gamma\vecb{e}_2+[\matr{U}]_\gamma\vecb{r}$ in the commitment space $\mathcal{C}_{ck}=\GG_\gamma^2$ and an opening $Op=\vecb{r}$.
		\item $\mathsf{GS}.\algVrfy_{ck}([\vecb{c}]_\gamma,Op)$ is a deterministic algorithm which, on input the commitment key $ck=\bmatr{U}_\gamma$, a commitment $\bvecb{c}_\gamma$,  a message 
		$m \in \mathcal{M}_{ck}$ and an opening $Op$, outputs $1$ if $\bvecb{c}_\gamma=\GS.\Com_{ck}(m;Op)$
		and $0$ otherwise.
	\end{itemize}
\end{definition}

We will instantiate this commitment scheme with two different matrix distributions which give rise to two different commitment keys: the \emph{perfectly binding} and the \emph{perfectly hiding} commitment keys. The matrix distribution for perfectly binding commitment keys is defined as
$$
\mathcal{B}:\matr{U}=(\vecb{u}_1||\vecb{u}_2),\text{ where }\vecb{u}_2\gets\distlin_1\text{ and } \vecb{u}_1 :=\mu\vecb{u}_2, \mu\gets\Z_q,
$$
and the matrix distribution for perfectly hiding commitment keys is defined as
$$
\mathcal{H}:\matr{U}=(\vecb{u}_1||\vecb{u}_2),\text{ where } \vecb{u}_2\gets\distlin_1\text{ and } \vecb{u}_1 :=\mu\vecb{u}_2-\vecb{e}_2, \mu\gets\Z_q.
$$

\begin{theorem}[\cite{SIAMJC:GroSah12}] If $ck\gets\algK(gk,\mathcal{B})$ (resp. $ck\gets\algK(gk,\mathcal{H})$), where $gk\gets\G_a(1^\lambda)$ , the Groth-Sahai commitment scheme is perfectly binding (resp. computationally binding if the DL assumption relative to $\G_a$ holds) and computationally hiding if the SXDH assumption relative to $\G_a$ holds (resp. perfectly hiding).

\end{theorem}

\subsection{The Scheme} \label{sec:gs-proofs-scheme}
Next, we give a description of the Groth-Sahai proof system in the SXDH instantiation.
\begin{description}
\item[$\algK(gk)$:]  On input the group key $gk$ pick $\matr{U},\matr{V}\gets\mathcal{B}$ and define $ck_1 := [\matr{U}]_1$ and $ck_2:=[\matr{V}]_2$.
   The common reference string is:
   $\crs_\GS:=(gk,[\vecb{u}_1]_1,[\vecb{u}_2]_1,[\vecb{v}_1]_2,[\vecb{v}_2]_2)$ and is known as the \emph{perfectly binding CRS}.
The CRS defines some associated maps:
\begin{align*}
&\iota_1: \Gr \cup \Z_q \rightarrow \Gr^2,& &\iota_1([x]_1):=([x]_1,[0]_1)^\top,& &\iota_1(x):=x[\vecb{u}_1]_1. \\
&\iota_2: \Hr \cup \Z_q \rightarrow \Hr^2,& &\iota_2([y]_2):=([y]_2,[0]_2)^\top,& &\iota_2(y):=y[\vecb{v}_1]_2.\\
&\iota_T: \begin{matrix}\Z_q\cup\Gr\cup\\\Hr\cup\GG_T\end{matrix} \rightarrow \GG_T^{2\times2}, &
    &\iota_T(t):=\pmatri{t & 0 \\ 0 & 0},& & \iota_T(x) := \iota_1(x)\iota_2(1)^\top,\\
&                                       & &\iota_T([x]_1):=\iota_1([x]_1)\iota_2(1)^\top,& &\iota_T([y]_2) := \iota_1(1)\iota_2([y]_2)^\top.
\end{align*}
The maps $\iota_X$ $X \in \{1,2\}$ can be naturally extended to column vectors of arbitrary length, and we write $\iota_X(\boldsymbol \delta^{\top})$ for 
$(\iota_X(\delta_1)|| \ldots ||\iota_X(\delta_r))$.

\item[$\algP(\crs_\GS,\mathsf{eq},(\varx_1,\ldots,\varx_{m_x}),(\vary_1,\ldots,\vary_{m_y}))$:] Given some equation $\mathsf{eq}$ of the form \ref{gseq} and solutions to the equation
$\varx_1,\ldots,\varx_{m_x}$ and $\vary_1,\ldots,\vary_{m_y}$, the prover proceeds as follows:
\begin{itemize}
\item Commit to all $\varx_i \in A_1$ computing $([\vecb{c}_i]_1,\vecb{r}_i)\gets\GS.\Com_{ck_1}(\varx_i)$ and commit to all $\vary_i \in A_2$ computing $([\vecb{d}_i]_2,\vecb{s}_i)\gets\GS.\Com_{ck_2}(\vary_i)$.
\item Let $\matr{R} := (\vecb{r}_1||\ldots||\vecb{r}_{m_x})$ and $\matr{S} := (\vecb{s}_1||\ldots||\vecb{s}_{m_y})$. Compute 
\begin{eqnarray*}
[\matr{\Pi}]_2 
& := & \iota_2(\boldsymbol \beta^{\top}) \matr{R}^\top +\iota_2(\vecb{y}^{\top}) \matr{\EscQE}^\top  \matr{R}^{\top}+
[\matr{V}]_2\matr{S} \matr{\EscQE}^\top \matr{R}^\top - [\matr{V}]_2 \matr{T}^\top,\\
\  [\matr{\Theta}]_1 &  := &\iota_1(\boldsymbol \alpha^{\top}) \matr{S}^\top +\iota_1(\vecb{x}^{\top}) \matr{\EscQE} \matr{S}^{\top}+ [\matr{U}]_1 \matr{T}.
 \end{eqnarray*}
\end{itemize}
\item[{$\algV(\crs_\GS,\mathsf{eq},\{[\vecb{c}_i]_1:i\in[m_x]\},\{[\vecb{d}_i]_2:i\in[m_y]\},[\matr{\Theta}]_1,[\matr{\Pi}]_2)$}:] Check if
\begin{eqnarray*}
\sum_{i \in m_x} [\vecb{c}_i]_1 \iota_2(\beta_i)^{\top} + \sum_{j \in m_y} \iota_1(\alpha_j) [\vecb{d}_j]_2^{\top} 
+ \sum_{i \in m_x}\sum_{j \in m_y}\escQE_{i,j} [\vecb{c}_i]_1 [\vecb{d}_j]_2^{\top} = \\ \iota_T(t) +[\matr{\Theta}]_1[\matr{V}]_2^\top + [\matr{U}]_1[\matr{\Pi}]_2^\top.
\end{eqnarray*}
\item[$\algS_1(gk)$:]  On input the group key $gk$ pick $\matr{U},\matr{V}\gets\mathcal{H}$ and define $ck_1 := [\matr{U}]_1$ and $ck_2:=[\matr{V}]_2$.
   The common reference string is:
   $\crs_\GS:=(gk,[\vecb{u}_1]_1,[\vecb{u}_2]_1,[\vecb{v}_1]_2,[\vecb{v}_2]_2)$ and is known as the \emph{perfectly hiding CRS}. This algorithm also outputs the trapdoor $\tau:=(\matr{U},\matr{V})$.
\item[$\algS_2(\crs_\GS,\mathsf{eq},\tau)$:] If $\mathsf{eq}$ allows simulation, it defines the new equation
\begin{equation}
\mathsf{eq}':=\sum_{j=1}^{m_y} f(\alpha_j, \vary_j)+\sum_{i=1}^{m_x} f(\varx_i, \beta_i)-f(\varx',t)+\sum_{i=1}^{m_x} \sum_{j=1}^{m_y}  f(\varx_i,\escQE_{i,j} \vary_j)=0,
\end{equation}
and runs the GS prover for equation $\mathsf{eq}'$ for solutions $\varx_1=\varx_2=\ldots=\varx_{m_x}=\vary_1=\ldots=\vary_{m_y}=0$ and $\varx'=1$. The simulated proof is $([\matr{\Theta}]_1,[\matr{\Pi}]_2)$, the proof of the GS prover for the modified equation.

\end{description}
The reference string $\crs_\GS$ chosen by algorithm $\algK$ defines perfectly binding commitments and the proof system has perfect soundness. Furthermore, there exists some extraction trapdoor which allows to compute a function of the witness (this is the perfect \emph{F-Knowledge} property \cite{PKC:EscGro14}). More specifically, the extraction trapdoor
allows to compute maps $p_1: \Gr^2\to \Gr$, $p_2:\Hr^2 \to \Hr$, and $p_T:\GG_T^{2 \times 2} \to  \GG_T$, such that:
\begin{enumerate}[label=(\alph*)]
\item for each $X\in\{1,2,T\}$, $p_X\circ\iota_X$ is the identity map,
\item for all $\varx\in\Am_1,\vary\in\Am_2$, $\iota_1(\varx)\iota_2(\vary)^\top = \iota_T(f(\varx,\vary))$ and for all $[\vecb{x}]_1\in\Gr^2,[\vecb{y}]_2\in\Hr^2$,
$f(p_1([\vecb{x}]_1),p_2([\vecb{y}]_2))=p_T([\vecb{x}]_1[\vecb{y}]_2^\top)$,
\item for each $i\in[2]$, $p_1([\vecb{u}_i]_1)=0$ and $p_2([\vecb{v}_i]_2)=0$.
\end{enumerate}

\begin{theorem}[\cite{EC:GroSah08}] If $\eq$ allows simulation then the Groth-Sahai proof system is a composable zero-knowledge proof system with perfect completeness, perfect soundness, and computational zero-knowledge based on the SXDH assumption. Otherwise, is a composable witness-indistinguishable proof system with perfect completeness, perfect soundness, and computational witness indistinguishability based the SXDH assumption.
\end{theorem}



    \section{QA-NIZK Arguments of Membership in Subspaces of $\GG_1$ or $\GG_2$} \label{sect:QANIZKlinspace}

        In this section we recall the two constructions of QA-NIZK arguments of membership in linear spaces given by 
Kiltz and Wee \cite{EC:KilWee15}, for the language:
 $$\mathcal{L}_{[\matr{M}]_1}:=\{ [\vecb{x}]_1 \in \Gr^{n}:  \exists \vecb{w} \in \Z_q^{t}, \  [\vecb{x}]_1=[\matr{M}]_1\vecb{w} \}.$$ 
\noindent Algorithm $\algK_0(1^\lambda)$ just outputs $gk := (q,\Gr,\Hr,\GG_T,e,\mathcal{P}_1,\mathcal{P}_2) \leftarrow \ggen_a(1^{\lambda})$, the rest of the algorithms are described in Fig.~\ref{fig:QANIZKlinear}. 

\begin{figure} 
$$
\begin{array}{ll}
\begin{array}{l}
\underline{\algK_1(gk,[\matr{M}]_1,n) \qquad  (\mathsf{S}_1(gk,[\matr{M}]_1,n)})\\[.1cm]

\matr{A} \gets \widetilde{\dist_{k}}, \matr{\Delta} \gets \Z_q^{\tilde{k} \times n}\\
{[\matr{A}_{\Delta}]_2:= \matr{\Delta}^\top[\matr{A}]_2, [\matr{M}_{\Delta}]_1:=\matr{\Delta} [\matr{M}]_1} \\
\text{Return} \ \mathsf{crs}:=([\matr{M}_{\Delta}]_1, [\matr{A}_\Delta]_2, [\matr{A}]_2)\\ [.1cm]
(\tau_{sim}:=\matr{\Delta})
\end{array}
&
\begin{array}{l}
\underline{\algP(\mathsf{crs},[\vecb{x}]_1, \vecb{w}) \  \backslash \backslash [\vecb{x}]_1=[\matr{M}]_1 \vecb{w}} \\[.1cm]

 \text{Return } [\grkb{\sigma}]_1:=[\matr{M}_{\Delta}]_1 \vecb{w}.
\\
\\
\\
\vspace*{.2cm}
\end{array}
\\
\\
\begin{array}{l}
\underline{\mathsf{S}_2(\crs,[\vecb{x}]_1,\tau_{sim})}   \\ [.1cm]
  \text{Return} \ [\grkb{\sigma}]_1:= \matr{\Delta} [\vecb{x}]_1
\end{array}
&
\begin{array}{l}
\underline{\algV(\mathsf{crs},[\vecb{x}]_1,[\grkb{\sigma}]_1)}\\ [.1cm]
\text{Return} \ ([\vecb{x}]_1^\top [\matr{A}_\Delta]_2 = [\grkb{\sigma}]_1^\top[\matr{A}]_2)
\end{array}
\end{array}
$$
\caption{The figure describes $\ps$ when $\widetilde{\dist_{k}}=\dist_k$ and  $\tilde{k}=k+1$ and $\psws$ when $\widetilde{\dist_{k}}=\overline{\dist}_k$ and  $\tilde{k}=k$. Both are QA-NIZK arguments for $\mathcal{L}_{[\matr{M}]_1}$. 
$\ps$  is the construction of \cite[Sect.~3.1]{EC:KilWee15},
  which is a generalization 
of Libert \textit{et al}'s QA-NIZK \cite{EC:LPJY14} to any $\dist_k\mbox{-}\fmdh_{\Hr}$ Assumption. $\psws$ is the construction of  \cite[Sect.\ 3.2.]{EC:KilWee15}%
  .}
\label{fig:QANIZKlinear}
\end{figure}

\begin{theorem}[Theorem 1 of \cite{EC:KilWee15}] If $\widetilde{\dist_{k}}=\dist_k$ and $\tilde{k}=k+1$,  Fig. \ref{fig:QANIZKlinear} describes a QA-NIZK
proof system with perfect completeness, computational adaptive soundness based on the  $\dist_{k}$-$\fmdh{}_{\Hr}$ Assumption, perfect zero-knowledge, and proof size $k+1$. 
\label{theo:qanizk1}
\end{theorem}

\begin{theorem}[Theorem 2 of \cite{EC:KilWee15}] If $\widetilde{\dist_{k}}=\overline{\dist}_k$ and $\tilde{k}=k$,  and $\dist_{gk}$ is a witness samplable distribution, Fig. \ref{fig:QANIZKlinear} describes a QA-NIZK
proof system with perfect completeness, computational adaptive soundness based on the  $\dist_{k}$-$\fmdh{}_{\Hr}$ Assumption, perfect zero-knowledge, and proof size~$k$. 
\label{theo:qanizk2}
\end{theorem}


    
\chapter{QA-NIZK Arguments of Membership in Subspaces of $\GG_1\times\GG_2$}\label{sec:agg-asym}
    
    
In this chapter we construct three QA-NIZK \emph{constant-size} arguments of membership in different subspaces of $\GG^{m}_1 \times \GG^{n}_2$. Their soundness relies on the Split Kernel Assumption. We then show that similar techniques allow to give a \emph{constant-size} proof of satisfiability of many linear equations (aggregation of Groth-Sahai proofs). Finally, we show that the same techniques also allow to build \emph{Structure Preserving Linearly Homomorphic Signatures} where messages can be elements of $\GG_1^m\times\GG_2^n$, an extension of the signature scheme introduced by Libert et al. \cite{C:LPJY13}. 
\section{Introduction}
A recent line of work 
  \cite{AC:JutRoy13,C:JutRoy14,EC:KilWee15,EC:LPJY14} 
has succeeded in constructing constant-size  
  arguments for very specific statements, namely, for membership in subspaces of $\Gr^{m}$, 
  where $\Gr$ is some group equipped with a bilinear map where the discrete logarithm is hard. 
The soundness of the schemes is based on standard, falsifiable assumptions 
  and the proof size is independent of both $m$ and the witness size.  These improvements are in a  \textit{quasi-adaptive} 
  model (QA-NIZK, \cite{AC:JutRoy13}).  This means that the common reference string of these proof systems is 
  specialized to the linear space where one wants to prove membership.
  
Interestingly, Jutla and Roy  \cite{C:JutRoy14} also showed that their techniques to construct 
  constant-size NIZK in linear spaces can be used to aggregate the GS proofs of $m$ equations in $n$ variables, that is,  the total proof size can be reduced to $\Theta(n)$.  Aggregation is also quasi-adaptive, 
which means that the common reference string depends on the set of equations one wants to aggregate.   Further, it is only possible if the equations meet some restrictions. The first one is that only linear equations can be aggregated. The second one is that, in asymmetric bilinear groups, the equations must be one-sided linear, i.e. linear equations 
  which have variables in only one of the $\Z_q$ modules $\Gr,\Hr$, 
  or $\Z_q$.\footnote{Jutla and Roy show how to aggregate two-sided linear equations in
  symmetric bilinear groups. The asymmetric case is not discussed, 
  yet for one-sided linear equations it can be easily  derived from
  their results. 
  This is not the case for two-sided ones, see Sect.\ \ref{sec:gsasym}.} 


Thus, it is worth to investigate if we can develop new techniques to aggregate 
other types of equations and recover all the aggregation results of \cite{C:JutRoy14} (in particular, for two-sided linear equations) in asymmetric bilinear groups. The latter (Type III bilinear groups, according to the classification of \cite{DAM:GalPatSma08}) are the most 
attractive 
from the perspective of a performance and security trade off, specially since the recent attacks on discrete logarithms in finite fields by Joux \cite{SAC:Joux13} and subsequent improvements. Considerable research effort 
(e.g. \cite{C:AGOT14a,EC:Freeman10})
has been put into translating pairing-based cryptosystems from a setting with more structure in which design is simpler (e.g. composite-order or symmetric bilinear groups) to a more efficient setting (e.g. prime order or asymmetric bilinear groups). In this line, we aim not only at obtaining new results in the asymmetric setting but also to translate known results and develop new tools specifically designed for it which might be of independent interest.




    \section{Argument of Membership in Subspace Concatenation}\label{sec:concat}

        Figure \ref{fig:QANIZKtwogroups} describes a QA-NIZK argument of membership in the language 
$$\mathcal{L}_{[\matr{M}]_1,[\matr{N}]_2}:=\{ ([\vecb{x}]_1,[\vecb{y}]_2): \exists \vecb{w} \in \Z_q^{t}, \ \vecb{x}=\matr{M}\vecb{w},   \vecb{y}=\matr{N}\vecb{w} \} \subseteq \GG_1^{m} \times \GG_2^{n},$$
where $([\matr{M}]_1,[\matr{N}]_2)\gets\dist_{gk}$ for some matrix distribution $\dist_{gk}.$

We refer to this as the \textit{concatenation language}, because 
if we define $\matr{P}$ as the concatenation of $[\matr{M}]_1,[\matr{N}]_2$, that is $\matr{P}:=\smallpmatrix{[\matr{M}]_1 \\ [\matr{N}]_2}$, then  $([\vecb{x}]_1,[\vecb{y}]_2) \in \mathcal{L}_{[\matr{M}]_1,[\matr{N}]_2}$ iff $\smallpmatrix{[\vecb{x}]_1 \\ [\vecb{y}]_2}$ is in the span of $\matr{P}$.
\begin{figure}
$$
\begin{array}{l}
\begin{array}{l}
    \underline{\algK(gk,[\matr{M}]_1,[\matr{N}]_2,m,n) \quad  (\mathsf{S}_1(gk,[\matr{M}]_1,[\matr{N}]_1,m,n))}\\[.1cm]
    \matr{A} \gets \widetilde{\dist_{k}},\matr{\Lambda} \gets \Z_q^{\tilde{k} \times m}, \matr{\Xi}\gets \Z_q^{\tilde{k}\times n}, \matr{Z} \gets \Z_q^{\tilde{k} \times t}\\
    {[\matr{A}_{\Lambda}]_2:= \matr{\Lambda}^\top [\matr{A}]_2},{[\matr{A}_{\Xi}]_1:= \matr{\Xi}^\top[\matr{A}]_1}\\
    {[\matr{M}_{\Lambda}]_1:=\matr{\Lambda} [\matr{M}]_1+[\matr{Z}]_1},{[\matr{N}_{\Xi}]_2:=\matr{\Xi} [\matr{N}]_2-[\matr{Z}]_2} \\ [.1cm]
    {\text{Return } \ \mathsf{crs}:=([\matr{M}_{\Lambda}]_1, [\matr{A}_\Lambda]_2 , [\matr{A}]_2,[\matr{N}_{\Xi}]_2,}{[\matr{A}_{\Xi}]_1 , [\matr{A}]_1).}\\
    (\tau_{sim}:=(\matr{\Lambda},\matr{\Xi}).)
    \\
    \\    
\end{array}
\\
\begin{array}{l}
    \underline{\algP(\mathsf{crs}, [\vecb{x}]_1, [\vecb{y}]_2, \vecb{w})}\backslash \backslash ([\vecb{x}]_1=[\matr{M}]_1 \vecb{w}, [\vecb{y}]_2=[\matr{N}]_2 \vecb{w})
    \\
    \matr{z} \gets \Z_q^{\tilde{k}}\\
    {[\grkb{\rho}]_1:=[\matr{M}_{\Lambda}]_1 \vecb{w}+ [\vecb{z}]_1},{[\grkb{\sigma}]_2:=[\matr{N}_{\Xi}]_2 \vecb{w}- [\vecb{z}]_2}\\
     \text{Return } \  ([\grkb{\rho}]_1,[\grkb{\sigma}]_2).
    \\
    \\
\end{array}
\\
\begin{array}{l}
    \underline{\algV(\mathsf{crs},([\vecb{x}]_1,[\vecb{y}]_2), ([\grkb{\rho}]_1,[\grkb{\sigma}]_2))}\\ [.1cm]
    \text{Return } \ ([\vecb{x}^\top]_1 [\matr{A_\Lambda}]_2  - [\grkb{\rho^\top}]_1[\matr{A}]_2=  [\grkb{\sigma^\top}]_2 [\matr{A}]_1-[\vecb{y}^\top]_2 [\matr{A_\Xi}]_1).
    \\
    \\
\end{array}
\\
\begin{array}{l}
    \underline{\mathsf{S}_2(\crs,([\vecb{x}]_1,[\vecb{y}]_2),\tau_{sim})}\\[.1cm]
    \vecb{z} \gets \Z_q^{\tilde{k}} \\ 
    {[\grkb{\rho}]_1:=\matr{\Lambda} [\vecb{x}]_1+ [\vecb{z}]_1},{[\grkb{\sigma}]_2:=\matr{\Xi} [\vecb{y}]_2- [\vecb{z}]_2}\\
    {\text{Return }  ([\grkb{\rho}]_1,[\grkb{\sigma}]_2).}
\end{array}
\\
\end{array}
$$
\caption{Two QA-NIZK arguments for  $\mathcal{L}_{[\matr{M}]_1,[\matr{N}]_2}$. $\sps$ is 
 defined for $\widetilde{\dist_{k}}=\dist_k$ and $\tilde{k}=k+1$, and is a generalization of Kiltz and Wee's construction \cite[Section~3.1]{EC:KilWee15}  in two groups. The second construction $\spsws$ corresponds to $\widetilde{\dist_{k}}=\overline{\dist_k}$ and $\tilde{k}=k$, and is a generalization of Kiltz and Wee's contraction \cite[Section~3.2]{EC:KilWee15} in two groups. Computational soundness is based on the $\dist_{k}$-\skermdh{} assumption. 
$\matr{M}$ is matrix of size $m\times t$, $\matr{N}$ is of size $n\times t$, $\matr{A}$ is of size $\tilde \times k$, $\matr{\Lambda}$ is of size $\tilde{k}\times m$, $\matr{\Xi}$ is of size $\tilde{k}\times n$, $\matr{M}_\Lambda$ is of size $\tilde{k}\times t$, $\matr{N}_\Xi$ is of size $\tilde{k}\times t$, $\matr{A}_\Lambda$ is of size $m\times k$, and $\matr{A}_\Xi$ is of size $n\times k$.
The CRS size is $(\tilde{k}k+\tilde{k}t+mk)\sG+(\tilde{k}k+\tilde{k}t+nk)\sH$ and the proof size $\tilde{k}\s$. Verification requires $2\tilde{k}k+(m+n)k$ pairing computations. We denote by $\Sps$ the most efficient instantiation of $\sps$, which happens when $\dist_k=\bar{\distlin}_2$. \label{fig:QANIZKtwogroups} }
\end{figure}

\paragraph{Soundness Intuition.}   If we ignore for a moment that $\Gr, \Hr$ are different groups, $\sps$ (resp. $\spsws$) is almost identical to $\ps$ (resp. to $\psws$), as defined in Section~\ref{sect:QANIZKlinspace}, for the language $\mathcal{L}_{[\matr{P}]_1}$, and $\matr{\Delta}:=(\matr{\Lambda}\cat\matr{\Xi})$, where  $\matr{\Lambda} \in \Z_q^{\tilde{k} \times m},\matr{\Xi} \in \Z_q^{\tilde{k} \times n}$. Further, the information that an unbounded adversary can extract from the CRS about $\matr{\Delta}$ is:
 \begin{enumerate}
 \item $\Big\{\matr{P}_\Delta= \matr{\Lambda} \matr{M}+ \matr{\Xi}\matr{N}, \matr{A}_{\Delta} = \matr{\Delta}^{\top}\matrA =\begin{pmatrix} \matr{\Lambda}^{\top}\matr{A} \\ \matr{\Xi}^{\top}\matr{A} \end{pmatrix}\Big\}$ from $\mathsf{crs}_{\ps}$, 
 \item $\Big\{ \matr{M}_\Lambda = \matr{\Lambda} \matr{M}+\matr{Z}, \matr{N}_\Xi = \matr{\Xi}\matr{N}-\matr{Z},  \begin{pmatrix}\matr{A}_\Lambda\\\matr{A}_\Xi\end{pmatrix} = \begin{pmatrix} \matr{\Lambda}^{\top}\matr{A} \\ \matr{\Xi}^{\top}\matr{A} \end{pmatrix} \Big\}$ from $\mathsf{crs}_{\sps}$. 
 \end{enumerate}
Given that the matrix $\matr{Z}$ is uniformly random,  $\mathsf{crs}_{\ps}$ and $\mathsf{crs}_{\sps}$
reveal the same information about $\matr{\Delta}$ to an unbounded adversary. Therefore, as the proof of soundness is essentially based on the fact that parts of $\matr{\Delta}$ are information theoretically hidden to the adversary, the original proof of Kiltz and Wee can be easily adapted for the new arguments. 

\begin{theorem} If $\widetilde{\dist_{k}}=\dist_k$ and $\tilde{k}=k+1$,  Fig. \ref{fig:QANIZKtwogroups} describes a QA-NIZK
proof system with perfect completeness, computational adaptive soundness based on the  $\dist_{k}$-$\skermdh$ assumption, and perfect zero-knowledge. 
\label{theo:membtwogroups1}
\end{theorem}

\begin{proof} Through this proof we define $\matr{P}:=\pmatri{\matr{M}\\\matr{N}}$, $\matr{\Delta}:=\pmatri{\matr{\Lambda}& \matr{\Xi}}$ and $\tilde{m}:=m+n$.

(Completeness.) Follows from the fact that
\begin{eqnarray*}
\vecb{x}^\top\matr{A}_\Lambda+\vecb{y}^\top\matr{A}_\Xi
& = &
\left(
\matr{P}\vecb{w}
\right)^\top
\matr{\Delta}^\top
\matr{A}\\
& = &
\left(
\begin{pmatrix}\matr{\Lambda}&\matr{\Xi}\end{pmatrix}
\begin{pmatrix}\matr{M}\\\matr{N}\end{pmatrix}\vecb{w}
\right)^\top
\matr{A}\\
& = &
\grkb{\rho}^\top\matr{A}+\grkb{\sigma}^\top\matr{A}.
\end{eqnarray*}
(Soundness.) $\advB$ receives a challenge $([\matr{A}]_1,[\matr{A}]_2)$, $\matrA \gets \dist_k$, and then it chooses $\matr{\Lambda}\gets\Z_q^{(k+1)\times m},\matr{\Xi}\gets\Z_q^{(k+1)\times n}$, samples $([\matr{M}]_1,[\matr{N}]_2) \gets \dist_{gk}$, $[\matr{M}]_1\in\GG_1^{m\times t},[\matr{N}]_2\in\GG_2^{n\times t}$, and computes 
\begin{align*}
\mathsf{crs}:=&([\matr{M}_{\Lambda}]_1, [\matr{A}_\Lambda]_2 ,[\matr{A}]_2, [\matr{N}_{\Xi}]_2,[\matr{A}_{\Xi}]_1,[\matr{A}]_1)\\
&\in\GG_1^{(k+1)\times t}\times\GG_2^{m\times k}\times \GG_2^{(k+1)\times k}\times \GG_2^{(k+1)\times t}\times \GG_1^{n\times k}, \GG_1^{(k+1)\times k}
\end{align*}
in the natural way.
An adversary $\Forger$ against the soundness property outputs a vector $([{\vecb{x}}^*]_1,[\vecb{y}^*]_2) \notin \mathcal{L}_{[\matr{M}]_1,[\matr{N}]_2}$ and a valid proof
$([{\boldsymbol \rho}^{*}]_1, [{\boldsymbol \sigma}^{*}]_2)$. At this point, $\advB$ computes its own proof $([{\boldsymbol \rho}^{\dagger}]_1, [{\boldsymbol \sigma}^{\dagger}]_2)$ using $\matr{\Lambda}$ and $\matr{\Xi}$. The adversary $\advB$ will output as a response to the $\dist_k$-\skermdh{} challenge the pair $([\vecb{r}]_1,[\vecb{s}]_2):=([{\boldsymbol \rho}^{*}]_1-[{\boldsymbol \rho}^{\dagger}]_1,[{  \boldsymbol \sigma}^{\dagger}]_2-[{\boldsymbol \sigma}^{*}]_2)$. We now see that with all but probability $1/q$, this is a valid 
solution. Indeed, if $\vecb{r} \neq \vecb{s}$, we are done, because since both are valid proofs, subtraction of the verification equations yields
\begin{equation*}
([{\boldsymbol \rho}^{*}]_1-[{\boldsymbol \rho}^{\dagger}]_1)^{\top} [\matr{A}]_2=  ([{\boldsymbol \sigma}^{\dagger}]_2-[{\boldsymbol \sigma}^{*}]_2)^{\top} [\matr{A}]_1.
\end{equation*}
By definition $\vecb{r} \neq \vecb{s}$  if and only if  $\boldsymbol \rho^{*}+\boldsymbol \sigma^{*} \neq 
\boldsymbol \rho^{\dagger}+ \boldsymbol \sigma^{\dagger}$.
Note that 
\begin{equation*}
\boldsymbol \rho^{\dagger}+ \boldsymbol \sigma^{\dagger}=\matr{\Lambda}  \vecb{x}^*+ \matr{\Xi} \vecb{y}^*=
\matr{\Delta} \vecb{t},\text{ where }\vecb{t}:=\begin{pmatrix} \vecb{x}^* \\ \vecb{y}^* \end{pmatrix}.
\end{equation*}
Since $\matr{Z}$ is a uniform random value, the CRS reveals (information theoretically) only $\left\{\matr{\Delta}\matr{P}:=\matr{P}_\Delta,  \matr{\Delta}^{\top}\matrA:=\matr{A}_\Delta \right\}$ about $\matr{\Delta}$. From $\matr{\Delta}^\top \matr{A} = \matr{A}_\Delta$, an (unbounded) adversary might only deduce that
$$
\matr{\Delta}_1 = (\matr{A}_1^\top)^{-1}(\matr{A}_\Delta^\top-\matr{A}_2^\top\matr{\Delta}_2),
$$
where $\matr{\Delta}_1\in\Z_q^{k\times \tilde{m}},\matr{\Delta}_2\in\Z_q^{1\times\tilde{m}},\matr{A}_1\in\Z_q^{k\times k},\matr{A}_2\in\Z_q^{k+1\times 1}$ and $\matr{\Delta}=\pmatri{\matr{\Delta}_1\\\matr{\Delta}_2},\matr{A}=(\matr{A}_1 \matr{A}_2)$. Therefore, $\matr{\Delta}_2$ remains completely hidden given only $\matr{A}_\Delta$. Note also that the first $k$ rows of $\matr{P}_\matr{\Delta}$ (i.e.~$\matr{\Delta}_1\matr{P}$) are completely determined by the last row of $\matr{P}_\Delta$ (i.e.~$\matr{\Delta}_2\matr{P}$) and the CRS, since
$$
\matr{\Delta}_1\matr{P}=(\matr{A}_0^\top)^{-1}(\matr{A}_\Delta^\top\matr{P}-\matr{A}_1^\top\matr{\Delta}_2\matr{P}).
$$
Let $\matr{P^\perp}\in\Z_q^{\tilde{m}\times(\tilde{m}-r)}$, where $r=\mathsf{rank}(\matr{P})$, a basis of the kernel of $\matr{P}$.
The row vector $\matr{\Delta}_2$ can be always written as $\matr{\Delta}_2^\top=\matr{P}\vecb{w}_1+\matr{P}^\perp\vecb{w}_2$, where $\vecb{w}_1\in\Z_q^{t},\vecb{w}_2\in\Z_q^{\tilde{m}-r}$ are uniformly random vectors. It follows that $\vecb{w}_2$ is completely hidden given only the CRS since $\matr{\Delta}_2\matr{P}=\vecb{w}_1^\top\matr{P}^\top\matr{P}$. Since $\vecb{t}\notin\mathsf{span}(\matr{P})$, there exists some $\vecb{w}'_1\in\Z_q^t,\vecb{w}'_2\in\Z_q^{\tilde{m}-r}$ such that $\vecb{w}'_2\neq \vecb{0}$ and $\vecb{t} = \matr{P}\vecb{w}'_1+\matr{P^\perp}\vecb{w}'_2$.
Finally, note that
\begin{align*}
\matr{\Delta}_2\vecb{t} &=(\matr{P}\vecb{w}_1+\matr{P^\perp}\vecb{w}_2)^\top(\matr{P}\vecb{w}'_2+\matr{P^\perp}\vecb{w}'_2)\\
&= \vecb{w}_1^\top\matr{P}^\top\matr{P}\vecb{w}'_1+\vecb{w}_2^\top(\matr{P}^\perp)^\top\matr{P^\perp}\vecb{w}'_2
\end{align*}
where the non-zero component of $\vecb{w}'_2$ is multiplied by a component of (the random vector) $\vecb{w}_2$ and thus, $\matr{\Delta}_2\vecb{t}$ is uniformly distributed over $\Z_q$. It follows that $\Pr[\matr{\Delta}\vecb{t}\neq\grkb{\rho}^*+\grkb{\rho}^*]\geq 1- 1/q$.

(Zero-Knowledge.) It is direct from the construction that the simulated proof follows the same distribution as an honestly computed proof.
\end{proof}


\begin{theorem} If $\widetilde{\dist_{k}}=\overline{\dist}_k$ and $\tilde{k}=k$,  and $\dist_{gk}$ is a witness samplable distribution, Fig. \ref{fig:QANIZKtwogroups}
describes a QA-NIZK proof system with perfect completeness, computational adaptive strong soundness based on the  $\dist_{k}$-$\skermdh$ assumption, and perfect zero-knowledge. 
\label{theo:membtwogroups2}
\end{theorem}

\begin{proof} (Completeness and Zero-Knowledge.) Equal as in Theorem \ref{theo:membtwogroups1}.

(Soundness.)
Define $\tm:=m+n$ and $\matr{P}:=\smallpmatrix{\matr{M}\\\matr{N}}$. 
An adversary $\advB$ against  
$\dist_{k}$-$\skermdh$ assumption receives a challenge $([\matr{A}]_1,[\matrA]_2)$, $\matrA \gets \dist_k$. 
It samples 
$([\matr{M}]_1,[\matr{N}]_2,\matr{M},\matr{N}) \in \R_{par}$ and computes  $\matr{P}^{\perp} \in \Z_q^{\tm \times (\tm-r)}$, where $r=\mathsf{rank}(\matr{P})$, a basis of the kernel of $\matr{P}^{\top}$. 
By definition, $\matr{P}^{\top}=(\matr{M}^{\top} \cat\matr{N}^{\top})$ and $\matr{P}^{\top}\matr{P}^{\perp} =\matr{0}$, thus we can write $\matr{P}^{\perp}= \smallpmatrix{\matr{E} \\ \matr{F}}$, for some matrices such that $\matr{M}^{\top}\matr{E}=-\matr{N}^{\top} \matr{F}$.

Adversary $\advB$ samples 
$\matr{R} \in \Z_q^{(\tm-r-1) \times (k+1)}$ and defines 
\begin{align*}
&[\matr{A}']_1:=\begin{pmatrix} [\matr{A}]_1 \\ \matr{R} [\matr{A}]_1 \end{pmatrix} \in \Gr^{(k+\tm-r) \times k},&
&[\matr{A}']_2:=\begin{pmatrix} [\matr{A}]_2 \\ \matr{R} [\matr{A}]_2 \end{pmatrix} \in \Hr^{(k+\tm-r) \times k}.
\end{align*} 

Then $\advB$ samples $(\widetilde{\matr{\Lambda}}\cat\widetilde{\matr{\Xi}}) \gets \Z_q^{k \times \tm}$.  Let $\matr{A}_0$ be the first $k$ rows of $\matrA'$ (or $\matr{A}$) and $\matrA_1'$ the rest of the rows, and $\matr{T}_{\matrA'}=\matrA'_1 \matrA_0^{-1}$. Then $\advB$ implicitly sets 
$(\matr{\Lambda} \cat \matr{\Xi}):= (\widetilde{\matr{\Lambda}} \cat \widetilde{\matr{\Xi}})+ \matr{T}^{\top}_{\matrA'} (\matr{E}^{\top} \cat \matr{F}^{\top})$, and computes:
\begin{equation}
\begin{pmatrix}
[\matrA_{\Lambda}]_2\\
[\matrA_{\Xi}]_1
\end{pmatrix}=
\begin{pmatrix}
\matr{\Lambda}^{\top} [\matrA_0]_2\\
\matr{\Xi}^{\top} [\matrA_0]_1
\end{pmatrix}:=
\begin{pmatrix}
(\widetilde{\matr{\Lambda}}^{\top}+ \matr{E} \matr{T}_{\matrA'}) [\matrA_0]_2 \\
(\widetilde{\matr{\Xi}}^{\top}+ \matr{F} \matr{T}_{\matrA'}) [\matrA_0]_1
\end{pmatrix}
=
\begin{pmatrix}
(\widetilde{\matr{\Lambda}}^{\top}\cat \matr{E}) [\matrA']_2 \\
(\widetilde{\matr{\Xi}}^{\top}\cat \matr{F} ) [\matrA']_1
\end{pmatrix}
\end{equation}
So far the argument is very similar to Kiltz and Wee's \cite[Section~3.2]{EC:KilWee15}, now comes an important difference. 
Adversary $\advB$ also needs to compute $\matr{\Lambda}[\matr{M}]_1+[\matr{Z}]_1$ and 
$\matr{\Xi}[\matr{N}]_2-[\matr{Z}]_2$. Although the adversary $\advB$ does not know how to 
compute $\matr{\Xi}\matr{N}$ or $\matr{\Lambda}\matr{M}$, it can compute their sum in $\Z_q$ as:
 $$\matr{\Xi}\matr{N}+\matr{\Lambda}\matr{M}=
    \left(
        (\widetilde{\matr{\Lambda}} \cat \widetilde{\matr{\Xi}})
        + \matr{T}^{\top}_{\matrA'}(\matr{E}^{\top}\cat\matr{F}^{\top})
    \right)
 \begin{pmatrix}
    \matr{M} \\ \matr{N}
 \end{pmatrix} =\widetilde{\matr{\Lambda}} \matr{M}+\widetilde{\matr{\Xi}} \matr{N}=:\matr{T}.  
 $$
Thus, $\advB$ picks $\matr{Z} \gets \Z_q^{k \times t}$ and outputs 
$[\matr{N}_{\Xi}]_2:=[\matr{T}]_2-[\matr{Z}]_2$ and  $[\matr{M}_{\Xi}]_1:=[\matr{Z}]_1$.
Now, when $\Forger$ outputs a valid proof for some $([\vecb{x}]_1,[\vecb{y}]_2) \notin \mathcal{L}_{[\matr{M}]_1,[\matr{N}]_2}$, it holds that:
\begin{align*}
[\vecb{x}^\top]_1 [\matr{A}_\Lambda]_2  - [\grkb{\rho}^\top]_1[\matr{A}_0]_2=  [\grkb{\sigma}^\top]_2 [\matr{A}_0]_1-[\vecb{y}^\top]_2 [\matr{A}_\Xi]_1   \Longleftrightarrow\\
[\vecb{x}^\top]_1 (\widetilde{\matr{\Lambda}}^{\top}\cat \matr{E})[\matr{A}']_2  - ([\grkb{\rho}^{\top}]_1\cat[\matr{0}_{1\times(\tm-r)}]_1)^\top[\matr{A}']_2 = \\
( [\grkb{\sigma}^\top]_2 \cat [\matr{0}_{1\times(\tm-r)}]_2)  [\matr{A}']_1-[\vecb{y}^\top]_2 (\widetilde{\matr{\Xi}}^{\top}\cat \matr{F} ) [\matrA']_1  \Longleftrightarrow\\
[\vecb{c}^{\top}]_1 [\matrA']_2=[\vecb{d}^{\top}]_2 [\matrA']_1,
\end{align*}
 where $[\vecb{c}^{\top}]_1:=([\vecb{x}^{\top}]_1\widetilde{\matr{\Lambda}}-[\grkb{\rho}^{\top}]_1\cat [\vecb{x}^{\top}]_1 \matr{E})$ and $ [\vecb{d}^{\top}]_2:=([\grkb{\sigma}^{\top}]_2-[\vecb{y}^{\top}]_2\widetilde{\matr{\Xi}}\cat -[\vecb{y}^{\top}]_2 \matr{F}).$

Obviously 
$$\vecb{c}-\vecb{d} \in \ker((\matrA')^{\top}) 
\Longleftrightarrow (\vecb{c}-\vecb{d})^{\top} \matr{A}'=0 \Longleftrightarrow (\vecb{c}_1^{\top}+\vecb{c}_2^{\top} \matr{R})-(\vecb{d}_1^{\top}+\vecb{d}_2^{\top} \matr{R})
 \in \ker(\matrA^{\top}).$$
while, by assumption,  $([\vecb{x}]_1,[\vecb{y}]_2) \notin \mathcal{L}_{[\matr{M}]_1,[\matr{N}]_2}$ and thus
$[\vecb{x}^{\top}]_1 \matr{E} \neq -[\vecb{y}^{\top}]_2 \matr{F}$.
 We conclude with an information-theoretic argument. Because $\matr{R}$ is only revealed to $\advB$ through $\matr{A}_\matr{R}:=\matr{R}\matr{A}$ it holds that
$$
\matr{R}_0 := \matr{A}_0^{-1}(\matr{A}_\matr{R}-\matr{r}_1\matr{A}_1),
$$
where $\matr{R}_0\in\Z_q^{(\tilde{m}-r-1)\times k},\matr{r}_1\in\Z_q^{\tilde{m}-r-1}$, $\matr{R}=(\matr{R}_0\cat\vecb{r}_1)$, $\matr{A}=\pmatri{\matr{A}_0\\\matr{A}_1}$, and $\vecb{r}_1$ remains completely hidden to the adversary.
Therefore,
\begin{align*}
\Pr[\vecb{c}_1^{\top}+\vecb{c}_2^{\top} \matr{R} = \vecb{d}_1^{\top}+\vecb{d}_2^{\top} \matr{R}] &=
\Pr[(\vecb{c}_2^\top-\vecb{d}_2^\top)\matr{R} = \vecb{d}^\top_1-\vecb{c}^\top_1]\\
&= \Pr[(\vecb{c}_2^\top-\vecb{d}_2^\top)\matr{R}_0,(\vecb{c}_2^\top-\vecb{d}_2^\top)\vecb{r}_1)=(\grkb{\alpha}^\top,\beta)]\\
&\leq \Pr[(\vecb{c}_2^\top-\vecb{d}_2^\top)\vecb{r}_1 = \beta]\\
&= 1/q
\end{align*}
where $\grkb{\alpha}\in\Z_q^k,\beta\in\Z_q$, and $(\grkb{\alpha}^\top,\beta):=\vecb{c}_2^\top-\vecb{d}_2^\top$.

We conclude that, with high probability, $([\vecb{c}_1^{\top}]_1+[\vecb{c}_2^{\top}]_1 \matr{R}),([\vecb{d}_1^{\top}]_2+[\vecb{d}_2^{\top}]_2 \matr{R})$ 
solves $\dist_{k}$-\skermdh{}.

This proves standard soundness. Strong soundness follows from the fact that the argument is essentially information theoretic. In particular, the knowledge of $(\matr{M},\matr{N})$ does not reveal additional information about $\matr{R}$. 
\end{proof}


    \section{Argument of Sum in Subspace}\label{sec:sum}
        
        We can adapt the previous construction to the \textit{Sum in Subspace} Language, 
 $$\mathcal{L}_{[\matr{M}]_1,[\matr{N}]_2,+}:=\{ ([\vecb{x}]_1,[\vecb{y}]_2) \in \GG_1^{m} \times \GG_2^{m} :  \exists \vecb{w} \in \Z_q^{t}, \  \vecb{x}+\vecb{y}=(\matr{M}+\matr{N}) \vecb{w}\}.$$
 We define two proof systems $\spsmas$, $\spswsmas$ as in Fig. \ref{fig:QANIZKtwogroups}, but now 
with $\matr{\Lambda}=\matr{\Xi}$. Also, we define $\Psi_\sfsum:=\Psi_\sfsum(\overline{\dist}_k)$ when $\dist_k=\distlin_2$.

Completeness and Zero-Knowledge are straightforward. Soundness follows from the same argument as before with the following differences. First, note that in the subspace concatenation case the information revealed to the adversary in the CRS was $\matr{\Lambda}\matr{M}+\matr{\Xi}\matr{N}$, $\matr{\Lambda}^\top\matr{A}$, and $\matr{\Xi}^\top\matr{A}$, and the information revealed now is $\matr{\Lambda}(\matr{M}+\matr{N}),\matr{\Lambda}^{\top}\matr{A}$. In the soundness proof for $\dist_k$ the key point was that if $([\vecb{x}^*]_1,[\vecb{y}^*]_2)\notin\Lang_{[\matr{M}]_1,[\matr{N}]_2,+}$, then $\matr{\Delta}\pmatri{\vecb{x}^*\\\vecb{y}^*}$, where $\matr{\Delta}=(\matr{\Lambda}||\matr{\Xi})$, was information theoretically hidden to the adversary. In this case $\vecb{x}^*+\vecb{y}^*\notin\Span(\matr{M}+\matr{N})$ and thus $\vecb{x}^*+\vecb{y}^*=(\matr{M}+\matr{N})\vecb{w}+\matr{P}^\bot\vecb{w}'$, where $\matr{P}^\bot$ is a basis of the kernel of $\matr{M}+\matr{N}$ and $\vecb{w}'\neq 0$. Then, $\matr{\Delta}\pmatri{\vecb{x}^*\\\vecb{y}^*}=\matr{\Lambda}(\vecb{x}^*+\vecb{y}^*)$ is information theoretically hidden because $\matr{\Lambda}\matr{P}^\bot$ is information theoretically hidden. In the soundness proof for $\overline{\dist}_k$, one needs to compute $\matr{P}^\bot$, a basis for the kernel of $\matr{M}+\matr{N}$, and defines $\matr{E}:=\matr{P}^\bot$ and $\matr{F}:=\matr{P}^\bot$. From this definitions it follows that $\matr{\Lambda}(\matr{M}+\matr{N})=\widetilde{\matr{\Lambda}}(\matr{M}+\matr{N})$ and that $([\vecb{x}]_1,[\vecb{y}]_2)\notin\Lang_{[\matr{M}]_1,[\matr{N}]_2,+}$ implies that $\vecb{x}^\top\matr{E}=-\vecb{y}^\top\matr{F}$. Given that the same information is revealed in the CRS, the proof of soundness follows as in Theorem \ref{theo:membtwogroups2}.


    \section{Argument of Equal Opening in Different Groups} \label{sec:aggcommit}

        Given $\sps$ or $\spsws$, it is direct to construct constant-size NIZK arguments of membership in:
$$\mathcal{L}_{\mathsf{com},[\matr{U}]_1,[\matr{V}]_2,\nu}:=\left\{([\vecb{c}]_1,[\vecb{d}]_2) \in \GG_1^{m} \times \GG_2^{n}:
\begin{array}{l}
    \exists (\vecb{w},\vecb{r},\vecb{s}) \text{ s.t. }\\
    {[\vecb{c}]_1= [\matr{U}]_1 \begin{pmatrix} \vecb{w}\\ \vecb{r} \end{pmatrix} \text{ and }}\\
    {[\vecb{d}]_2=[\matr{V}]_2 \begin{pmatrix} \vecb{w} \\ \vecb{s} \end{pmatrix}}
\end{array}
\right\},$$
where $[\matr{U}]_1\in \GG_1^{m \times \tilde{m}}$, 
$[\matr{V}]_2 \in \GG_2^{n \times \tilde{n}}$
and $\vecb{w} \in \Z_q^{\nu}$. The witness is 
 $(\vecb{w},\vecb{r},\vecb{s}) \in \Z_q^{\nu} \times \Z_q^{\tilde{m}-\nu} \times \Z_q^{\tilde{n}-\nu}$. This language is interesting because it can express the fact that 
$([\vecb{c}]_1,[\vecb{d}]_2)$ are commitments to the same vector 
$\vecb{w} \in \Z_q^{\nu}$ in different groups.
 
The construction is an immediate consequence of the observation 
that $\mathcal{L}_{\mathsf{com},[\matr{U}]_1,[\matr{V}]_2,\nu}$  can be rewritten as some concatenation language $\mathcal{L}_{[\matr{M}]_1,[\matr{N}]_2}$.
Denote by $\matr{U}_1$ the first $\nu$ columns of $\matr{U}$  and $\matr{U}_2$ the remaining ones, and $\matr{V}_1$ the first $\nu$ columns of $\matr{V}$ and $\matr{V}_2$ the remaining ones. If we define: 
\begin{equation*}
 \matr{M} := (\matr{U}_1||\matr{U}_2||\matr{0}_{m\times (\tilde{n}-\nu)}) \qquad
\matr{N} := (\matr{V}_1||\matr{0}_{n\times (\tilde{m}- \nu)}||\matr{V}_2).
\end{equation*}
then it is immediate to verify that $\mathcal{L}_{\mathsf{com},[\matr{U}]_1,[\matr{V}]_2,\nu}=\mathcal{L}_{[\matr{M}]_1,[\matr{N}]_2}$.

We denote as $\spswscomm$ the proof 
system for $\mathcal{L}_{\mathsf{com},\hmatr{U},\cmatr{V},\nu}$ which corresponds to $\spsws$ for $\mathcal{L}_{[\matr{M}]_1,[\matr{N}]_2}$, where $[\matr{M}]_1,[\matr{N}]_2$ are the matrices defined above. Also, we denote the proof system as $\Psi_\sfcom$ when we use $\Psi_\mathsf{spl}$. Note that for commitment schemes we can generally assume $[\matr{U}]_1,[\matr{V}]_2$ to be drawn from some witness samplable distribution. Therefore, it follows from Theorem~\ref{theo:membtwogroups2} that $\spswscomm$ satisfies the notion of strong soundness.  
 

    \section{Aggregation of Groth-Sahai Proofs}\label{sec:agg-gs}

        In this section we discuss two different ways to aggregate GS equations. The first is a direct application of the proof of equal commitment opening and is only valid for two-sided linear equations in $\Z_q$, the second is an extension of the results of Jutla and Roy for all other types of linear equations.

\subsection{Aggregating Two-Sided Linear Equations in $\Z_q$}
\label{sec:ts-zq}
We note that proving that $n$ pairs of GS commitments open (pairwise) to the same elements in $\Z_q$ is simply a special case of 
the proof of equal commitment opening in Section~\ref{sec:aggcommit}. Indeed, the concatenation of $n$ GS commitments is just a commitment to a vector of scalars. In particular,  given $\mathsf{crs}_\GS=({gk},[\vecb{u}_1]_1,[\vecb{u}_2]_1,[\vecb{v}_1]_2,[\vecb{v}_2]_2)$,  it is easy to see that  $n$ commitments to $x_i \in \Z_q$, which are of the form:
 $[\vecb{c}_i]_1= x_i [\vecb{u}_1]_1 +r_i [\vecb{u}_2]_1$ for some $r_i \in \Z_q$ (recall that $\iota_1(x_i)= x_i [\vecb{u}_1]_1$), can be written as  
 $$\begin{pmatrix} [\vecb{c}_1]_1 \\ \vdots \\ [\vecb{c}_n]_1 \end{pmatrix}=   \begin{pmatrix} [\vecb{u}_1]_1 & \ldots & [\vecb{0}]_1\\ \vdots & \ddots & \vdots \\   [\vecb{0}]_1 & \ldots & [\vecb{u}_1]_1  \end{pmatrix} \begin{pmatrix} x_1 \\ \vdots \\ x_n \end{pmatrix}
 + \begin{pmatrix} [\vecb{u}_2]_1 & \ldots & [\vecb{0}]_1\\ \vdots & \ddots & \vdots \\  [\vecb{0}]_1 & \ldots & [\vecb{u}_2]_1  \end{pmatrix}
 \begin{pmatrix} r_1 \\ \vdots \\ r_n \end{pmatrix},
 $$
and similarly the concatenation of $n$ commitments $[\vecb{d}_i]_2$, $i \in [\ell]$ can be written as $[\matr{V}^1]_2\vecb{y}+[\matr{V}^2]_2 \vecb{s}$, where $[\matr{V}^i]_2$ is the block-wise concatenation of $n$ copies of $[\vecb{v}_i]_2$.  

In particular, proving that $n$ GS commitments open to the same value can be also seen as the aggregation of the proof  of $n$ GS equations of the form $\varx_\ell -\vary_\ell=0$. The aggregation of any other set of two-sided linear equations in $\Z_q$
easily reduces to this case using the homomorphic properties of GS commitments. Indeed, given $n$ equations of the form:
 $$  \grkb{\alpha}_{\ell}^{\top} \vvary+ \vvarx^{\top}\boldsymbol \beta_{\ell} =t_\ell, \ \ell \in [n],$$
 and the commitments to a satisfying assignment (where the commitments to every coordinate of $\vvarx$ (resp. $\vvary$) are in $\Gr$ (resp. $\Hr$), it is easy to derive a commitment to $\vvarx^{\top} \grkb{\beta}_{\ell}- t_{\ell}$ in $\Gr$ and a commitment to  $\grkb{\alpha}_{\ell}^{\top} \vvary$ in $\Hr$ for all $\ell \in [n]$. Obviously, the equations are satisfied if for each $\ell$, these commitments open to the same value. 

%We insist that two-sided linear equations in $\Z_q$ are essential to prove quadratic statements in asymmetric bilinear groups. In particular, this result can be used to reduce the proof size that $n$ commitments open to a bit-string from $6n\s$ to $(4n+2)\s$.

\subsection{QA Aggregation of Other Equation Types} \label{sec:jutroyaggasym}
Jutla and Roy \cite{C:JutRoy14} show how to aggregate GS proofs of 
two-sided linear equations in symmetric bilinear groups. In the original construction of \cite{C:JutRoy14} soundness is based on a decisional assumption (a weaker variant of the $\lin{2}$ Assumption). Its natural generalization in asymmetric groups (where soundness is based on the SXDH Assumption) only enables to aggregate the proofs of one-sided linear equations. 

In this section, we revisit their construction. We give an alternative, simpler, proof of soundness under a computational assumption which avoids altogether the ``Switching Lemma" of \cite{C:JutRoy14}. Further, we extend it to two-sided equations in the asymmetric setting. For one-sided linear equations we can prove soundness under any kernel assumption and for two-sided linear equations, under any split kernel assumption.\footnote{The results of  \cite{C:JutRoy14} are based on what they call  the ``Switching Lemma". As noted in \cite{EPRINT:MorRafVil15}, it is implicit in the proof of this lemma that the same results can be obtained under computational assumptions.}

Let $A_1,A_2,A_T$ be $\Z_q$-vector spaces compatible with some 
Groth-Sahai equation as detailed in Section
\ref{sec:gs-proofs}.  Let $\dist_{{gk}}$ be a witness samplable distribution which outputs $n$ pairs of vectors
$(\grkb{\alpha}_\ell, \grkb{\beta}_\ell) \in A_1^{m_y} \times A_2^{m_x}$, $\ell \in [n]$, for some 
$m_x,m_y \in \mathbb{N}$. Given some fixed pairs $(\grkb{\alpha}_\ell,
\grkb{\beta}_\ell)$, we define, for each $\vecb{\tilde{t}} \in A_{T}^{n}$, the set of equations $\mathcal{S}_{\vecb{\tilde{t}}}$ as:
$$\mathcal{S}_{\vecb{\tilde{t}}}=\left\{ E_{\ell}(\vvarx,\vvary)=\tilde{t}_{\ell}: \ell \in [n]\right\},  \quad 
E_{\ell}(\vvarx,\vvary):=\sum_{j\in[m_y]} f(\alpha_{\ell,j}, \vary_j) + 
\sum_{i\in[m_x]}  f(\varx_i, \beta_{\ell,i}).$$  

We note that, as in \cite{C:JutRoy14}, we only achieve \textit{quasi-adaptive aggregation}, that is, the common reference string is specific to a particular set of equations. More specifically, it depends on the constants $\grkb{\alpha}_{\ell},\grkb{\beta}_{\ell}$ (but not on $\tilde{t}_{\ell}$, which can be chosen by the prover) and it can be used to aggregate the proofs of 
$\mathcal{S}_{\vecb{\tilde{t}}}$, for any~$\vecb{\tilde{t}}$.  

Given the equation types for which we can construct NIZK GS proofs (and not only NIWI proofs), there always exists (1) $t_{\ell} \in A_1$, such that $\tilde{t}_{\ell}=f(t_{\ell},\mathsf{base}_{2})$ or 
 (2) ${t}_{\ell} \in A_2,$ such that $\tilde{t}_{\ell}=f(\mathsf{base}_{1},t_{\ell})$, where $\mathsf{base}_{i}=1$ if $\Am_i=\Z_q$ and $\mathsf{base}_i=\mathcal{P}_i$ if $A_i=\GG_i$, $i \in \{1,2\}$ \cite{SIAMJC:GroSah12}. For simplicity, 
 in the construction we assume that (1) is the case, otherwise 
 change $\iota_2(a_{\ell,i}),  \iota_1(t_{\ell})$ for $\iota_1(a_{\ell,i}),  \iota_2(t_{\ell})$ in the construction below. 

\begin{description}
%\item[$\algK_0(1^\lambda)$:]  Return ${gk} := (q,\Gr,\Hr,\GG_T,e,\mathcal{P}_1,\mathcal{P}_2) \leftarrow \ggen_a(1^{\lambda})$.

%\item[$\dist_{gk}$:] $\dist_{gk}$ is some distribution over $n$ pairs of vectors  $(\grkb{\alpha}_{\ell}$, $\grkb{\beta}_{\ell}) \in A_1^{m_x} \times A_2^{m_y}$.

\item[$\algK({gk}, \mathcal{S}_{\vecb{\tilde{t}}})$:]
Let $\matr{A}=(a_{i,j}) \gets \dist_{n,k}$. Define 
$$\crs:=\left(\crs_{\GS}, \left\{\sum_{\ell\in[n]}\iota_1(a_{\ell,i} \grkb{\alpha}_\ell), \sum_{\ell\in[n]} \iota_2(a_{\ell,i} \grkb{\beta}_\ell), \big\{\iota_{2}(a_{\ell,i}): \ell \in [n]\big\}: i \in [k]\right\}\right)$$
\item[$\algP({gk}, \mathcal{S}_{\vecb{\tilde{t}}},\vecb{x},\vecb{y})$:] 
Given a solution $\vvarx=\vecb{x}$,  $\vvary=\vecb{y}$ to $\mathcal{S}_{\vecb{\tilde{t}}}$, the prover proceeds as follows:
\begin{itemize}
\item Commit to all $x_j \in A_1$ as $[\vecb{c}_j]_1\gets\mathsf{GS.Com}(x_j)$, and to all 
$y_j\in\Am_2$ as $[\vecb{d}_j]_2\gets\mathsf{GS.Com}(y_j)$.

\item For each $i \in [k]$, run the GS prover for the equation $\sum_{\ell\in[n]} a_{\ell,i} E_{\ell}(\vvarx,\vvary)= \sum_{\ell \in [n]} f(t_{\ell}, a_{\ell,i})$ to obtain the proof, which is a pair  $([\matr{\Theta}_i]_1,[\matr{\Pi}_i]_2)$.
\end{itemize}
Output 
$(\{[\vecb{c}_j]_1 : j \in [m_x]\}, \{[\vecb{d}_j]_2: j \in [m_y]\}, \{([\matr{\Pi}_{i}]_2,[\matr{\Theta}_{i}]_1) : i \in [k]\})$.
\item[$\algV(\crs,\mathcal{S}_{\vecb{\tilde{t}}},{\{[\vecb{c}_j]_1 : j\in[m_x]\},\{[\vecb{d}_j]_2:j\in[m_y]\}, \{[\matr{\Theta}_{i}]_1,[\matr{\Pi}_{i}]_2:i \in [k]\}})$:] For each $i \in [k]$, run the GS verifier for equation
$$\sum_{\ell\in[n]} a_{\ell,i} E_{\ell}(\vvarx,\vvary)= \sum_{\ell \in [n]} f(t_{\ell}, a_{\ell,i}).$$
\end{description}  

\begin{theorem}
The above protocol is a QA-NIZK proof system for two-sided linear equations.
\end{theorem}  
\begin{proof}\ 
(Completeness.)  Observe that for each $i\in[k]$
\begin{equation}
\sum_{\ell\in[n]} a_{\ell,i} E_{\ell}(\vvarx,\vvary)= \sum_{\ell\in[n]} a_{\ell,i}\tilde{t}_\ell=\sum_{\ell\in[n]}f(t_\ell,a_{\ell,i}).
\end{equation}  Completeness follows 
from the observation that to efficiently compute the proof, the GS Prover only needs, apart from a satisfying assignment to the equation, the randomness used in the commitments plus a way to compute the inclusion map of all involved constants, in this case $\iota_1(a_{\ell,i} \alpha_{\ell,j})$,
$\iota_2(a_{\ell,i} \beta_{\ell,j})$, and the latter is part of the CRS.
 
\noindent{(Soundness.)} We change to a game $\mathsf{Game}_{1}$ where we know the discrete logarithm of the GS commitment key, as well as the discrete logarithms of $(\grkb{\alpha}_{\ell},\grkb{\beta}_{\ell})$, $\ell \in [n]$. This is possible because they are both chosen from a witness samplable distribution.

We now prove that an adversary against the soundness in $\mathsf{Game}_{1}$ can be used to construct an adversary $\advB$ against the 
$\dist_{n,k}\mbox{-}\skermdh$ Assumption, where 
$\dist_{n,k}$ is the matrix distribution used in the CRS generation. 

$\advB$ receives a challenge $([\matr{A}]_1,[\matr{A}]_2)\in\Gr^{n\times k}\times\Hr^{n\times k}$. Given all the discrete logarithms that $\advB$ knows, it can compute a properly distributed CRS even without knowledge of the discrete logarithm of $[\matr{A}]_1$. The  soundness adversary outputs commitments $\{[\vecb{c}_j]_1:j\in[m_x]\},\{[\vecb{d}_{j}]_2:j\in[m_y]\}$ together with proofs $\{[\matr{\Theta}_{i}]_1,[\matr{\Pi}_{i}]_2:i \in [k]\}$, which are accepted by the verifier.

The adversary $\advB$ can use the discrete logarithm of the commitment keys to compute openings of $\{[\vecb{c}_j]_1:j\in[m_x]\},\{[\vecb{d}_j]_2:j\in[m_y]\}$ (or the corresponding translation to $\GG_s$ when $A_s=\mathbb{Z}_q, s\in\{1,2\}$). Let $[\vecb{x}]_1$ and $[\vecb{y}]_2$ the vectors of these commitments.
We claim that the pair  $([\grkb{\rho}]_1,[\grkb{\sigma}]_2) \in \Gr^{n} \times \Hr^n$, 
$[\grkb{\rho}]_1:=(\grkb{\beta}_1^\top[\vecb{x}]_1 -[t_1]_1,\ldots,
\grkb{\beta}_{n}^\top[\vecb{x}]_1 -[t_{n}]_1),  
[\grkb{\sigma}]_2:=(\grkb{\alpha}_1^\top[\vecb{y}]_2,\ldots,
\grkb{\alpha}_{n}^\top[\vecb{y}]_2)$, solves the  $\dist_{n,k}\mbox{-}\skermdh$ challenge and can be efficiently computed by $\advB$. 

First, observe that if the adversary is successful in breaking the soundness property, then $\grkb{\rho} \neq \grkb{\sigma}$. Indeed, 
if this is the case there is some index $\ell \in [n]$ such that 
$E_{\ell}(\vecb{x},\vecb{y}) \neq \tilde{t}_{\ell}$, which means that 
$\sum_{j\in[m_y]} f(\alpha_{\ell,j}, \vary_j) \neq 
\sum_{j\in[m_x]}  f(\varx_j, \beta_{\ell,j}) - f(t_{\ell},\mathsf{base}_{2})$.  
If we take discrete logarithms in each side of the equation, this inequality is exactly equivalent to
$\grkb{\rho} \neq \grkb{\sigma}$.


Further, because GS proofs have perfect soundness, $\vecb{x}$ and $\vecb{y}$ satisfy 
the equation $\sum_{\ell\in[n]} a_{\ell,i} E_{\ell}(\vvarx,\vvary)= \sum_{\ell \in [n]} f(t_{\ell}, a_{\ell,i})$, for all $i \in [k]$.
Thus, for all $i\in[k]$, 
\begin{equation}
\sum_{\ell\in[n]} [{a}_{\ell,i}]_2 \left(\grkb{\beta}_\ell^\top[\vecb{x}]_1 - [{t}_{\ell}]_1 \right)  = \sum_{\ell \in [n]} [{a}_{\ell,i}]_1
\left(\grkb{\alpha}^\top_\ell[\vecb{y}]_2\right),
\label{eq:gs-ker}
\end{equation}
which implies that $[\grkb{\rho}]_1[\matr{A}]_2=[\grkb{\sigma}]_2[\matr{A}]_1$.

\noindent{(Zero-Knowledge.)}  The same simulator of GS proofs can be used. Specifically
the simulated proof corresponds to $k$ simulated GS proofs.
\end{proof}

\subsubsection{One-Sided Equations.} In the case when $\grkb{\alpha}_{\ell}=\matr{0}$ and $\tilde{t}_{\ell}=f(t_{\ell},\mathsf{base}_{2})$ for some $t_{\ell} \in A_1$, for all $\ell\in[n]$, proofs can be aggregated under a standard Kernel Assumption (and thus, in asymmetric bilinear groups we can choose $k=1$). Indeed, 
in this case, in the soundness proof, the adversary $\advB$ receives $[\matr{A}]_2\in\Hr^{n\times k}$, an instance of the $\dist_{n,k}\mbox{-}\kermdh_{\Hr}$ problem. The adversary $\advB$ outputs $[\grkb{\rho}]_1:=(\grkb{\beta}_1^\top[\vecb{x}]_1 -[{t}_{1}]_1,\ldots,
\grkb{\beta}_{n}^\top[\vecb{x}]_1 -[{t}_{n}]_1) $ as a solution to the challenge. To see why this works, note that, when $\grkb{\alpha}_{\ell}=\matr{0}$ for all $\ell\in[n]$, equation (\ref{eq:gs-ker}) reads $\sum_{\ell\in[n]} [{a}_{\ell,i}]_2 \left(\grkb{\beta}_\ell^\top [\vecb{x}]_1 - [{t}_{\ell}]_1 \right)  = [{0}]_T$ and thus $[\grkb{\rho}]_1[\matr{A}]_2=[\matr{0}]_T$.  The case when $\grkb{\beta}_\ell=\matr{0}$ and $\tilde{t}_{\ell}=f(\mathsf{base}_{1},t_{\ell})$ for some $t_{\ell} \in A_2$, for all $\ell\in[n]$, is analogous. 

\subsubsection{Public Parameters.} The size of the CRS of the construction above depends on the number of elements needed to represent $[\matr{A}]_2$. In this sense, it is interesting to sample $[\matr{A}]_2$ from some family of matrix assumptions with good representation size. As we assume that $n>k$, it is interesting to instantiate this scheme with the \textit{Circulant Matrix Distribution} of \cite{EPRINT:MorRafVil15}, which has a representation size of $n$ --- independent of $k$. 


%\paragraph{Symmetric bilinear groups.}  The size of the CRS of the construction below depends on the number of elements needed to represent $\hmatr{A}$, while in \cite{C:JutRoy14}, $\hmatr{A}$ is a uniform matrix. This generalization allows to have shorter public parameters than the  \cite{C:JutRoy14} in the symmetric case if  $\hmatr{A}$  is sampled from a family of matrix assumptions with good representation size. In particular, if $\hmatr{A}$ is sampled from the circulant matrix dis

 
%In summary, this means that with the techniques of \cite{C:JutRoy14} and our ``Split'' type of assumptions, we can aggregate two-sided equations of all types under the same restrictions as \cite{C:JutRoy14} and the proof size is $k$ times the proof size of a single equation. 




    \section{Structure Preserving Linearly Homomorphic Signatures}

        \label{def:splh}
Linearly-homomorphic structure preserving signatures  \cite{C:AFGHO10,PKC:BFKW09} enable to sign group elements 
in $G$, where $G$ is a group and to publicly derive signatures of new elements which are a linear combination of other signed messages. We take Libert et al.'s definition \cite{C:LPJY13}, except that we do not identify the elements of $G$ with vectors in $\Gr^n$, for some group $\Gr$. The reason is that $G$ might be some space of the form $\Gr^m \times \Hr^n$. 

\begin{definition}[SPLHS scheme]
A linearly homomorphic structure-preserving signature scheme over the group $G$  consists
of a tuple of efficient algorithms $\SPLHSinst$$=$$(\SG,$ $\SN,$ $\SD,$ $\SV)$ for which the message space
is $\mathcal{M} := G$, with the following specifications.

\begin{description}

\item[$\SG(gk, n):$] is a randomized algorithm that takes as input a group key $gk$ and an integer $n$ and  outputs a key pair $(\pk, \sk)$. The public key $pk$ specifies a $\Z_q$ vector space $G$ of dimension $n$. 

\item[$\SN(\sk,\vecb{m})$:] is a possibly probabilistic algorithm that takes as input a private key $\sk$
 and $\vecb{m}\in G$. It outputs a signature $\grkb{\sigma}\in G$.

\item[$\SD(\pk,\{\omega_i,\grkb{\sigma}_i,\vecb{m}_i \}_{i\in[\ell]})$:] is a possibly probabilistic signature derivation algorithm. It
takes as input a public key $\pk$ as well as $\ell$ pairs $(\omega_i,\grkb{\sigma}_i)$, each of which
consists of a weight $\omega_i\in\Z_q$ and a signature $\grkb{\sigma_i}\in G$. The output is a signature
$\grkb{\sigma}\in G$ on the vector $\vecb{m} = \sum_{i\in[\ell]} \omega_i\vecb{m}_i $.

\item[$\SV(\pk, \vecb{m},  \grkb{\sigma})$:] is a deterministic algorithm that takes in a public key $\pk$,
a signature $\grkb{\sigma}$, and a vector $\vecb{m}$. It outputs 1 if $\grkb{\sigma}$ is deemed valid and 0 otherwise.
\end{description}
\end{definition}

Correctness is expressed by imposing that, for all security parameters $\lambda\in\N$, all integers $n\in\poly(\lambda)$
and all pairs $(\pk,\sk) \gets \SG(gk, n)$, the following holds:
\begin{enumerate}
\item For all $\vecb{m} \in G$, if $\grkb{\sigma} = \SN(\sk, \vecb{m})$, then we have $\SV(\pk, \vecb{m}, \grkb{\sigma}) = 1$.
\item For any $\ell > 0$ and any set of triples $\{(\omega_i, \grkb{\sigma}_i, \vecb{m}_i)\}_{i\in[\ell]}$,
if $\SV(\pk, \vecb{m}_i, \grkb{\sigma}_i) = 1$ for each $i \in [\ell]$, then
$\SV(
    \pk,
    \sum_{i\in[\ell]} \omega_i\vecb{m}_i,
    \SD(
        \pk,
        \{(\omega_i, \grkb{\sigma}_i)\}
        )
    )=1
$
\end{enumerate}

\label{splhs-unforgeability}

In order to get a uniform definition for different types of forgery, we will say that a pair
$(\vecb{m}^*,\grkb{\sigma}^*)$ is a forgery if $P(\vecb{m}^*, Q)=1$, where
$P$ is a predicate on $(\vecb{m}^*, Q)$ and $Q$ is the set of reveal queries
made by the adversary. We stress that the predicate $P$ is not always efficiently computable. For instance, for the scheme of Libert \textit{et al.} (\cite{C:LPJY13}), this predicate is $1$ iff $\vecb{m}^*$ is outside the linear span of 
previous queries, and this is, in general, hard to decide in the group $G$ (although it might be easy for some set $Q$).  

\begin{definition} A SPLHS scheme $\SPLHSinst = (\SG, \SN, \SV, \SD)$ is secure against type P adversaries if no PPT
adversary has non-negligible advantage in the game below:
\begin{enumerate}
\item The adversary $\advA$ chooses an integer $n \in \N$ and sends it to the challenger who runs
$\SG(gk, n)$
and obtains $(\pk,\sk)$ before sending $\pk$ to $\advA$.
\item On polynomially-many occasions, $\advA$ can interleave the following kinds of queries.
\begin{description}

\item[Signing queries:] $\advA$ chooses a vector $\vecb{m}\in G^n$.
The challenger picks a handle
$h$ and computes $\grkb{\sigma}\gets\SN(\sk, \vecb{m})$. It stores $(h,\vecb{m}, \grkb{\sigma})$
in a table $T$ and returns $h$.

\item[Derivation queries:] $\advA$ chooses a vector of handles $\vecb{h} = (h_1, \ldots , h_\ell)$ and a set of coefficients
$\{\omega_i\}_{i\in[\ell]}$. The challenger retrieves the tuples $\{(h_i,\vecb{m}_i, \grkb{\sigma}_i)\}_{i\in[\ell]}$
from $T$ and returns $\perp$ if one
of these does not exist. Otherwise, it computes
$\vecb{m} = \sum_{i\in[\ell]} \omega_i\vecb{m}_i$ and runs
$\grkb{\sigma}\gets\SD(\pk,\{(\omega_i,\grkb{\sigma}_i)\}_{i\in[\ell]})$.
It also chooses a handle $h$, stores $(h,\vecb{m}, \grkb{\sigma})$ in $T$ and returns $h$
to $\advA$.

\item[Reveal queries:] $\advA$ chooses a handle $h$. If no tuple of the form $(h,\vecb{m},\grkb{\sigma})$
exists in $T$, the challenger returns $\perp$. Otherwise, it returns $\grkb{\sigma}$
to $\advA$ and adds $(\vecb{m}, \grkb{\sigma})$ to the set $Q$.
\end{description}

\item $\advA$ outputs a signature $\grkb{\sigma}^*$ and a vector $\vecb{m}^*$.
The adversary $\advA$ wins if $P(\vecb{m}^*,Q)=1$.
\end{enumerate}
$\advA$’s advantage is its probability of success taken over all coin tosses.
\end{definition}

Libert \textit{et al.} also used a set $\mathcal{T}$ of tags in order to add up many instances of their signature scheme
in only one. For simplicity, we omit this parameter.


\subsection{One-Time LHSPS Signatures in Different Groups} \label{sec:newhomtwogroups}

The one-time linearly homomorphic signature of Libert, Peters and Yung \cite{EC:LPJY14}  implies a QA-NIZK argument for linear spaces. Similarly, our constructions of QA-NIZK proofs for membership in concatenated subspace and for sum in subspace (in the case where the space is not from a witness samplable distribution) is implied by a one-time structure preserving signature scheme with different security properties. 

In particular, for subspace concatenation, ``one-time" means that the adversary is unable to sign vectors which are not in the span of previously signed vectors, namely,
the adversary cannot output a signature for a pair
$([\vecb{x}]_1^*,[\vecb{y}]_2^*) \in \Gr^{m} \times \Hr^{n}$ if $((\vecb{x}^*)^{\top}\cat(\vecb{y}^*)^{\top})$ is
linearly independent from the 
vectors $(\vecb{x}_i^{\top}\cat\vecb{y}_i^{\top})$, $i \in [q_s]$, (the concatenation of two vectors), where $([\vecb{x}_i]_1,[\vecb{y}_i]_2)$ are the signing queries of the adversary. The discussion for the scheme which results from our 
Sum-in-Subspace QA-NIZK proof, results in a different notion of ``one-time" --- this is captured in the security definition by a different predicate $P$ ---, see discussion below. 

In either case, the size of the resulting signatures is $(k+1)\s$ under the $\skermdh$ assumption, but if security against random message attacks is sufficient (meaning that the signatures in the set $Q$ which are seen by the adversary are sampled uniformly at random), the signature size can be reduced to $k\s$ (essentially, in this case one can sample $\matrA$ from $\overline{\dist}_k$). This is inspired by the one-time constructions of structure-preserving signatures of Kiltz et al.~secure against random message attacks\cite{C:KilPanWee15}. 
 We omit any further discussion of this case, as it is a straightforward generalization of our QA-NIZK proofs in the witness samplable setting using the ideas of Kiltz et al. 


% For subspace concatenation, ``one-time" means that the adversary is unable to sign vectors which are not in the span of previously signed vectors. For the one-time signature scheme derived from the sum in subspace proof, ``one-time" refers to the fact that the adversary cannot obtain signatures for vectors which are  half of the components in $\Gr$ and other half in $\Hr$ and such that the sum of the vecto


Our construction is based on the $\skermdh$ assumption introduced in Section~\ref{sec:mddh}. Following the syntactic definition of Section~\ref{sec:mddh},
our scheme assumes $\Group = \Gr^m\times\Hr^n$ and the length of the messages is $n+m$.
 
\begin{itemize}
\item $\SG(gk,m,n)$:
Choose $\matr{A} \leftarrow \dist_{k}$,
$\matr{\Lambda},\matr{\Xi} \leftarrow \Z_q^{(k+1) \times m}$, $\vecb{A}_{\Lambda}:=\matr{\Lambda}^\top \matr{A}$, 
$\vecb{A}_{\Xi}:=\matr{\Xi}^\top \matr{A}$
The secret key is $\mathsf{sk}=(\matr{\Lambda},\matr{\Xi})$, while the public key is defined to be
$$\mathsf{pk}=([\vecb{A}]_1, [\vecb{A}_{\Xi}]_1,[\vecb{A}]_2, [\vecb{A}_{\Lambda}]_2) \in  \Gr^{(k+1) \times k} \times \Gr^{m \times k} \times \Hr^{(k+1) \times k} \times \Hr^{m \times k}.$$
\item $\SN(\mathsf{sk},([\vecb{x}]_1,[\vecb{y}]_2))$: To sign a vector $([\vecb{x}]_1,[\vecb{y}]_2) \in \Gr^{m} \times \Hr^{m}$, pick 
$\vecb{z} \leftarrow \Z_q^{(k+1)}$ and output the pair 
$([\grkb{\rho}]_1, [\grkb{\sigma}]_2) \in \Gr^{(k+1)} \times \Hr^{(k+1)}$, defined as:
$$[\grkb{\rho}]_1:=   \matr{\Lambda}[\vecb{x}]_1 + [\vecb{z}]_1,
                            \qquad \qquad 
[\grkb{\sigma}]_2 := \matr{\Xi}[\vecb{y}]_2 - [\vecb{z}]_2.$$   
 \item  $\SD(\mathsf{pk},\{(\omega_i,[\grkb{\rho}_i]_1, [\grkb{\sigma}_i]_2)\}_{i=1}^{\ell})$: given the public key $pk$, and $\ell$ tuples 
 $(\omega_i,[\grkb{\rho}_i]_1,$ $[\grkb{\sigma}_i]_2)$, output the pair 
 $(\sum_{i=1}^{\ell} \omega_i [\grkb{\rho}_i]_1, \sum_{i=1}^{\ell} \omega_i  [\grkb{\sigma}_i]_2) \in \Gr^{(k+1)} \times \Hr^{(k+1)}$. 
 \item $\SV(\mathsf{pk},([\vecb{x}]_1,[\vecb{y}]_2), ([\grkb{\rho}]_1, [\grkb{\sigma}]_2))$  is a deterministic algorithm, that takes as input a public key $\mathsf{pk}$,  a signature $([\grkb{\rho}]_1, [\grkb{\sigma}]_2)$ and returns $1$ if and only if 
$([\grkb{\rho}]_1, [\grkb{\sigma}]_2)$ satisfies
$$ [\grkb{\rho}^{\top}]_1 [\vecb{A}]_2+ [\grkb{\sigma}^{\top}]_2 [\vecb{A}]_1= [\vecb{x}^{\top}]_1 [\vecb{A}_{\Lambda}]_2+[\vecb{y}^{\top}]_2[\vecb{A}_{\Xi}]_1. $$
\end{itemize}

\paragraph{Correctness.} If a signature is correctly generated then 
$$[\grkb{\rho}^{\top}]_1 [\matr{A}]_2-[\vecb{x}^{\top}]_1 [\matr{A}_{\Lambda}]_2= [\vecb{z}^{\top}]_1 [\matr{A}]_2 \qquad \qquad  [\grkb{\sigma}^{\top}]_2[\matr{A}]_1 -[\vecb{y}^{\top}]_2  [\matr{A}_{\Xi}]_1= -[\vecb{z}^{\top}]_2 [\matr{A}]_1.$$
Therefore the verification algorithm outputs $1$ on a correctly generated signature. The proof of correctness  of the signature derivation algorithm follows a similar argument.  

Let $Q=\{ ([\vecb{x}_i]_1,[\vecb{y}_i]_2) \}_{i \in [q_s]}$ be some set of elements of $\Gr^{m} \times \Hr^n$. We define the predicate $P$ as $P(([\vecb{x}]_1,[\vecb{y}]_2),Q)=1$ iff $(\vecb{x}^{\top}\cat\vecb{y}^{\top}) \in \Z_q^{2m}$ is not in the space spanned by $\{(\vecb{x}_i^{\top}\cat\vecb{y}_i^{\top}) :i \in [q_s]\}$. 

\begin{theorem} The signature scheme is type P unforgeable if the $\skermdh$ assumption holds in $\Gr,\Hr$.
\end{theorem}
%\textcolor{red}{seguramente sacar si ya es evidente por lo otro}
The argument is almost identical to Libert et al.'s \cite{C:LPJY13}.
\begin{proof} We show how to construct an algorithm $\advB$ which takes as input an instance $([\matr{A}]_1,[\matr{A}]_2)$ of the $\skermdh$ assumption and outputs a pair of vectors $([\vecb{r}]_1,[\vecb{s}]_2) \in \Gr^{3} \times \Hr^{3}$, $\vecb{r} \neq \vecb{s}$, such that 
$[\vecb{r}^{\top}]_1 [\matr{A}]_2=[\vecb{s}^{\top}]_2 [\matr{A}]_1$ given oracle access to a forger $\Forger$ against the signature scheme
(see Section~\ref{splhs-unforgeability}). 

Algorithm $\advB$ starts by honestly running the key generation algorithm 
using a randomly chosen $\mathsf{sk}=(\matr{\Lambda},\matr{\Xi})$. Any signature query of $\Forger$ on a vector
$([\vecb{x}]_1,[\vecb{y}]_2)$ is honestly answered by $\advB$, by running the signing algorithm.
The game ends with $\Forger$ outputting a vector $([\vecb{x}]_1^*,[\vecb{y}]_2^*)$ 
with a valid signature $([\grkb{\rho}^{*}]_1, [\grkb{\sigma}^{*}]_2)$. At this point, $\advB$ computes its own signature $([\grkb{\rho}^{\dagger}]_1, [\grkb{\sigma}^{\dagger}]_2)$ using the secret key $\mathsf{sk}:=(\matr{\Lambda},\matr{\Xi})$. The adversary $\advB$ will output as a response to the $\skermdh$ challenge the pair $([\grkb{\rho}^{*}]_1-[\grkb{\rho}^{\dagger}]_1,[ \grkb{\sigma}^{\dagger}]_2-[\grkb{\sigma}^{*}]_2)$.

We now see that, with overwhelming probability, this is a valid answer to the $\skermdh$ challenge.
Indeed, since both signatures satisfy the verification equation, we can subtract the verification equation of each pair, obtaining:
\begin{equation*}
([\grkb{\rho}^{*}]_1-[\grkb{\rho}^{\dagger}]_1)^{\top} [\matr{A}]_2=  ([\grkb{\sigma}^{\dagger}]_2-[\grkb{\sigma}^{*}]_2)^{\top} [\matr{A}]_1
\end{equation*}
Therefore, all we need to argue is that $\grkb{\rho}^{*}-\grkb{\rho}^{\dagger} \neq \grkb{\sigma}^{\dagger}-\grkb{\sigma}^{*}$
with overwhelming probability. This is equivalent to show that  the probability that $\grkb{\rho}^{*}+\grkb{\sigma}^{*} = 
\grkb{\rho}^{\dagger}+ \grkb{\sigma}^{\dagger}$ is negligible.  
The key point of the argument is that 
 \begin{equation*}
 \grkb{\rho}^{\dagger}+ \grkb{\sigma}^{\dagger}=\matr{\Lambda}\vecb{x}^*+\matr{\Xi}\vecb{y}^*= 
  \begin{pmatrix}  
 {\matr{\Lambda}} & {\matr{\Xi}}
 \end{pmatrix}
 \begin{pmatrix}\vecb{x}^*\\ \vecb{y}^* \end{pmatrix}
  \end{equation*}
is information theoretically hidden to $\Forger$. 

The rest of the argument is identical to Libert et al.'s. The argument goes as follows: since, by assumption, $\begin{pmatrix}\vecb{x}^*\\ \vecb{y}^* \end{pmatrix}$ is independent of all  previous queries, then  there is some information about $\begin{pmatrix}  
 {\matr{\Lambda}} & {\matr{\Xi}}
 \end{pmatrix}$ 
which is information theoretically hidden. Thus,  $\grkb{\rho}^{\dagger}+ \grkb{\sigma}^{\dagger}$ is information theoretically hidden and from the adversary's point of view it is equally likely that it has any out of $q$ potential values.  
\end{proof}

\paragraph{Signing the Sum of Two Linear Spaces.} \label{sec:newhom} When $m=n$, we can adapt the previous construction to a different forgery condition namely, we 
can prove security against a different type of adversary. Namely, Libert et al.'s scheme is secure against an adversary whose goal is to output a forgery for a message which is linearly independent from all of its signing queries. In our case, we require that the adversary cannot output a signature for a pair
$([\vecb{x}]_1^*,[\vecb{y}]_2^*) \in \Gr^{m} \times \Hr^{m}$ if $\vecb{x}^*+\vecb{y}^*$ is
linearly independent from the 
vectors $\vecb{x}_i+\vecb{y}_i$, $i \in [q_s]$, where $([\vecb{x}_i]_1,[\vecb{y}_i]_2)$ are the signing queries of the adversary. 

Our construction is like the previous one taking $\matr{\Xi}=\matr{\Lambda}$. Indeed, in this case the adversary only learns $\matr{\Lambda} \vecb{x}^*+ \matr{\Xi} \vecb{y}^*=\matr{\Lambda} (\vecb{x}^*+\vecb{y}^*)$, and identically the same argument follows. 


\chapter {QA-NIZK Arguments for Bit-Strings} \label{sec:bits}

    In this chapter construct a {constant-size proof} that a set of $n$ commitments to elements in some field $\Z_q$ open to 0 or 1. Equivalently, we construct a constant size proof for the satisfiability of the equations $b_1(b_1-1)=0,\ldots,b_n(b_n-1)=0$.
Although solutions for this problem can be easily derived from general results of constant-size NIZK for any NP language \cite{EC:GGPR13,AC:DFGK14,EC:Groth16}, they would rely on strong and controversial assumptions, namely non-falsifiable assumptions. Therefore, it is an open question how to build constant-size proofs for this statement using only standard falsifiable assumptions. 

A set of $n$ commitments $\vecb{c}_1,\ldots,\vecb{c}_n$ to elements of $\Z_q$, each commitment of size $s$, defines a single commitment $\vecb{c}=(\vecb{c}_1,\ldots,\vecb{c}_n)$ to an element of $\Z_q^n$, of size $n\cdot s$. Alternatively, one can define the commitment $\vecb{c}$ so that its size may be $<n\cdot s$ and, depending on the size of $\vecb{c}$, there may or not be a unique opening. Thereby, we distinguish two different cases:

\begin{description}
\item[Perfectly Binding Commitment:] The commitment defines a unique vector $\vecb{x}\in\Z_q^n$. It must hold that $|\vecb{c}|\geq n\log q = \Omega(n)$.
\item[Computationally Binding Commitment:] In this case $\vecb{c}$ can be opened to many values and it is possible that $|\vecb{c}|= o(n)$.
\end{description}

In the second case it is not clear what a proof that the openings are in $\bits$ means. For example, in the case of perfectly hiding commitments such as {multi-Pedersen} commitments -- where a commitment to $\vecb{x}\in\Z_q^n$ is $[c]_1=\sum_{i\in[n]}x_i[g_i]_1+r[g_{n+1}]_1\in\GG_1$ -- each $[c]_1$ can be opened to any $\vecb{x}'\in\Z_q^n$ since
$$[c]_1=\sum_{i\in[n]}x'_i[g_i]_1+\left(\left(c-\sum_{i\in[n]}x'_ig_i\right)g_{n+1}^{-1}\right)[g_{n+1}]_1$$
(in particular to 0 or 1) and thus the proof is trivial. Although it does makes sense to do a \emph{proof of knowledge}, where one can extract a witness, we do not know how construct such proof system using only falsifiable assumptions.

For this reason, in \cite{AC:GonHevRaf15} we first concentrated in the perfectly binding case, specifically Groth-Sahai commitments (and also other related perfectly-binding commitment scheme). We find two interesting applications of this proof system: more efficient signature schemes, with emphasis on the case of {ring signatures}, and more efficient {threshold Groth-Sahai proofs}.
 
Later, in \cite{ACNS:GonRaf16}, we tackle the computationally binding case for a commitment scheme which is an ``hybrid'' between multi-Pedersen commitments and Groth-Sahai commitments. We call this commitment scheme \emph{extended multi-Pedersen commitments}. What is interesting about extended multi-Pedersen commitments is that they can be perfectly hiding but they can also be perfectly binding at one (and only one) coordinate, depending on the the commitment key distribution. Furthermore, the different commitment key distributions are computationally indistinguishable and, thereby, one can randomly choose an index which remains hidden to the adversary such that $b_i$, the opening at coordinate $i$, is uniquely defined. Unlike the NIZK proof for multi-Pedersen commitment, our NIZK proof for extended multi-Pedersen commitments implies that $b_i\in\bits$ which is not trivially true.

Extended multi-Pedersen commitments bears some similarities with \emph{somewhere statistically binding hashing} \cite{ITCS:HubWic15} and \emph{vector commitments} \cite{PKC:CatFio13}. See Section~\ref{sec:mp-vs-others} for a more detailed comparison. 

To exemplify the usefulness of extended multi-Pedersen commitments, we build a proof for the perfectly binding case. Given a perfectly binding commitment $[\vecb{c}]_1$ to $\vecb{b}\in\Z_q^n$ compute a proof that $\vecb{b}\in\bits^n$ as follows:
\begin{enumerate}
\item Compute and extended multi-Pedersen commitment $[\vecb{c}']_1$ to $(b_1,\ldots,b_n)$.
\item Show that $[\vecb{c}]_1$ and $[\vecb{c}']_1$  can be opened to the same value.
\item Show that $[\vecb{c}']_1$ can be opened to and element from $\bits^n$. \label{finochio}
\end{enumerate}
Soundness follows from soundness of the proof from step \ref{finochio} as follows. Suppose that $\vecb{b}\notin \bits^n$, i.e.~there is some $i^*$ such that $b_{i^*}\notin\bits$. By choosing a random index $i$ from $[n]$ and picking the commitments keys such that the extended multi-Pedersen commitments are perfectly binding at coordinate $i$, we have that with probability $1/n$, $i^*=i$. Given that $[\vecb{c}]_1$ and $[\vecb{c}']_1$ can be opened to the same value and the opening of $[\vecb{c}']_1$ at coordinate $i$ is uniquely defined, such opening must be equal to $b_{i^*}\notin\bits$ and we can break soundness of the proof from step \ref{finochio} with probability at least $1/n$.
 
While we use essentially the same techniques from the perfectly binding case to build the proof system for the computationally binding case (step \ref{finochio}), this new approach can be applied to more diverse scenarios. Indeed, it is a key ingredient in Chapter \ref{sec:shuf-rp} where we construct {aggregated set-membership proofs} and more efficient {range proofs} and {proofs of correctness of a shuffle}.

In Section~\ref{sec:bits-binding} we describe our results for the perfectly binding case and the applications, and in Section~\ref{sec:bits-non-binding} we describe our results for the computationally binding case.


    \section{The Perfectly Binding case} \label{sec:bits-binding} 
    
        In this section we construct a constant-size proof that a perfectly binding commitment in $\Gr$ opens to an element of $\bits^n$.   
Such a construction was unknown even in symmetric bilinear groups (yet, it can be easily generalized to this setting \cite[Appendix C]{EPRINT:GonHevRaf15}).
More specifically, we prove membership in 
$$\mathcal{\mathcal{L}}_{ck,\sfbits} := 
    \{[\vecb{c}]_1 \in \Gr^{n+m}: \exists \vecb{b}\in\bits^n,\vecb{w}\in\Z_q^m \text{ s.t. }
        [\vecb{c}]_1:=  \Com_{ck}(\vecb{b};\vecb{w})
    \},$$
 where $ck:=([\matr{U}_1]_1,[\matr{U}_2]_1)\in\Gr^{(n+m)\times n}\times\Gr^{(n+m)\times m}$ define a perfectly binding and computationally hiding commitment to $\vecb{b}$ which is computed as $\Com_{ck}(\vecb{b};\allowbreak\vecb{w}):=[\matr{U}_1]_1\vecb{b}+[\matr{U}_2]\vecb{w}$. Specifically, we give instantiations for $m=1$ (when $[\vecb{c}]_1$ is a single commitment to $\vecb{b}$), and $m=n$ (when $[\vecb{c}]_1$ is the concatenation  of $n$ Groth-Sahai commitments).

We stress that although our proof is constant-size, we need the commitment to be perfectly binding, thus the size of the commitment is linear in $n$.  The common reference string  which we need for this construction is quadratic in the size of the bit-string. Our proof is compatible with proving linear statements about the bit-string, for instance,  
that $\sum_{i \in [n]} b_i=t$ by adding a linear number (in $n$) of elements to the CRS (see Section~\ref{sec:linear-eqs-bitstrings}). We observe that in the special case where $t=1$ the common reference string can be linear in $n$. The costs of our constructions and the cost of GS proofs are summarized in Table \ref{table:eff1}.

Our results rely solely on falsifiable assumptions. More specifically, in the asymmetric case we need some assumptions which are weaker 
than the symmetric external DH assumption %\cite{ManualBib_SIAMJC:GroSah12}
plus the $\SSDP$ assumption. Interestingly, the translation of our construction to the symmetric setting relies on assumptions which are all weaker than the $\lin{2}$ assumption \cite[Appendix C]{EPRINT:GonHevRaf15}.

We combine the QA-NIZK argument for $\mathcal{L}_{[\matr{M}]_1,[\matr{N}]_2,+}$ from Section~\ref{sec:sum} with decisional assumptions in $\Gr$ and $\Hr$. We do this with the purpose of using QA-NIZK arguments even when $\matr{M}+\matr{N}$ has full rank. In this case, strictly speaking ``proving membership in the space'' looses all meaning, as every vector in $\Gr^m\times \Hr^m$ is in the space. However, using decisional assumptions, we can argue that the generating matrix of the space is indistinguishable from a lower rank matrix which spans a subspace in which it is meaningful to prove membership.  

Finally, in Section~\ref{sec:bits-applications} we discuss some applications of our results.  
In particular, our results provide shorter %imply prove the  
signature size of several schemes,
  more efficient ring signatures,
  more efficient set membership proofs,
  and improved threshold GS proofs for pairing product equations.
%\textcolor{red}{acabar}


\begin{table}[h]
\begin{center}
\begin{minipage}{\textwidth}
\begin{center}
\begin{tiny}
             %   proof  % CRS   %
% group type %  Gr % Hr % Gr Hr % #Pairings
\begin{tabular}{|l||c|c|c|c|c|}
\hline
                                                    & Comms     & Proof           & CK                   & CRS($\rho$)       & \#Pairings
\\ \hline\hline
\rule{0pt}{2.5ex}GS \cite{EC:GroSah08}              & $(2n,2n)$    & $(4n,4n)$          & $(4,4)$                & $0$                 & $28n$   \\ \hline
\rule{0pt}{2.5ex}GS + $\QANIZKcomms$                & $(2n,2n)$    & $(2n+2,2n+2)$      & $(4,4)$                & \begin{tabular}{c}
                                $(10n+4,$\\
                                $10n+4)$
                              \end{tabular}         & $20n+8$  \\ \hline
\rule{0pt}{2.5ex}$\Pi_\bit$ $m=1$                   & $(n+1,0)$& $(10,10)$          & $(n + 1,0)$         & \begin{tabular}{c}
                        $(6n^2 +11n+34,$\\
                        $6n^2+11n+34)$
                      \end{tabular}                 & $n+55$   \\ \hline
\rule{0pt}{2.5ex}$\Pi_\bit$ $m=n$ (i)               & $(2n,0)$    & $(10,10)$         & $(4,0)$               & \begin{tabular}{c}
                                                                                                             $(12n^2+14n+22,$\\
                                                                                                             $12n^2+13n+24)$
                                                                                                           \end{tabular}       & $2n+52$  \\ \hline
\rule{0pt}{2.5ex}$\Pi_\bit$ $m=n$ (ii)              & $(2n,0)$    & $(10,10)$         & $(4,0)$               & \begin{tabular}{c}
                                                                                                             $(6n^2+16n+32,$\\
                                                                                                             $6n^2+12n+32)$
                                                                                                           \end{tabular}       & $4n+52$  \\ \hline

\rule{0pt}{2.5ex}$\Pi_\bit$ weight 1, $m=1$         & $(n+1,0)$& $(10,10)$          & $(n+1,0)$           & \begin{tabular}{c}
                                                                                                             $(18n+32,$\\
                                                                                                             $19n+34)$
                                                                                                           \end{tabular}       & $n+55$   \\ \hline
\rule{0pt}{2.5ex}$\Pi_\bit$ weight 1, $m=n$         & $(2n,0)$   & $(10,10)$          & $(4,0)$               & \begin{tabular}{c}
                                                                                                             $(20n+32,$\\
                                                                                                             $18n+32)$
                                                                                                           \end{tabular}       & $4n+52$   \\ \hline
%\rule{0pt}{2.5ex}Partial Sat. \cite{TCC:Rafols15}   &           &                 &                      &                     &          \\ \hline
\end{tabular}
\end{tiny}
\end{center}
\caption{Comparison for proofs of membership in $\Lang_{ck,\sfbits}$ between GS proofs 
 and our different constructions. Our NIZK construction for bit-strings is denoted by $\Pi_\bit$ and the construction for proving that two sets of commitments open to the same value $\QANIZKcomms$. Row ``$\Pi_\bit$ $m=1$'' is for our construction for a single commitment of size $n+1$ to a bit-string of size $n$. Rows ``$\Pi_\bit$ $m=n$ (i)'' and ``$\Pi_\bit$ $m=n$ (ii)'' are for our construction for $n$ concatenated GS commitments, using the two different CRS distributions described in Section~\ref{sec:bits-instantiations}. Rows ``$\Pi_\bit$ weight 1, $m=1$'' and ``$\Pi_\bit$ weight 1, $m=n$'' are 
for our constructions for bit-strings of weight 1 with $m=1$ and $m=n$, respectively.
Column ``Comms'' contains
the size of the commitments, ``CK'' the size of the commitment keys in the CRS, and ``CRS($\rho$)''
the size of the language dependent part of the CRS. Notation $(a,b)$ means $a$ elements of $\GG_1$ and $b$ elements of $\GG_2$. The table is computed for $\dist_k=\distlin_2$, the 2-Linear matrix distribution. \label{table:eff1}  } 
\end{minipage}
\vspace{-0.54cm}

%\begin{minipage}{\textwidth}
%\medskip
%\begin{center}
%\begin{tabular}{|l||c|c|c|c|c|}
%\hline
%                                                    & Comms                 & Proof           & CK                   & CRS($\rho$)       & \#Pairings
%\\ \hline\hline
%\rule{0pt}{2.5ex}GS \cite{EC:GroSah08}              & $2m_x\sG+2m_y\sH$     & $2n\s$          & $4\s$                & $nm_y\sG+nm_x\sH$   & $16n$   \\ \hline
%\rule{0pt}{2.5ex}$\Pi_\sfts$                        & $2m_x\sG+2m_y\sH$     & $2\s$           & $4\s$                & $2nm_y\sG+2nm_y\sH$ & $8n+4$  \\ \hline
%\end{tabular}
%\end{center}
%\caption{Comparison for two-sided Multi-scalar Multiplication Equations $\cvecb{y}^\top\hgrkb{\alpha}_\ell+\hvecb{x}^\top\cgrkb{\beta} = 0$,
%for $\ell\in[n]$, $\hvecb{x}\in\Gr^{m_x}$, $\cvecb{y}\in\Hr^{m_y}$, $\hgrkb{\alpha}\in{\Gr}^{m_y}$, and $\cgrkb{\beta}\in\Hr^{m_y}$,
%between GS proofs and our aggregation techniques.\label{table:eff2}}
%\end{minipage}
\end{center}
\end{table}
%\begin{table}\begin{center}
%\end{minipage}\end{center}
% Column ``Comms'' contain
%the size of the commitments, ``CK'' the size of the commitment keys in the CRS, and ``CRS($\Gamma$)'' the size of the language dependent part of the CRS.
%The size of elements in $\Gr$ and $\Hr$ is $\sG$ and $\sH$ respectively, and $\s = \sG+\sH$.
%. \label{table:eff}}
%\end{table}




        \subsection{Intuition} \label{sec:bits-intuition}

            To prove that a commitment in $\Gr$ opens to a vector of bits $\vecb{b}$, the usual strategy is to compute another commitment $[\vecb{d}]_2\in\Hr^{\bar{n}}$ to a vector $\bar{\vecb{b}}\in\Z_q^n$ and prove 
  (1) $b_i(\overline{b}_i-1)=0$, for all $i \in [n]$, and 
  (2) $b_i-\overline{b}_i=0$, for all $i \in [n]$. 
For statement  (2), since $[\matr{U}]_1$ is witness samplable, we can use our most efficient QA-NIZK from Sect.\ \ref{sec:aggcommit} for equal opening in different groups.  Under the $\SSDP$ Assumption, which is the $\skermdh$ Assumption of minimal size conjectured to hold in asymmetric groups, the proof is of size $2\s$. Thus, the challenge is to aggregate $n$ equations of the form $b_i(\overline{b}_i-1)=0$. We note that this is a particular case of the problem of aggregating proofs of quadratic equations, which was left open in \cite{C:JutRoy14}.

We finally remark that the proof must include $[\vecb{d}]_2$ and thus it may be not of size independent of $n$. However, it turns out that $[\vecb{d}]_2$ needs not be perfectly binding, in fact $\bar{n}=2$ suffices.

\subsubsection{Our Approach}

A prover wanting to show satisfiability of the equation  $\varx(\vary-1)=0$ using GS proofs, will commit to a solution  
$\varx=b$ and $\vary=\overline{b}$ as $[\vecb{c}]_1=b [\vecb{u}_1]_1 + r [\vecb{u}_2]_1$ and 
 $[\vecb{d}]_2=\overline{b} [\vecb{v}_1]_2 + s [\vecb{v}_2]_2$, for $r,s \leftarrow \Z_q$, and then give a pair $([\grkb{\theta}]_1,[\grkb{\pi}]_2)
\in \Gr^{2} \times \Hr^{2}$ which satisfies the following verification equation\footnote{For readers familiar with the Groth-Sahai notation, equation (\ref{eq:gsver}) corresponds to 
$\vecb{c} \bullet \left(\vecb{d}-\iota_2(1)\right) = \vecb{u}_2 \bullet \grkb{\pi} + \grkb{\theta} \bullet \vecb{v}_2$.}:
\begin{equation} \label{eq:gsver}
[\vecb{c}]_1 \left([\vecb{d}]_2-
 [{\vecb{v}}_1]_2\right)^{\top}=[{\vecb{u}}_2]_1  
[{\boldsymbol \pi}^{\top}]_2+ [{\boldsymbol \theta}]_1   [\vecb{v}_2^{\top}]_2. 
\end{equation}
The reason why this works is that, if we express both sides of the equation in the basis of 
$\GG_T^{2\times 2}$ given by 
$\{[\vecb{u}_1]_1[\vecb{v}_1^\top]_2,[\vecb{u}_2]_1[\vecb{v}_1^\top]_2,[\vecb{u}_1]_1[\vecb{v}_2^\top]_2,[\vecb{u}_2]_1[\vecb{v}_2^\top]_2\}$, the coefficient of 
$[\vecb{u}_1]_1[\vecb{v}_1^\top]_2$ is $b(\overline{b}-1)$ on the left side and $0$ on the right side (regardless of
$([\grkb{\theta}]_1,[\grkb{\pi}]_2)$).
Our observation is that the verification equation can be abstracted as saying:
\begin{equation}\label{eq:gsabstracted}
[\vecb{c}]_1 \left([\vecb{d}]_2-[\vecb{v}_1]_2\right)^{\top} \ \in \mathsf{Span}([\vecb{u}_2]_1[\vecb{v}_1^\top]_2,[\vecb{u}_1]_1[\vecb{v}_2^\top]_2,[\vecb{u}_2]_1[\vecb{v}_2^\top]_2) \subset  \GG_T^{2 \times 2}. 
\end{equation}

Now consider commitments to $(b_1,\ldots,b_n)$ and $(\overline{b}_1,\ldots,\overline{b}_n)$ constructed with some commitment key $\{([\vecb{g}_i]_1,[\vecb{h}_i]_2): i \in [n+1]\}\subset\Gr^{\nb}\times\Hr^\nb$, for some $\nb\in\N$, to be determined later, and defined as $[\vecb{c}]_1:=\sum_{i \in [n]} b_i [\vecb{g}_i]_1 + r [\vecb{g}_{n+1}]_1$, 
$[\vecb{d}]_2:=\sum_{i \in [n]} \overline{b}_i [\lrck_i]_2 + s [{\lrck}_{n+1}]_2$, $r,s \gets \Z_q$. Suppose for a moment that 
$\{ [\vecb{g}_{i}]_1 [\vecb{h}_{j}^\top]_2 : i,j \in [n+1]\}$ 
is a set of linearly independent vectors. Then,  
\begin{equation} \label{eq:gsveri}
[\vecb{c}]_1 \left([\vecb{d}^\top]_2-
\sum_{j\in[n]} [\lrck_j^\top]_2\right) \in \Span\{ [\vecb{g}_i]_1[\vecb{h}_j^\top]_2: i\neq j\text{ when } i,j\neq n+1 \} 
\end{equation}
if and only if $b_i(\overline{b}_i-1)=0$ for all $i \in [n]$,
because $b_i(\overline{b}_i-1)$ is the coordinate of 
$[\vecb{g}_i]_1[\vecb{h}_i^\top]_2$ in the left side of the equation.


Equation \ref{eq:gsveri} suggests to use one of the constant-size QA-NIZK Arguments for linear spaces to get a constant-size proof that $b_i(\overline{b}_i-1)=0$ for all $i \in [n]$. Unfortunately, these arguments are only defined for membership in subspaces  in 
$\Gr^m$ or $\Hr^m$ but not in $\GG_T^m$. Our solution is to include information in the CRS to ``bring back'' 
  this statement from $\GG_T$ to $\Gr$, i.e.\ 
  the matrices   $[\Lqmatr_{i,j}]_1:=[\vecb{g}_i]_1\lrck_j^{\top}$, where $i\neq j$ when $i,j\neq n+1$. We denote this set of matrices as
 $\Qspace:=\{[\Lqmatr_{i,j}]_1: i\neq j\text{ when } i,j\neq n+1\}$.   
Then, to prove that $b_i(\overline{b}_i-1)=0$ for all $i\in [n]$, the prover computes 
$[\matr{\Theta}]_1$ as a linear combination (with coefficients which depend on
 $\vecb{b},\vecb{\bb},r,s$) of the matrices in $\Qspace$.
Then the verifier checks that
\begin{equation}
[\vecb{c}]_1\left([\vecb{d}]_2-
\sum_{j\in[n]}[\lrck_j]_2\right)^{\top}=
[\matr{\Theta}]_1[\matr{I}_{\nb}]_2,
\end{equation}
 and finally the prover gives a QA-NIZK proof of  $[\matr{\Theta}]_1 \in \mathsf{Span}(\Qspace)$.

This reasoning assumes that $\{[\vecb{g}_i]_1 \lrck_j^{\top}\}$ (or equivalently, $\{[\matr{C}_{i,j}]_1\}$) are linearly independent,  which can only happen if 
$\nb \geq n+1$. If that is the case, the proof cannot be constant because $[\matr{\Theta}]_1 \in \Gr^{\nb\times\nb}$ and this matrix is part of the proof.
Instead, we choose $\vecb{g}_1,\ldots,\vecb{g}_{n+1} \in \Gr^{2}$ and $\vecb{h}_1,\ldots,\vecb{h}_{n+1} \in \Hr^{2}$, so that 
$\{[\matr{C}_{i,j}]_1\} \subseteq \Gr^{2 \times 2}$.  Intuitively, this should still work because the prover receives these vectors as part of the CRS and he does not know their discrete logarithms, so to him, they behave as linearly independent vectors.  

Nevertheless, the statement $[\matr{\Theta}]_1\in\Span(\Qspace)$ seems no longer meaningful, as $\Span(\Qspace)$ is all of $\Gr^{2\times 2}$ with overwhelming probability. But this is not the case, because by means of decisional assumptions in $\Gr$ and in $\Hr$, we switch to a game where the matrices
$[\matr{C}_{i,j}]_1$ span a non-trivial space of $\Gr^{2 \times 2}$. Specifically, to a game where $[\matr{C}_{i^*,i^*}]_1\notin\Span(\Qspace)$, were $i^*$ is a random integer in $[n]$ which remains hidden to the adversary. Once we are in such a game, perfect soundness is guaranteed for equation $b_{i^*}(\bbb_{i^*}-1)=0$ and a cheating adversary is caught with probability at least $1/n$. We think this technique might be of independent interest.

The last obstacle is that, 
  using decisional assumptions on the set of vectors 
  $\{[\lrck_{j}]_2\}_{j\in[n+1]}$ is incompatible with using the discrete logarithms of $[\vecb{h}_j]_2$ to compute the matrices $\Qmatr_{i,j}:=[\vecb{g}_i]_1 \lrck_j^{\top}$ given in the CRS. 
To account for the fact that, in some games,
  we only know $\vecb{g}_i \in \Z_q$ and, in some others,
  only $\lrck_j \in \Z_q$, we replace each matrix 
  $[\Lqmatr_{i,j}]_1$ by a pair 
  $([\matr{C}_{i,j}]_1,[\matr{D}_{i,j}]_2)$ which is uniformly 
  distributed conditioned on 
  $\matr{C}_{i,j}+\matr{D}_{i,j}=\vecb{g}_i \lrck_j^{\top}$.
This randomization completely hides the group in which we can compute 
  $\vecb{g}_i \lrck_j^{\top}$. 
  Finally, we use our QA-NIZK Argument for sum in a subspace (Sect.\ \ref{sec:sum}) to prove membership in this space.

 

        \subsection{Instantiations} \label{sec:bits-instantiations}

            We discuss in detail two particular cases of languages $\mathcal{\mathcal{L}}_{ck,\sfbits}$. First, in Section~\ref{sec:bits-scheme} we discuss the case when 
\begin{itemize}
\item[(a)] $[\vecb{c}]_1$ is a vector in $\Gr^{n+1}$,  $\vecb{u}_{n+1} \leftarrow \distlin_{n+1,1}$ and
 $\matr{U}_1:=\begin{pmatrix}{\matr{I}_{n\times n}}\\{\matr{0}_{1\times n}}\end{pmatrix} \in \Z_q^{(n+1) \times n}, \matr{U}_2:=\vecb{u}_{n+1} \in \Z_q^{n+1}$, $\matr{U}:=(\matr{U}_1\cat \matr{U}_2)$.    
\end{itemize}
In this case, the vectors $[\vecb{g}_i]_1$ in the intuition are defined as $[\vecb{g}_i]_1=\matr{\Delta} [\vecb{u}_i]_1$, where $\matr{\Delta}\gets\Z_q^{2\times(n+1)}$, and the commitment 
to $\vecb{b}$ is computed as $[\vecb{c}]_1:=\sum_{i\in[n]}b_i[\vecb{u}_i]_1+w[\vecb{u}_{n+1}]_1$.
Then in Section~\ref{sec:bits-extensions} we discuss how to generalize the construction for a) to 
\begin{itemize}
 \item[(b)] $[\vecb{c}]_1$ is the concatenation of $n$ GS commitments. That is, given the  GS CRS   $\mathsf{crs}_\GS=(gk,[\vecb{u}_1]_1,[\vecb{u}_2]_1,[\vecb{v}_1]_2,[\vecb{v}_2]_2)$, we define,
$$\matr{U}_1:=  \begin{pmatrix} \vecb{u}_1 & \ldots & \vecb{0}\\ \vdots & \ddots & \vdots \\   \vecb{0} & \ldots & \vecb{u}_1  \end{pmatrix} \in \Z_q^{2n \times n},  \matr{U}_2:= \begin{pmatrix} \vecb{u}_2 & \ldots & \vecb{0}\\ \vdots & \ddots & \vdots \\  \vecb{0} & \ldots & \vecb{u}_2  \end{pmatrix} \in \Z_q^{2n \times n}.$$ 
\end{itemize}

%\paragraph{Remark.}
Although the proof size is constant, in both of our instantiations the commitment size is $\Theta(n)$. Specifically, $(n+1)\sG$ for case a) and $2n\sG$ for case b).
%Furthermore, one might relax the requirement of perfectly binding commitments as long as one is able to extract a witness from $[\vecb{c}]_1$. However, if for example $[\vecb{U}]_1$ defines a distribution over hard instances of the Subset-Sum language, impossibility results from $\cite{STOC:GenWic11}$ imply that $[\vecb{c}]_1$ cannot be of size $\mathsf{poly}(\lambda)n^{o(1)}$.

%Although this is a limitation for some applications, \textit{e.g.} range proofs, this is a requirement for some others (like proving membership in a list).   



        \subsection{The Scheme} \label{sec:bits-scheme}

            %\input{bitstrings/proofasym-fig.tex}
\begin{description} \label{bits-proof-system}

%\item[$\algK_0(1^\lambda)$:]  Return $gk := (q,\Gr,\Hr,\GG_T,e,\mathcal{P}_1,\mathcal{P}_2) \leftarrow \ggen_a(1^{\lambda})$.

%\item[$\dist_{gk}$:] The distribution $\dist_{gk}$ over $\Gr^{(n+1) \times (n+1)}$ is some witness samplable distribution which 
%defines the relation $\R_{gk} = \{\R_{[\matr{U}]_1}\} 
%\subseteq \Gr^{n+1}\times(\bits^{n}\times\Z_q)$,
%where $[\matr{U}]_1\gets\dist_{gk}$,
%such that $([\vecb{c}]_1,\langle\vecb{b}, w\rangle)\in\R_{[\matr{U}]_1}$ iff
%$[\vecb{c}]_1=[\matr{U}]_1\binom{\vecb{b}}{w}$ and $\vecb{b}\in\bits^n$. The relation $\R_{par}$ consists of pairs $([\matr{U}]_1,\matr{U})$ where $[\matr{U}]_1 \gets \dist_{{gk}}$.
\item[$\algK({gk}, {[\matr{U}]_1})$:]
Let $\lrck_{n+1}\gets \Z_q^2$
and for all $i \in [n]$, $\lrck_{i}:=\epsilon_{i}\lrck_{n+1}$, where
$\epsilon_{i} \leftarrow \Z_q$. Define
$\Rck := ([\vecb{h}_1]_2|| \ldots ||[\vecb{h}_{n+1}]_2)$.
Choose 
$\matr{\Delta} \leftarrow \Z_q^{2 \times (n+1)}$,
define $\Lck := \matr{\Delta}[\matr{U}]_1$
and $[\llck_{i}]_1:=\matr{\Delta} [\vecb{u}_i]_1 \in \Gr^2$, for all $i \in [n+1]$. 
Let $\vecb{a} \leftarrow \distlin_{1}$ and define $[\vecb{a}_{\Delta}]_2:=\matr{\Delta}^\top[\vecb{a}]_2 \in \Hr^{n+1}$. 
For any pair $(i,j) \in \indexSet{n}{1}$ (as defined in Sect. \ref{sec:notation}), let 
$\matr{T}_{i,j}\gets\Z_q^{2\times2}$ and set:
$$[\matr{C}_{i,j}]_1:=[\vecb{g}_i]_1 \lrck_j^{\top} - [\matr{T}_{i,j}]_1  \in \Gr^{2 \times 2},
\qquad \qquad 
[\matr{D}_{i,j}]_2:=[\matr{T}_{i,j}]_2 \in \Hr^{2 \times 2}.$$ 
Note that $[\matr{C}_{i,j}]_1$ can be efficiently computed 
as $\lrck_j \in \Z_q^{2}$ is the vector of discrete logarithms of $[\vecb{h}_j]_1$.

Let ${\Psi_\sfsum}$ be the proof system for Sum in Subspace 
(Sect. \ref{sec:sum}) and ${\Psi_\sfcom}$
be an instance of the proof system for Equal Opening (Sect. \ref{sec:aggcommit}).

Let
$\crs_{\Psi_\sfsum} \gets \algK_1({gk}, \{[\matr{C}_{i,j}]_1,[\matr{D}_{i,j}]_2:(i,j)\in\indexSet{n}{1}\})$ and \footnote{We identify
matrices in $\Gr^{2 \times 2}$ (resp. in $\Hr^{2 \times 2}$) with vectors in $\Gr^{4}$ (resp. in $\Hr^{4}$).}  
 $\crs_{\Psi_\sfcom} \gets \algK_1({gk}, \Lck,\Rck,n)$. The common reference string is given by:
\begin{eqnarray*}
\mathsf{crs}_P&:=&\left( [\matr{U}]_1,  \Lck,
    [\Lrck]_1, \{[\matr{C}_{i,j}]_1,[\matr{D}_{i,j}]_2 :(i,j) \in \indexSet{n}{1}\},\crs_{\Psi_\sfsum},\crs_{\Psi_\sfcom} \right), \\
\mathsf{crs}_V&:=&\left([\vecb{a}]_2, [\vecb{a}_\Delta]_2, \crs_{\Psi_\sfsum},\crs_{\Psi_\sfcom} \right). 
 \end{eqnarray*}
\item[{$\algP(\mathsf{crs}_P, [\vecb{c}]_1, \langle \vecb{b}, w_g \rangle)$:}]
Pick $w_h \gets \Z_q$,  $\matr{R} \gets \Z_q^{2\times 2}$ and then: 
\begin{enumerate}
\item Define 
$$[\vecb{c}_{\Delta}]_1 := \Lck \begin{pmatrix} \vecb{b} \\ w_g \end{pmatrix},
\qquad [\vecb{d}]_2 := \Rck \begin{pmatrix} \vecb{b} \\ w_h \end{pmatrix}.$$ 
\item Compute 
 $([\matr{\Theta}]_1, [\matr{\Pi}]_2) :=$
\begin{eqnarray} \label{eq:ThetaPi}
& &
    \sum_{i \in [n]}\left(
        b_i w_h ([\matr{C}_{i,n+1}]_1,[\matr{D}_{i,n+1}]_2)+
        w_g(b_i-1) ([\matr{C}_{n+1,i}]_1, [\matr{D}_{n+1,i}]_2)\right)
        \nonumber\\ & &           +
       \sum_{i \in [n]}  \sum_{\substack{j \in [n]\\ j\neq i}} b_i (b_j-1) ([\matr{C}_{i,j}]_1, [\matr{D}_{i,j}]_2)\nonumber\\
       & &
     +
    w_gw_h ([\matr{C}_{n+1,n+1}]_1, [\matr{D}_{n+1,n+1}]_2) +  ([\matr{R}]_1,-[\matr{R}]_2).
 \end{eqnarray}

\item Compute a proof $\pi_\sfsum$
that $\matr{\Theta}+\matr{\Pi}$
belongs to the space spanned by $\{\matr{C}_{i,j}+\matr{D}_{i,j}:(i,j)\in\indexSet{n}{1}\}$,
 and a proof 
$\pi_\sfcom$
that
$([\vecb{c}_\Delta]_1,[\vecb{d}]_2)$ open to the same value,
using $\vecb{b},w_g$, and $w_h$. 
\end{enumerate}

\item[{$\algV(
    \mathsf{crs}_V,
    [\vecb{c}]_1,
    [\vecb{c}_{\Delta}]_1, [\vecb{d}]_2,
    ([\matr{\Theta}]_1, [\matr{\Pi}]_2), 
    \pi_\sfsum,\pi_\sfcom)$:}] ~
\begin{enumerate}
\item  Check if $[\vecb{c}]_1^\top[\vecb{a}_\Delta]_2 = [\vecb{c}_\Delta]_1^\top[\vecb{a}]_2$. 
\item Check if 
\begin{equation}\label{eq:ver1}[\vecb{c}_{\Delta}]_1
\left(
    [\vecb{d}]_2-
    \sum_{j \in [n]} [\vecb{h}_{j}]_2
\right)^{\top} =
    [\matr{\Theta}]_1[\matr{I}_{2 \times 2}]_2 +
    [\matr{I}_{2 \times 2}]_1[\matr{\Pi}]_2.
    \end{equation}  
  \item Verify that $\pi_\sfsum,\pi_\sfcom$ are valid proofs for 
  $([\matr{\Theta}]_1,[\matr{\Pi}]_2)$
        and $([\vecb{c}_\Delta]_1,[\vecb{d}]_2)$ using $\Psi_\sfsum$ and $\Psi_\sfcom$ respectively.
\end{enumerate}
If any of these checks fails, the verifier outputs $0$, else it outputs $1$.
\item[{$\mathsf{S}_1({gk},[\matr{U}]_1)$:}] The simulator receives as input a description of an asymmetric bilinear group ${gk}$ and a matrix $[\matr{U}]_1 \in \Gr^{(n+1) \times (n+1)}$ sampled according to distribution $\dist_{gk}$. It generates and outputs the CRS in the same way as $\algK_1$, but additionally it also  outputs the simulation trapdoor 
$$\tau=\left(\Lrck, \matr{\Delta}, \tau_{\Psi_\sfsum}, \tau_{\Psi_\sfcom}\right),$$
where $\tau_{\Psi_\sfsum}$ and $\tau_{\Psi_\sfcom}$ are, respectively, ${\Psi_\sfsum}$'s and ${\Psi_\sfcom}$'s simulation trapdoors.
\item[{$\mathsf{S}_2(\crs_P,[\vecb{c}]_1,\tau)$:}] Compute $[\vecb{c}_{\Delta}]_1:=\matr{\Delta}[\vecb{c}]_1$.
      Then pick random $\overline{w}_h \gets \Z_q$, $\matr{R} \gets \Z_q^{2 \times 2}$ and define 
 $\vecb{d}:= \overline{w}_{h} \lrck_{n+1}.$
 Then set:
\begin{align*} 
[\matr{\Theta}]_1 & :=  [\vecb{c}_\Delta]_1 \left(\vecb{d}-\sum_{i \in [n]} \lrck_i\right)^\top + [\matr{R}]_1,
    &
    [\matr{\Pi}]_2 & := - [\matr{R}]_2.
\end{align*}
Finally, simulate proofs $\pi_\sfsum,\pi_\sfcom$ using $\tau_{\Psi_\sfsum}$ and $\tau_{\Psi_\sfcom}$.
\end{description}



\subsection{Proof of Security}

\begin{theorem}
The proof system from Section~\ref{bits-proof-system} is QA-NIZK proof system for the language $\Lang_{ck,\sfbits}$ with perfect completeness, computational soundness, and perfect zero-knowledge
\end{theorem}

\begin{proof}
(Completeness.) It is obvious by definition that for any $[\vecb{c}]_1\in \Lang_{ck,\sfbits}$
the vector $[\vecb{c}_\Delta]_1$
generated by an honest prover passes the verification test described in 1).

Note that,
by definition of $[\matr{C}_{i,j}]_1$ and $[\matr{D}_{i,j}]_2$, 
$[\matr{C}_{i,j}]_1[\matr{I}_{2\times2}]_2 + [\matr{I}_{2\times2}]_1 [\matr{D}_{i,j}]_2
= [\vecb{g}_{i}]_1[\vecb{h}_j^\top]_2$.  Since $b_i(b_i-1) = 0$ for each $i\in[n]$,
\begin{align*}
&[\vecb{c}_{\Delta}]_1 \left( [\vecb{d}]_2 - \sum_{i\in[n]} [\vecb{h}_{i}]_2 \right)^\top \\
  =& 
    \sum_{i \in [n]}\left(
        b_i w_h[\vecb{g}_{i}]_1[\vecb{h}_{n+1}^{\top}]_2 +w_g(b_i-1) [\vecb{g}_{n+1}]_1[\vecb{h}_i^{\top}]_2 +
        \sum_{j \in [n]} b_i (b_j-1) [\vecb{g}_{i}]_1[\vecb{h}_{j}^{\top}]_2
    \right) +\\
&
    w_gw_h [\vecb{g}_{n+1}][\vecb{h}_{n+1}^{\top}]_2
\\  = & 
    \sum_{i \in [n]}\left(
        b_i w_h[\vecb{g}_{i}]_1[\vecb{h}_{n+1}^\top]_2 +
        w_g(b_i-1) [\vecb{g}_{n+1}]_1[\vecb{h}_{i}^\top]_2 +
        \sum_{\substack{j \in [n]\\ j\neq i}} b_i (b_j-1) [\vecb{g}_{i}]_1[\vecb{h}_{j}^\top]_2
    \right)
\\  &
    + w_gw_h [\vecb{g}_{n+1}]_1[\vecb{h}_{n+1}^\top]_2 +
    [\matr{R}]_1[\matr{I}_{2\times2}]_2 - [\matr{I}_{2\times2}]_1[\matr{R}]_2
\\  = &
    [\matr{\Theta}]_1[\matr{I}_{2\times2}]_2 +
    [\matr{I}_{2\times2}]_1[\matr{\Pi}]_2.
\end{align*}
Finally, the rest of the proof follows from completeness of ${\Psi_\sfcom}$ and ${\Psi_\sfsum}$. 

(Soundness.) Soundness is proven in Theorem \ref{teo:bits-bind-soundness}.

(Zero-Knowledge.) First, note that the vector $[\vecb{d}]_2 \in \Hr^2$ output by the prover and the vector output by $\mathsf{S}_2$ follow exactly the same distribution. This is because the rank of $\Rck$ is $1$. In particular, although the simulator $\mathsf{S}_2$ does not know the opening of $[\vecb{c}]_1$, which is some $\vecb{b} \in \{0,1\}^{n}$, 
there exists $w_h \in \Z_q$ such that $[\vecb{d}]_2=\Rck\smallpmatrix{\vecb{b}\\ w_h}$. 
Since $\matr{R}$ is chosen uniformly at random in $\Z_q^{2 \times 2}$, the proof $([\matr{\Theta}]_1, [\matr{\Pi}]_2)$ is uniformly distributed conditioned on satisfying check 2) of algorithm $\algV$.
Therefore, these elements of the simulated proof have the same distribution as in a real proof. This fact combined with the perfect zero-knowledge property of ${\Psi_\sfsum}$  and ${\Psi_\sfcom}$ concludes the proof. 
\end{proof}
 
\begin{theorem} Let $\mathsf{Adv}_{\mathcal{PS}}(\advA)$ 
be the advantage of an adversary $\advA$ against the soundness of 
the proof system  described above. There exist PPT adversaries
$\advB_1,\advB_2,\advB_3,\advSound_1,\advSound_2$ such that 
\begin{eqnarray*}
\mathsf{Adv}_{\mathcal{PS}}(\advA) & \leq 
n& \left(6/q+ \mathsf{Adv}_{\mathcal{U}_1,\Gr}(\advB_1)
+  \mathsf{Adv}_{\mathcal{U}_1,\Hr}(\advB_2)
+  \mathsf{Adv}_{\SP_{\Hr}}(\advB_3)\right. \\
& & \mbox{ } 
+  \left.\mathsf{Adv}_{{\Psi_\sfsum}}(\advSound_1)
+
 \mathsf{Adv}_{{\Psi_\sfcom}}(\advSound_2)\right).
\end{eqnarray*}
\label{teo:bits-bind-soundness}
\end{theorem}

The proof follows from the indistinguishability of the following games:
\begin{itemize}
\item [$\mathsf{Real}$] This is the real soundness game. 
 The output is $1$ if  the adversary breaks the soundness,
i.e. the adversary submits
some $[\vecb{c}]_1 = [\matr{U}]_1\left(\begin{smallmatrix}\vecb{b}\\ w_g\end{smallmatrix}\right)$, for some
$\vecb{b}\notin \bits^n$ and $w \in \Z_q$, and
the corresponding proof which is accepted by the verifier.
\item[$\mathsf{Game}_0$] This game is identical to 
$\mathsf{Real}$ except that algorithm $\algK$ does not receive $[\matr{U}]_1$ as a input but it samples
$([\matr{U}]_1,\matr{U}) \in \mathcal{R}_{par}$
itself according to $\dist_{{gk}}$.
\item[$\mathsf{Game}_1$] This game is identical to 
$\mathsf{Game_0}$ except that the simulator picks a random $i^* \in [n]$, and uses $\matr{U}$ to check  
    if the output of the adversary $\advA$ is such that 
    $b_{i^*}\in \bits$.  It aborts if  $b_{i^*}\in \bits$.
\item[$\mathsf{Game}_{2}$] This game is identical to 
$\mathsf{Game}_1$ except that now the vectors $[\vecb{g}_{i}]_1$, $i \in [n]$ and $i \neq i^*$,
are uniform vectors in the space spanned by $[\vecb{g}_{n+1}]_1$.   
\item[$\mathsf{Game}_{3}$] This game is identical to 
$\mathsf{Game}_2$ except that now the vector $[\vecb{h}_{i^*}]_2$ is 
a uniform vector in $\Hr^2$, sampled independently of 
$[\vecb{h}_{n+1}]_2$.     
\end{itemize}
It is obvious that the first two games are indistinguishable. 
The rest of the argument goes as follows. 

\begin{lemma} $\Pr\left[ \mathsf{Game}_1(\advA)=1\right]\geq\dfrac{1}{n}\Pr\left[\mathsf{Game}_0(\advA)=1\right].$
\label{lemma:bits1}
\end{lemma}

\begin{proof}  The probability that
 $\mathsf{Game}_1(\advA)=1$ is the probability that  a) $\mathsf{Game}_0(\advA)=1$ and
b)  $b_{i^*} \notin \bits$. The view of adversary $\advA$ is independent of $i^*$, while, if $\mathsf{Game_0}(\advA)=1$, then there is at least one index $\ell \in [n]$ such that  
such that  $b_{\ell} \notin \bits$. Thus, 
the probability that the event described in b) occurs conditioned on $\mathsf{Game_0}(\advA)=1$, is greater than or equal to $1/n$ and the lemma follows.
\end{proof}

\begin{lemma} There exists a~$\mathcal{U}_1$-$\mddh_{\Gr}$ adversary $\advB$ such that
$|\Pr\left[\mathsf{Game}_{1}(\advA)=1\right]$ $-\Pr\left[\mathsf{Game}_{2}(\advA)=1\right]|$ $\leq
    \mathsf{Adv}_{\mathcal{U}_1,\Gr}(\advB) + 2/q.$
\label{lemma:bits2}
\end{lemma}
\begin{proof}
The adversary $\advB$ receives $([\vecb{s}]_1, [\vecb{t}]_1)$ an instance of the $\mathcal{U}_1$-$\mddh_{\Gr}$ problem.
$\advB$ defines all the parameters honestly except that
it embeds the $\mathcal{U}_1$-$\mddh_{\Gr}$ challenge in the matrix 
$\Lck$.

Let $[\matr{E}]_1:=([\vecb{s}]_1||[\vecb{t}]_1)$. $\advB$ picks $i^*\gets[n]$, $\matr{W}_0\gets\Z_q^{2\times(i^*-1)}$,
$\matr{W}_1\gets\Z_q^{2\times(n-i^*)}$,
$[\vecb{g}_{i^*}]_1\gets\Gr^{2}$,
and defines $\Lck := ([\matr{E}]_1\matr{W}_0||[\vecb{g}_{i^*}]_1||[\matr{E}]_1\matr{W}_1|| [\vecb{s}]_1)$. 
In the real algorithm $\algK$, the generator picks the matrix $\matr{\Delta} \in \Z_q^{2 \times (n+1)}$.
Although $\advB$ does not know $\matr{\Delta}$,  it can compute $[\matr{\Delta}]_1$ as $[\matr{\Delta}]_1= \Lck\matr{U}^{-1}$,
given that $\matr{U}$ is full rank and was  sampled 
by $\advB$, so it can compute the rest of the elements of the
common reference string  using the discrete logarithms of $[\matr{U}]_1$, $\Rck$ and $[\vecb{a}]_2$.  

In case $[\vecb{t}]_1$ is uniform over $\Gr^{2}$, by the Schwartz-Zippel lemma $\det([\matr{E}]_1) = 0$ with probability at most $2/q$.
Thus, with probability at least $1-2/q$, the matrix $[\matr{E}]_1$ is full-rank and $\Lck$ is uniform over $\Gr^{2\times(n+1)}$ as in
$\sfGame_1$.
On the other hand, in case $[\vecb{t}]_1=\gamma [\vecb{s}]_1$, all of $[\vecb{g}_{i}]_1$, $i\neq i^*$, are in the space
spanned by $[\vecb{g}_{n+1}]_1$ as in $\sfGame_2$.
\end{proof}

\begin{lemma} There exists a $\mathcal{U}_1$-$\mddh_{\Hr}$ adversary $\advB$ such that
$|\Pr\left[\mathsf{Game}_{2}(\advA)=1\right]$ $-\Pr\left[\mathsf{Game}_{3}(\advA)=1\right]|$ $\leq
\mathsf{Adv}_{\mathcal{U}_1,\Hr}(\advB).$
\label{lemma:bits3}
\end{lemma}

\begin{proof}
The adversary $\advB$ receives an instance of the $\mathcal{U}_1$-$\mddh_{\GG_2}$ problem, which is a pair
$([\vecb{s}]_2, [\vecb{t}]_2)$, where $[\vecb{s}]_2$ is a uniform vector 
of $\Hr^{2}$ and $[\vecb{t}]_2$ is either a uniform vector in $\Hr^2$ or 
$[\vecb{t}]_2=\gamma[\vecb{s}]_2$, for random $\gamma\in\Z_q$.      
 
Adversary $\advB$ defines 
$[\vecb{h}_{n+1}]_2:= [\vecb{s}]_2$ and the rest of the columns of $\Rck$ are honestly sampled
with the sole exception of $[\vecb{h}_{i^*}]_2$, which is set to $[\vecb{t}]_2$.

Given that adversary $\advB$ can only compute $\llck_{i}[\vecb{h}_j^\top]_2\in\Hr^{2\times2}$,
it defines $[\matr{D}_{i,j}]_2 := \llck_{i}[\vecb{h}_j^\top]_2 - [\matr{T}_{i,j}]_2$ and
$[\matr{C}_{i,j}]_1:=[\matr{T}_{i,j}]_1$, for $\matr{T}_{i,j}\gets\Z_q^{2\times 2}$ and $(i,j)\in\indexSet{n}{1}$. Note 
that this does not change the distribution of $([\matr{D}_{i,j}]_2,[\matr{C}_{i,j}]_1)$, which is the uniform one conditioned
on $\matr{C}_{i,j}+\matr{D}_{i,j}= \llck_i\lrck_j^\top.$

The rest of the parameters are computed using $\vecb{a}\gets\distlin_1$,
the matrix $\matr{\Delta} \in \Z_q^{2 \times (n+1)}$ and the discrete logarithms
of $\Lck$.
It is immediate to see that adversary $\advB$ perfectly simulates $\sfGame_2$ when $[\vecb{t}]_2=\gamma[\vecb{s}]_2$ and $\sfGame_3$ when $[\vecb{t}]_2$ is uniform.  
\end{proof}

\begin{lemma} There exists a $\SP_{\Hr}$ adversary $\mathcal{\advB}$, a soundness adversary $\advSound_1$  for ${\Psi_\sfsum}$ and 
a strong soundness adversary $\advSound_2$ for ${\Psi_\sfcom}$  such that
$$\Pr\left[\mathsf{Game}_3(\advA)=1\right] \leq 4/q + \adv_{\SP_{\Hr}}(\advB) +
\adv_{\Psi_\sfsum}(\advSound_1)+\adv_{\Psi_\sfcom}(\advSound_2).$$  
\label{lemma:G3}
\end{lemma}
\begin{proof}
$\Pr[\det((\llck_{i^*}||\llck_{n+1}))=0]=\Pr[\det((\lrck_{i^*}||\lrck_{n+1}))=0]\leq2/q$, by the Schwartz-Zippel lemma. Then, with probability at least $1-4/q$, $\llck_{i^*}\lrck_{i^*}^\top$ is linearly independent from
$\{\llck_{i}\lrck_j^\top:(i,j)\in[n+1]^2\setminus\{(i^*, i^*)\}\}$ which implies that $\llck_{i^*}\lrck_{i^*}^\top\notin\Span(\{\matr{C}_{i,j}+\matr{D}_{i,j}:(i,j)\in\indexSet{n}{1}\})$. 
Additionally  $\sfGame_3(\advA)=1$ implies that $b_{i^*} \notin \{0,1\}$
while the verifier accepts the proof  produced by $\advA$, which is
$ (
        [\vecb{c}_{\Delta}]_1, [\vecb{d}]_2,
        ([\matr{\Theta}]_1, [\matr{\Pi}]_2), 
        \pi_\sfsum,\pi_\sfcom
).$ Since $\{[\vecb{h}_{i^*}]_2,[\vecb{h}_{n+1}]_2\}$ is a basis of $\Hr^2$,
we can define $\overline{w}_h,\overline{b}_{i^*}$ as the unique coefficients in $\Z_q$ such that $[\vecb{d}]_2= \overline{b}_{i^*} [\vecb{h}_{i^*}]_2 + \overline{w}_h [\vecb{h}_{n+1}]_2$.
We distinguish three cases:
\begin{itemize}
\item[1)] If $[\vecb{c}_{\Delta}]_1 \neq \matr{\Delta} [\vecb{c}]_1$, we can construct an adversary 
$\advB$ against the $\SP_{\Hr}$ assumption that outputs 
$[\vecb{c}_{\Delta}]_1-\matr{\Delta} [\vecb{c}]_1\in\ker([\matr{a}]_2^\top)$.
\item[2)] If $[\vecb{c}_{\Delta}]_1 = \matr{\Delta} [\vecb{c}]_1$ but $b_{i^*} \neq \overline{b}_{i^*}$. Given that $[\vecb{c}]_1$ is perfectly binding and that $\bb_{i^*}\neq b_{i^*}$ is the unique opening of $[\vecb{c}_\Delta]_1$ at coordinate $i^*$, both commitments can not be opened to the same value. Therefore, the adversary $\advSound_2$ against the strong soundness of ${\Psi_\sfcom}$
outputs $\pi_\sfcom$ which is a fake proof for 
$([\vecb{c}_\Delta]_1,[\vecb{d}]_2)$. Note that strong soundness is required since, in order to compute $\{[\matr{C}_{i,j}]_1,[\matr{D}_{i,j}]_2:(i,j)\in\indexSet{n}{1}\}$, $\advSound_2$ requires the discrete logs of either $[\matr{G}]_1$ or $[\matr{D}]_2$.
\item[3)] If $[\vecb{c}_{\Delta}]_1 = \matr{\Delta} [\vecb{c}]_1$ and $b_{i^*} = \overline{b}_{i^*}$, then 
$b_{i^*}(\overline{b}_{i^*} -1) \neq 0$.
If we express $\matr{\Theta}+\matr{\Pi}$
as a linear combination of $\llck_{i}\lrck_{j}^{\top}$, the coordinate of
$\llck_{i^*}\lrck_{i^*}^\top$ is $b_{i^*}(\overline{b}_{i^*}-1)\neq 0$ and thus $\matr{\Theta}+\matr{\Pi}\notin\Span(\{\matr{C}_{i,j}+\matr{D}_{i,j}:(i,j)\in\indexSet{n}{1}\})$.
The adversary $\advSound_1$ against ${\Psi_\sfsum}$  outputs  $\pi_\sfsum$
which is a fake proof for $([\matr{\Theta}]_1, [\matr{\Pi}]_2)$. \footnote{The proof system $\Psi_\sfsum$ is constructed for matrices $\{(\matr{C}_{i,j},\matr{D}_{i,j}):(i,j)\in\indexSet{n}{1}\}$ sampled from some distribution $\dist_{{gk}}$, which in this case depends on the distribution of $\matr{G}$ and $\matr{H}$. We assume that the adversary $\advSound_2$ against $\Psi_\sfsum$ receives the common reference string of $\Psi_\sfsum$ as described in Section~\ref{sec:sum} and additionally the matrices $[\matr{G}]_1$ and $[\matr{H}]_2$ which defines the language, i.e. the distribution of $\matr{C}_{i,j}$, $\matr{D}_{i,j}$ (this is necessary so that $\advSound_2$ can simulate the crs for adversary $\advA$). We stress this additional information to describe the language does not affect the soundness proof for Theorem \ref{theo:membtwogroups1} (in particular,  $[\matr{G}]_1$ and $[\matr{H}]_2$ are independent of $(\matr{\Lambda}, \matr{\Xi})$).}
\end{itemize}
\end{proof}


        \subsection{Extensions} \label{sec:bits-extensions}

            \subsubsection{CRS Generation for Individual Commitments.}
A natural way to extend our construction to individual commitments (distribution (b) from Section~\ref{sec:bits-instantiations}) is the following. The only change is that the matrix $\matr{\Delta}$ is sampled uniformly from  $\Z_q^{2 \times 2n}$ (the distribution of $\Rck$ is not changed). Thus, the matrix 
$[\matr{G}]_1:=\matr{\Delta} [\matr{U}]_1$ has $2n$ columns instead of $n+1$ and 
$[\vecb{c}_{\Delta}]_1 := \Lck \smallpmatrix{\vecb{b} \\ \vecb{w}_g}$ for some $\vecb{w}_g \in \Z_q^{n}$.   
In the soundness proof, the only change is that in $\mathsf{Game}_2$, the extra columns are also changed to span a one-dimensional space, \textit{i.e.} in this game $[\vecb{g}_{i}]_1$, $i \in [2n-1]$ and $i \neq i^*$, are uniform vectors in the space spanned by $[\vecb{g}_{2n}]_1$.
With this approach, the proof size is still constant and the changes to the original construction are minimal but the CRS is considerably larger. Further, we do not know how to make the CRS linear for bit-strings of weight $1$. 

Therefore, we propose an alternative way to extend our result to individual commitments.
%\footnote{This is essentially the proof system from Section~\ref{sec:matr-bits} for $m=1$. {\color{red} Deber\'ia ir ac\'a esto? o basta con referenciar la sec no binding?}}
In this new construction, the matrix $\Lck$ is independent from $[\matr{U}]_1$ and for all $i \in [n]$, $[\vecb{g}_{i}]_1 = \mu_i [\vecb{g}_{n+1}]_1$, $\mu_i \gets \Z_q$ and   $[\vecb{g}_{n+1}]_1\gets\Z_q^2$. 

The proof is defined in a slightly different way. Now one computes $[\vecb{c}_\Delta]_1:=\Lck\smallpmatrix{\vecb{b}\\w'_g}$, $w'_g\gets\Z_q$, and one proves that the three commitments, $[\vecb{c}]_1,[\vecb{c}_\Delta]_1,[\vecb{d}]_2$ open to the same value.  Intuitively, this replaces in the original construction the proofs that $\matr{\Delta}[\vecb{c}]_1=[\vecb{c}_\Delta]_1$ and that $\matr{\Delta}[\vecb{c}]_1$ and $[\vecb{d}]_2$ open to the same value. More specifically, this is proven by showing that $(\smallpmatrix{{[\vecb{c}]_1}\\{[\vecb{c}_\Delta]_1}}, [\vecb{d}]_2)\in\Lang_{[\matr{M}]_1,[\matr{N}]_2}$, where.
\begin{align*}
&\matr{M} := 
\pmatri
{
    \matr{U}_1 & \matr{U}_2           & \matr{0}_{2n\times 1} & \matr{0}_{2n\times 1}\\
    \vecb{G}_1      & \matr{0}_{2\times n} & \vecb{g}_{n+1}             & \matr{0}_{2\times 1}\\
}
\text{ and }
\matr{N} :=
\pmatri{
    \vecb{H}_1      & \matr{0}_{2\times n} & \matr{0}_{2\times 1}  & \vecb{h}_{n+1}
}.
\end{align*}
The advantage of this alternative approach is that the matrix $\Lck$ has now $n+1$ columns as in the original construction as opposed to $2n$ in the first extension to individual commitments.  


The proof of soundness must be modified in the following way.  In the proof of Lemma \ref{lemma:bits2} one sets $[\vecb{g}_{n+1}]_1:=[\vecb{s}]_1$ and $[\vecb{g}_{i^*}]_1:=[\vecb{t}]_1$, similarly as done in Lemma \ref{lemma:bits3}. This guarantees that, as in the original construction, in the last game  $[\vecb{g}_{i^*}]_1$ (resp. $[\vecb{h}_{i^*}]_2$) is linearly independent of the rest of columns of $\Lck$ (resp. $\Rck$). In the last game we need to show that $[\vecb{c}_{\Delta}]_1=b_{i^*}[\vecb{g}_{i^*}]_1+\tilde{w}_g[\vecb{g}_{n+1}]_1$ and $[\vecb{d}]_2=b_{i^*}[\vecb{h}_{i^*}]_2+\tilde{w}_h[\vecb{h}_{n+1}]_2$, for some $\tilde{w}_g,\tilde{w}_h\in\Z_q$ and that $b_{i^*} \in \{0,1\}$. Note that the fact that $(\smallpmatrix{{[\vecb{c}]_1}\\{[\vecb{c}_\Delta]_1}}, {[\vecb{d}]_2})\in\Lang_{[\matr{M}]_1,[\matr{N}]_2}$ implies that there is some $\grkb{\gamma}\in\Z_q^{2n+2}$ such that $\smallpmatrix{{[\vecb{c}]_1}\\{[\vecb{c}_\Delta]_1}\\{[\vecb{d}]_2}}=\smallpmatrix{{[\matr{M}]_1}\\{[\matr{N}]_2}}\grkb{\gamma}$, and the fact that $[\vecb{c}]_1$ is perfectly binding together with the form of $[\matr{M}]_1,[\matr{N}]_2$ implies that $\grkb{\gamma}=\smallpmatrix{\vecb{b}\\\grkb{\gamma}'}$. In particular, $[\vecb{c}_\Delta]_1 = \Lck\smallpmatrix{\vecb{b}\\\gamma_{2n+1}}=b_{i^*}+\tilde{w}_g[\vecb{g}_{n+1}]_1$ and $[\vecb{d}]_2=\Rck\smallpmatrix{\vecb{b}\\\gamma_{2n+2}}=b_{i^*}[\vecb{h}_{i^*}]_2+\tilde{w}_h[\vecb{h}_{n+1}]_2$ for some unique $b_{i^*}$. To conclude the proof of soundness we just need to argue that $b_{i^*} \neq \{0,1\}$, leads to a contradiction. This follows from the same argument as the original proof. 


For zero-knowledge, observe that $[\vecb{c}_{\Delta}]_1$ is just a uniform vector in $\Span([\vecb{g}_{n+1}]_1)$. The simulator just picks a random $[\vecb{c}_{\Delta}]_1$ and simulates the proof that $\left(\smallpmatrix{{[\vecb{c}]_1}\\{[\vecb{c}_\Delta]_1}}\right.,\allowbreak\left.\allowbreak {[\vecb{d}]_2}\right)\in\Lang_{[\matr{M}]_1,[\matr{N}]_2}$ with the appropriate trapdoor. The rest of the proof is identical to the simulated proof in the original construction.  


\subsubsection{Linear Equations Satisfied by Bit-Strings}\label{sec:linear-eqs-bitstrings}
Because of the homomorphic properties of the commitments, 
we can easily extend it to prove that the bit-string $ \vecb{b}$ satisfies $\sum_{i \in [n]} \beta_i b_i=t$, for some $\grkb{\beta} \in \Z_q^{n}, t\in \Z_q$. 
If the commitment $[\vecb{c}]_1$ is a concatenation of GS commitments to $b_i$, this can be done in the usual way with GS proofs. 
But if $\matr{U}$ is drawn from distribution (a) (see Section~\ref{sec:bits-instantiations})
this can also be done as follows. 
Define $\matr{B}:=\smallpmatrix{ \beta_1 & \ldots & \beta_n & 0\\ 0 & \ldots & 0 &1}
\in \Z_q^{2\times (n+1)}$. %and let $\vecb{e}_2^{\ell}$ denote the $i$ th vector of the canonical basis of $\Z_q^\ell$.  
We claim the following:
$$\matr{B}[\vecb{c}]_1- t [\vecb{e}^{2}_{1}] \in \Span\left(\matr{B}[\vecb{u}_{n+1}]_1\right) \Leftrightarrow \sum_{i \in [n]}\beta_i  b_i=t.$$
This is justified because
$\matr{B}\vecb{u}_i = \matr{B}\vecb{e}^{n+1}_i=\beta_i\vecb{e}_1^{2}$, and then
$ \matr{B}\vecb{c}- t \vecb{e}^{2}_{1}=w  \matr{B} \vecb{u}_{n+1}+ \sum_{i \in [n]} b_i \matr{B} \vecb{u}_{i} - t \vecb{e}^{2}_1=w\matr{B}\vecb{u}_{n+1}+\left(\sum_{i\in[n]}\beta_i b_i-t\right)\vecb{e}_1^2$.
So to be able to prove that $\sum_{i \in [n]} \beta_i b_i=t$, we just need to add to the CRS the necessary elements to prove membership 
in $\mathcal{L}_{B[\vecb{u}_{n+1}]_1}:=\{ [\vecb{x}]_1 \in \Gr^2: \exists w \in \Z_q,  \vecb{x} = w\matr{B}\vecb{u}_{n+1} \}$ using one of the constructions of Section~
\ref{sect:QANIZKlinspace}.

\subsubsection{Bit-Strings of Weight 1}  In the special
case when the bit-string has only one $1$ (this case is useful in some applications, see Section~\ref{sec:bits-applications}),  the size of the CRS can be made linear in $n$, instead of quadratic.
To prove this statement we would combine our proof system for bit-strings of Section~\ref{sec:bits-scheme} and a proof that  $\sum_{i \in [n]} b_i=1$ as described above
when $m=1$ or using GS-proofs when $m=n$.
In the definition of $(\matr{\Theta},\matr{\Pi})$ in Eq.~\ref{eq:ThetaPi}, one 
sees that for all pairs $(i,j) \in [n] \times [n]$, the coefficient of $(\vecb{C}_{i,j},\vecb{D}_{i,j})$ is $b_i(b_j-1)$.
If $i^*$ is the only index such that $b_{i^*}=1$, then we have:
$$\sum_{i \in [n]} \sum_{j \in [n]} b_i(b_j-1) (\vecb{C}_{i,j},\vecb{D}_{i,j}) = \sum_{j \neq i^*} (\vecb{C}_{i^*,j},\vecb{D}_{i^*,j})=: (\vecb{C}_{i^*,\neq},\vecb{D}_{i^*,\neq}).$$
Therefore, one can replace in the CRS the pairs of matrices  $ (\vecb{C}_{i,j},\vecb{D}_{i,j})$ by  $(\vecb{C}_{i,\neq},\vecb{D}_{i,\neq})$, $i \in [n]$. The resulting CRS is linear in $n$.



        \subsection{Applications} \label{sec:bits-applications}

            Many protocols use proofs that a commitment opens to a bit-string as a 
  building block. 
Since our commitments are still of size $\Theta(n)$,
  our results may not apply to some of these protocols. 
Yet, there are several applications where 
  bits need to be used independently and our results provide 
  significant improvements.
Table \ref{table:app} summarizes them.
 
\subsubsection{Signatures} Some application examples are the signature schemes of  \cite{PKC:BFPV11,SCN:BlaPoiVer12,RSA:Camacho13,AFRICACRYPT:EscHerMor11}. For example, in the revocable attribute-based signature scheme of Escala \textit{et. al} \cite{AFRICACRYPT:EscHerMor11}, every signature includes a proof that a set of GS commitments, whose size is the number of attributes, opens to a bit-string. 

Further, the set membership proof discussed below can also be used to reduce the size of the Ring Signature scheme of  \cite{ICALP:ChaGroSah07}, which is the most efficient ring signature in the standard model. Indeed, to sign a message $m$, among other things, the signer picks a one-time signature key and certifies the one-time verification key by signing it with a Boneh-Boyen signature under $vk_\alpha$. Then, the signer commits to $vk_\alpha$ and shows that $vk_\alpha$ belongs to the set of Boneh-Boyen verification keys $(vk_1,\ldots,vk_n)$ of the parties in the ring $R$. 


\begin{table}[h]
\begin{center}
\begin{minipage}{\textwidth}
\begin{center}
\begin{small}
             %   proof  % CRS   %
% group type %  Gr % Hr % Gr Hr % #Pairings
\begin{tabular}{|l||l|l|}
\hline
Proof System & Author                                                    & Proof Size 
\\ \hline\hline
\multirow{3}{*}{Threshold GS} & R\`afols \cite{TCC:Rafols15} (1) & $(m_x + 3(n-t) + 2\bar{n},0)$     \\
\cline{2-3}                   & R\`afols \cite{TCC:Rafols15} (2) & $2(n-t+1,n)$         \\
\cline{2-3}                   & This work                         & $(2n+12,10)$             \\
\hline
\multirow{3}{*}{\minitbl{Set-Membership proof}{(Ring Signature)}} & Chandran et al. \cite{ICALP:ChaGroSah07}  
&  $(16\sqrt{n}+4,16\sqrt{n}+4)$    \\
\cline{2-3}
                                         & R\`afols \cite{TCC:Rafols15}              &  $(8\sqrt{n}+6,12\sqrt{n})$    \\
\cline{2-3} 
                                         & This work                                 & $(4 \sqrt{n}+14,8\sqrt{n}+14)$ \\ \hline
\multirow{2}{*}{\minitbl{Set-Membership proof}{(fixed set)}}  & This work (first scheme)                  & $(4\sqrt{n}+16,2\sqrt{n}+22)$\\
\cline{2-3}
                                         & This work (second scheme)                 & {$(6 \sqrt[3]{n}+36,6\sqrt[3]{n}+60)$}\\ \hline
% using proof of memb in a list again \minitbl{$(8 \sqrt[3]{n}+6 \sqrt[6]{n}+28)\sG+$}{$(2\sqrt[3]{n}+6\sqrt[6]{n}+38)\sH$}
\end{tabular}
\end{small}
\end{center}
\caption{Comparison of the application of our techniques and results from the literature. Notation $(a,b)$ means $a$ elements of $\GG_1$ and $b$ elements of $\GG_2$. In rows labeled  as ``Threshold GS'' 
we give the size of the proof of satisfiability of $t$-out-of-$n$ sets $\mathcal{S}_i$, where $m_x$ is the sum of the number of variables in $\Gr$ in each set $\mathcal{S}_i$, and $\bar{n}$ is the total number of two-sided and quadratic equations in $\bigcup_{i\in[n]}\mathcal{S}_i$. For all rows, we must add to the proof size the cost of a GS proof of each equation in one of the sets $\mathcal{S}_i$. In the other rows $n$ is the size of the set.\label{table:app}}
\end{minipage}
\end{center}
\end{table}


\subsubsection{Threshold GS Proofs for PPEs} There are two approaches to construct threshold GS proofs for PPEs, i.e. proofs of satisfiability of $t$-out-of-$n$ equations. One is due to \cite{AC:Groth06} and consists of compiling the $n$ equations into a single equation which is satisfied only if $t$ of the original equations are satisfied. For the case of PPEs, this method adds new variables and proves that each of them opens to a bit.  Our result reduces the cost of this approach, but we omit any further discussion as it is quite inefficient because the number of additional variables is $\Theta(m_{var}+n)$, where $m_{var}$ is the total number of variables in the original $n$ equations.

The second approach is due to R\`afols \cite{TCC:Rafols15}. The basic idea behind \cite{TCC:Rafols15}, which extends \cite{C:GroOstSah06}, follows from the observation that for each GS equation type 
$\mathsf{tp}$,  the CRS space $\mathcal{K}$ is partitioned into a perfectly sound CRS space $\mathcal{K}^b_\mathsf{tp}$ and a perfectly witness indistinguishable CRS space $\mathcal{K}^{h}_\mathsf{tp}$.

In particular, to prove satisfiability of $t$-out-of-$n$ sets of equations from $\{\mathcal{S}_i:i\in[n]\}$ of type $\mathsf{tp}$, it suffices to construct an algorithm $\mathsf{K}_{\mathsf{corr}}$ which on input
$\mathsf{crs}_{\GS}$ and some set of indexes $A \subset [n]$, $|A|=t$, generates $n$ GS common reference strings 
$\{\mathsf{crs}_i, i \in [n]\}$ and simulation trapdoors $\tau_{i,sim}$, $i \in A^c$, in a such a way that\footnote{More technically, this is the notion of \textit{Simulatable Verifiable Correlated Key Generation} in \cite{TCC:Rafols15}, which extends the definition of Verifiable Correlated Key Generation of \cite{C:GroOstSah06}.}:
\begin{itemize}
\item[a)] it can be publicly verified the set of perfectly sound keys, 
$\{\mathsf{crs}_i : \mathsf{crs}_i \in \mathcal{K}_{\mathsf{tp}}^{b}\}$ is of size at least $t$,
\item [b)] there exists a simulator $\mathsf{S}_{\mathsf{corr}}$ who outputs $(\mathsf{crs}_i,\tau_{i,sim})$ for all $i \in [n]$, and the distribution of $\{\mathsf{crs}_i : i \in [n]\}$ is the same as the one of the keys  output by $\mathsf{K}_{\mathsf{corr}}$ when $\mathsf{crs}_{\GS}$ is the perfectly witness-indistinguishable CRS.
\end{itemize}
The prover of $t$-out-of-$n$ satisfiability can run $\mathsf{K}_{\mathsf{corr}}$ and, for all $i \in [n]$, compute a real (resp. simulated) proof for 
$\mathcal{S}_i$ with respect to $\mathsf{crs}_i$
when $i \in A$ (resp. when $i \in A^c$).

R\`afols gives two constructions for PPEs, the first one can be found in \cite{TCC:Rafols15}, App. C and the other follows from \cite[Section~7]{TCC:Rafols15}\footnote{The construction in \cite[Section~7]{TCC:Rafols15} is for other equation types but can be used to prove that $t$-out-of-$n$ of $\mathsf{crs}_1,\ldots,\mathsf{crs}_n$ are perfectly binding for PPEs.
}. 
Our algorithm $\mathsf{K}_\mathsf{corr}$ for PPEs\footnote{Properly speaking the construction is for PPEs which are left-simulatable in the terminology of \cite{TCC:Rafols15}.} goes as follows:

\begin{itemize}
\item Define $(b_1,\ldots,b_n)$ as $b_i=1$
if $i \in A$ and $b_i=0$ if $i \in A^c$. For all $i \in [n]$, let $[\vecb{z}_i]_1:=\mathsf{GS.Comm}_{ck}(b_i)=b_i [\vecb{u}_1]_1+ r_i [\vecb{u}_2]_1$, $r_i \in \Z_q$, and define $\tau_{sim,i}=r_i$, for all $i \in A^c$. Define $\mathsf{crs}_i:=(\Gamma,[\vecb{z}_i]_1,[\vecb{u}_2]_1,[\vecb{v}_1]_2,[\vecb{v}_2]_2)$. 
\item Prove that $([\vecb{c}_1]_1,\ldots,[\vecb{c}_n]_1)$ opens to $\vecb{b} \in \{0,1\}^{n}$ and that $\sum_{i\in[n]} b_i=t$.
\end{itemize}
The simulator just defines $\vecb{b}=\vecb{0}$. The reason why this works is that when $b_i=1$, $([\vecb{z}_i]_1-[\vecb{u}_1]_1) \in \mathsf{Span}( [\vecb{u}_2]_1)$, therefore $\mathsf{crs}_i \in \mathcal{K}^{b}_{PPE}$ and when
$b_i=0$,  $([\vecb{z}_i]_1-[\vecb{u}_1]_1) \notin \mathsf{Span}( [\vecb{u}_2]_1)$ so $\mathsf{crs}_i \in \mathcal{K}^{h}_{PPE}$.

\subsubsection{More Efficient Set-Membership Proofs} Chandran \textit{et al.} construct a ring signature of size $\Theta(\sqrt{n})$ \cite{ICALP:ChaGroSah07}, which is the most efficient ring signature in the standard model. Their construction uses as a subroutine a non-interactive proof of membership in some set $S=([s_1]_1,\ldots,[s_n]_1)$ which is of size $\Theta(\sqrt{n})$.  The trick of Chandran \textit{et al.} to achieve this asymptotic complexity is to view $S$ as a matrix $[\matr{S}]_1\in\Gr^{m\times m}$, for $m=\sqrt{n}$, where the $i,j$ th element of $[\matr{S}]_1$ is $[s_{i,j}]_1 := [s_{(i,j)}]_1$ and $(i,j):=(i-1)m+j$. Given a commitment $[\vecb{c}]_1$ to some element $[s_{\alpha}]_1$, where $\alpha = (i_\alpha,j_\alpha)$, their construction in asymmetric bilinear groups works as follows :
\begin{enumerate}
\item Compute GS commitments in $\Hr$ to $b_1\ldots,b_m$ and $b'_1,\ldots,b'_m$,
      where $b_i = 1$ if $i=i_\alpha$ and $0$ otherwise, and $b'_{j}=1$ if $j=j_\alpha$, and $0$ otherwise.
\item Compute a GS proof that $b_i \in \{0,1\}$ and $b'_j \in \{0,1\}$ for all $i,j\in[m]$, and that $\sum_{i\in[m]}b_i=1$, and $\sum_{j\in[m]}b'_j=1$.
\item Compute GS commitments to $[x_1]_1:=[s_{(i_\alpha,1)}]_1,\ldots,[x_m]_1:=[s_{(i_\alpha,m)}]_1$.
\item Compute a GS proof that $[x_j]_1 = \sum_{i\in[m]}b_i[s_{(i,j)}]_1$, for all $j\in[m]$, is satisfied. 
\item Compute a GS proof that $[s_{\alpha}]_1 = \sum_{j\in[m]}b'_j[x_j]_1$ is satisfied.
\end{enumerate}
With respect to the naive use of GS proofs, Step $2$ was improved by R\`afols \cite{TCC:Rafols15}.  Using our proofs for bit-strings of weight $1$ from Section~\ref{sec:bits-extensions}, we can further reduce the size of the proof in step 2, 
see table.

We note that although in step 4 the equations are all two-sided linear equations, proofs can only be aggregated if the set comes from a witness samplable distribution and the CRS is set to depend on that specific set. 
This is not useful for the application 
to ring signatures, since the CRS should be independent of the ring $R$ (which defines the set). If aggregation is possible then the size of the proof in step 4 is reduced from $(2\sG+4\sH)\sqrt{n}$ to $4\sG+8\sH$.

Next we describe the construction generalized to vectors (which will be useful in the next construction) and then we show that,
when the CRS depends on the set and the set is witness samplable, the proof can be further reduced to $\Theta(\sqrt[3]{n})$.

\subsubsection{More Efficient Set-Membership proof for Set of Vectors}  
Here we describe a more efficient version of Chandran's \textit{et al.} Set-Membership proof,  
which is also extended to vectors --i.e. to the case where $S=([\vecb{s}_1]_1,\ldots,[\vecb{s}_n]_1)$ is a set of vectors of length $\ell$. In such a proof, we show that some commitment $[\vecb{c}]_1$ opens to a vector $[\vecb{s}_{\alpha}]_1$, where $\alpha=(i_\alpha,j_\alpha)$ (recall that $(i,j)=\sqrt{n}(i-1)+j$).

\begin{enumerate}
\item Compute GS commitments in $\Hr$ to $b_1\ldots,b_m$ and $b'_1,\ldots,b'_m$,
      where $b_i = 1$ if $i=i_\alpha$ and $0$ otherwise, and $b'_{j}=1$ if $j=j_\alpha$, and $0$ otherwise.
\item Compute a proof that $b_i \in \{0,1\}$ and $b'_j \in \{0,1\}$ for all $i,j\in[m]$, using the proof system of Section~
~ \ref{sec:bits-scheme}.
\item Compute GS proofs that $\sum_{i\in[m]}b_i=1$ and $\sum_{j\in[m]}b'_j=1$. 
\item Compute GS commitments to each coordinate of $[\vecb{x}_1]_1:=[\vecb{s}_{(i_\alpha,1)}]_1,\ldots,[\vecb{x}_m]_1:=[\vecb{s}_{(i_\alpha,m)}]_1$.
\item Compute an aggregated GS proof that the equations $[\vecb{x}_j]_1 = \sum_{i\in[m]}b_i[\vecb{s}_{(i,j)}]_1$, for all $j\in[m]$, are satisfied, as detailed in Section~\ref{sec:aggcommit}.
\item Compute a GS proof that $[\vecb{s}_{\alpha}]_1 = \sum_{j\in[m]}b'_j[\vecb{x}_j]_1$ is satisfied.
\end{enumerate}

We emphasize that the CRS depends on the set $S$. This is necessary to aggregate the proofs as in Step 4. More  specifically, to aggregate the proof of the equations  $[\vecb{x}_j]_1 = \sum_{i\in[m]}b_i [\vecb{s}_{(i,j)}]_1$ , $j\in[m]$ (that is, a total of $\ell k$ equations), we need to include in the CRS some information which depends on the coordinates of $[\vecb{s}_{(i,j)}]_1$.

\begin{theorem}\label{theo:listsquarevect}
If $S$ is witness samplable, the above protocol is a perfectly complete, computationally sound, and computationally zero-knowledge proof system for the language of commitments to elements from the set $S$.
\end{theorem}

\begin{proof} Completeness follows directly from the completeness of the 
building blocks. Soundness follows directly from the perfect soundness of GS proofs together with the computational soundness of aggregation of GS proofs. For computational zero-knowledge, if  
$\crs_\GS:=(\Gamma,[\vecb{u}_1]_1,[\vecb{u}_2]_1,[\vecb{v}_1]_2,[\vecb{v}_2]_2)$ is the GS common reference string in the soundness setting as defined in Section~ \ref{sec:gs-proofs}, switch to a game where $[\vecb{v}_1]_2= \epsilon [\vecb{v}_2]_2$. Under the DDH Assumption in $\Hr$, the new CRS is computationally indistinguishable from the original CRS. In a simulated proof, commit to $b_i=0$, $b'_j=0$ for all $i,j \in [m]$. In step 2, simply compute a real proof. In step 3, use the GS simulation algorithm  (with trapdoor $\epsilon$) to simulate the proof. In Step 4, set $[\vecb{x}_j]_1=[\vecb{0}]_1$. Finally, in step 6, simulate a proof using $\epsilon$. It is not hard to see that such a proof can be simulated even without knowledge of an opening of $[\vecb{c}]_1$. 
\end{proof}

\subsubsection{A $\Theta(\sqrt[3]{n})$ Set-Membership proof for Witness Samplable non-Adaptive Sets} We give Set-Membership proof with improved asymptotic proof size when the set is drawn from a witness samplable distribution and the CRS depends on the set (two different sets require two different CRS).

The main idea is to combine the previous Set-Membership proof with a Split Kernel Assumption. Specifically, the CRS includes a matrix $[\matr{A}]_2$, $\matr{A} \gets\dist_{m,2}$, whose rows are denoted $[\grkb{a}_1]_2,\ldots,[\grkb{a}_m]_2$ and a set 
$$S':=\left(\sum_{i\in[m]}\grkb{a}_i[s_{(i,1,1)}]_1,\sum_{i\in[m]}\grkb{a}_i[s_{(i,1,2)}]_1,\ldots,\sum_{i\in[m]}\grkb{a}_i[s_{(i,m,m)}]_1\right)\in\Gr^{2\times m^2},$$
where $m:=\sqrt[3]{n}$, $(i,j,k)=m^2(i-1)+m(j-1)+k\in[n]$. As before, the goal to prove is that a commitment $[\vecb{c}]_1$ opens to some $[s_{\alpha}]_1 \in S=\{[s_1]_1,\ldots,[s_n]_1\}$ and  $(i_{\alpha},j_{\alpha},k_{\alpha})$ are such that $\alpha=(i_{\alpha},j_{\alpha},k_{\alpha})$.

\begin{enumerate}
\item Commit to $[\vecb{y}]_1:=\sum_{i\in[m]}\grkb{a}_i[s_{(i,j_\alpha,k_\alpha)}]_1$ such that $\alpha=(i_\alpha,j_\alpha,k_\alpha)$.
\item Using the Set-Membership proof for vectors to show that $[\vecb{y}]_1$ is an element of $S'$.
\item Compute commitments to $[z_i]_1:=[s_{(i,j_\alpha,k_\alpha)}]_1$, for each $i\in[m]$.
\item Compute a GS proof for the equations $[\vecb{y}]_1[h]_2=\sum_{i\in[n]}[\grkb{a}_i]_2 [z_i]_1$. 
\item Compute GS commitments in $\Hr$ to $b_1\ldots,b_m\in\bits$,
      where $b_i = 1$ if $i=i_\alpha$ and $0$ otherwise.
\item Using our proof system from Section~\ref{sec:bits-scheme} prove that $b_i\in\bits$ for all $i\in[m]$.
\item Compute GS proofs for the satisfiability of equations $\sum_{i\in[m]}b_i=1$ and $[s_{\alpha}]_1=\sum_{i\in[m]}b_i[z_i]_1$.
\end{enumerate}

The first step is to commit to $[\vecb{y}]_1:= \sum_{i\in[m]} \grkb{a}_i [s_{(i,j_\alpha,k_\alpha)}]_1$ and use the previous proof system to prove $[\vecb{y}]_1 \in S'$. The next step is to commit to $[z_i]_1:=[s_{(i,j_\alpha,k_\alpha)}]_1$ and prove that $\sum_{i\in[m]}[\grkb{a}_i]_2[z_i]_1 = [h]_2[\vecb{y}]_1$ holds. Finally, steps 5 and 6 prove that $[s_{\alpha}]_1$ is an element of the set $([z_1]_1,\ldots,[z_m]_1)$. For the last statement, compute GS commitments to $b_i$, $i \in [m]$, 
and prove that $\sum_{i\in[m]}b_i[z_i]_1=[s_{\alpha}]_1$, $\sum_{i\in[m]}b_i=1$ and $b_i \in \{0,1\}$.\footnote{Such statement can also be proven using again the Set-Membership proof, and the proof will be of size $\Theta(\sqrt[6]{n})$. Note this is not exactly a Set-Membership proof, since only the commitments to the elements in the set are public. However, it is not hard to construct a proof system for that statement using the same ideas as Chandran \textit{et al.} }

\begin{theorem}
If $S$ is witness samplable, the above protocol is a perfectly complete, computationally sound, and computationally zero-knowledge proof system for the language of commitments to elements from the set $S$.
\end{theorem}

\begin{proof} Completeness follows directly from the completeness of the 
building blocks.  Soundness can be argued as follows. If the set is witness samplable, the CRS can be generated given an instance of the $\dist_{m,2}\mbox{-}\skermdh$ Assumption, $([\matr{A}]_1, [\matr{A}]_2)$. By the soundness of the extension to vectors of the proof of Chandran \textit{et al.}, it holds that $[\vecb{y}]_1 = \sum_{i\in[m]}\grkb{a}_i[s_{(i,j,k)}]_1$ for some $j,k\in[m]$.
Because of the perfect soundness of GS proofs it must hold that $\sum_{i\in[m]}[\grkb{a}_i]_2[z_i]_1=[\vecb{y}]_1[h]_2=\sum_{i\in[m]}[\grkb{a}_i]_2[s_{(i,j,k)}]_1$. It must also be the case that $[z_1]_1=[s_{(1,j,k)}]_1,\allowbreak\ldots,[z_m]_1=[s_{(m,j,k)}]_1$, because otherwise the pair $([\grkb{\rho}]_1,[\grkb{0}]_2)$, where 
$[\grkb{\rho}]_1:=([z_1]_1-[s_{(1,j,k)}]_1,\allowbreak\ldots,[z_m]_1-[s_{(m,j,k)}]_1)$ is a solution to the $\dist_{m,2}\mbox{-}\skermdh$ challenge, as $[\grkb{\rho}]_1[\matr{A}]_2=[\grkb{0}]_2[\matr{A}]_1$. Soundness of the last step implies that $b_i\in\bits$, for all $i\in[m]$, and that $\sum_{i\in[m]} b_i=1$. Therefore, there exists a unique $i\in[m]$ such that $b_i=1$. Finally, $[s_{\alpha}]_1=\sum_{i\in[m]}b_i[z_i]_1$ implies that $[\vecb{c}]_1$ opens to $[s_{\alpha}]_1=[z_i]_1=[s_{(i,j,k)}]_1$.
Zero-knowledge follows from the same argument as in the proof of Theorem \ref{theo:listsquarevect}.
\end{proof} 



    \section{The Non-binding Case} \label{sec:bits-non-binding}

        In this section we construct a constant-size proof that a non-binding commitment to a vector from $\Z_q^n$ opens to an element from $\bits^n$. We remark that in this case the size of the commitment is independent of $n$. 

In Section~\ref{sec:ext-mp} we introduce a new commitment scheme, \emph{Extended Multi-Pedersen Commitments}, which is an ``hybrid'' between Groth-Sahai commitments and Multi-Pedersen commitments.
Then, in Section~\ref{sec:bits-scheme-nb}, we construct a QA-NIZK argument that an extended Multi-Pedersen commitment opens to an element from $\bits^n$. Finally, in Section~\ref{sec:matr-bits} we show how to give a constant-size proof that many Multi-Pedersen commitments open to bit-strings. Using this last proof system one can derive our construction for the perfectly-binding case as a simple corollary and, further, it helps to construct the efficient constructions from Chapter \ref{sec:shuf-rp}.


        \subsection{Extended Multi-Pedersen Commitments} \label{sec:ext-mp}

            In this section  we introduce a new commitment scheme which is a generalization of multi-Pedersen commitments and which was implicitly used in Section~\ref{sec:bits-scheme}. 

Given a vector $\vecb{m}\in\Z_q^m$, the multi-Pedersen commitment in $\GG_{\gamma}$ is a single group element $[c]_\gamma:=\sum_{i\in [m]} m_i[g_i]_\gamma+r[g_{m+1}]_\gamma \in\GG_\gamma$, where $[g_i]_{\gamma}\in\GG_\gamma$, $i\in[m+1]$, and $r\gets\Z_q$. \footnote{Written in the usual multiplicative notation $c=\prod_{i\in[m]}g_i^{m_i} \cdot g_{m+1}^r$.}  The $(k+1)$-dimensional multi-Pedersen commitments 
differs only in that the keys and the resulting commitments are in 
$\GG_{\gamma}^{k+1}$, for $k\geq 1$. 


While the original MP commitments are perfectly hiding, the interest of the new commitments is that, if the keys come from the distribution $\distink$ defined in Section~\ref{sec:mddh}, they are perfectly binding at coordinate $i$. Intuitively, the new commitment is defined in a larger space so that not all the information about the witness is destroyed (in an information-theoretic sense). 

\begin{definition} The $(k+1)$-dimensional multi-Pedersen commitment scheme in the group $\GG_\gamma$ 
%is parameterized by a matrix distribution $\dist_{2,nm+1}$ and 
is specified by the following three algorithms 
	$\mathsf{MP}=(\mathsf{MP}.\algK,\mathsf{MP}.\Com,$ $ \mathsf{MP}.\algVrfy)$:
	\begin{itemize} 
		\item  $\mathsf{MP}.\algK$ is a randomized algorithm, which on input the group key $gk$, a natural number $m \in \N$, and the description of some matrix distribution $\dist_{k+1,m+k}$, 
		outputs a commitment key $ck:=[\matr{G}]_\gamma$, where $\matr{G} \gets \dist_{k+1,m+k}$.
		\item $\mathsf{MP}.\Com$ is a randomized algorithm which, om input a commitment key $ck=[\matr{G}]_\gamma$, and a message 
		$\vecb{m}$ in the message space $\mathcal{M}_{ck}=\Z_q^{m}$, samples $\vecb{r} \gets \Z_q^k$ and outputs a commitment $\bvecb{c}_\gamma:=\bmatr{G}_\gamma\smallpmatrix{\vecb{m} \\ \vecb{r}}$ in the commitment space $\mathcal{C}_{ck}=\GG_\gamma^{k+1}$ and an opening $Op=\vecb{r}$, 
		\item $\mathsf{MP}.\algVrfy$ is a deterministic algorithm which, on input the commitment key $ck=\bmatr{G}_\gamma$, a commitment $\bvecb{c}_\gamma$,  a message 
		$\vecb{m} \in \Z_q^{m}$ and an opening $Op=\vecb{r}\in\Z_q^k$, outputs $1$ if $\bvecb{c}_\gamma=\bmatr{G}_\gamma\smallpmatrix{\vecb{m} \\ \vecb{r}}$
		and $0$ otherwise. 
	\end{itemize}
\end{definition}

\begin{theorem} \label{theo:mp} The $\MP$ scheme is computationally binding if  the discrete logarithm assumption holds in $\GG_\gamma$. Further, if 
$\dist_{k+1,m+k}=\distink$, it holds that: 
\begin{itemize}
\item If $i=0$,  then $\MP$ is perfectly hiding,
\item If $i \in [m]$, then $\MP$ is statistically binding at coordinate $i$, which means that for each $[\vecb{c}]_\gamma \in \GG_{\gamma}^{k+1}$,
there exists a unique $\tilde{m}_i \in\Z_q$ such that for all $\vecb{m} \in\Z_q^m, \vecb{r} \in\Z_q^{k}$ such that  $\bvecb{c}_{\gamma}=\bmatr{G}_{\gamma}\smallpmatrix{\vecb{m}\\\vecb{r}}$, $m_i=\tilde{m}_i$. Further, the scheme is perfectly hiding at the rest of coordinates. 
\end{itemize}
\end{theorem}

\begin{proof}
(Computationally binding.) (This follows a proof due to Jorge Villar). Let $[a]_\gamma\in\GG_{\gamma}$ be the discrete logarithm challenge. To sample the commitment key according to $\distink$, choose $\matr{G}_{2} \gets \dist_k$, and define the last $k$ columns of $[\matr{G}]_\gamma$ as $[\matr{G}_{2}]_{\gamma}$. For the rest of the columns of $[\matr{G}]_\gamma$, independently  for each $j \in [m]$, $i \neq j$, sample a pair $\vecb{\alpha}_j,\vecb{\beta}_j$ and define $[\vecb{g}_{j}]_{\gamma}=[\matr{G}_{2} (a \vecb{\alpha}_j+\beta_j)]_\gamma$,
which can be computed as $[a]_\gamma \matr{G}_{2}\vecb{\alpha}_j+[ \matr{G}_{2} \beta_j]_\gamma$. If $i \neq 0$, set $\vecb{g}_{i} \gets \Z_q^{k+1}$. In this case, with overwhelming probability,  $\vecb{g}_{i}$ is linearly independent of the rest of the columns and we will assume so in the following. 
The commitment key is then given to the adversary against the binding property of the scheme, and it outputs a commitment $[\vecb{c}]_{\gamma}$, together with two valid openings 
$(\vecb{m},\vecb{r}), (\vecb{m}',\vecb{r}')$ such that $\vecb{m}\neq \vecb{m}'$. It follows that $[\vecb{c}]_{\gamma}=[\matr{G}]_{\gamma} \smallpmatrix{\vecb{m} \\ \vecb{r}} = \bmatr{G}_{\gamma}\smallpmatrix{\vecb{m}' \\ \vecb{r}'}$, which implies that $[\vecb{0}]_{\gamma}=[\matr{G}]_{\gamma} \smallpmatrix{\vecb{m}-\vecb{m}' \\ \vecb{r}-\vecb{r}'}$. Further, because $\vecb{g}_i$ is linearly independent of the rest of the columns, it holds that:
\begin{equation}\label{recovera}
a \left(\matr{G}_{2} (\sum_{j \neq i} (m'_j-m_j)  \boldsymbol{\alpha}_j) \right) = \left(  \matr{G}_{2}  (\vecb{r}-\vecb{r}' + \sum_{j \neq i} \boldsymbol{\beta}_j  (m_j-m_j'))\right).
\end{equation}
W.l.o.g we can assume that $\matr{G}_{2}$ has full rank (it can be shown that if $\dist_k$-$\mddh$ is a generically hard assumption in $k$-linear groups, then matrices sampled from $\dist_k$ have full rank with overwhelming probability).  Then, we can recover $a \in \Z_q$ from equation \ref{recovera} except if $\sum_{j \neq i} (m'_j-m_j)  \boldsymbol{\alpha}_j = \vecb{0}$. But since, for all $j$, $\boldsymbol{\alpha}_j$ is information theoretically hidden from the adversary, the probability of this event is at most 
$1/q^k$. 

(Perfectly binding at coordinate $i$.) With overwhelming probability, $\vecb{g}_i$ is linearly independent of the rest of the columns of $\matr{G}$. Therefore, given any $[\vecb{c}]_{\gamma} \in \GG_{\gamma}^{k+1}$, if $\vecb{m}\in\Z_q^m, \vecb{r} \in\Z_q^k$ are such that $\vecb{c}=\matr{G}\smallpmatrix{\vecb{m}\\ \vecb{r}}$, there exists a unique $\tilde{m}_i \in \Z_q$ such 
that $m_i=\tilde{m}_i$. 

(Perfectly hiding at coordinate $j$, $j \neq i$.) This follows immediately from the fact that $\vecb{g}_j$ is in the image 
of $\matr{G}_{2}$. 
\end{proof}




        \subsection{The Scheme} \label{sec:bits-scheme-nb}

            We construct a QA-NIZK argument of membership in the language
$$
\Lang_{ck,\sfbits} := \{[\vecb{c}]_1\in\GG_1^{k+1} : \exists \vecb{b}\in\bits^m,\vecb{r}\in\Z_q^k \text{ s.t. } [\vecb{c}]_1 = \MP.\Com_{ck}(\vecb{b};\vecb{r})\},
$$
where $ck:=[\matr{G}]_1$ and $\matr{G}$ is a matrix sampled from 
some distribution $\distink$ (as defined on Sect. \ref{sec:mddh}). For simplicity, in the exposition we restrict ourselves to the case $\dist_k=\distlin_{1}$ so  $\matr{G}$ is sampled from $\distlininone$, for some $0 \leq i \leq m$.

It is important to note that, as an extended MP commitment is at best only binding at one coordinate, a priori showing that it opens to $\vecb{b} \in \{0,1\}^m$ is not very meaningful, as it does open to other values as well. However, when combined with external protocols that univocally define $\vecb{b}$, it becomes a key building block to obtain the the results of chapter \ref{sec:shuf-rp}.

The argument is implicit in Sect. \ref{sec:bits-binding}, where we construct a QA-NIZK argument for proving that a perfectly binding commitment opens to a bit-string. More technically, to prove that a  perfectly binding commitment $[\vecb{c}']_1$ opens to a bit-string $\vecb{b}$, the argument in Sect. \ref{sec:bits-binding} takes the following steps:
\begin{enumerate}
\item Construct two MP commitments $[\vecb{c}]_1$, 
$[\vecb{d}]_2$ to $\vecb{b}$. 
\item Prove that $[\vecb{c}]_1$ and $[\vecb{c}']_1$ open to the same string. 
\item Prove that the two MP commitments $[\vecb{c}]_1$ and $[\vecb{d}]_2$ open to the same string.
\item Prove that $\vecb{c}(\vecb{d}-\sum_{j \in [m]}
\vecb{h}_j)^\top\in\Span(\{\vecb{g}_i\vecb{h}_j^\top:i,j\in[m+1]\}\setminus\{\vecb{g}_i\vecb{h}_i^\top:i\in[m]\})$, where $ck:=[(\vecb{g}_1,\ldots,\vecb{g}_{m+1})]_1$ and $ck':=[(\vecb{h}_1,\ldots,\vecb{h}_{m+1})]_2$.
\end{enumerate}
%The last step guarantees that 
%$b_i(b_i-1)=0$ for all $i \in [m]$. Indeed, 
%$\vecb{c}(\vecb{d}-\sum_{j \in [m]}
%\vecb{h}_j)^\top$ can be written as a linear combination of the vectors $\{\vecb{g}_i\vecb{h}_j^\top\}$ where the coefficient of $\vecb{g}_i\vecb{h}_i^{\top}$ is $b_i(b_i-1)$. Intuitively, an adversary will be able to prove that $\vecb{c}(\vecb{d}-\sum_{j \in [m]}
%\vecb{h}_j)^\top$ is in the span of the vectors  $\{\vecb{g}_i\vecb{h}_j^\top\}$ without those pairs where $i=j$ only if $b_i(b_i-1)=0$ for all $i \in [m]$. 


The argument we need for our results eliminates the perfectly binding commitment, which of course also means that step 2 disappears. Additionally, in the original scheme from Sect. \ref{sec:bits-binding}, the distribution of $ck=[\matr{G}]_1$ is uniform over $\GG_1^{2\times(n+1)}$, while in our argument of membership in 
$\Lang_{ck,\sfbits}$, $\matr{G}$ can follow any distribution $\distlininone$ for some $0 \leq i \leq m$. 
However, it is not hard to adapt the original proof to these distributions (in fact, in the soundness proof of 
Lemma \ref{lemma:bits1}, there is a game where the distribution of $\matr{G}$ is changed to $\distlininone$, for some $i \gets [m]$). The proof that $\Lang_{ck,\sfbits}$ admits a constant-size QA-NIZK argument essentially reuses parts of the proof of Theorem \ref{teo:bitstr-soundness}.  In summary, we prove the following result. 

\begin{theorem} \label{theo:bits}
The proof system described on Sect. \ref{sec:bits-scheme} is a QA-NIZK proof system with Perfect Completeness, Computational Soundness, and Perfect Zero-Knowledge.
\end{theorem}	
\begin{proof}
We remark that proof of Completeness and Zero-Knowledge is the same for any distribution $\distlininone$.
\begin{description}
\item[Perfect Completeness:]
Note that,
by definition of $\matr{C}_{i,j}$ and $\matr{D}_{i,j}$, 
$[\matr{C}_{i,j}]_1[\matr{I}]_2+[\matr{I}]_1[\matr{D}_{i,j}]_2=$
$[\vecb{g}_{i}]_1[\vecb{h}_j]_2^\top$.  Since $b_i(b_i-1) = 0$ for each $i\in[m]$,
\begin{eqnarray*}
\lefteqn{
[\vecb{c}]_1( [\vecb{d}]_2 - \sum_{i\in[m]} [\vecb{h}_{i}]_2)^\top}\\
& = & 
    \sum_{i \in [m]}\left(
        b_i s[\vecb{g}_{i}]_1[\vecb{h}_{m+1}]_2^{\top}+
        r(b_i-1)[\vecb{g}_{m+1}]_1[\vecb{h}_i]_2^{\top}+
        \sum_{j \in [m]} b_i (b_j-1)[\vecb{g}_{i}]_1[\vecb{h}_{j}]_2^{\top}
    \right)
\\ & & \mbox{ }
    + rs[\vecb{g}_{m+1}]_1[\vecb{h}_{m+1}]_2^{\top}\\
& = & 
    \left(\sum_{i\in[m]}b_is[\vecb{g}_{i}]_1\vecb{h}_{m+1}^{\top}+r(b_i-1)[\vecb{g}_{m+1}]_1\vecb{h}_i^{\top}+
        \sum_{\substack{j \in [m]\\j\neq i}}b_i(b_j-1)[\vecb{g}_{i}]_1\vecb{h}_{j}^{\top}\right)[\matr{I}]_2\\
& & \mbox{ }
    +rs[\vecb{g}_{m+1}]_1\vecb{h}_{m+1}^{\top}[\matr{I}]_2
    +[\matr{R}]_1[\matr{I}]_2 + [\matr{I}]_1[-\matr{R}]_2
\\ & = &
    [\matr{\Theta}]_1[\matr{I}]_2+
    [\matr{I}]_1[\matr{\Pi}]_2.
\end{eqnarray*}
Finally, the rest of the proof follows from completeness of $\Pi_\sfsum$ and $\Pi_\sfcom$. 

\item[Soundness:] When $\matr{G}$ is sampled from $\distlinizeroone$ it suffices to prove that the commitment $[\vecb{c}]_1$ output by the adversary is in $\Span([\matr{G}]_1)$ since, by the perfect hiding property, $[\vecb{c}]_1$ can be opened to any $\vecb{b}\in\bits^m$ thus $[\vecb{c}]_1\in\Lang_{ck,\sfbits}$. If $[\vecb{c}]_1\notin\Span([\matr{G}]_1)$, then we can break the (strong) soundness of the proof that $[\vecb{c}]_1$ and $[\vecb{d}]_2$ open to the same value, since that proof implies that there exist $\vecb{x},r,s$ such that $[\vecb{c}]_1=[\matr{G}]_1\smallpmatrix{\vecb{x}\\r}$ and $[\vecb{d}]_2=[\matr{H}]_2\smallpmatrix{\vecb{x}\\s}$. Therefore, we construct an adversary $\advB$ against the strong soundness of $\Pi_\sfcom$ that simulates $\advA$ until it halts and outputs $([\vecb{c}]_1,[\vecb{d}]_2,\pi_\sfcom)$. Note that, in order to simulate the CRS $\advB$ requires $\matr{H}$, but this is not a problem since is part of the input in the strong soundness game.
 
When $\matr{G}$ is sampled from $\distlinisnone$, ${i^*}>0$, the proof follows from the indistinguishability of the following three games:
\begin{itemize}
\item[$\mathsf{Real}$:] This is the real Soundness game. The output is 1 if the adversary submits some $[\vecb{c}]_1\notin\Lang_{ck,\sfbits}$ and the corresponding proof which is accepted by the verifier.
\item[$\sfGame_0$:] This identical to $\mathsf{Real}$, except that $\algK_1$ does not receive $[\matr{G}]_1$ as a input but
it samples $\matr{G}$ itself according to $\distlinisnone$.
\item[$\sfGame_1$:] This game is identical to $\sfGame_0$ except that now $\matr{H}\gets\distlinisnone$.
\end{itemize}

It is obvious that the first two games are indistinguishable. The rest of the argument goes as follows.

\begin{lemma} There exists a\ $\distlin_1$-$\mddh_{\GG_2}$ adversary $\advD$ such that
$|\Pr\left[\mathsf{Game}_{0}(\advA)=1\right]$ $-\Pr\left[\mathsf{Game}_{1}(\advA)=1\right]|$ $\leq \mathsf{Adv}_{\distlin_1,\ggen_a}(\advD).$
\end{lemma}
\begin{proof}
We construct an adversary $\advD$ that receives 
a challenge $([\matr{A}]_2,[\vecb{u}]_2)$ of the 
$\distlin_1$-$\mddh_{\GG_2}$ Assumption. From this challenge, $\advD$ just defines the matrix  $[\matr{H}]_2\in\GG_2^{2\times(m+1)}$ as the matrix whose last column consists of $[\matr{A}]_2$, the ith column consists of $[\vecb{u}]_2$ and the rest of the columns are random vectors in the image of $[\matrA]_2$. 
Obviously, when $[\vecb{u}]_2$ is sampled from 
the image of $[\matr{A}]_2,$ $\matr{H}$ follows the distribution $\distlinizeroone$, while if $[\vecb{u}]_2$ is a uniform element of $\GG^2$, $\matr{H}$ follows the distribution $\distlinisnone$. 
 
Adversary $\advD$ samples
$\matr{G} \gets \distlinisnone$. Given that $\advD$ does not know the discrete logarithms of $[\matr{H}]_2$, it cannot compute the pairs $(\matr{C}_{i,j},\matr{D}_{i,j})$ exactly as in $\sfGame_0$. Nevertheless, for each $(i,j)\in\indexSet{m}{1}$ it can compute identically distributed pairs by picking $\matr{T}\gets\Z_q^{2\times 2}$ and defining
$$
([\matr{C}_{i,j}]_1,[\matr{D}_{i,j}]_2):=([\matr{T}]_1,\vecb{g}_i[\vecb{h}_j]_2^\top-[\matr{T}]_2).
$$
The rest of the elements of the CRS, namely $\crs_\sfcom$ and $\crs_\sfsum$, are honestly computed. When $\matr{H}\gets\distlinizeroone$, $\advD$ perfectly simulates $\sfGame_0$, and when $\matr{H}\gets\distlinisnone$, $\advD$ perfectly simulates $\sfGame_1$, which concludes the proof. 
\end{proof}

\begin{lemma}
There exist adversaries $\advB_1$, against the strong soundness of $\Pi_\sfcom$, and $\advB_2$, against the soundness of $\Pi_\sfsum$, such that $\Pr[\sfGame_1(\advA)=1]\leq 4/q+ \adv_{\Pi_\sfcom}(\advB_1)+\adv_{\Pi_\sfsum}(\advB_2)$.
\end{lemma}
\begin{proof}
With probability $1-4/q$, $\{\vecb{g}_{i^*},\vecb{g}_{m+1}\}$ and $\{\vecb{h}_{i^*},\vecb{h}_{m+1}\}$ are both bases of $\Z_q^2$,
we can define $b_{i^*},\overline{w}_g,\overline{w}_h,\overline{b}_{i^*}$ as the unique coefficients in $\Z_q$ such that $\vecb{c}=b_{i^*}\vecb{g}_{i^*} + \overline{w}_g \vecb{g}_{m+1}$ and $\vecb{d}= \bb_{i^*} \vecb{h}_{i^*} + \overline{w}_h \vecb{h}_{m+1}$.

In particular, if $\advA$ breaks soundness, this implies that $b_{i^*} \notin \{0,1\}$ (since for $i\neq i^*$, 
$\vecb{c}$ can always be opened to 
  choose $b_i=0$). Further, the verifier accepts the proof proof:
$ (
        [\vecb{d}]_2,
        ([\matr{\Theta}]_1, [\matr{\Pi}]_2), 
        \pi_\sfcom,\pi_\sfsum )$
  produced by $\advA$.
We distinguish two cases:
\begin{description}
\item[$b_{i^*} \neq \overline{b}_{i^*}$:] Given that $[\vecb{c}]_1$ and $[\vecb{d}]_2$ are perfectly binding at coordinate $i^*$, if $b_{i^*}\neq\bb_{i^*}$ it is not possible that $[\vecb{c}]_1$ and $[\vecb{d}]_2$ open to the same value. We construct an adversary $\advB_1$ against the strong soundness 
of $\Pi_\sfcom$ that simulates game $\sfGame_1$ with $\advA$ (using $\matr{H}$ to simulate the CRS) until it halts and outputs $([\vecb{c}]_1,[\vecb{d}]_2,\pi_\sfcom)$. If $b_{i^*}\neq\bb_{i^*}$, $\pi_\sfcom$ is a fake proof for $([\vecb{c}]_1,[\vecb{d}]_2)$ opening to the same value and then $\advB_1$ breaks the strong soundness of $\Pi_\sfcom$.
\item[$b_{i^*} = \overline{b}_{i^*}$, 
$b_{i^*}(\overline{b}_{i^*} -1) \neq 0$:]
If we express $\matr{\Theta}+\matr{\Pi}$
as a linear combination of $\{\vecb{g}_{i}\vecb{h}_{j}^{\top}:i,j\in[n+1]\}$, the coordinate of
$\vecb{g}_{i^*}\vecb{h}_{i^*}^\top$ is $b_{i^*}(\bb_{i^*}-1)\neq 0$ and thus $\matr{\Theta}+\matr{\Pi}\notin\Span(\{\matr{C}_{i,j}+\matr{D}_{i,j}:(i,j)\in\indexSet{m}{1}\})$. We construct an adversary $\advB_2$ against the soundness of $\Pi_\sfsum$ that simulates game $\sfGame_1$ with $\advA$ until it halts and outputs $([\matr{\Theta}]_1,[\matr{\Pi}]_2,\pi_\sfsum)$. If $b_{i^*} = \overline{b}_{i^*}$ but $b_{i^*}(\overline{b}_{i^*} -1) \neq 0$, $\pi_\sfsum$ is a fake proof for $([\matr{\Theta}]_1,[\matr{\Pi}]_2)$ and then $\advB_2$ breaks the soundness of $\Pi_\sfsum$.
\end{description}
\end{proof}

\item[Perfect Zero-Knowledge:] First, note that the vector $[\vecb{d}]_2 \in \GG_2^2$ output by the prover and the vector output by $\algS_2$ follow exactly the same distribution. This is because $\matr{H}\gets\distlinizeroone$ defines perfectly hiding commitments. In particular, although the simulator $\algS_2$ does not know $\vecb{b} \in \{0,1\}^{m}$ such that $[\vecb{c}]_1=[\matr{G}]_1\smallpmatrix{\vecb{b}\\r}$, for some $r\in\Z_q$, 
there exists $s \in \Z_q$ such that $[\vecb{d}]_2=[\matr{H}]_2\smallpmatrix{\vecb{b}\\ s}$. 

Since $\matr{R}$ is chosen uniformly at random in $\Z_q^{2 \times 2}$, the proof $([\matr{\Theta}]_1, [\matr{\Pi}]_2)$ is uniformly distributed conditioned on satisfying check 1) of algorithm $\algV$.
 Finally, the rest of the proof follows from Zero-Knowledge of $\Pi_\sfsum$ and $\Pi_\sfcom$.
\end{description}
\end{proof}


        \subsection{Constant-Size Argument for $\Lang_{ck,\sfbits}^n$} \label{sec:matr-bits}

            We give a QA-NIZK argument of membership in the language $\Lang_{ck,\sfbits}^n  = \Lang_{ck,\sfbits} \times \ldots \times \Lang_{ck,\sfbits}$ with a proof size which is independent of $n$ (but with a loss factor in the proof of soundness of $n$). The result will be crucial to get improved proof sizes for more complex statements. 

\subsection{Intuition}
We would like to prove that some tuple
$([\vecb{c}_1]_1,\ldots, [\vecb{c}_n]_1) \in \Lang_{ck,\sfbits}^n$, where $[\vecb{c}_j]_1=\MP.\Com_{ck}(\vecb{b}_j;\vecb{r}_j)$, $j \in [n]$, $ck:=[\matr{G}]_1$, and $\matr{G} \gets \distink$, for some $0 \leq i \leq n$. 
Denote $\vecb{c}=\vecb{c}_1 \oplus \ldots \oplus \vecb{c}_n$, $\vecb{b}=\vecb{b}_1 \oplus \ldots \oplus \vecb{b}_n$ and $\vecb{r}=\vecb{r}_1 \oplus \ldots \oplus \vecb{r}_n$ (concatenation of column vectors as defined in Section~\ref{secc:prelim}). The proof system works as follows:
\begin{enumerate}
\item It defines  
 $\overline{ck}:=[\overline{\matr{G}}]_1\gets\MP.\algK(1^\lambda,mn)$ for computing Multi-Pedersen commitments to vectors of size $mn$, where $\overline{\matr{G}} \leftarrow 
\distzeronmk$, and  $[\ovecb{c}]_1\gets\MP.\Com_{\overline{ck}}(\vecb{b};\vecb{s})$ for some 
randomness $\vecb{s}$, 
\item it proves that $[\ovecb{c}]_1\in\Lang_{\overline{ck},\sfbits}$ with the proof system $\Pi_\sfbits$,
\item it  proves that there exists an equal opening of $[\vecb{c}]_1$ and $[\ovecb{c}]_1$ with  $\Pi_\sfcom$.
\end{enumerate}
First, note that in step 3, we can use the proof system $\Pi_\sfcom$, as both $[\vecb{c}]_1$ and $[\ovecb{c}]_1$ are commitments of the required form, as if $\matr{G}_2$ denotes the last $k$ columns 
of $\matr{G}$ and $\matr{G}_1$ the rest,
$\vecb{c}=\matr{G}_1^n\vecb{b}+\matr{G}_2^n\vecb{r}.
$
%$$[\vecb{c}]_1=\left(\begin{smallmatrix}
%\matr{G}_1 & \ldots & \vecb{0}  \\
%\vdots     & \ddots & \vdots  \\
%\vecb{0}   & \ldots & \matr{G}_1
%\end{smallmatrix}\right) \vecb{b}+ \left(\begin{smallmatrix}
% \matr{G}_{2} & \ldots & \vecb{0}\\
% \vdots         & \ddots & \vdots\\
% \vecb{0}       & \ldots & \matr{G}_{2}
%\end{smallmatrix}\right) \vecb{r}.$$

We give some intuition on why is the above scheme sound. For the case $\matr{G} \gets \distzeronk$ it suffices to note that the proof that  $[\vecb{c}]_1$ and $[\ovecb{c}]_1$ share an opening implies in particular that they are both valid commitments. But if 
$[\ovecb{c}]_1$ is a valid commitment then $([\vecb{c}_1]_1,\ldots, [\vecb{c}_n]_1) \in \Lang_{ck,\sfbits}^n$ because for this distribution the commitments are perfecly hiding. 

For the case where $\matr{G} \gets \distink$, $i>0$, 
recall that the MP commitment with this key is 
perfectly binding at coordinate $i$. In particular, this implies that if some $[\vecb{c}_j]_1 \notin \Lang_{ck,\sfbits}$, the $ith$ coordinate of $\vecb{b}_j$, denoted $b_{i,j}$, satisfies that $b_{i,j} \notin \{0,1\}$. Therefore, given some $j^* \gets [n]$, if the adversary breaks soundness, then, with probability at least $1/n$, $b_{i,j^*} \notin \{0,1\}$. In the soundness proof, we switch to a game where the distribution of $\overline{\matr{G}}$ is changed so that now 
$\MP.\Com_{\overline{ck}}$ is perfectly binding for $b_{i,j^*}$. Now it is easy to prove that if $b_{i,j^*} \notin \{0,1\}$, the soundness of $\Pi_\sfbits$ or of $\Pi_\sfcom$ is broken, because this is incompatible with $[\vecb{c}]_1$ and $[\ovecb{c}]_1$ sharing an opening and $[\ovecb{c}]_1 \in\Lang_{\overline{ck},\sfbits}$.



\subsection{The scheme}
%   If $\matr{G}\gets\dist_{2,m+1}^0$ the proof that $[\vecb{c}]_1$ and $[\ovecb{c}]_1$ open to the same value implies that $\ovecb{c}_j\in\Span(\matr{G})$ for all $j\in[n]$ and, by the perfect hiding property, $([\vecb{c}_1]_1,\ldots,[\vecb{c}_n]_1)$ can be opened to any $\matr{B}\in\bits^{m\times n}$ thus $([\vecb{c}_1]_1,\ldots,[\vecb{c}_n]_1)\in\Lang_{ck,\sfbits}^n$. If $\matr{G}\gets\dist_{2,m+1}^{i^*}$ and $([\vecb{c}_1]_1,\ldots,[\vecb{c}_n]_1)\notin\Lang_{ck,\sfbits}^n$, then the $i^*$ th row of $\matr{B}$ is not in $\bits^{1\times n}$ which implies that there is some $j\in[n]$ such that $[\vecb{c}_j]_1\notin\Lang_{ck,\sfbits}$. If we pick $j^*\gets[n]$ and $\overline{\matr{G}}\gets\dist_{2,mn+1}^{m(i^*-1)+j^*}$, soundness of the proof that $[\ovecb{c}]_1\in\Lang_{\overline{ck},\sfbits}$ is violated with probability at least $1/n$.
\begin{figure}
\begin{\algSize}
$$
\begin{array}{ll}
\begin{array}{l}
\algK_1(\gk,[\matr{G}]_1,n)\quad (\mathsf{S}_1(\gk,[\matr{G}]_1,n))\\
\hline
{[\overline{\matr{G}}]_1 \gets \MP.\algK(1^\lambda,mn,\distlin_1^{mn,0})}\\
\crs_\sfcom\gets\Pi_\sfcom.\algK_1(\gk,[\vecb{G}^n_1||\vecb{g}_{n+1}^n]_1,[\ovG]_1, mn)\\
\crs_\sfbits\gets\Pi_\sfbits.\algK_1(\gk,[\overline{\matr{G}}]_1)\\
\text{Return } \ \mathsf{crs}:=(\crs_\sfcom,\crs_\sfbits).\\
(\tau_\sfcom\gets\Pi_\sfcom.\mathsf{S}_1(\gk,[\matr{G}^n]_1,[\ovG]_1,mn)\\
\tau_\sfbits\gets\Pi_\sfbits.\mathsf{S}_1(\gk,[\overline{\matr{G}}]_1).\\
\tau := (\vecb{a},\tau_\sflin,\tau_\sfbits)).\\
\\
\end{array}
&
\begin{array}{l}
{\algP(\mathsf{crs}, ([\vecb{c}]_1,\ldots,[\vecb{c}_n]_1), \langle (\vecb{b}_1,\ldots,\vecb{b}_n), \vecb{w}\rangle)}\\
\hline
{[\ovecb{c}]_1 :=\MP.\Com_{[\ovG]_1}(\vecb{b};\overline{w})},\overline{w}\gets\Z_q\\
{\pi_\sfcom \gets \Pi_{\sfcom}.\algP(\crs_\sfcom,[\vecb{c}]_1,[\ovecb{c}]_1,}{\langle\vecb{b},\vecb{w},\overline{w}\rangle)}\\
\pi_\sfbits \gets \Pi_\sfbits.\algP(\crs_\sfbits,[\ovc]_1,\langle \vecb{b},\overline{w}\rangle)\\
\text{Return } \  ([\ovc]_1,\pi_\sfcom,\pi_\sfbits). \\
\\
\\
\\
\\
\end{array}\\
\begin{array}{l}
{\algV(\mathsf{crs},([\vecb{c}_1]_1,\ldots,[\vecb{c}_n]_1),([\ovc]_1,\pi_\sfcom,\pi_\sfbits))}\\
\hline
\mathsf{ans}_1 \gets \Pi_\sfcom.\algV(\crs_\sfcom,[\vecb{c}]_1,[\ovc]_1,\pi_\sfcom)\\
\mathsf{ans}_2 \gets \Pi_\sfbits.\algV(\crs_\sfbits,[\ovc]_1,\pi_\sfbits)\\
\text{Return } \ \mathsf{ans}_1\wedge\mathsf{ans}_2.
\\
\\
\end{array}
&
\begin{array}{l}
{\mathsf{S}_2(\crs,([\vecb{c}_1]_1,\ldots,[\vecb{c}_n]_1),[\matr{D}]_1,\tau)}\\
\hline
{[\ovc]_1 \gets \MP.\Com_{[\ovG]_1}(\vecb{0}_{mn\times 1})}\\
\pi_\sfcom\gets \Pi_\sfcom.\algS_2(\crs_\sfcom,[\vecb{c}]_1,[\ovc]_1,\tau_\sfcom)\\
\pi_\sfbits \gets \Pi_\sfbits.\algS_2(\crs_\sfbits,[\ovc]_1,\tau_\sfbits)\\
\text{Return }  ([\ovc]_1,\pi_\sfcom, \pi_\sfbits).
\end{array}
\end{array}$$
\end{\algSize}
\caption{The proof system for the language $\Lang_{[\matr{G}]_1,\sfbits}^n$. $\Pi_\sfbits$ is the proof system from Sect. \ref{sec:bits-scheme}. The matrix $\matr{G}$ is parsed as $\vecb{G}=(\vecb{G}_1||\vecb{g}_{n+1})$, and $\vecb{c}:=\vecb{c}_1\oplus\ldots\oplus\vecb{c}_n$ and $\vecb{b}:=\vecb{b}_1\oplus\ldots\oplus\vecb{b}$.
\label{fig:bitsn}
}
\end{figure}
%\vspace*{-1cm}

The description of the protocol is in Fig. \ref{fig:bitsn} and  we prove that:

\begin{theorem} \label{theo:bitsnm} The proof system from Fig. \ref{fig:bitsn} is a QA-NIZK proof system for the language $\Lang_{ck,\sfbits}^n$ with proof size  
$10|\GG_1|+10|\GG_2|$, perfect completeness, perfect-zero knowledge, and computational soundness. 
\end{theorem}

\begin{proof}
(Completeness.)
Follows from the fact that $([\vecb{c}]_1,[\ovc]_1)\in\Lang_{\sfcom,[\matr{G}^n]_1,[\ovG]_1,mn}$ and that $[\ovc]_1\in\Lang_{[\ovG]_1,\sfbits}$.

(Soundness.)
When $\matr{G}\gets\distlinizeroone$ the proof follows from the proof that $[\vecb{c}]_1$ and $[\ovecb{c}]_1$ open to the same value.
When $\matr{G}\gets\distlinisnone$, the proof follows from the indistinguishability of the following games.

\begin{description}
\item[$\mathsf{Real}$:] This is the real soundness game. The adversary wins if it outputs $([\vecb{c}_1]_1,\ldots,[\vecb{c}_n]_1)\notin\Lang_{ck,\sfbits}^n$ and the corresponding proof which is accepted by the verifier.
\item[$\sfGame_0$:] This game is exactly as $\mathsf{Real}$ except that $\algK_1$ does not receive $[\matr{G}]_1$ as a input but it samples $\matr{G}$ itself according to $\distlinisnone$.
\item[$\sfGame_1$:] This game is exactly as $\sfGame_0$ except that the simulator picks a random $j^*\in[n]$ and uses $\matr{G}$ to check whether $\vecb{c}_{j^*}=b_{i^*,j^*}\vecb{g}_{i^*}+\tilde{w}\vecb{g}_{n+1}$ such that $b_{i^*,j^*}\notin\bits$. It aborts if this is not the case.
\item[$\sfGame_2$:] This game is exactly as $\sfGame_1$ except that $\overline{\matr{G}}\gets\distlin_1^{mn, m(i^*-1)+j^*}$.
\end{description}

It is obvious that the first two games are indistinguishable. 
The rest of the argument goes as follows. 

\begin{lemma} $\Pr\left[ \mathsf{Game}_1(\advA)=1\right]\geq\dfrac{1}{n}\Pr\left[\mathsf{Game}_0(\advA)=1\right].$
\end{lemma}

\begin{proof}  The probability that
 $\mathsf{Game}_1(\advA)=1$ is the probability that  a) $\mathsf{Game}_0(\advA)=1$ and
b)  $b_{i^*,j^*} \notin \bits$. The view of adversary $\advA$ is independent of $j^*$, while, if $\mathsf{Game_0}(\advA)=1$, then there is at least one index $\ell \in [n]$ such that $[\vecb{c}_\ell]_1\notin\Lang_{[\matr{G}]_1,\sfbits}\Longrightarrow b_{i^*,\ell} \notin \bits$. Thus, 
the probability that the event described in b) occurs conditioned on $\mathsf{Game_0}(\advA)=1$, is greater than or equal to $1/n$ and the lemma follows.
\end{proof}

\begin{lemma} There exists a\ $\dist_1$-$\mddh_{\GG_1}$ adversary $\advD$ such that
$|\Pr\left[\mathsf{Game}_{1}(\advA)=1\right]$ $-\Pr\left[\mathsf{Game}_{2}(\advA)=1\right]|$ $\leq
    \mathsf{Adv}_{\distlin_1,\ggen_a}(\advD).$
\end{lemma}
\begin{proof}
We construct an adversary $\advD$ that receives 
a challenge $([\matr{A}]_1,[\vecb{u}]_1)$ of the 
$\distlin_1$-$\mddh_{\GG_1}$ Assumption. From this challenge, $\advD$ just defines the matrix  $[\overline{\matr{G}}]_1\in\GG_1^{2\times(mn+1)}$ as the matrix whose last column consists of $[\matr{A}]_1$, the ith column consists of $[\vecb{u}]_1$ and the rest of the columns are random vectors in the image of $[\matrA]_1$. Then $\advD$ honestly simulates the rest of the CRS, gives it as input to $\advA$, and outputs whatever $\advA$ outputs.

Obviously, when $[\vecb{u}]_1$ is sampled from 
the image of $[\matr{A}]_1,$ $\overline{\matr{G}}$ follows the distribution $\distlinizeroone$ and $\advD$ perfectly simulates $\sfGame_1$, while if $[\vecb{u}]_1$ is a uniform element of $\GG^2_1$, $\overline{\matr{G}}$ follows the distribution $\distlinisnone$ and $\advD$ perfectly simulates $\sfGame_2$. 
%It is immediate to see that adversary $\advD$ perfectly simulates $\sfGame_1$ when $\overline{\matr{G}}\gets\dist_{2,mn+1}^0$ and $\sfGame_2$ when $\overline{\matr{G}}\gets\dist_{2,mn+1}^{m(i^*-1)+j^*}$. The rest of the proof follows from Lemma \ref{lemma:dist-i}.  
\end{proof}

\begin{lemma}
There exists adversaries $\advB_1,\advB_2$ such that $\Pr[\sfGame_2(\advA)=1]\leq\adv_{\Pi_\sfcom}(\advB_1)+\adv_{\Pi_\sfbits}(\advB_2)$.
\end{lemma}

\begin{proof}
If $\sfGame(\advA)=1$, then $b_{i^*,j^*}\notin\bits$ while all the verification equations are accepted. Given that $\ovecb{g}_{m(i^*-1)+j^*}$ is linearly independent from $\{\ovecb{g}_j:j\neq m(i^*-1)+j^*\}$, it holds that $\{\ovecb{g}_{m(i^*-1)+j^*},\ovecb{g}_{mn+1}\}$ is a basis for $\Z_q^2$ and thus we can define $\bb_{i^*,j^*},\overline{w}_{h,i}$ as the unique coefficients in $\Z_q$ such that $\ovecb{c} = \bb_{i^*,j^*}\ovecb{g}_{m(i^*-1)+j^*}+\overline{w}_{h,i}\ovecb{g}_{mn+1}$. If $b_{i^*,j^*}\neq\bb_{i^*,j^*}$, then $([\vecb{c}]_1,[\ovecb{c}]_1)$ can not open to the same value and we can construct an adversary $\advB_1$ against $\Pi_\sfcom$. Else, it must be the case that $\bb_{i^*,j^*}=b_{i^*,j^*}\notin\bits$. Therefore, if an adversary $\advB_2$ simulates $\sfGame_2$ until $\advA$ halts and outputs $([\ovecb{c}]_1,\pi_\sfbits)$, then $\advB_2$ breaks soundness of $\Pi_\sfbits$. 
\end{proof}

(Zero-Knowledge.) Given that $\overline{\matr{G}}$ defines perfectly hiding commitments, $[\ovecb{c}]_1$ can be opened to any value. Therefore $[\ovecb{c}]_1$ and $[\vecb{c}]_1$ share a common opening and $[\ovecb{c}]_1\in\Lang_{\overline{ck},\sfbits}$, and thus $\pi_\sfcom$ and $\pi_\sfbits$ are correctly distributed.  
\end{proof}





\chapter{New Techniques for Non-Interactive Shuffle and Range Arguments} \label{sec:shuf-rp}

    This chapter focuses on obtaining efficiency improvements for non-interactive arguments: \emph{Range proofs} and \emph{Proof of Correctness of a Shuffle}, based only on falsifiable assumptions.  To derive efficiency improvements these arguments we develop a new cryptographic primitive which we call \emph{Aggregated Zero-Knowledge Set-Membership Proof} (aZKSMP). An aZKSMP allows to show that the openings of many commitments belong to a public set, and we say that the proof is \emph{aggregated} because its size does not depend on the number of commitments.

Our resulting proofs are more efficient in terms of proof size and are based on more standard assumptions, but they have a rather large common reference string. They build on the recent arguments for membership in linear spaces of \cite{EC:LPJY14,C:JutRoy14,EC:KilWee15} and the argument for proving that some commitment to a vector of integers in $\Z_q^{n}$ opens to $\{0,1\}^n$ due to \cite{AC:GonHevRaf15}. 
 



    \section{Related Work} \label{sec:set-memb-rel-work}

        \subsubsection{Zero Knowledge Set Membership Arguments.}
Camenisch et al. constructed $\Theta(1)$ interactive zero-knowledge set membership arguments using Boneh-Boyen Signatures, and they prove them secure under the $q$-SDH assumption \cite{AC:CamChaShe08}. Bayer and Groth constructed $\Theta(\log |S|)$ interactive zero-knowledge arguments for polynomial evaluation, which can be used to construct set membership arguments, relying only on the discrete logarithm assumption \cite{EC:BayGro13}.
However, none of the previous constructions has addressed the problem of aggregating many proofs, and a direct use of them will end up with a proof of size $\Omega(n)$.

\subsubsection{NIZK Shuffle and Range Arguments.}
The most efficient NIZK shuffle argument under falsifiable assumptions is the one from Groth and Lu \cite{AC:GroLu07}, which works for BBS ciphertexts. The proof size is linear in the number of ciphertexts, specifically $15n + 120$ group elements in type I groups. The security of their construction relies on two assumptions: the \emph{pairing product assumption} and the \emph{permutation pairing assumption}. The first assumption is a $\dist_{n,2}\mbox{-}\kermdh$ assumption, when $\matr{M}\gets\dist_{n,2}$ is of the form $\matr{M}^\top:=\pmatri{x_1,\ldots,x_n\\x_1^2,\ldots,x_n^2}$ for $x_i\gets\Z_q$, $i\in[n]$. The second assumption is proven generically secure by Groth and Lu, but it seems to be unrelated with any other assumption.

Using non-falsifiable assumptions (i.e. knowledge of exponent type of assumptions), Lipmaa and Zhang \cite{SCN:LipZha12} constructed a shuffle argument with communication $6n\sG+11\sH$, and recently Fauzi and Lipmaa  \cite{EPRINT:FauLip15} constructed a shuffle argument with communication $(5n+2)\sG+2n\sH$.

Rial, Kohlweiss, and Preneel constructed a range argument in $[0,2^n-1]$ with communication $\Theta(\frac{n}{\log n -\log\log n})$ and prove it secure under the $q$-HSDH assumption \cite{PAIRING:RiaKohPre09}. One might get rid of the $q$-HSDH assumption replacing the \emph{P-signature} with any \emph{structure preserving signature}, but, since the proof requires $\frac{n}{\log n-\log \log n}$ Groth-Sahai proofs of satisfiability of the signature's verification equation and the signature's size is at least 6 group elements \cite{EPRINT:JutRoy17}, the resulting protocol is far less efficient.
Using non-falsifiable assumptions, Chaabouni, Lipmaa, and Zhang constructed a range argument with constant communication~\cite{FC:ChaLipZha12}. 

A detailed comparison of our shuffle and range arguments with the most efficient constructions under falsifiable assumptions is depicted in Table \ref{table:eff}.


\begin{table}[h]
\begin{center}
\begin{minipage}{\textwidth}
\begin{center}
\begin{scriptsize}
\begin{tabular}{|l|ll|ll|}
\hline
                                                   & \multicolumn{2}{c|}{Shuffle Argument} & \multicolumn{2}{c|}{Range Argument} \\
                                                   & \cite{AC:GroLu07}          
%& \cite{EPRINT:FauLip15}
 & $\Pi_\mathsf{shuffle}$
                                                   & \cite{PAIRING:RiaKohPre09} & $\Pi_{\mathsf{range}\mbox{-}\mathsf{proof}}$ 
\\ \hline\hline
\rule{0pt}{2.5ex}CRS size                          & $2n + 8$                   
%& $8n + 17$
              & $(n^2+24n+36,23n+37)$                
                                                   & $\Theta(\frac{n}{\log n-\log\log n})$ & $(6n^2,6n^2)$ \\
% (6n^2+13n+n+\frac{n}{klogn}+2klogn,6n^2+13n+n+\frac{n}{klogn}+34)
\rule{0pt}{2.5ex}Proof size                        & $15n + 120$                
%& $(5n+2,2n)$
            & $(4n+17,14)$
                                                   & $\Theta(\frac{n}{\log n-\log\log n})$ & $(\frac{2n}{k\log n},10)$ \\
%$(\frac{2n}{k\log n}+2k\log n+11,10)$
\rule{0pt}{2.5ex}$\algP$'s comp.                   & $51n + 246$               
%& $22n + 11$
             & $11n+17$
                                                   & $\Theta(\frac{n}{\log n-\log\log n})$ & $2n$ \\
%2n+\frac{3n}{klogn}+3k\log n+2
\rule{0pt}{2.5ex}$\algV$'s comp.                   & $75n + 282$               
%& $18n + 6$
              & $13n+55$
                                                   & $\Theta(\frac{n}{\log n-\log\log n})$ & $\frac{4n}{k\log n}$ \\
%\frac{n}{k\log n}+6k\log n+62
\rule{0pt}{2.5ex}Assumption                        & PP                        
%& KE
                    & SXDH+SSDP
                                                   & $q$-HSDH                   & SXDH+SSDP \\\hline 
\end{tabular}
\end{scriptsize}
\end{center}
\caption{Comparison of our shuffle, $\Pi_\mathsf{shuffle}$, and range, $\Pi_{\mathsf{range}\mbox{-}\mathsf{proof}}$, arguments with the literature. To increase readability, for $\Pi_{\mathsf{range}\mbox{-}\mathsf{proof}}$ we include only the leading part of the sizes, that is, we write $f(n)$ and we mean $f(n)+o(f(n))$. Notation $(x,y)$ means $x$ elements of $\GG_1$ and $y$ elements of $\GG_2$. ``PP'' stands for the permutation pairing assumption.
%and ``KE'' for knowledge of exponent assumption ({\color{red} ver cual}).
The prover's computation is measured by the number of exponentiations (i.e. $z[x]_i$) and the verifier's computation is measured by the number of pairings.\label{table:eff}  } 
\end{minipage}
\vspace{-0.54cm}

\end{center}
\end{table}




    \section{Overview} \label{sec:shuf-rp-overview}

        Our starting point is the observation that range and shuffle proofs can be constructed using an aZKSMP as a common building block by slightly modifying some previous strategies used for shuffle and range proofs. Before moving to shuffles and range proofs, we need to define in more detail what an aZKSMP is.

Given some publicly known set $S$, an aZKSMP allows to prove that $n$ commitments $c_1,\ldots,c_n$ open to values $x_1,\ldots,x_n \in S$.  The set $S$ is of polynomial size and is either $[0,d-1]\subset\Z_q$ or a subset of $\GG_\gamma$, $\gamma \in \{1,2\}$.  
In other words, an aggregated set membership argument proves that each $c_1,\ldots,c_n$ is in the language
$$
\Lang_{ck,S}:=\{c: \exists x\in S, \vecb{w}\in\Z_q^r \text{ s.t. } c=\Com_{ck}(x;\vecb{w})\}\text{, where }ck\gets\distk,
$$
and $c=\Com_{ck}(x;\vecb{w})$ is a Groth-Sahai commitment to $x$ with randomness $\vecb{w}$.

The proof that we construct in Section~\ref{sec:aZKSMP} is quasi-adaptive, in the sense that the language and the common reference string depends on $ck$ and $S$. Further, the marginal distribution of $ck$ is witness samplable, that is, it can be sampled along with its discrete logarithms. The argument is \textit{aggregated} because the size of the proof is independent of $n$ ($\Theta(\log d)$ when $S=[0,d-1]$ and $\Theta(|S|)$ when $S\subset\GG_\gamma$). However, in the soundness proof we will loose a factor of $n$ in the reduction. 

%Before discussing how to construct such an argument, we come back on showing how to use it as a building block for range and shuffle proofs.  
\subsection{Range Argument}
A range argument is a tool often required in e-voting and e-cash scenarios, with the purpose of showing that the opening $y$ of some commitment $c$ is an integer in some interval $[A,B]$. For simplicity, the range considered is usually $[0,2^n-1]$ since a proof in any interval can be reduced to a proof in this interval.

Let $n,d\in\mathbb{N}$, $m:=\log d$, and $\ell:=n/m$. A commitment $c$ opens to a integer $x$ in the range $[0,2^n-1]$ if $\exists x_1,\ldots,x_\ell \in[0,d-1]$ and  $x=\sum_{i\in[\ell]}x_id^{i-1}$. Indeed, since $x_i\in[0,d-1]$
\begin{eqnarray*}
x & = & \sum_{i\in[\ell]} x_i d^{i-1}
   \in  [0,d^\ell-1]  =  [0,(d^{1/\log d})^n-1] = [0,2^n-1].
\end{eqnarray*}
The statement $\exists x_1,\ldots,x_\ell \in[0,d-1]$ can be proven by showing that $(c_1,\ldots,c_n)\in\Lang_{ck,[0,d-1]}^\ell$, where $c_i=\Com_{ck}(x_i)$, with an aggregated set membership proof, and the statement $x=\sum_{i\in[\ell]}d^{i-1}x_i$ can be proven using standard techniques. 

While this way of constructing range arguments has been widely used in the literature \cite{AC:CamChaShe08,PAIRING:RiaKohPre09}, with the addition of our techniques we get a smaller proof size. Indeed, the total cost of the range proof is $\Theta(\ell)+\Theta(m)$ ($\ell$ is due to the size of the commitments $c_1,\ldots,c_\ell$ and $m$ to the size of an aggregated proof of membership in $\Lang_{ck,[0,d-1]}^\ell$).  Setting $d=n^{k}$ for arbitrary $k$ leads to a proof size of $\Theta(\frac{n}{k \log n})$. Compared to previous approaches, the novelty of ours is that the cost of proving that $x_1,\ldots,x_\ell\in[0,d-1]$ is significantly reduced.

\subsection{Shuffle Argument}
An argument of correctness of a shuffle is an essential tool in the construction of \emph{mix-nets} \cite{CACM:Chaum81}. A mix-net, in turn, is a distributed protocol between many \emph{mixers}, where each mixer receives as input a set of $n$ ciphertexts and  outputs a \emph{shuffle} of the input ciphertexts. That is, a \emph{re-randomization} of the set of ciphertexts obtained after applying a \emph{random permutation} to the input set of ciphertexts. To enforce the honest behavior of mixers they are required to produce a zero-knowledge argument that the shuffle was correctly computed.  

Our proof is partially inspired by the non-interactive shuffle of Groth and Lu \cite{AC:GroLu07}. The statement we want to prove in a correctness of a shuffle argument is : ``Given two vectors of ciphertexts which open, respectively, to vectors of plaintexts $[\vecb{m}_1]_2, [\vecb{m}_2]_2$, prove that 
 $[\vecb{m}_2]_2$ is a permutation of $[\vecb{m}_1]_2$''.  Roughly, our strategy is the following:  
\begin{itemize}
\item[1)] Publish some vector of group elements $[\vecb{s}]_1 =([s_1]_1,\ldots,[s_n]_1)^\top$ (which we identify with the set $S$ of its components) in the common reference string, where $\vecb{s}$ is sampled from some distribution $\dist_{n,1}$.
\item[2)] The prover commits to $[\vecb{x}]_1=([x_1]_1,\ldots,[x_n]_1)^\top$, a permutation of the set $S$ and proves that the commitments to $[\vecb{x}]_1$ are in $\mathcal{L}^{n}_{ck,S}$.
\item[3)] The prover proves that $\sum_{i \in [n]} [x_i]_1 =\sum_{i \in [n]} [s_i]_1$.
\item[4)] Finally, the prover outputs a proof that:\footnote{This is a slightly oversimplified explanation. 
Actually, a prover (a mixer) does not know the randomness nor the decryptions of the ciphertexts but only the randomness of the re-encryptions, so it cannot prove exactly this statement.} 
\begin{equation}\label{shuffle:ker}[\vecb{s}^{\top}]_1 [\vecb{m}_1]_2 =[\vecb{x}^{\top}]_1 [\vecb{m}_2]_2.
\end{equation}
\end{itemize}
The underlying computational assumption is that it is infeasible to find a non-trivial combination of elements of $S$ which adds to $0$, that is, given $[\vecb{s}]_1$ it is infeasible to find $[\vecb{k}]_2 \neq [\vecb{0}]_2$ such that
$\vecb{s}^{\top} \vecb{k}=\vecb{0}$ (this is the $\dist_{n,1}$-$\kermdh$ assumption from Section~\ref{sec:mddh}). 

Soundness goes as follows. First, by the soundness of the aggregated set membership proof, $[\vecb{x}]_1 \in S^{n}$ and from the fact that 
 $\sum_{i \in [n]} x_i =\sum_{i \in [n]} s_i$, it holds that if 
 $\vecb{x}$ is not a permutation of $\vecb{s}$, then one can extract in the soundness game (assuming the extractor knows $ck$) a non-trivial linear combination of elements of $S$ which adds to $0$, which contradicts the security assumption. 
Finally, if $\vecb{x}$ is a permutation of $\vecb{s}$,  then equation (\ref{shuffle:ker}) implies that the shuffle is correct, or, again, 
one can extract from   $[\vecb{m}_1]_2,[\vecb{m}_2]_2$ the coefficients of some non-trivial combination of elements of $S$ which is equal to $0$ (breaking the $\dist_{n,1}$-$\kermdh$ assumption). 

This soundness argument is an augmentation and translation into asymmetric groups of the argument of Groth and Lu \cite{AC:GroLu07}. Essentially, the argument there also consists of two parts: one devoted to proving that some GS commitments open to a permutation of some set in the CRS (Groth and Lu prove this using the (non-standard) pairing permutation assumption), while the second part (Step 4) is proven very similarly (in particular, its soundness also follows from some kernel assumption secure in symmetric bilinear groups). 

We note that it is crucial for our soundness argument that it is possible to decrypt the ciphertexts (otherwise we cannot extract solutions to the kernel problems). This is possible in our case because public key for encryption is assumed to be witness-samplable and the argument is quasi-adaptive. This explains why we do not refer to the notion of culpable soundness, as done by Groth and Lu \cite{AC:GroLu07} and by Fauzi and Lipmaa \cite{EPRINT:FauLip15}.



    \section{Aggregated NIZK Set Membership Arguments} \label{sec:aZKSMP}

        In this section we construct a QA-NIZK argument that many commitments open to elements in a set $[0,d-1] \subset \Z_q$ or  $S \subset \GG_{\gamma}$. We say that the argument is aggregated beacuse the size of the proof does not depend on the number of commitments.

Before we move to the aggregated case, we study the case of a single set membership proof.

 \subsection{Set Membership Proofs}
We want to show that a single commitment belongs to the language
$$
\Lang_{ck,S}:=\{c: \exists x\in S, \vecb{w}\in\Z_q^r \text{ s.t. } c=\Com_{ck}(x;\vecb{w})\}\text{, where }ck\gets\distk,
$$
and $c=\Com_{ck}(x;\vecb{w})$ is a Groth-Sahai commitment to $x$ with randomness $\vecb{w}$.


We observe that membership in $S$ can be written as:
\begin{itemize}
\item If $S \subset \GG_{\gamma}$, and we identify $S$ with $[\vecb{s}]_\gamma=([s_1]_\gamma,\ldots,[s_m]_\gamma)^\top$ then, 
$c \in \Lang_{ck,S}$ if and only if $\exists \vecb{b} \in \Z_q^{m}$ such that:
\begin{enumerate}
    \item $\vecb{b} \in \{0,1\}^{m}$,
    \item $c=\GS.\Com_{ck}(x;w)$,
    \item $x=\vecb{s}^{\top} \vecb{b}$,
    \item $\sum_{i \in [m]} b_i=1$.
\end{enumerate}
\item If $S=[0,d-1]$ and $m:=\log d$, then
$c \in \Lang_{ck,S}$ if and only if $\exists \vecb{b} \in \Z_q^{m}$ such that:
\begin{enumerate}   
    \item $\vecb{b} \in \{0,1\}^{m}$,
    \item $c=\GS.\Com_{ck}(x;w)$,
    \item $x=(1,2,\ldots,2^{m-1}) \vecb{b}$.
\end{enumerate}
\end{itemize} 
That is, both languages can be written in a similar way, except that when $S \subset \GG_{\gamma}$ there is an additional linear constraint that $\vecb{b}$ must satisfy (condition 4)). 

 To avoid distinguishing all the time between both types of subsets, we note that both languages can be seen as special case of the language 
 $\Lang_{[\matr{M}]_1,[\matr{N}]_1,\matr{\Lambda},\grkb{\alpha}}\subseteq\GG_1^\la$, as defined below.

\begin{definition}
Denote by $\Lang_{[\matr{M}]_1,[\matr{N}]_1,\matr{\Lambda},\grkb{\alpha}}\subseteq\GG_1^\la$ the language parameterized by $[\matr{M}]_1 \in\GG_1^{\la\times \lb},\matr{N}\in\GG_1^{\la\times \lc},\matr{\Lambda}\in\Z_q^{\ld\times\lb},$ and $\grkb{\alpha}\in\Z_q^\ld$ such that
\begin{equation}\label{eq:definition1}
[\vecb{c}]_1\in\Lang_{[\matr{M}]_1,[\matr{N}]_1,\matr{\Lambda},\grkb{\alpha}} \Longleftrightarrow \exists \vecb{b}\in\bits^\lb,\vecb{w}\in\Z_q^\lc \text{ s.t. }
\pmatri
{
    \vecb{c}\\
    \grkb{\alpha}
}
=
\begin{pmatrix}
    \matr{M}       & \matr{N}\\
    \matr{\Lambda} & \matr{0}_{\ld\times \lc}
\end{pmatrix}
\pmatri
{
    \vecb{b}\\
    \vecb{w}
}.
\end{equation}
Additionally, we require $(\matr{N},[\matr{N}]_1)$ to be efficiently samplable and that membership in $\Lang_{[\matr{M}]_1,[\matr{N}]_1,\matr{\Lambda},\grkb{\alpha}}$ is efficiently testable with the trapdoor $\matr{N}$, that is, that there exists an efficient algorithm $\algF$ such that $\algF([\matr{M}]_1,\matr{N},[\vecb{c}]_1)=1\Longleftrightarrow [\vecb{c}]_1\in\Lang_{\matr{M},\matr{N},\matr{\Lambda},\grkb{\alpha}}.$. 
\end{definition}
The additional condition is key in the proof of soundness and allows to tell apart fake from honest proofs.
\begin{example}
The language of GS commitments to group elements in the set $S:=\{[s_1]_1,\ldots,[s_\lb]_1\} \subset \GG_1$, $\Lang_{ck,S}$, where $ck:=([\vecb{u}_1]_1||[\vecb{u}_2]_1)$, is equal to $\Lang_{[\matr{M}]_1,[\matr{N}]_1,\matr{\Lambda},\grkb{\alpha}}$, where
$\matr{M}:=\smallpmatrix{s_1 & \cdots & s_\lb \\ 0 &\cdots & 0}$, $\matr{N}:=(\vecb{u}_1-\vecb{e}_1||\vecb{u}_2)$, $\alpha=1$, and $\matr{\Lambda}=(1,\ldots,1)$. Membership in $S$ is efficiently testable given $\vecb{u}_1,\vecb{u}_2\in\Z_q^2$ and assuming $|S|=\poly(\lambda)$.
\end{example}

\begin{example}
The language of GS commitments to integers in the range $[0,d-1]$, $\Lang_{ck,[0,d-1]}$, where $ck:=([\vecb{u}_1]_1||[\vecb{u}_2]_1)$, is equal to $\Lang_{[\matr{M}]_1,[\matr{N}]_1,\matr{\Lambda},\grkb{\alpha}}$, where
$\matr{M}:=\vecb{u}_1(2^0,2^1,\ldots,2^{\log d-1})\in\Z_q^{2\times \log d}$, $\matr{N}:=\vecb{u}_2\in\Z_q^{2}$, and $\ld:=0$. Membership in $\Lang_{ck,[0,d-1]}$ is easily testable given $\vecb{u}_2 \in \Z_q^2$ and assuming $d=\poly(\lambda)$. 
\end{example}

\subsubsection{Proof Strategy} The most efficient strategy we are aware of for proving membership in $\Lang_{[\matr{M}]_1,\matr{N},\matr{\Lambda},\grkb{\alpha}}$ follows a 
 commit-and-prove approach. Namely, to prove that $\vecb{b},\vecb{w}$ exist, one computes 
GS commitments $[\vecb{d}_i]_1$, $i \in [m]$, to all coordinates of $\vecb{b}$ and then it proves two independent statements, namely that:
\vspace{-0.2cm}
\begin{enumerate}
\item $\exists \vecb{b}\in \Z_q^\lb, \vecb{r} \in \Z_q^m$ such that  
    \begin{enumerate}
    \item $ \vecb{b}\in \{0,1\}^\lb$ and 
    \item $\forall i \in [\lb], \vecb{d}_i=\begin{pmatrix} \vecb{u}_1 &   \vecb{u}_2 \end{pmatrix}   \smallpmatrix{b_i  \\ r_i}$.
    \end{enumerate}
\item   $\exists \widetilde{\vecb{b}} \in \Z_q^\lb,  \widetilde{\vecb{r}} \in \Z_q^m, \vecb{w} \in\Z_q^\lc$ such that
    \begin{enumerate}
    \item $\smallpmatrix
{
    \vecb{c}\\
    \grkb{\alpha}
}
=
\smallpmatrix{
    \matr{M}       & \matr{N}\\
    \matr{\Lambda} & \matr{0}_{\ld\times \lc}
}
\smallpmatrix
{
    \widetilde{\vecb{b}}\\
    \vecb{w}
}$ and 
    \item $\forall i \in [m], \vecb{d}_i=\begin{pmatrix} \vecb{u}_1 &   \vecb{u}_2 \end{pmatrix}   \smallpmatrix{
\widetilde{b}_i  \\ r_i}$.
    \end{enumerate}
\end{enumerate}
For the first statement, one can use the QA-NIZK argument for bit-strings of \cite{AC:GonHevRaf15}, and for the second, the QA-NIZK argument for linear spaces of \cite{C:JutRoy14,EC:KilWee15} (for the latter, note that conditions 2.a) and 2.b) can be written down as a single system of equations with a large matrix $\widetilde{\matr{M}}$ and then satisfiability of 2.a) and 2.b) is equivalent to  $(\vecb{c}^{\top},\grkb{\alpha}^{\top},\vecb{d}_1^{\top}, \ldots, \vecb{d}_m^{\top})^\top$ being in the span 
of this matrix $\widetilde{\matr{M}}$).

Since both proofs are constant-size, the resulting proof size is dominated by the cost of the commitments to $b_i$, which is $\Theta(m)$. 
For soundness, the important point here is that we never prove that $\vecb{b}=\widetilde{\vecb{b}}$, but, since GS commitments are perfectly binding (or, said otherwise, because $\begin{pmatrix} \vecb{u}_1 &   \vecb{u}_2 \end{pmatrix}$
has full rank), equality holds. This immediately proves that the statement is in the language.  
\subsection{Aggregated Set Membership Proofs}
An aggregated set membership proof amounts to proving membership in $\Lang_{[\matr{M}]_1,[\matr{N}]_1,\matr{\Lambda},\grkb{\alpha}}^n$. By definition, $([\vecb{c}_1]_1,\ldots, [\vecb{c}_n]_1) \in\Lang_{[\matr{M}]_1,[\matr{N}]_1,\matr{\Lambda},\grkb{\alpha}}^n$ if and only if  $\forall j \in [n], \exists \vecb{b}_j\in \Z_q^\lb,\vecb{w}_j\in\Z_q^\lc$ such that
$$
 1) \vecb{b}_j\in \{0,1\}^\lb
  \wedge \ 2)
\smallpmatrix
{
    \vecb{c}_j\\
    \grkb{\alpha}
}
=
\smallpmatrix
{
    \matr{M}       & \matr{N}\\
    \matr{\Lambda} & \matr{0}_{\ld\times \lc}
}
\smallpmatrix
{
    \vecb{b}_j\\
    \vecb{w}_j
}.
$$
Recall that we want a proof size independent of $n$. This rules out the naive approach of computing GS commitments to all the coordinates of $\vecb{b}_j$, for all $j \in [n]$, as the cost is $\Theta(nm)$. Therefore, to improve on the asymptotic size of the proof, we are forced to use shrinking commitments to $b_{i,j}$. We stress that it is far from clear how to do this, as it might break down the soundness argument completely (e.g. in the single proof, we used in a fundamental way the uniqueness of the commitment openings). In fact, overcoming this problem is one of the main technical contributions of this chapter. 

Our idea is to use as a shrinking commitment a two-dimensional Multi-Pedersen commitment as defined on Sect. \ref{sec:ext-mp}. Given some matrix $\matr{G} \in \Z_q^{2 \times (n+1)}$ sampled from some distribution $\dist_{2,n+1}$, the commitment to $\vecb{y}\in\Z_q^n$ using randomness $r\in\Z_q$ is computed as $\mathsf{MP}.\Com(\vecb{y};r):=[\matr{G}]_1 \smallpmatrix{\vecb{y} \\ r}$. The special thing about these commitments is that one can set a ``hidden'' linearly independent column of 
$\matr{G}$, and thus commitments are perfectly binding at some coordinate $j^*\in[n]$ which is computationally hidden to the adversary.

Define the matrix $\matr{B}=(\vecb{b}_1|| \ldots || \vecb{b}_n) \in \{0,1\}^{\lb \times n}$ and let $\vecb{b}_i^*$ be the $i$th row of $\matr{B}$. To prove $([\vecb{c}_1]_1,\ldots, [\vecb{c}_n]_1) \in\Lang_{[\matr{M}]_1,[\matr{N}]_1,\matr{\Lambda},\grkb{\alpha}}^n$, we first compute MP commitments $[\vecb{d}_i]_1$, $i \in [\lb]$, to $\vecb{b}_i^*$.  As before, 
the proof actually consists of two independent statements:
\begin{enumerate}
\item $\exists \vecb{r} \in \Z_q^\lb, \matr{B} \in \Z_q^{\lb \times n}$ such that  
    \begin{enumerate}
    \item $\matr{B} \in \{0,1\}^{\lb \times n}$ and
    \item $\forall i \in [m], \vecb{d}_i=\matr{G}\smallpmatrix{\vecb{b}_i^*  \\ r_i}$,
    \end{enumerate}
\item $\exists \widetilde{\vecb{r}}\in\Z_q^\lb, \vecb{w}_1,\ldots,\vecb{w}_n \in\Z_q^\lc, \widetilde{\matr{B}} \in \Z_q^{\lb \times n}$, (whose rows are denoted as $\widetilde{\vecb{b}}_i^*$, $i \in [\lb]$, and the columns $\widetilde{\vecb{b}}_j$, $j \in [n]$), such that  
   \begin{enumerate}
   \item $\forall i\in[n], \smallpmatrix
{
    \vecb{c}_j\\
    \grkb{\alpha}
}
=
\smallpmatrix
{
    \matr{M}       & \matr{N}\\
    \matr{\Lambda} & \matr{0}_{\ld\times \lc}
}
\smallpmatrix
{
    \vecb{b}_j\\
    \vecb{w}_j
}$ and 
    \item $\forall i \in [\lb], \vecb{d}_i=\matr{G}   \smallpmatrix{\widetilde{\vecb{b}}^*_i  \\ \widetilde{r}_i}$.
    \end{enumerate}
\end{enumerate}
For the first we use a our aggregated proof that many commitments opens to bit-strings from Sect. \ref{sec:matr-bits} and for the second, (after rewriting the equations) a QA-NIZK argument for linear spaces. With this approach, the proof remains of size $\Theta(m)$, the size of the commitments, while the rest of the proof is constant. 

The interesting part is the soundness argument. The previous reasoning for the non-aggregated case (when $n=1$) fails here because now there is no guarantee that 
 $\matr{B}=\widetilde{\matr{B}}$ (as the openings of $[\vecb{d}_i]_1$ are not unique).  However, as we said, the distribution of the MP commitment key can be chosen so that it is binding at some coordinate $j^*$. This implies that for all $i$, the $j^*$th coordinate of $\vecb{b}_i^*$ and $\widetilde{\vecb{b}}_i^*$ is equal, i.e. the $j^*$th column of $\matr{B}$ and $\widetilde{\matr{B}}$ must be equal.  

Thus, we have that for the coordinate $j^*$, the proof is sound (because $\vecb{b}_j^*$ is uniquely determined, which was the uniqueness of openings which was necessary to prove soundness for $n=1$). That is, the adversary cannot break soundness for any tuple $([\vecb{c}_1]_1,\ldots, [\vecb{c}_n]_1)$ such that $[\vecb{c}_j^*]_1 \notin \Lang_{[\matr{M}]_1,[\matr{N}]_1,\matr{\Lambda},\grkb{\alpha}}$. But since $j^*$ is computationally hidden 
from the adversary, we can prove soundness with a loss in the reduction of $1/n$. 





        \subsection{QA-NIZK Argument of Membership in $\Lang_{[\matr{M}]_1,[\matr{N}]_1,\matr{\Lambda},\grkb{\alpha}}^n$} \label{sec:bin-lan-constr}

            The main result of this Section is a proof, of size $2\lb|\GG_1|+\Theta(1)$, that  $([\vecb{c}_1]_1,\allowbreak\ldots,\allowbreak[\vecb{c}_n]_1)$ is in $\Lang_{[\matr{M}]_1,[\matr{N}]_1,\matr{\Lambda},\grkb{\alpha}}^n$.

For all $j \in [n]$, let $(\vecb{b}_j,\vecb{w}_j) \in \{0,1\}^{\lb} \times \Z_q^{\lc}$ be the witness of $\vecb{c}_j \in \Lang_{[\matr{M}]_1,[\matr{N}]_1,\matr{\Lambda},\grkb{\alpha}}.$ Let $\matr{B}=(\vecb{b}_1|| \ldots ||\vecb{b}_n)$ and let $\vecb{b}^*_{i}$, $i \in [m]$ the ith row of $\matr{B}$. To get a proof of size independent of $n$ we commit to $\matr{B}$ ``compressing the rows'', that is, the proof includes MP commitments $[\vecb{d}_i]_1$, $i \in [n]$ to $\vecb{b}_i^*$.\footnote{To get a constant-size proof, it would be tempting to compress the commitments to all of $\matr{B}$, but we do not know how to prove soundness in this case.} Further, as announced in Sect. \ref{sec:shuf-rp-overview}, the proof consists of two independent statements:
\begin{itemize}
\item $\exists \vecb{r} \in \Z_q^\lb, \matr{B} \in \Z_q^{\lb \times n}$ such that  
$\text{1'')} \matr{B} \in \{0,1\}^{\lb \times n}$ and $\text{3'')} \forall i \in [\lb]: \vecb{d}_i=\matr{G}\smallpmatrix{\vecb{b}_i^*  \\ r_i}$,
\item $\exists \widetilde{\vecb{r}} \in \Z_q^\lb, \vecb{w}_1,\ldots,\vecb{w}_n \in\Z_q^\lc, \widetilde{\matr{B}} \in \Z_q^{\lb \times n}$ such that  
   $\text{2'')}\forall j \in [n], \smallpmatrix
{
    \vecb{c}_j\\
    \grkb{\alpha}
}
=
\smallpmatrix
{
    \matr{M}       & \matr{N}\\
    \matr{\Lambda} & \matr{0}
}
\smallpmatrix
{
    \widetilde{\vecb{b}}_j\\
    \vecb{w}_j
}$ and $3'') \forall i \in [\lb], \vecb{d}_i=\matr{G}   \smallpmatrix{\widetilde{\vecb{b}}^*_i  \\ \widetilde{r}_i}  $.
\end{itemize} 
For the first statement we use the constant-size argument for $\Lang_{ck,\sfbits}^m$ of Sect. \ref{sec:matr-bits}. For the second statement, we write conditions 2''), 3'') as a single system of equations and use $\Pi_\sflin$ to prove that it can be satisfied. 

The soundness argument follows from the arguments exposed in Sect. \ref{sec:shuf-rp-overview}. The full description of the argument is in Fig. \ref{fig:bin-leng-nizk} and security folows from Theorem \ref{theo:aggset}.
\begin{figure}
\begin{\algSize}
$$
\begin{array}{ll}
\begin{array}{l}
\algK_1(\gk,[\matr{M}]_1,[\matr{N}]_1,n)
\quad (\mathsf{S}_1(\gk,[\matr{M}]_1,[\matr{N}]_1,n))
\\
\hline
[\matr{G}]_1 \gets \MP.\algK(1^\lambda,n)\\
{[\matr{\Xi}]_1 := [\matr{\Xi}(\matr{M},\matr{N},\matr{\Lambda},\matr{G})]_1}\\
\crs_\sflin\gets\Pi_\sflin.\algK_1(\gk,[\matr{\Xi}]_1)\\
\crs_\sfbits\gets\Pi_\sfbits.\algK_1(\gk,[\matr{G}]_1,\lb)\\
\text{Return } \ \crs:=(\crs_\sflin,\crs_\sfbits).\\
(\tau_\sflin\gets\Pi_\sflin.\algS_1(\gk,[\matr{\Xi}]_1)\\
\tau_\sfbits\gets\Pi_\sfbits.\algS_1(\gk,[\matr{G}]_1,\lb).\\
\tau := (\tau_\sflin,\tau_\sfbits)).\\
\\
\end{array}
&
\begin{array}{l}
{\algP(\mathsf{crs}, \{[\vecb{c}_j]_1,\langle \vecb{b}_j,\vecb{w}_j\rangle:j\in[n]\})}\\
\hline
{[\vecb{d}_i]_1} := \MP.\Com_{[\matr{G}]_1}(\vecb{b}_i^*;r_i),\\
r_i \gets\Z_q, \forall i\in[\lb]\\
\pi_\sflin \gets 
    \Pi_\sflin.\algP
    (
        \crs_\sflin,
            [\vecb{y}]_1,
            \vecb{v}
    )\\
\pi_\sfbits \gets
    \Pi_\sfbits.\algP
    (
        \crs_\sfbits,
        \{[\matr{d}_i]_1,\\
\qquad
        \langle\matr{b}^*_i,r_i\rangle:i \in[\lb]\}
    )\\
\text{Return } \  ([\vecb{d}]_1,\pi_\sflin,\pi_\sfbits). \\
\\
\\
\\
\end{array}\\
\begin{array}{l}
{\algV(\mathsf{crs},\{[\vecb{c}_j]_1:j\in[n]\},([\vecb{d}]_1,\pi_\sflin,\pi_\sfbits))}\\
\hline
\mathsf{ans}_1 \gets
    \Pi_\sflin.\algV
    (
        \crs_\sflin,
            [\vecb{y}]_1,
        \pi_\sflin
    )\\
\mathsf{ans}_2 \gets \Pi_\sfbits.\algV(\crs_\sfbits,\{[\vecb{d}_i]_1:i\in[\lb]\},\pi_\sfbits)\\
\text{Return } \ \mathsf{ans}_1\wedge\mathsf{ans}_2.
\\
\\
\\
\\
\end{array}
&
\begin{array}{l}
{\mathsf{S}_2(\crs,[\vecb{c}]_1,\tau)}\\
\hline
{[\vecb{d}_i]_1} := \MP.\Com_{[\matr{G}]_1}(\matr{0}_{n\times 1};\tilde{{r}}_i)\\
\tilde{{r}}_i\gets\Z_q, \forall i\in[\lb]\\
\pi_\sflin \gets 
    \Pi_\sflin.\algS
    (
        \crs_\sflin,
            [\vecb{y}]_1,
       \tau_\sflin
    )\\
\pi_\sfbits \gets
    \Pi_\sfbits.\algS
    (
        \crs_\sfbits,\\
\qquad\quad  \{[\vecb{d}_i]_1:i\in[\lb]\},
        \tau_\sfbits
    )\\
\text{Return } \  ([\vecb{d}]_1,\pi_\sflin,\pi_\sfbits). \\
\end{array}
\end{array}$$
\end{\algSize}
\caption{Proof system for the language $\Lang_{\matr{M},\matr{N},\matr{\Lambda},\grkb{\alpha}}^n$, where $\Pi_\sfbits$ is the proof system for $\Lang_{ck,\sfbits}^m$ from Sect. \ref{sec:matr-bits}, $\vecb{d}:=\vecb{d}_1\oplus\ldots\oplus\vecb{d}_\lb$, and $\vecb{c}:=\vecb{c}_1\oplus\ldots\oplus\vecb{c}_n$. The proof size is $(2\lb+11)|\GG_1|+10|\GG_2|$.\label{fig:bin-leng-nizk}}
\end{figure}



\begin{theorem} \label{theo:aggset} There exists a QA-NIZK argument $\Pi_\sfset$ for membership in the language $\Lang_{[\matr{M}]_1,[\matr{N}]_1,\matr{\Lambda},\grkb{\alpha}}^n$ with proof size  $(2\lb+11)|\GG_1|+10|\GG_2|$, perfect completeness, perfect-zero knowledge and computational soundness. 
\end{theorem}



    \section{Proof of Correctness of a Shuffle} \label{sec:shuffle}

        In a NIZK Shuffle argument one wants to prove that two lists of ciphertexts open to the same values when second list is permuted under some hidden permutation.
We represent each list of ciphertexts as a matrix in $\GG_2^{2\times n}$ where each column is an El-Gamal ciphertext under public key $pk:=[\vecb{v}]_2\in\GG^2_2$ and we write $\enc_{pk}([\vecb{m}^\top]_2;\vecb{r}^\top):=(\enc_{pk}([m_1]_2;r_1)||\cdots||\enc_{pk}([m_n]_2;r_n))$, where $[\vecb{m}]_2\in\GG_2^n$, $\vecb{r}\in\Z_q^n$, and $\enc_{pk}([m]_2;r):=[m]_2\vecb{e}_2+r[\vecb{v}]_2$. Similarly, through this section we will sometimes write $\GS.\Com_{ck}([\vecb{x}^\top]_\gamma;\matr{R}):=$\-$(\GS.\Com_{ck}([x_1]_\gamma;\vecb{r}_1)||\cdots||$$\GS.\Com_{ck}($$[x_n]_\gamma;$$\vecb{r}_n))$, where $\matr{R}=(\vecb{r}_1||\cdots||\vecb{r}_n)\in\Z_q^{2\times n}$.

The language of correct shuffles under public key $pk:=[\vecb{v}]_2\in\GG^2_2$ can can be defined as 
\begin{align*}
\Lang_{pk,n,\mathsf{shuffle}}:=\{([\matr{C}]_2,&[\matr{D}]_2)\in\GG_2^{2\times n}\times \GG_2^{2\times n} :\\
                                                         &\exists \matr{P}\in\mathcal{S}_n,\grkb{\delta}\in\Z_q^n \text{ s.t. } {[\matr{C}]_2\matr{P}-[\matr{D}]_2 = \Enc_{pk}([\vecb{0}_{1\times n}]_2;\grkb{\delta}^\top)}\},
\end{align*}
where $\mathcal{S}_n$ is the set of permutation matrices of size $n\times n$. This definition can be generalized for any ``El-Gamal like'' encryption scheme as, for example, the BBS encryption scheme from \cite{C:BonBoySha04}.

%We construct a NIZK Shuffle argument with linear proof size, specifically $4n+17$ elements of $\GG_1$, $14$ elements of $\GG_2$, and 1 element of $\Z_q$ in type III groups, and $6n+34$ group elements in Type I groups. The security is based on assumptions which are weaker than \sxdh~plus \SSDP~Assumption in  Type III groups. In Type I groups our construction can be based on assumptions which are all weaker than \lin{2}.\footnote{In the symmetric case, the proof system from Sect. \ref{sec:bits} can be based on assumptions which are all weaker than $\lin{2}$, similarly as done in \cite[Appendix C]{EPRINT:GonHevRaf15}.}
 
\subsection{Our construction}

Our proof system builds on a proof that a set of GS commitments open to elements in the set $S=\{[s_1]_1,\ldots,[s_n]_1\}$, where $\vecb{s}:=(s_1,\ldots,s_n)^\top\gets\dist_{n,1}$ and the $\dist_{n,1}\mbox{-}\kermdh$ Assumption holds in $\GG_1$. Given $[\matr{F}]_1\in\GG_1^{2\times n}$, where the $i$ th column is $[\vecb{f}_i]_1\gets\GS.\Com([x_i]_1)$, let $\vecb{x}:=({x}_1,\ldots,{x}_n)^\top=\matr{P}\vecb{s}$, for some permutation matrix $\matr{P}$, and given a commitment to $[y]_1:=[\vecb{s}^\top]_1\grkb{\delta}$, we prove that $([\matr{C}]_2,[\matr{D}]_2)\in\Lang_{pk,n,\mathsf{shuffle}}$ as follows:
\begin{enumerate}[label=\alph*)]
\item Show that $[\matr{F}]_1\in\Lang_{ck,S}^n$, where $ck\gets\GS.\algK(\gk)$.\label{shuffle:a}
\item Give a GS proof for the satisfiability of $\sum_{i\in[n]}[s_i]_1-\sum_{j\in[n]}[{x}_j]_1=[0]_1$.\label{shuffle:b}
\item Give a GS proof for the satisfiability of
$
[\vecb{x}^\top]_1[\matr{C}^\top]_2-[\vecb{s}^\top]_1[\matr{D}^\top]_2=[y]_1[\vecb{v}^\top]_2.\label{shuffle:c}
$\footnote{While using slightly different assumptions, in \cite{AC:GroLu07} this step is also done.}
\end{enumerate}

\subsubsection{Soundness Intuition.} Conditions \ref{shuffle:a} and \ref{shuffle:b} implies that $\vecb{x}$ is a permutation of $\vecb{s}$ or equivalently, $\matr{x}=\matr{P}\vecb{s}$ and $\matr{P}$ is a permutation matrix. Note that $\matr{P}$ is a permutation matrix iff $\matr{P}$ is a binary matrix and for each row and column there is at most one 1. Let's see in more detail why $\vecb{x}$ is a permutation of $\vecb{s}$. Condition \ref{shuffle:a} implies that each $x_i$ is an element from $\{s_1,\ldots,s_n\}$, which can be written as $\vecb{x}=\matr{P}\vecb{s}$, $\matr{P}\in\bits^{n\times n}$, where each row of $\matr{P}$ has at most one 1. But, given that there might be repeated elements, there might be also more than one 1 in some column of $\matr{P}$. For example, if $S=\{s_1,s_2,s_3\}$, it may be that $\vecb{x}=\smallpmatrix{s_2\\s_3\\s_1}=\smallpmatrix{0&1&0\\0&0&1\\1&0&0}\smallpmatrix{s_1\\s_2\\s_3}$ but also $\vecb{x}=\smallpmatrix{s_2\\s_3\\s_3}=\smallpmatrix{0&1&0\\0&0&1\\0&0&1}\smallpmatrix{s_1\\s_2\\s_3}$. Condition \ref{shuffle:b} implies that there are no repeated $x_i$s unless one can break the $\dist_{n,1}\mbox{-}\kermdh$ assumption. Indeed, there are repeated $x_i$s iff  $(1,\ldots,1)\matr{P}$ (the row vector of ``frequencies" of $\vecb{x}$, which in the first example is $(1,1,1)$ and in the second $(0,1,2)$) is not equal to $ (1,\ldots,1)$. Given that \ref{shuffle:b} is equivalent to $((1,\ldots,1)-(1,\ldots,1)\matr{P})[\vecb{s}]_1=[0]_1$, then $((1,\ldots,1)-(1,\ldots,1)\matr{P})^\top$ is solution to the $\dist_{n,1}\mbox{-}\kermdh$ problem. We conclude that $\matr{P}$ is a permutation matrix and thus $\vecb{x}$ is a permutation of $\vecb{s}$.

The remainder of the proof follows essentially the proof from \cite{AC:GroLu07}. Suppose that $[\matr{C}]_2=\Enc_{[\vecb{v}]_2}([\matr{m}^\top]_2)$ and $[\matr{C}]_2=\Enc_{[\vecb{v}]_2}([\matr{n}^\top]_2)$. Let $\vecb{k}=(-v_2/v_1,1)^\top$ the ``decryption key'' (i.e. $\vecb{v}^\top\vecb{k}=0$ and $(0,1)\vecb{k}=1$)\footnote{The availability of the decryption key $\vecb{k}$ in the soundness reduction is possible since the reduction samples by itself the language parameter $\vecb{v}$. Correspondingly Groth and Lu \cite{AC:GroLu07} proved \emph{Culpable Soundness} (also called co-soundness), which essentially requires the soundness adversary to produce the decryption key.}, we multiply by $\matr{k}$, on the right, the equation from condition $\ref{shuffle:c}$ to ``decrypt'' $[\matr{C}]_2$ and $[\matr{D}]_2$. We get that
$[\vecb{s}^\top]_1\matr{P}^\top[\matr{m}]_2-[\vecb{s}^\top]_1[\matr{n}]_2=[0]_T$, which implies that $\matr{P}^\top[\matr{m}]_2=[\matr{n}]_2$ unless $\matr{P}^\top[\matr{m}]_2-[\matr{n}]_2$ is a solution to the $\dist_{n,1}\mbox{-}\kermdh$. Finally this implies that $[\matr{C}]_2\matr{P}-[\matr{D}]_2$ is an encryption of $[\matr{0}_{n\times 1}]_2$ and thus $([\matr{C}]_2,[\matr{D}]_2)\in\Lang_{pk,n,\mathsf{shuffle}}$.

A detailed description of our construction is in Fig. \ref{fig:shuffles} and the proof of security follows from Theorem \ref{theo:shuffle}
%\vspace*{-1cm}
\begin{figure}
\begin{\algSize}
$$
\begin{array}{l}
\begin{array}{l}
\algK(\gk,[\vecb{v}]_2,n)\quad (\mathsf{S}_1(\gk,[\vecb{v}]_2,n))\\
\hline
S:=\{[s_1]_1,\ldots,[s_n]_1\}, \vecb{s}^\top\gets\distlin_{1,n}\\
\crs_\GS\gets\GS.\algK(\gk)\\
{\crs_\sfset\gets\Pi_\sfset.\algK(\gk,[\matr{M}]_1,[\matr{N}]_1,[\matr{\Lambda}]_1,n)}\\
\text{Return } \ \crs:=([\vecb{v}]_2,\crs_\GS,\crs_\sfset).\\
(\tau_\GS\gets\GS.\algS_1(\gk)\\
\tau_\sfset\gets\Pi_\sfset.\algS_1(\gk,L,[\matr{M}]_1,[\matr{N}]_1,[\matr{\Lambda}]_1,n).\\
\tau := (\vecb{s},\tau_\GS,\tau_\sfset)).\\
\\
\end{array}
\\
\begin{array}{l}
{\algP(\crs, [\matr{C}]_2, [\matr{D}]_2,\langle \matr{P}, \grkb{\delta}\rangle)}\\
\hline
[\vecb{x}]_1 := \matr{P}[\vecb{s}]_1,
[y]_1 := [\vecb{s}^\top]_1\grkb{\delta}\\
\pi_\GS\gets\GS.\algP(\crs_\GS,\{\eq_1,\eq_2\},\langle[\vecb{x}]_1,[y]_1\rangle)\\
\pi_\sfset \gets \Pi_\sfset.\algP(\crs_\sfset, [\matr{F}]_1,\langle\matr{P}^\top,\matr{R}\rangle)\\
//{[\matr{F}]_1} = \GS.\Com_{\crs_\GS}([\vecb{x}^\top]_1;\matr{R})\\
\text{Return } \  (\pi_\GS,\pi_\sfset). \\
\\
\end{array}\\
\begin{array}{l}
{\algV(\crs,[\matr{C}]_2,[\matr{D}]_2,(\pi_\GS,\pi_\sfset))}\\
\hline
\mathsf{ans}_1 \gets \GS.\algV(\crs_\GS,\{\eq_1,\eq_2\},\pi_\GS)\\
\mathsf{ans}_2 \gets \Pi_\sfset.\algV(\crs_\sfset,[\matr{F}]_1,\pi_\sfset)\\
\text{Return } \ \mathsf{ans}_1\wedge\mathsf{ans}_2.
\\
\\
\end{array}
\\
\begin{array}{l}
{\mathsf{S}_2(\crs,[\matr{C}]_1,[\matr{D}]_1,\tau)}\\
\hline
\pi_\GS\gets \GS.\algS_2(\crs_\GS,\{\eq_1,\eq'_2\},\tau_\GS)\\
\pi_\sfset\gets \Pi_\sfset.\algS_2(\crs_\sfset,[\matr{F}]_1,\tau_\sfset)\\
\text{Return }  (\pi_\GS,\pi_\sfset).
\end{array}
\end{array}
$$
\end{\algSize}
\caption{The proof system $\Pi_\mathsf{shuffle}$ for the language $\Lang_{[\vecb{v}]_2,n,\mathsf{shuffle}}$. $\Pi_\sfset$ is the proof system from Sect. \ref{sec:bin-lan-constr}. The matrices $\matr{M},\matr{N},\matr{\Lambda}$ are defined as
$\matr{M}:=\binom{\vecb{s}^\top}{\matr{0}_{1\times n}}$,
$\matr{N}:=(\vecb{u}_1||\vecb{u}_2),\matr{\Lambda}:= (1,\ldots,1)$, where $\vecb{u}_1,\vecb{u}_2$ are the GS commitment keys from $\crs_\GS$, and the equations are defined as $\eq_1:= \sum_{i\in[n]}[s_i]_1-\sum_{j\in[n]}[{x}_i]_1 = [0]_1$, $\eq_2:= [\vecb{x}^\top]_1[\matr{C}^\top]_2-[\vecb{s}^\top]_1[\matr{D}^\top]_2=[y]_1[\vecb{v}^\top]_2$, and $\eq'_2:= [\vecb{x}^\top]_1[\matr{C}^\top]_2-[1]_1[\vecb{s}^\top\matr{D}^\top]_2=[y]_1[\vecb{v}^\top]_2$   
. The proof size is $(4n+17)|\GG_1|+14|\GG_2|+1|\Z_q|$
%$|\pi_\GS|+|\pi_\sfset|=(2n+6)|\GG_1|+4|\GG_2|+1|\Z_q|+(2n+11)|\GG_1|+10|\GG_2|=(4n+17)|\GG_1|+14|\GG_2|+1|\Z_q|$.
\label{fig:shuffles}}
\end{figure}


\begin{theorem} \label{theo:shuffle}
The proof system from Fig. \ref{fig:shuffles} is a QA-NIZK proof system for the language $\Lang_{pk,n,\mathsf{shuffle}}$ with perfect completeness, computational soundness, and computational zero-knowledge.
\end{theorem}

\begin{proof}
(Completeness.) If $\matr{P}$ is a permutation matrix and $\vecb{x}=\matr{P}\vecb{s}$, then $\GS.\Com(\vecb{x}^\top)\in\Lang_{S,\vecb{u}_1,\vecb{u}_2}^n$ and $\sum_{i\in[n]}[s_i]_1-\sum_{j\in[n]}[{x}_j]_1=[0]_1$. If $[y]_1=[\vecb{s}^\top]_1\grkb{\delta}$ then
\begin{eqnarray*}
    [\vecb{x}^\top]_1[\matr{C}^\top]_2-
    [\vecb{s}^\top]_1[\matr{D}^\top]_2
 = 
    [\vecb{s}^\top]_1([\matr{C}]_2\matr{P}-[\matr{D}]_2)^\top
 = 
    [\vecb{s}^\top]_1([\vecb{v}]_2\grkb{\delta}^\top)^\top
 =
    [y]_1[\vecb{v}^\top]_2.
\end{eqnarray*}

(Soundness.)
We will show that for any adversary $\advA$ against the soundness of the proof system from Figure \ref{fig:shuffles}, there exist an adversary $\advB_1$ against soundness of $\Pi_{\sfset}$ and an adversary $\advB_2$ against the $\dist_{n,1}\mbox{-}\kermdh$ Assumption such that
$$\adv(\advA) \leq \adv_{\Pi_{\sfset}}(\advB_1) + \adv_{\dist_{n,1}\mbox{-}\kermdh}(\advB_2).
$$

The adversary $\advB_1$ receives as input $\crs_\sfset$ and honestly samples the rest of the CRS. Then $\advB_1$ runs $\advA$ until it halts and outputs $[\matr{F}]_1$ with the proof $\pi_\sfset$.

The adversary $\advB_2$ receives as input $[\vecb{s}]_1\in\GG_1^{n}$, samples $\vecb{u}_1,\vecb{u}_2,\vecb{v}\gets \Z_q^2$, honestly simulates the rest of the CRS, and runs $\advA$ until it halts. It extracts $[\vecb{x}^\top]_1$, the opening of $[\matr{F}]_1$, using $\vecb{u}_1,\vecb{u}_2$, and aborts if $[\matr{F}]_1\notin\Lang_{\vecb{u}_1,\vecb{u}_2,S}^n$. Else $[\vecb{x}]_1=\matr{B}[\vecb{s}]_1$, where $\matr{B}\in\bits^{n\times n}$ and $\matr{B}\matr{1}_{n\times 1}=\matr{1}_{n\times 1}$. If there are repeated $x_i$s, $\advB$ outputs $(\matr{1}_{1\times n}-\matr{1}_{1\times n}\matr{B})^\top$. Else, using $\vecb{v}$, $\advB_2$ decrypts $[\matr{C}]_2$ and $[\matr{D}]_2$ obtaining $[\vecb{m}^\top]_2\in\GG_2^{1\times n}$ and $[\vecb{n}^\top]_2\in\GG_2^{1\times n}$, respectively, and returns $\matr{B}^\top[\vecb{m}]_2-[\vecb{n}]_2$.

Let $E_1$ the event where $([\matr{C}]_2,[\matr{D}]_2)\in\Lang_{pk,n,\mathsf{shuffle}}$, $E_2$ the event where $[\matr{F}]_1\in\Lang_{\vecb{u}_1,\vecb{u}_2,S}^n$, and $E_3$ the event where $\matr{1}_{1\times n}=\matr{1}_{1\times n}\matr{B}\wedge\matr{B}^\top[\vecb{m}]_2-[\vecb{n}]_2=0$. Note that $E_2\wedge E_3\Longrightarrow E_1$ since $E_2$ implies that $\vecb{x}=\matr{B}\vecb{s}$, where $\matr{B}\in\bits^{n\times n}$ and $\matr{B}\matr{1}_{n\times 1}=\matr{1}_{n\times 1}$, and together with $E_3$ implies that $\matr{B}$ is a permutation. Note also that $\eq_1\wedge\eq_2\wedge\neg E_3$ implies that $(\matr{1}_{1\times n}-\matr{B}\matr{1}_{1\times n})^\top$ or $\matr{B}^\top[\vecb{m}]_1-[\vecb{n}]_2$ are solutions to the $\dist_{1,n}\mbox{-}\kermdh$. Then it holds that
\begin{eqnarray*}
\adv(\advA) & = &
    \Pr[\neg E_1 \wedge \algV(\crs,([\matr{C}]_2,[\matr{D}]_2),\pi)=1]\\
& = &
    \Pr[\neg E_1 \wedge \algV(\crs,([\matr{C}]_2,[\matr{D}]_2),\pi)=1 \wedge \neg E_2]+\\
& &
    \Pr[\neg E_1 \wedge \algV(\crs,([\matr{C}]_2,[\matr{D}]_2),\pi)=1 \wedge E_2]\\
& \leq &
    \Pr[\neg E_2 \wedge \Pi_\sfset.\algV(\crs_\sfset,[\matr{F}]_2,\pi_\sfset)=1]+
    \Pr[\neg E_1 \wedge \eq_1 \wedge \eq_2 \wedge E_2]\\
& \leq &
    \adv_{\Pi_\sfset}(\advB_1)+\Pr[\neg E_1 \wedge \eq_1 \wedge \eq_2 \wedge E_2\wedge E_3] + \Pr[\neg E_1 \wedge \eq_1 \wedge \eq_2 \wedge E_2\wedge \neg E_3]\\
& \leq &
    \adv_{\Pi_\sfset}(\advB_1) + \Pr[\neg E_1\wedge E_2 \wedge E_3]+
    \Pr[\eq_1\wedge\eq_2\wedge\neg E_3]\\
& = &
    \adv_{\Pi_\sfset}(\advB_1)+0+\adv_{\dist_{1,n}\mbox{-}\kermdh}(\advB_2).
\end{eqnarray*} 

(Zero-Knowledge.) We need to check that the inputs to the simulators are true statements and, for the GS simulator, that the equations allow simulation. This is certainly true for $\mathsf{eq}_1$ and, if $([\matr{C}]_1,[\matr{D}]_1)\in\Lang_{pk,n,\mathsf{shuffle}}$, then is also true for $\mathsf{eq}'_2$. Furthermore, it is guaranteed that $\pi_\GS$ is computationally indistinguishable from a real proof for $\{\eq_1,\eq'_2\}$, which is identically distributed to a real proof for $\{\eq_1,\eq_2\}$ since $\eq'_2$ and $\eq_2$ accepts the same set of solutions.
Finally, since the perfectly hiding $\crs_\GS$ is such that $\rank(\vecb{u}_1||\vecb{u}_2)=2$, then $\Lang_{\vecb{u}_1,\vecb{u}_2,S}^n=\GG_1^{2\times n}$ and thus $[\matr{F}]_1\in\Lang_{\vecb{u}_1,\vecb{u}_2,S}^n$ is always true.
\end{proof}


    \section{Range Argument} \label{sec:range-proof}

        We want to prove that a GS commitment $[\vecb{c}]_1$ opens to some integer $y$ in the range $[0,2^n-1]$. That is, construct a NIZK proof system for the language
$$
\Lang_{ck,[0,2^n-1]} := \{[\vecb{c}]_1\in\GG_1^2: \exists y,r\in\Z_q\text{ s.t. }[\vecb{c}]_1=\GS.\Com(y;r)\wedge y\in[0,2^n-1]\},
$$
where $ck:=([\vecb{u}_1]_1,[\vecb{u}_2]_1)\gets\GS.\algK(1^\lambda)$.
%The most efficient Range Proof under falsifiable assumptions, specifically the $q$-Hidden Strong Diffie-Hellman Assumption, is of size $\Theta(\frac{n}{\log n-\log\log n})$ \cite{PAIRING:RiaKohPre09}.
%In general, the approach from \cite{PAIRING:RiaKohPre09}, which closely follows \cite{AC:CamChaShe08}, can be summarized as follows:
Our proof is as follows:
\begin{enumerate}[label=\alph*)]
\item Commit to $y_1,\ldots y_\ell$.
\item Show that $y_i\in[0,d-1]$, for each $i\in[\ell]$. \label{rp:b}
\item Show that $y=\sum_{i\in[\ell]}y_id^{i-1}$.
\end{enumerate}
%Given that the maximum $y$ expressible as $\sum_{i\in[\ell]}x_id^{i-1}$ is $\sum_{i\in[\ell]}(d-1)d^{i-1}=d^\ell-1$,
Given that it must hold that $\ell=n/\log d$, the total size of the proof is $\mathsf{S}_{[0,d-1]}(\ell)+\Theta(\ell)$, where $\mathsf{S}_{[0,d-1]}(\ell)$ is the size of $\ell$ Range Proofs in the interval $[0,d-1]$.

\subsection{Our Construction}
Note that \ref{rp:b} is equivalent to show that $(\GS.\Com(y_1)||\cdots||\GS.\Com(y_\ell))\in\Lang_{ck,[0,d-1]}^\ell$. Thus, using the proof system from Sect. \ref{sec:bin-lang} we are able to aggregate $\ell$ Range Proofs in the interval $[0,d-1]$ into a single proof of size $\Theta(\log d)$. Choosing $d=n^k$ we get that $\mathsf{S}_{[0,d-1]}(\ell)=\Theta(k\log n)$ and $\ell=n/\log n^k=\frac{n}{k\log n}$, and thus the size of our Range Proof is $\Theta(\frac{n}{k\log n})$ for an arbitrarily chosen $k\in\mathbb{N}$. One would be tempted to choose $d=2^{\sqrt{n}}$ to obtain a proof of size $\Theta(\sqrt{n})$. However, the proof system from Sect. \ref{sec:bin-lang} requires membership in $\Lang_{ck,[0,d-1]}$ to be efficiently testable, which seems to be infeasible as when $d=2^{\sqrt{n}}$.

A detailed description of our proof system is in Fig. \ref{fig:rp}.
\begin{figure} 
\begin{\algSize}
$$
\begin{array}{ll}
\begin{array}{l}
\algK_1(\gk,[\vecb{u}]_1,[\vecb{u}_2]_1,n)\quad (\mathsf{S}_1(\gk,[\vecb{u}_1]_1,[\vecb{u}_2]_1,n))\\
\hline
d:=n^k, m:=\log d, \ell := n/m\\
\crs_\GS\gets\GS.\algK_1(\gk)\\
{\crs_\sfset\gets\Pi_\sfset.\algK_1(\gk,[\matr{M}]_1,[\matr{N}]_1,n)}\\
\text{Return } \ \crs:=([\vecb{u}_1]_1,[\vecb{u}_2]_1,\crs_\GS,\crs_\sfset).\\
(\tau_\GS\gets\GS.\algS_1(\gk)\\
\tau_\sfset\gets\Pi_\sfset.\algS_1(\gk,[\matr{M}]_1,[\matr{N}]_1,n).\\
\tau := (\tau_\GS,\tau_\sfset)).\\
\\
\end{array}
&
\begin{array}{l}
{\algP(\crs, [\matr{c}]_1, \langle y, r\rangle)}\\
\hline
\vecb{y}\in\Z_q^{\ell} \text{ is s.t. } y=\sum_{i\in[\ell]}{y}_id^{i-1}\\
\matr{B}\in\bits^{m\times\ell} \text{ is s.t. }\\
\qquad\vecb{y}^\top = (2^0,\ldots,2^{m-1})\matr{B}\\
\pi_\GS\gets\GS.\algP(\crs_\GS,\eq,\langle \vecb{y},r\rangle)\\
\pi_\sfset \gets \Pi_\sfset.\algP(\crs_\sfset, [\matr{X}]_1,\langle\matr{B},\matr{R}\rangle)\\
//{[\matr{X}]_1} = \GS.\Com_{\crs_\GS}(\vecb{y}^\top;\matr{R})\\
\text{Return } \  (\pi_\GS,\pi_\sfset). \\
\\
\end{array}\\
\begin{array}{l}
{\algV(\crs,[\vecb{c}]_1,(\pi_\GS,\pi_\sfset))}\\
\hline
\mathsf{ans}_1 \gets \GS.\algV(\crs_\GS,\eq,\pi_\GS)\\
\mathsf{ans}_2 \gets \Pi_\sfset.\algV(\crs_\sfset,[\matr{X}]_1,\pi_\sfset)\\
\text{Return } \ \mathsf{ans}_1\wedge\mathsf{ans}_2.
\\
\\
\end{array}
&
\begin{array}{l}
{\mathsf{S}_2(\crs,[\vecb{c}]_1,\tau)}\\
\hline
\pi_\GS\gets \GS.\algS_2(\crs_\GS,\eq,\tau_\GS)\\
\pi_\sfset\gets \Pi_\sfset.\algS_2(\crs_\sfset,[\matr{X}]_1,\tau_\sfset)\\
//{[\matr{X}]_1=\GS.\Com_{\crs_\GS}(\vvar{Y};\matr{R})}\\
\text{Return }  (\pi_\GS,\pi_\sfset).
\end{array}
\end{array}$$
\end{\algSize}
\caption{The proof system $\Pi_\rp$ for the language $\Lang_{ck,[0,2^n-1]}$. $\Pi_\sfset$ is the proof system from Sect. \ref{sec:bin-lan-constr}. The matrices $\matr{M},\matr{N}$ are defined as $\matr{M}:=\vecb{u}'_1(2^0,2^1,\ldots,2^{m-1}),\matr{N}:=\vecb{u}'_2$, where $\vecb{u}'_1,\vecb{u}'_2$ are the GS commitment keys from $\crs_\GS$. The equation $\eq$ is defined as $[\vecb{c}]_1-\sum_{i\in[\ell]}{y}_id^{i-1}[\vecb{u}_1]_1=r[\vecb{u}_2]_1$.
\label{fig:rp}}
\end{figure}




%\chapter{Other Contributions} \label{sec:extras}

%    In this chapter we describe a more efficient ring signature and an improved aggregated set-membership proof.

Our ring signature is the first of size $\Theta(\sqrt[3]{n})$ without random oracles. We highlight that this is the first asymptotic improvement since Chandran et al.'s ring signature \cite{ICALP:ChaGroSah07}. To prove security we make use of the \emph{Permutation Pairing assumption}, introduced by Groth and Lu \cite{AC:GroLu07}, which is defined on symmetric groups (or type I). Although it might be possible to find a natural asymmetric counterpart of the assumption, we choose the simple solution and construct our ring signature on symmetric groups. We note that this is the only construction on symmetric groups given in this work.

In the second part of this section we construct a QA-NIZK proof system for the language
\[
\Lang_{ck,\mathsf{set}}^n:=\left\{\begin{array}{l}
([\grkb{\zeta}_1]_1,\ldots,[\grkb{\zeta}_n]_1,S):\exists w_1,\ldots,w_n\in\Z_q \text{ s.t. } S\subset\Z_q\\
\text{ and } \forall i\in [n]\ [\grkb{\zeta}_i]_1=\GS.\Com_{ck}(x_i;w_i)\wedge x_i\in S
\end{array}\right\},
\]
with proof size $\Theta(\log n)$. In Section~\ref{sec:improved-aZKSMP-group-case} we show how to extend these ideas to show membership in the language
\[
\Lang_{ck,S}^n:=\left\{\begin{array}{l}
([\grkb{\zeta}_1]_1,\ldots,[\grkb{\zeta}_n]_1):\exists w_1,\ldots,w_n\in\Z_q \text{ s.t. } S\subset\GG_1 \\
\text{ and } \forall i\in [n]\ [\grkb{\zeta}_i]_1=\GS.\Com_{ck}([x_i]_1;w_i)\wedge [x_i]_1\in S
\end{array}\right\},
\]
with proof-size $\Theta(\log n)$, for any \(S\subset\GG_1\).

The first case is the more general form of a set-membership proof where the set is dynamically chosen. In the second case each instance of the proof system is fixed to a specific set (encoded in the CRS) and is the same notion of the proofs for ``fixed sets'' from Section \ref{sec:bits-applications}. We note that the aggregated set-membership proofs for $S\subset\GG_s$ from Chapter \ref{sec:shuf-rp} are proofs of membership in $\Lang_{ck,S}^n$.

There is a straightforward application of the improved aZKSMP. In the proof of a Shuffle from Section \ref{sec:shuffle}, the size of the proof that $[\matr{F}]\in\Lang_{ck,S}^n$ can be reduced from $2n+\Theta(1)$ to $\Theta(\log n)$ and thus the total proof size is reduced from $4n+o(n)$ to $2n+o(n)$.

In Section \ref{sec:opt-rs} we consider another application of the improved aZKSMP: theoretical $\Theta(\log n)$ ring signatures without random oracles. Although the constants hidden in the asymptotic size of the proof are only polynomially bounded on the security parameter, we interpret this construction as a feasibility result for $\Theta(\log n)$ ring signatures (note that only $\Theta(\sqrt{n})$ ring signatures where known up to this work).




\chapter{Ring Signature of Size $\Theta(\sqrt[3]{n})$ without Random O\-ra\-cles} \label{sec:opt-rs}

    Ring signatures, introduced by Rivest, Shamir and Tauman, \cite{AC:RivShaTau01}, allow to anonymously sign a message on behalf of a \emph{ring} of users $P_1,\ldots,P_n$, only if the signer belongs to the ring. Although there are other cryptographic schemes that provides similar guarantees (e.g.~group signatures \cite{EC:ChaVan91}), ring signatures are not coordinated: each user generates secret/public keys on his own -- i.e.~no central authorities -- and might sign on behalf of a ring without the approval or assistance of the other members.

%the holder of the secret key $sk$ for a signature scheme to \emph{anonymously} sign a message on behalf of a \emph{ring} of users $P_1,\ldots,P_n$, where $vk_i$ is the verification key of party $P_i$, only if $sk$ is the corresponding secret key of $vk_i$ for some $i\in[n]$. We define $R:=\{vk_1,\ldots,vk_n\}$ the set of public keys associated to the ring $P_1,\ldots,P_n$.

The literature on ring signatures is vast, and, while there exist even constant size solutions \cite{EC:DKNS04}, most of them rely on the \emph{random oracle model} (ROM). The ROM idealises the behavior of hash functions and it has been showed to be flawed, in the sense that there are protocols secure in the ROM but insecure using any real hash functions \cite{STOC:CanGolHal98}. Without random oracles all the constructions have signatures of size linear in the size of the ring, being the the sole exception the $\Theta(\sqrt{n})$ ring signature of Chandran et al.~\cite{ICALP:ChaGroSah07} (already discussed, and optimized, in Section~\ref{sec:bits-applications}). 
We remark that no asymptotic improvements to Chandran et al.'s construction have been made since their introduction (only improvements in the constants by R\`afols \cite{TCC:Rafols15} and the improvements from Section~\ref{sec:bits-applications}). We note that although some previous works claim to construct signatures of constant \cite{ACISP:BosDasRan15} or logarithmic \cite{IET:GriSusPla16} size, they are either in a weaker security model or we can identify a flaw in the construction (see Section \ref{sec:rs-flawed}). 
%Moreover, in Section \ref{sec:rs-flawed} we show that two schemes which claim to obtain constant and logarithmic signature size, respectively, are flawed.


In this section we present the first ring signature whose signature size is asymptotically smaller than Chandran et al.'s. Specifically, our ring signature is of size $\Theta(\sqrt[3]{n})$. Interestingly, the security of our construction relies on a security assumption -- the \emph{permutation pairing assumption} -- introduced by Groth and Lu \cite{AC:GroLu07} in an unrelated setting: proofs of correctness of a shuffle. While the assumption is ``non-standard'', in the sense that is not a ``DDH like'' assumption, it is a falsifiable assumption and it was proven to be generically hard by Groth and Lu. For simplicity, we work on symmetric groups ($\GG_1=\GG_2$), but our techniques should be easily extended to asymmetric groups if a natural translation to asymmetric groups of the Groth and Lu's assumption is given.

%Our construction mixes Chandran et al.'s techniques with kernel assumptions in a novel way. Chandran et al.'s pointed out that the core problem in a ring signature is to show that some the opening of some commitment belongs to some set $S\subset\GG_1$ (a set-membership proof in our terminology). Chandran et al.'s solution arranges $S$ as a matrix of size $\sqrt{|S|}\times\sqrt{|S|}$


    
    \section{High Level Description}
        Our scheme follows the ring signature of Chandran et al. Given a {Boneh-Boyen signature scheme} (Section \ref{sec:bbs}), where the secret/verification keys are of the form $(sk,[vk])$ and $sk=vk\in\Z_q$, and given a {one-time signature scheme} (Section \ref{sec:ots}), the signature of the message $m$ for a ring $R=\{[vk_1],\ldots,[vk_n]\}$ is computed as follows:\footnote{We could replace the Boneh-Boyen signature scheme with any structure preserving signature scheme secure under milder assumptions (e.g.~\cite{EPRINT:JutRoy17}). We rather keep it simple and stick to Boneh-Boyen signature which, since the verification key is just one group element, simplifies the notation and reduces the size of the final signature.}
\begin{itemize}
\item[a)] pick a one-time signature key $(sk_\mathsf{ot},vk_\mathsf{ot})$, sign $m$ with the one-time signature, and sign the one-time verification key with $sk$,
\item[b)] commit to the signature of the one-time verification key and to $[vk]$ and show that it is a valid signature key using GS proofs,
\item[c)] show that $[vk]\in R$.
\end{itemize}
The most costly part is c) and our contribution is a proof of size $\Theta(\sqrt[3]{n})$ of c).

Our construction is similar to the proof system for set membership with proof size $\Theta(\sqrt[3]{n})$ from Section~\ref{sec:bits-applications}. However, note that the proof system from Section~\ref{sec:bits-applications} does not suffice for constructing a ring signature because the CRS is fixed to a specific set and thus, the resulting ring signature will be fixed to a specific ring. 

In our scheme the secret/verification keys of party $P$ are $(sk,\vecb{vk})$, where $\vecb{vk}=([vk],[\vecb{a}],\vecb{a}[vk])$, $(sk,[vk])$ are secret/verification keys of the Boneh-Boyen signature scheme, and $\vecb{a}\in\Z_q^2$ is chosen independently for each key from some distribution $\mathcal{Q}$ to be specified later. Suppose that $\vecb{vk}$ is the $\alpha$ th element in the ring $R=\{\vecb{vk}_{(1,1,1)},\ldots,\vecb{vk}_{(m,m,m)}\}$, where $\alpha=(i_\alpha,j_\alpha,k_\alpha):=(i_\alpha-1)m^2+(j_\alpha-1)m+k_\alpha$ for $i_\alpha,j_\alpha,k_\alpha\in[m],m:=\sqrt[3]{n}$. The prover commits to $[\vecb{x}]=[\vecb{s}_\mu]$ and $[\vecb{y}]=[\vecb{s}'_{\mu'}]$, for $\mu=\mu'=(j_\alpha-1)m+k_\alpha$, and shows, using the set-membership proof from Section \ref{sec:bits-applications} (which is the proof from Chandran et al.~adapted to work with vectors), that $[\vecb{x}]\in S$ and that $[\vecb{y}]\in S'$, 
where
\begin{align*}
&S:=\{[\vecb{s}_1],\ldots,[\vecb{s}_{n^{2/3}}]\}:=\left\{\sum_{i\in[m]}[\vecb{a}_{(i,1,1)}],\ldots,\sum_{i\in[m]}[\vecb{a}_{(i,m,m)}]\right\}\text{ and }\\
&S':=\{[\vecb{s}'_1],\ldots,[\vecb{s}'_{n^{2/3}}]\}:=\left\{\sum_{i\in[m]}\vecb{a}_{(i,1,1)}[vk_{(i,1,1)}],\ldots,\sum_{i\in[m]}\vecb{a}_{(i,m,m)}[vk_{(i,m,m)}]\right\}.
\end{align*}

The prover also needs to assure that $\mu=\mu'$, which can be done reutilizing the commitment to $\mu$ (in fact to its $m$-ary representation ) used in the proof that $[\vecb{x}]\in S$ in the proof that $[\vecb{y}]\in S'$. Since both sets are of size $n^{2/3}$, the two set membership proofs are of size $\Theta(\sqrt[3]{n})$.
 
Now that the prover has commited to elements $[\vecb{x}]=\sum_{i\in[m]}[\vecb{a}_{(i,j_\alpha,k_\alpha)}]$ and $[\vecb{y}]=\sum_{i\in[m]}\vecb{a}_{(i,j_\alpha,k_\alpha)}[vk_{(i,j_\alpha,k_\alpha)}]$, it additionally commits to $[\kappa_1]:=[vk_{(1,j_\alpha,k_\alpha)}],\allowbreak\ldots,\allowbreak[\kappa_m]:=[vk_{(m,j_\alpha,k_\alpha)}]$ and $[\vecb{z}_1]:=[\vecb{a}_{(1,j_\alpha,k_\alpha)}],\ldots,[\vecb{z}_m]:=[\vecb{a}_{(m,j_\alpha,k_\alpha)}]$. The prover now gives a proof that
\begin{equation}
\sum_{i\in[m]}[\vecb{z}_i][\kappa_i]=[\vecb{y}][1]. \label{eq:verif1}
\end{equation}

Provided that $\vecb{z}_1,\ldots,\vecb{z}_m$ is a permutation of $\vecb{a}_{(1,j_\alpha,k_\alpha)},\ldots,\vecb{a}_{(m,j_\alpha,k_\alpha)}$,
we can show that if $[\kappa_1],\ldots,[\kappa_m]$ is not a permutation of $[vk_{(1,j_\alpha,k_\alpha)}],\allowbreak\ldots,\allowbreak[vk_{(m,j_\alpha,k_\alpha)}]$, then we can extract an element from the kernel of the matrix $([\vecb{a}_{1,j_\alpha,k_\alpha}]\cdots\allowbreak [\vecb{a}_{m,j_\alpha,k_\alpha}])$. Thereby, provided the corresponding kernel assumption holds, the prover can simply select the $i_\alpha$ th element from $[\kappa_1],\ldots,\allowbreak [\kappa_m]$ which is guaranteed to be an element from the ring.

Therefore, it is only left the to show that $\vecb{z}_1,\ldots,\vecb{z}_m$ is a permutation of $\vecb{a}_{(1,j_\alpha,k_\alpha)},\ldots,\vecb{a}_{(m,j_\alpha,k_\alpha)}$. To do so we will use the following assumption introduced by Groth and Lu \cite{AC:GroLu07}.
\begin{definition}[Permutation Pairing Assumption]
Let $\mathcal{Q}_{m}=\underbrace{\mathcal{Q}\cat\ldots\cat\mathcal{Q}}_{m\text{ times}}$, where concatenation of matrix distributions is defined in the natural way and 
$$\mathcal{Q}: \vecb{a}=\pmatri{x\\x^2},\quad x\gets\Z_q.$$
We say that the $m$-permutation pairing assumption holds relative to $\G_s$ if for any adversary $\advA$
$$
\Pr\left[
\begin{array}{l}
gk\gets\G_s(1^k);\matr{A}\gets\mathcal{Q}_{m};[\matr{Z}]\gets\advA(gk,[\matr{A}]):\\
\mathrm{(i)} \sum_{i\in[m]}[\vecb{z}_i]=\sum_{i\in[m]}[\vecb{a}_i], \mathrm{(ii)}\ \forall i\in[m]\ [z_{2,i}][1]=[z_{1,i}][z_{1,i}],\\
\text{ and }\matr{Z}\text{ is not a permutation of the columns of }\matr{A}
\end{array}
\right],
$$
where $[\matr{Z}]=[(\vecb{z}_1,\ldots,\vecb{z}_m)],[\vecb{A}]=[(\vecb{a}_1,\ldots,\vecb{a}_m)]\in\GG^{2\times m}$,
is negligible in $k$.
\end{definition}

If the prover additionally proves that equations (i) and (ii) are satisfied for $\matr{A}:=(\vecb{a}_{(1,j_\alpha,k_\alpha)},\ldots,\vecb{a}_{(m,j_\alpha,k_\alpha)})$, which can be done with $\Theta(m)$ group elements using Groth-Sahai proofs, the assumption is guaranteeing that the columns of $\matr{Z}$ are a permutation of the columns of $\matr{A}$, for some permutation $\pi\in S_m$. Therefore, equation (\ref{eq:verif1}) implies that
\begin{align*}
\sum_{i\in[m]}[\vecb{z}_i][\kappa_i]&=\sum_{i\in[m]}[\vecb{a}_{(\pi(i),j_\alpha,k_\alpha)}][\kappa_i]=\sum_{i\in[m]}[\vecb{a}_{(i,j_\alpha,k_\alpha)}][\kappa_{\pi^{-1}(i)}]\\
&=\sum_{i\in[m]}[\vecb{a}_{(i,j_\alpha,k_\alpha)}][vk_{(i,j_\alpha,k_\alpha)}].
\end{align*}
Then $\kappa_1,\ldots,\kappa_m$ is a permutation of $\vecb{a}_{(1,j_\alpha,k_\alpha)},\ldots,\vecb{a}_{(m,j_\alpha,k_\alpha)}$ (the same defined by $\vecb{z}_1,\ldots,\vecb{z}_m$), unless $(\kappa_{\pi^{-1}(1)}-{vk_{(1,j_\alpha,k_\alpha)}),\ldots,\kappa_{\pi^{-1}(m)}-vk_{(m,j_\alpha,k_\alpha)})})^\top$ is in the kernel of $\matr{A}$. Groth and Lu showed the hardness of finding an element from $\ker(\matr{A})$, when $\matr{A}$ is sampled from $\mathcal{Q}_m$, in the generic group model. They called this assumption the \emph{simultaneous pairing assumption} and it corresponds to the $\mathcal{Q}_m^\top\mbox{-}\kermdh$ assumption.


\paragraph{Remark.}
A natural question is if this technique can be applied once again. That is, to compute a $\Theta(\sqrt[4]{n})$  proof, compute commitments to an element from $S=\{\sum_{i\in[m]}\vecb{a}_{(i,1,1,1)}[vk_{(i,1,1,1)}],\ldots,\sum_{i\in[m]}\vecb{a}_{(i,m,m,m)}[vk_{(i,m,m,m)}]\}$ and $S'=\allowbreak\{\sum_{i\in[m]}[\vecb{a}_{(i,1,1,1)}],\ldots,\sum_{i\in[m]}[\vecb{a}_{(i,m,m,m)}]\}$, and then prove that they belong to the respective sets with the proof of size $\Theta(\sqrt[3]{n})$. Since $|S|=|S'|=n^{3/4}$, proof will be of size $\Theta(\sqrt[3]{n^{3/4}})=\Theta(\sqrt[4]{n})$. However, this is not possible since the $\Theta(\sqrt[3]{n})$ proof is not a set membership proof for arbitrary sets, but only for sets where each element is of the form $([vk],\vecb{a}[vk],[\vecb{a}])$.



    \section{Preliminaries}
        
        \subsection{Groth-Sahai Proofs in the $\lin{2}$ Instantiation}
        
            In our ring signature we use as primitive Groth-Sahai proofs, which we have only instantiated in the SXDH setting. Since the SXDH (and DDH) is false in symmetric groups, in order to use Groth-Sahai proofs in symmetric groups one usually instantiates them using the $\lin{2}$ assumption. Following \cite{EC:GroSah08}, in symmetric groups and using the $\lin{2}$ assumption, GS commitments are vectors in $\GG^3$ of the form
$$
\GS.\Com_{ck}([x];\vecb{r})=\pmatri{{[0]}\\{[0]}\\{[x]}}+r_1[\vecb{u}_1]+{r}_2[\vecb{u}_2]+r_3[\vecb{u}_3]
$$
where $ck:=([\vecb{u}_1]||[\vecb{u}_2][\vecb{u}_3])$, $(\vecb{u}_2||\vecb{u}_3)\gets\distlin_2$ and $\vecb{u}_1:=w_1\vecb{u}_2+w_2\vecb{u}_3$ in the perfectly binding setting, and $\vecb{u}_1:=w_1\vecb{u}_2+w_2\vecb{u}_3-\vecb{e}_3$ in the perfectly hiding setting, for $w_1,w_2\gets\Z_q$. Security of GS commitments follows from the hardness of the $\lin{2}$ assumption in symmetric groups.

As consequence of the enlargement of commitments, proofs are also larger (for example, 9 group elements for pairing product equations). The form of the proofs is similar  to the asymmetric case (see Section \ref{sec:gs-proofs-scheme}), but we do not give the full detail since it does not help for understanding our ring signature nor for proving its security. 




        \subsection{Flawed Constructions}\label{sec:rs-flawed}
    
            In \cite{ACISP:BosDasRan15}, Bose et al.~claim to construct a constant-size ring signature in the standard model. However, they construct a weak Ring Signature where: a) the public keys are generated all at once in a correlated way; b) the set of parties which are able to participate in a ring is fixed as well as the maximum ring size; and c) the key size is linear in the maximum ring size. In the work of Chandran et al.~and also in our setting: a) the key generation is independently run by the user using only the CRS as input; b) any party can be member of the ring as long as she has a verification key, and the maximum ring size is unbounded; and c) the key size is constant. These stronger requirements are in line with the original spirit of \emph{non-coordination} of  Rivest et al.~\cite{AC:RivShaTau01}.

In \cite{IET:GriSusPla16}, Gritti et al.~claim to construct a logarithmic ring signature in the standard model. However, their construction is completely flawed as explained below. 
In page 12 of \cite{IET:GriSusPla16}, Gritti et al.~define $v_{b_i} := v_{b_1\cdots b_i *}$, where $b_1\cdots b_i *$ is the set of all bit-strings of size $d:=\log n$ whose prefix is $b_1\cdots b_i$. From this, one have to conclude that $v_{b_i}$ is a set (or vector) of group elements of size $2^{d-i}$.
In the same page they define the commitment $D_{b_i} := v_{b_i}h^{s_{b_i}}$, for random $s_{b_i}\in\Z_q$, which, according to the previous observation, is the multiplication of a set (or vector) of group elements with a group element. Given that length reducing group to group commitments are known to not exist \cite{EC:AbeHarOhk12}, its representation require at least $2^{d-i}$ group elements . Since commitments $D_{b_0},\ldots,D_{b_d}$ are part of the signature, the actual signature size is $\Theta(2^d)=\Theta(n)$, rather than  $\Theta(d)=\Theta(\log n)$ as claimed by Gritti et al.\footnote{We used multiplicative notation for the group operations to keep the expressions as they appear in \cite{IET:GriSusPla16}.}



        \subsection{Definition}
    
            We follow the definitions from \cite{ICALP:ChaGroSah07} described below.

\begin{definition}[Ring Signature]
A ring signature scheme consists of a quadruple of
PPT algorithms $(\mathsf{CRSGen}, \KG, \mathsf{Sign}, \mathsf{Verify})$ that respectively, generate the common
reference string, generate keys for a user, sign a message, and verify the signature of a
message. More formally:
\begin{itemize}
\item $\mathsf{CRSGen}(gk)$, where $gk$ is the group key, outputs the common reference
string $\rho$.
\item $\KG(\rho)$ is run by the user. It outputs a public verification key $vk$ and a private
signing key $sk$.
\item $\mathsf{Sign}_{\rho,sk}(m, R)$ outputs a signature $\sigma$ on the message $m$ with respect to the ring
$R = \{vk_1,\ldots,vk_n\}$. We require that $(vk, sk)$ is a valid key-pair output by $\KG$
and that $vk \in R$.
\item $\mathsf{Verify}_{\rho,R}(m, \sigma)$ verifies a purported signature $\sigma$ on a message $m$ with respect to
the ring of public keys $R$.
\end{itemize}
The quadruple $(\mathsf{CRSGen}, \KG, \mathsf{Sign}, \mathsf{Verify})$ is a ring signature with perfect
anonymity if it has perfect correctness, computational unforgeability and perfect
anonymity as defined below.
\end{definition}

\begin{definition}[Perfect Correctness]
We require that a user can sign any message on behalf of a ring where she is a member. A ring signature $(\mathsf{CRSGen},\allowbreak \KG, \mathsf{Sign}, \mathsf{Verify})$
has perfect correctness if for all adversaries $\advA$ we have:
$$
\Pr\left[\begin{array}{l}
gk\gets\G(1^\lambda);\rho\gets\mathsf{CRSGen}(gk);(vk,sk)\gets\KG(\rho);\\
(m,R)\gets\advA(\rho,vk,sk);\sigma\gets\mathsf{Sign}_{\rho,sk}(m;R):\\
\mathsf{Verify}_{\rho,R}(m,\sigma)\text{ or }vk\notin R
\end{array}\right]=1
$$
\end{definition}

\begin{definition}[Computational Unforgeability]
A ring signature scheme $(\mathsf{CRSGen}, \KG, \mathsf{Sign}, \mathsf{Verify})$
is unforgeable if it is infeasible to forge a ring
signature on a message without controlling one of the members in the ring. Formally, it
is unforgeable when for any non-uniform polynomial
time adversaries $\advA$ we have that
$$
\Pr\left[\begin{array}{l}
gk\gets\G(1^\lambda);\rho\gets\mathsf{CRSGen}(gk);(m,R,\sigma)\gets\advA^{\mathsf{VKGen},\mathsf{Sign},\mathsf{Corrupt}}(\rho):\\
\mathsf{Verify}_{\rho,R}(m,\sigma)=1
\end{array}\right]
$$
is negligible in th security parameter, where

\begin{itemize}
\item $\mathsf{VKGen}$ on query number $i$ selects randomness $w_i$, computes $(vk_i,sk_i):= \KG(\rho; w_i)$
and returns $vk_i$.
\item $\mathsf{Sign}(\alpha, m, R)$ returns $\sigma \gets \mathsf{Sign}_{\rho,sk_\alpha}(m, R)$, provided $(vk_\alpha, sk_\alpha)$ has been generated
by $\mathsf{VKGen}$ and $vk_\alpha\in R$.
\item $\mathsf{Corrupt}(i)$ returns $w_i$ (from which $sk_i$ can be computed) provided $(vk_i, sk_i)$ has
been generated by $\mathsf{VKGen}$.
\item $\advA$ outputs $(m, R, \sigma)$ such that $\mathsf{Sign}$ has not been queried with $(*, m, R)$ and $R$
only contains keys $vk_i$ generated by $\mathsf{VKGen}$ where $i$ has not been corrupted.
\end{itemize}
\end{definition}

\begin{definition}[Perfect Anonymity]
A ring signature scheme
$(\mathsf{CRSGen},\allowbreak \KG,\allowbreak \mathsf{Sign}, \mathsf{Verify})$ has perfect anonymity, if a signature on a message
$m$ under a ring $R$ and key $vk_{i_0}$
looks exactly the same as a signature on the
message $m$ under the ring $R$ and key $vk_{i_1}$, where $vk_{i_0},vk_{i_1}\in R$. This means that the signer's key is hidden
among all the honestly generated keys in the ring. Formally, we require that for any unbounded
adversary $\advA$:
\begin{align*}
&\Pr\left[\begin{array}{l}
gk\gets\G_a(1^\lambda);\rho\gets\mathsf{CRSGen}(gk);\\
(m,i_0,i_1,R)\gets\advA^{\KG(\rho)}(\rho);\sigma\gets\mathsf{Sign}_{\rho,sk_{i_0}}(m,R):\\
\advA(\sigma)=1
\end{array}\right]
=\\
&\Pr\left[\begin{array}{l}
gk\gets\G_a(1^\lambda);\rho\gets\mathsf{CRSGen}(gk);\\
(m,i_0,i_1,R)\gets\advA^{\KG(\rho)}(\rho);\sigma\gets\mathsf{Sign}_{\rho,sk_{i_1}}(m,R):\\
\advA(\sigma)=1
\end{array}\right]
\end{align*}
where $\advA$ chooses $i_0, i_1$ such that $(vk_{i_0}, sk_{i_0}),(vk_{i_1}, sk_{i_1})$ have been generated by the
oracle $\KG(\rho)$.
\end{definition}



        \subsection{Boneh-Boyen Signatures} \label{sec:bbs}
    
            Boneh and Boyen described a short signature -- each signature consists of only one group element -- which is UF-CMA without random oracles \cite{EC:BonBoy04a}. Interestingly, the verification of the validity of any signature-message pair can be written as a set of pairing product equations. Thereby, using Groth-Sahai proofs one can show the possession of a valid signature without revealing the actual signature (as done in Chandran et al.'s ring signature and our ring signature).

The Boneh-Boyen signature is proven UF-CMA secure under the $m$-\emph{Strong Diffie-Hellman} assumption, which is described below.

\begin{definition}[$m\mbox{-}SDH$ assumption]
For any adversary $\advA$
$$
\Pr\left[gk\gets\G_s(1^\lambda),x\gets\Z_q:\advA(gk,[x],[x^2],\ldots,[x^m])=(c,\left[\frac{1}{x+c}\right])\right]
$$
is negligible in $\lambda$.
\end{definition}

The Boneh-Boyen signature scheme is described below.

\begin{description}
\item[$\mathsf{BB}.\KG$:] Given a group key $gk$, pick $vk\gets\Z_q$. The secret/public key pair is defined as $(sk,[vk]):=(vk,[vk])$.
\item[$\mathsf{BB}.\Sign$:] Given a secret key $sk\in\Z_q$ and a message $m\in\Z_q$, output the signature $[\sigma]:=\left[\frac{1}{sk+m}\right]$. In the unlikely case that $sk+m=0$ we let $[\sigma]:=[0]$.
\item[$\mathsf{BB}.\Ver$:] On input the verification key $[vk]$, a message $m\in\Z_q$, and a signature $[\sigma]$, verify that $[m+vk][\sigma]=[1]_T$.
\end{description} 



    \section{Our Construction}
    
        \begin{description}
\item[$\mathsf{CRSGen}(gk)$:] Pick a perfectly hiding CRS for the Groth-Sahai proof system $\crs_\GS$, and a CRS for the proof of the $\Theta(\sqrt{n})$ proof of membership in a set $\crs_\sfset$ (Section \ref{sec:bits-applications}), and output $\rho:=(gk,\crs_\GS,\crs_\sfset).$
\item[$\KG(\rho)$:] Pick $\vecb{a}\gets\mathcal{Q}$ and $(sk,[vk])\gets\mathsf{BB}.\KG(gk)$. The secret key is $sk$ and the verification key is $\vecb{vk}:=([vk],[\vecb{a}],\vecb{a}[vk])$.
\item[$\mathsf{Sign}_{\rho,sk}(m,R)$:]
\begin{enumerate}
\item Compute $(sk_\mathsf{ot},vk_\mathsf{ot})\gets\mathsf{OT}.\KG(gk)$ and $\sigma_\mathsf{ot}\gets\mathsf{OT}.\mathsf{Sign}_{sk_\mathsf{ot}}(m,R)$.
\item Compute $[\vecb{c}]:=\GS.\Com_{ck}([vk];r)$, $r\gets\Z_q$, $[\sigma]\gets\mathsf{BB}.\mathsf{Sign}_{sk}(vk_\mathsf{ot})$, $[\vecb{d}]:=\GS.\Com_{ck}([\sigma];s)$, and a GS proof $\pi_\GS$ that $\mathsf{BB}.\mathsf{Ver}_{[vk]}([\sigma],[vk_\mathsf{ot}])=1$ (which can be expressed as a set of pairing product equations).
\item Parse $R$ as $\{\vecb{vk}_{(1,1,1)},\ldots,\vecb{vk}_{(m,m,m)}\}$, where $m:=\sqrt[3]{|R|}$, and let $\alpha=(i_\alpha,j_\alpha,k_\alpha)$ the index of $\vecb{vk}$ in $R$. Define the sets $S_1=\{\sum_{i\in[m]}[\vecb{a}_{(i,1,1)}],\allowbreak\ldots,\sum_{i\in[m]}[\vecb{a}_{(i,m,m)}]\}$ and $S_2=\{\sum_{i\in[m]}\vecb{a}_{(i,1,1)}[vk_{(i,1,1)}],\allowbreak\ldots,\sum_{i\in[m]}\vecb{a}_{(i,m,m)}[vk_{(i,m,m)}]\}$.
\item Compute GS commitments to $[\vecb{x}]:=\sum_{i\in[m]}[\vecb{a}_{(i,j_\alpha,k_\alpha)}]$ and $[\vecb{y}]=\sum_{i\in[m]}\vecb{a}_{(i,j_\alpha,k_\alpha)}[vk_{(i,j_\alpha,k_\alpha)}]$, and compute proofs $\pi_1$ and $\pi_2$ that they belong to $S_1$ and $S_2$, respectively (recall that it is also proven that they appear in the same positions reutilising the commitments to $b_1,\ldots,b_{\sqrt{n}}$ and $b'_1,\ldots,b'_{\sqrt{n}}$).
\item Compute GS commitments to $[\kappa_1]:=[vk_{(1,j_\alpha,k_\alpha)}],\ldots,[\kappa_m]:=[vk_{(m,j_\alpha,k_\alpha)}]$ and $[\vecb{z}_1]:=[\vecb{a}_{(1,j_\alpha,k_\alpha)}],\ldots,[\vecb{z}_m]:=[\vecb{a}_{(m,j_\alpha,k_\alpha)}]$, and GS proof $\pi_\kappa$ that $\sum_{i\in[m]}[\kappa_i][\vecb{z}_i]=[\vecb{y}][1]$ and a GS proof $\pi_{z}$ that $\sum_{i\in[m]}[\vecb{z}_i]=[\vecb{x}]$ and $[z_{2,i}][1]=[z_{1,i}][z_{1,i}]$ for each $i\in[m]$.
\item Compute a proof $\pi_3$ that $[vk]$ belongs to $S_3=\{[\kappa_1],\ldots,[\kappa_m]\}$.
\item Return the signature $\grkb{\sigma}:=(vk_\mathsf{ot},\sigma_\mathsf{ot},[\vecb{c}],[\vecb{d}],\pi_1,\pi_2,\pi_3,\pi_\kappa,\pi_z)$. (GS proofs include commitments to variables).
\end{enumerate}
\item[$\mathsf{Verify}_{\rho,R}(m,\grkb{\sigma})$:] Verify the validity of the one-time signature and of all the proofs. Return 0 if any of these checks fails and 1 otherwise.
\end{description}

\begin{theorem}
The scheme presented in this section is a ring signature scheme
with perfect correctness, perfect anonymity and computational unforgeability under the
$m$-permutation pairing assumption, the $\mathcal{Q}_m^\top\mbox{-}\kermdh$ assumption, the $\lin{2}$ assumption, and the assumption
that the one-time signature and the Boneh-Boyen signature are unforgeable.
\end{theorem}
\begin{proof}
Perfect correctness follows directly from the definitions. Perfect anonymity follows from the fact that the perfectly hiding Groth-Sahai CRS defines perfectly hiding and perfect zero-knowledge proofs, information theoretically hiding any information about $\vecb{vk}$.

Computational unforgeability is proven as follows. The $\lin{2}$ assumption implies that $\crs_\GS$ can be changed to a perfectly binding CRS, at the cost of a negligible security loss. For the sake of contradiction, suppose that $\vecb{vk}\notin R$. Soundness of Groth-Sahai proofs and of the $\Theta(\sqrt{n})$ proof of set membership, implies that there exist $j_\alpha,k_\alpha\in[m]$ such that
\begin{align}
&\sum_{i\in[m]}[\kappa_i][\vecb{z}_i]=\sum_{i\in[m]}\vecb{a}_{(i,j_\alpha,k_\alpha)}[vk_{(i,j_\alpha,k_\alpha)}]\text{ and }\label{eq-rs-1}\\
&\sum_{i\in[m]}[\vecb{z}_i]=\sum_{i\in[m]}[\vecb{a}_{(i,j_\alpha,k_\alpha)}], \label{eq-rs-2}\\
&[z_{i,2}][1]=[z_{i,1}][z_{i,1}] \text{ for all }i\in[m],\label{eq-rs-3}
\end{align}
and that $[vk]=[\kappa_{i^*}]$, for some $i^*\in[m]$. Equations (\ref{eq-rs-2}) and (\ref{eq-rs-3}) imply that $(\vecb{z}_1,\ldots,\vecb{z}_m)$ is a permutation of $(\vecb{a}_{(1,j_\alpha,k_\alpha)}\ldots,\vecb{a}_{(m,j_\alpha,k_\alpha)})$ unless one can break the $m$-permutation pairing assumption. Therefore, equation (\ref{eq-rs-1}) implies that there is a permutation $\pi$ such that
\begin{align}
\sum_{i\in[m]}(\kappa_i-vk_{(\pi(i),j_\alpha,k_\alpha)})\vecb{a}_i=0.
\end{align}
Assuming $[\kappa_{i^*}]\notin R$, $([\kappa_1]-[vk_{(\pi(1),j_\alpha,k_\alpha)}],\ldots,[\kappa_m]-[vk_{(\pi(m),j_\alpha,k_\alpha)}])\neq [\vecb{0}]$ is a solution to the $\mathcal{Q}_m^\top\mbox{-}\kermdh$ problem. We conclude that $\vecb{vk}\in R$.

Finally, since $[\vecb{c}]$ is a commitment to a valid signature of $vk_\mathsf{ot}$ under $[vk]$ and given that $vk_\mathsf{ot}$ has not been previously used, $[\sigma]$ is a forgery of the Boneh-Boyen signature scheme. By the unforgeability of the Boneh-Boyen signature scheme, this happens only with negligible probability.
\end{proof}



\chapter{Improved aZKSMP} \label{improved-aZKSMP} 

    \newcommand{\setsize}{t}

In this chapter we construct a QA-NIZK proof system for the language
\[
\Lang_{ck,\mathsf{set}}^n:=\left\{\begin{array}{l}
([\grkb{\zeta}_1]_1,\ldots,[\grkb{\zeta}_n]_1,S):\exists w_1,\ldots,w_n\in\Z_q \text{ s.t. } S\subset\Z_q\\
\text{ and } \forall i\in [n]\ [\grkb{\zeta}_i]_1=\GS.\Com_{ck}(x_i;w_i)\wedge x_i\in S
\end{array}\right\},
\]
with proof size $\Theta(\log |S|)$. In Section~\ref{sec:improved-aZKSMP-group-case} we show how to extend these ideas to show membership in the language
\[
\Lang_{ck,S}^n:=\left\{\begin{array}{l}
([\grkb{\zeta}_1]_1,\ldots,[\grkb{\zeta}_n]_1):\exists w_1,\ldots,w_n\in\Z_q \text{ s.t. } S\subset\GG_1 \\
\text{ and } \forall i\in [n]\ [\grkb{\zeta}_i]_1=\GS.\Com_{ck}([x_i]_1;w_i)\wedge [x_i]_1\in S
\end{array}\right\},
\]
with proof size $\Theta(\log |S|)$, for any \(S\subset\GG_1\).

The first case is the more general form of a set-membership proof where the set is dynamically chosen. In the second case each instance of the proof system is fixed to a specific set (encoded in the CRS) and is the same notion of the proofs for ``fixed sets'' from Section \ref{sec:bits-applications}. We note that the aggregated set-membership proofs for $S\subset\GG_s$ from Chapter \ref{sec:shuf-rp} are proofs of membership in $\Lang_{ck,S}^n$.

In Section \ref{sec:improved-aZKSMP-intuition}, we start with an intuitive description for the case $S\subset\Z_q$ without aggregation. We note that even in this simpler case, to the best of our knowledge, the shortest non-interactive proof, under falsifiable assumptions and without assuming anything about $S$,\footnote{If $S=[a,b]\subset\Z_q$ and $a<b$ we can use range proofs.} that exists in the literature is the one of Chandran et al.~of size $\Theta(\sqrt{|S|})$.
Our approach is to commit to the binary representation $(b_1,\ldots,b_{\log t})\in\bits^{\log t}$ of the index of the purported $x\in S$, for $S=\{s_1,\ldots,s_t\}$ and where $b_1$ is the least significant bit, to select the the leaves under the paths $(b_{\log t}),(b_{\log t},b_{\log t-1}),\ldots,(b_{\log t},\ldots, b_1)$ in the binary tree whose leaves are (from left to right) $s_1,\ldots,s_t$. In order to keep a logarithmic proof, we commit to the selected leaves using MP commitments from Section \ref{sec:ext-mp} and show, for each $\ell\in[\log t]$, that the leaves under the path $(b_m,\cdots, b_{\ell})$ are equal the leftmost or rigmost, depending of $b_\ell$, leaves under the path $(b_{\log t},\cdots, b_{\ell-1})$. We use these ideas together with a clever usage of QA-NIZK proofs of membership in linear subspaces, Groth-Sahai proofs, and the proof systems from Chapters \ref{sec:agg-asym} and \ref{sec:bits}.
 %In the case of fixed sets $S\subset\Z_q$, Kohlweiss et al.~constructed a non-interactive proof of size $\Theta(1)$, using Boneh-Boyen signatures \cite{PAIRING:RiaKohPre09}.

In Section \ref{sec:log-set-memb-Z} we give a full description of the non-aggregated case and then we show how to extend this result to the case $S\subset\GG_s$. We use the ideas from Section \ref{sec:improved-aZKSMP-intuition} and aggregate many instances using similar techniques to those from Chapter \ref{sec:bits}. We note that, to the best of our knowledge, there is no aggregated proof in the literature (i.e.~all proofs are of size $\Omega(n)$) with the sole exception of our proof from Section \ref{sec:shuf-rp} which is of size $\Theta(|S|)$.  
Our proof bears some similarities with the work of Groth and Kohlweiss \cite{EC:GroKoh15} -- both allow to construct proofs of membership in a set of logarithmic size using the binary encoding of the element index -- but they are in general incomparable. Indeed, Groth and Kohlweiss's construction is on a different setting (interactive, without pairings) and does not support aggregation of many proofs.

There is a straightforward application of the improved aZKSMP. In the proof of a Shuffle from Section \ref{sec:shuffle}, the size of the proof that $[\matr{F}]\in\Lang_{ck,S}^n$ can be reduced from $2n+\Theta(1)$ to $\Theta(\log n)$ and thus the total proof size is reduced from $4n+o(n)$ to $2n+o(n)$.
%Furhter, in Section \ref{sec:log-ring-signature} we consider another application of the improved aZKSMP: theoretical $\Theta(\log n)$ ring signatures without random oracles. Although the constants hidden in the asymptotic size of the proof are only polynomially bounded in the security parameter, we interpret this construction as a feasibility result for $\Theta(\log n)$ ring signatures (note that only $\Theta(\sqrt{n})$ ring signatures where known up to this work).



        
\section{Intuition} \label{sec:improved-aZKSMP-intuition}
    
    For simplicity, we will restrict to the case \(S=\{s_1,\ldots,s_\setsize \}\subset\Z_q\) without aggregation, that is, there is a single commitment \([\vecb{c}]_1=\GS.\Com_{ck_\GS}(x;r)\) and we want to show that \(x=s_\alpha\), for some \(\alpha\in[\setsize ]\). In Section \ref{sec:log-set-memb-Z} we will show how to aggregate many proofs.

The (non-aggregated) proof from Section~\ref{sec:aZKSMP} essentially codifies the position \(\alpha\) as a weight 1 binary vector \(\vecb{b}\) of size \(\setsize \) such that \(x=\sum_{i\in[\setsize ]}b_is_i\) and
\(b_i=1\) if \(i=\alpha\) and \(0\) if not.\footnote{The case \(S\subset\Z_q\) is not really discussed in Section~\ref{sec:aZKSMP}, but it is straightforward that the same techniques from the case \(S\subset\GG_s\) apply.} A step further in efficiency was given by Chandran et al.~\cite{ICALP:ChaGroSah07} (already discussed in Section~\ref{sec:bits-applications}). There the position \(\alpha\) is codified as two weight 1 binary vectors \(\vecb{b}\) and \(\vecb{b}'\) of size \(\sqrt{\setsize }\) such that
\[
\begin{pmatrix}
x_1\\\vdots\\x_{\sqrt{\setsize }}
\end{pmatrix}
=\sum_{i=1}^{\sqrt{\setsize }}b_i
\begin{pmatrix}
s_{(i-1)\sqrt{\setsize }+1}\\\vdots\\s_{(i-1)\sqrt{\setsize }+\sqrt{\setsize }}
\end{pmatrix},
\text{ } x=\sum_{i=1}^{\sqrt{\setsize }}b'_ix_i,\]
and \(b_i=1\) iff \(i=i_\alpha\) and \(b'_j=1\) iff \(j=j_\alpha\), where \(\alpha=(i_\alpha-1)\sqrt{\setsize }+j_\alpha\). Since \(\sqrt{\setsize }\) new variables are added (variables \(x_1,\ldots,x_{\sqrt{\setsize }}\)), the proof must contain \(\sqrt{\setsize }\) new commitments to these variables. However, this does not not affect the asymptotic size of the proof, which is \(\Theta(\sqrt{\setsize })\) anyway.


Let $m:=\log \setsize$.\footnote{ W.l.o.g.~we assume that \(\log \setsize \in\mathbb{N}\), because we can always prove membership in the (multi-)set \(S'=S\biguplus_{i=1}^{2^{\lceil \log \setsize  \rceil}-\setsize}\{s_{\setsize} \}\) and it holds that $|S'|=2^{\lceil \log t \rceil}$ and that $x\in S \Longleftrightarrow x\in S'$.} The natural next step is to codify \(\alpha\) as \(m\) weight 1 binary vectors of size 2 (note that a weight 1 binary vector of size 2 can be always written as \((1-b,b)\), \(b\in\bits\) ) such that
\begin{align}
&\begin{pmatrix}
x_{\ell,1}\\\vdots\\x_{\ell,{2^{\ell-1}}}
\end{pmatrix}
=
(1-b_\ell)
\begin{pmatrix}
x_{\ell+1,1}\\\vdots\\x_{\ell+1,{2^{\ell-1}}}
\end{pmatrix}
+
b_\ell
\begin{pmatrix}
x_{\ell+1, 2^{\ell-1}+1}\\\vdots\\x_{\ell+1,2^{\ell}}
\end{pmatrix}
\text{ if } \ell \in[m],\label{eq-log-2}\\
&
x= x_{1,1}\label{eq-log-3},
\end{align}
where \(x_{m+1,i}:=s_i\), \(i\in[t]\), and \(\alpha=\sum_{i=1}^{m}b_i2^{i-1}+1\). Note that we have added the additional variables \(x_{\ell,i}\), \(\ell\in[m]\) and \(i\in[2^\ell]\).

Consider the binary tree whose leafs are $x_{m+1,1}=s_1,\ldots,x_{m+1,t}=s_{t}$, where the leftmost leaf is $s_1$ and the rightmost leaf is $s_{2^m}=s_t$. Intuitively, equation (\ref{eq-log-2}) for $\ell=m$ says that variables $x_{m,1},\ldots, x_{m,2^{m-1}}$ are the leafs of the subtree under the path $(b_m)$. For example, if $b_m=1$, the variables $x_{m,1},\ldots,x_{m,2^{m-1}}$ are equal to $s_{2^{m-1}+1},\ldots,s_{t}$, which are the leafs of the subtree under the path (1) as depicted below.
\begin{center}
\begin{tikzpicture}
[n/.style={draw=none},
 every node/.append style={inner ysep=+0pt,outer ysep=+0pt,minimum size=+0pt}]
%\hspace{-3.9cm}
\Tree 
    [.{}
        \edge node[auto=right] {\small 0};
        \qroof{\parbox{\widthof{$s_{2^{m-1}+1}\cdots s_{t}$}}{\vspace*{.1cm}$s_1$\hfill$\cdots$\hfill$s_{2^{m-1}}$}}.{}
        \edge node[auto=left] {\small 1};
        \qroof{\parbox{\widthof{$s_{2^{m-1}+1}\cdots s_{t}$}}{\vspace*{.1cm}$s_{2^{m-1}+1}\cdots s_{t}$}}.{}
    ]
\end{tikzpicture}
\end{center}
Similarly, equation (\ref{eq-log-2}) for $\ell=m-1$ says that variables $x_{m-1,1},\ldots,x_{m-1,2^{m-2}}$ are the leafs of the subtree under the path $(b_m,b_{m-1})$. For example, if $(b_m,b_{m-1})=(1,0)$, the variables $x_{m-1,1},\ldots,x_{m-1,2^{m-2}}$ are equal to $x_{m,1}=s_{2^{m-1}+1},\ldots,\allowbreak x_{m,2^{m-2}}=s_{2^{m-1}+2^{m-2}}$, which are the leafs of the subtree under the path (1,0) as depicted below.
\begin{center}
\begin{tiny}\begin{tikzpicture}
[n/.style={draw=none},
 every node/.append style={inner ysep=+0pt,outer ysep=+0pt,minimum size=+0pt}]
\Tree 
    [.{}
        \edge node[auto=right] {\small 0};
        [.{}
            \edge node[auto=right] {\small 0};
            \qroof{\parbox{\widthof{$s_{2^{m-1}+2^{m-2}+1}\cdots s_{2^{m}}$}}{\vspace*{.1cm}$s_1$\hfill$\cdots$\hfill$s_{2^{m-2}}$}}.{}
            \edge node[auto=left] {\small 1};
            \qroof{\parbox{\widthof{$s_{2^{m-1}+2^{m-2}+1}\cdots s_{2^{m}}$}}{\vspace*{.1cm}$s_{2^{m-2}+1}$\hfill$\cdots$\hfill$s_{2^{m-1}}$}}.{}
        ]
        \edge node[auto=left] {\small 1};
        [.{}
            \edge node[auto=right] {\small 0};
            \qroof{\parbox{\widthof{$s_{2^{m-1}+2^{m-2}+1}\cdots s_{2^{m}}$}}{\vspace*{.1cm}$s_{2^{m-1}+1}\cdots s_{2^{m-1}+2^{m-2}}$}}.{}
            \edge node[auto=left] {\small 1}; 
            \qroof{\parbox{\widthof{$s_{2^{m-1}+2^{m-2}+1}\cdots s_{2^{m}}$}}{\vspace*{.1cm}$s_{2^{m-1}+2^{m-2}+1}\cdots s_{2^{m}}$}}.{}
        ]
    ]
\end{tikzpicture}\end{tiny}
\end{center}

In general, the variables $x_{\ell,1},\ldots,x_{\ell,2^{\ell-1}}$ are equal to the leafs $s_{\sfleft},\ldots,\allowbreak s_{\sfright}$ under the path $(b_m,\ldots, b_\ell)$, where $\sfleft=\sum_{i=\ell}^m b_i 2^{i-1}+1$ and $\sfright=\sfleft+2^{\ell-1}-1$. 
Therefore, for $\ell=1$ equation (\ref{eq-log-2}) says that the variable $x_{1,1}$ is equal to the leaf $s_{\sfleft}=s_{\sfright}=s_\alpha$, since $\sfleft=\sfright=\sum_{i=1}^m b_i2^{i-1}+1=\alpha$, which is the unique leaf (and the unique node) in the subtree under the path $(b_m,\ldots, b_1)$.

Similarly as in Chandran et al.'s proof, for each new variable a new commitment must be added to the proof. But, in contrast with Chandran et al.'s proof, in this case the additional commitments do increase the asymptotic size of the proof. Indeed, the total number of new variables is \(2^{m-1}+2^{m-2}+\ldots+1=2^m-1=t-1\), and thus $t-1$ new commitments must be added.

One can reduce the total size of the commitments using the length reducing Multi-Pedersen commitments from Section~\ref{sec:ext-mp}. However, this must be done carefully in order to be able to express equation (\ref{eq-log-2}) with a Groth-Sahai proof of an equation that involves the MP commitments and the variables $b_1,\ldots,b_m$. For example, if one computes a single commitment to all variables $\MP.\Com_{ck}((x_{1,1},\ldots,x_{m,2^{m-1}})^\top;r)$ it is not clear how to obtain $b_\ell\MP.\Com_{ck}(\allowbreak (x_{\ell+1,1},\ldots,x_{\ell+1,2^\ell})^\top\allowbreak ;r_\ell)=\MP.\Com_{ck}(b_\ell(x_{\ell+1,1},\ldots,x_{\ell+1,2^\ell})^\top;b_\ell r_\ell)$ as required in equation \ref{eq-log-2}. Our solution is to compute a single MP commitment to each vector that appears in equation (\ref{eq-log-2}) in order to show with Groth-Sahai proofs that
\begin{small}
\begin{align*}
\MP.\Com_{ck_\ell}\left(
                    \pmatri{x_{\ell,1}\\\vdots\\x_{\ell,2^{\ell-1}}};r_\ell\right)
=\ &(1-b_\ell)\MP.\Com_{ck_\ell}\left(\pmatri{x_{\ell+1,1}\\\vdots\\x_{\ell+1,2^{\ell-1}}};r_{\ell,1}\right)+\\
& b_\ell\MP.\Com_{ck_\ell}\left(\pmatri{x_{\ell+1,2^{\ell-1}+1}\\\vdots\\x_{\ell+1,2^\ell}};r_{\ell,2}\right)+\\
& \MP.\Com_{ck_\ell}(\vecb{0};y_\ell),
\end{align*}\end{small}
for some $y_\ell\in\Z_q$. In this way, we only need $3m=3\log t$ additional commitments. The reason for using different commitment keys will be clear when we explain soundness.

Concretely, the prover computes
\begin{align*}
&[\vecb{c}_\ell]_1=\MP.\Com_{ck_m}((x_{\ell,1},\ldots,x_{1,2^{\ell-1}})^\top;r_\ell),\\
\end{align*}
for random $r_\ell\in\Z_q$ and $\ell\in[m]$, and 
\begin{align*}
&[\vecb{c}_{\ell,1}]_1=\MP.\Com_{ck_{\ell}}((x_{\ell+1,1},\ldots,x_{\ell+1,2^{\ell-1}})^\top;r_{\ell,1}),\\
&[\vecb{c}_{\ell,2}]_1=\MP.\Com_{ck_{\ell}}((x_{\ell+1,2^{m-1}+1},\ldots,x_{\ell+1,2^\ell})^\top;r_{\ell,2}),\\
\end{align*}
for random $r_{\ell,1},r_{\ell,2}\in\Z_q$ and $\ell\in[m-1]$. Note that the prover does not need to compute commitments to $(x_{m+1,1},\ldots,x_{m+1,t})^\top$ since they can be computed by the verifier.

Then, the prover shows the satisfiability of equation (\ref{eq-log-2}) with a GS proof of
the satisfiability of
\begin{align}
&[\vecb{c}_\ell]_1-(1-b_\ell)[\vecb{c}_{\ell,1}]_1-b_\ell[\vecb{c}_{\ell,2}]_1 = y_\ell[\vecb{g}_{\ell,2^{\ell-1}+1}]_1, \text{ for } \ell \in[m],  \label{eq-log-5}
\end{align}
where $[\vecb{c}_{m,1}]:=\MP.\Com_{ck_m}((s_1,\ldots,s_{2^{m-1}})^\top;0)$ and $[\vecb{c}_{m,2}]:=\allowbreak \MP.\Com_{ck_m}(\allowbreak (s_{2^{m-1}+1},\allowbreak\ldots,s_{t})^\top;0)$ can be directly computed by the verifier, and $y_\ell:=r_\ell-(1-b_\ell)r_{\ell,1}-b_\ell r_{\ell,2}$.

The prover also shows that equation (\ref{eq-log-3}) is satisfied with a QA-NIZK proof that
\begin{equation}
[\vecb{c}]_1\text{ and }[\vecb{c}_1]_1\text{ open to the same value}, \label{eq-log-6}
\end{equation}
using the proof system from Section \ref{sec:aggcommit}.

Note that variables $x_{\ell+1,1},\ldots,x_{\ell+1,2^{\ell-1}}$ appear in both $[\vecb{c}_{\ell,1}]_1$ and $[\vecb{c}_{\ell+1}]_1$, as well as $x_{\ell+1,2^{\ell-1}+1},\ldots,x_{\ell+1,2^\ell}$ appear in both $[\vecb{c}_{\ell,2}]_1$ and $[\vecb{c}_{\ell+1}]_1$. To get a sound proof, the prover needs to show that this redundancy is consistent. That is, the prover needs to show that $[\vecb{c}_{\ell,1}]_1$ and $[\vecb{c}_{\ell,2}]_1$ are commitments to the first and last halves of the opening of $[\vecb{c}_{\ell+1}]_1$.

For \(\ell\in[m]\), let \(ck_\ell:=([\matr{G}_\ell]_1,\allowbreak[\vecb{g}_{\ell,2^{\ell-1}+1}]_1)\in\GG_1^{2\times{2^{\ell-1}+1}}\) the commitment key of a MP commitment scheme and let
\begin{align*}
&\matr{G}_{\ell,1}:=
\begin{pmatrix}
    \vecb{g}_{\ell,1}&\cdots&\vecb{g}_{i,2^{\ell-2}}
\end{pmatrix},
&\matr{G}_{\ell,2}:=
\begin{pmatrix}
    \vecb{g}_{\ell,2^{\ell-2}+1}&\cdots&\vecb{g}_{\ell,2^{\ell-1}}
\end{pmatrix}\\
&\matr{G}_\ell:=\matr{G}_{\ell,1}||\matr{G}_{\ell,2}
\end{align*}
To prove consistency the prover will show that, for each $\ell\in [m-1]$, the following linear system is satisfied
{\begin{align}
&\begin{pmatrix}
\vecb{c}_{\ell+1}\\
\vecb{c}_{\ell,1}\\
\vecb{c}_{\ell,2}
\end{pmatrix}
=
&\left(\begin{array}{cc|ccc}
\matr{G}_{\ell+1,1}           & \matr{G}_{\ell+1,2}            & \vecb{g}_{\ell+1,2^\ell+1} & \vecb{0}                     & \vecb{0}\\
\matr{G}_{\ell}               & \matr{0}_{2\times2^{\ell-1}}   & \vecb{0}                   & \vecb{g}_{\ell,2^{\ell-1}+1} & \vecb{0} \\
\matr{0}_{2\times 2^{\ell-1}} & \matr{G}_{\ell}                & \vecb{0}                   & \vecb{0}                     & \vecb{g}_{\ell,2^{\ell-1}+1}
\end{array}\right)
\vecb{w},\label{eq-log-split}
\end{align}}%
for some \(\vecb{w}\in\Z_q^{2^\ell+3}\), which can be proven using the proof system from Section \ref{sec:concat}.

Intuitively, \(\vecb{w}\) should be equal to \((x_{\ell+1,1},\ldots,x_{\ell+1,2^{\ell}},\allowbreak r_{\ell+1},r_{\ell,1},r_{\ell,2})\) and thus 
\begin{align*}
&[\vecb{c}_{\ell,1}]_1=\MP.\Com_{ck_\ell}(\allowbreak(x_{\ell+1,1},\ldots,x_{\ell+1,2^{\ell-1}})^\top;r_{\ell,1})\text{ and }\\
&[\vecb{c}_{\ell,2}]_1=\MP.\Com_{ck_\ell}((x_{\ell+1,2^{\ell-1}+1},\ldots, x_{\ell+1,2^{\ell}})^\top;r_{\ell,2}).
\end{align*}
However, since Multi-Pedersen commitments have multiple openings it might be the case that the satisfying witness of the proof is different from \((x_{\ell+1,1},\ldots,\allowbreak x_{\ell+1,2^\ell},r_{\ell+1},r_{\ell,1},r_{\ell,2})\) and thus the intuitive reasoning is invalid.

Despite this flawed reasoning, we will show that the proof system is still sound.

\subsubsection{Soundness Intuition}
Suppose that an adversary against soundness outputs GS commitments to \(b_1,\ldots,b_{m}\in\Z_q\), outputs commitments \([\vecb{c}_\ell]_1\), $\ell\in[m]$, and \([\vecb{c}_{\ell,1}]_1,[\vecb{c}_{\ell,2}]\), $\ell\in[m-1]$, a GS proofs of the satisfiability of equation (\ref{eq-log-5}), and QA-NIZK proofs of (\ref{eq-log-6}) and (\ref{eq-log-split}) for each \(\ell\in[m-1]\).
We will simply assume that \(b_1,\ldots,b_m\in\bits\), since it can be proven (with perfect soundness) with a GS proof of size \(\Theta(\log \setsize )\) or using the more efficient proofs of Section~\ref{sec:bits} (with computational soundness).

Let $\alpha_\ell:=(\alpha-1 \mod 2^{\ell-1})+1$, the position of $s_\alpha$ respective to the leafs under the path $(b_m,\ldots, b_\ell)$, and note that $\alpha_\ell \in[2^{\ell-1}]$.
In the reduction we will guess the (sub-)path $(b_{m-1},\ldots, b_1)$ (it will be not necessary to guess first the edge of the path) chosen by the adversary, and for each $\ell\in[m]$, we will chose $\vecb{g}_{\ell,\alpha_\ell}$ linearly independent from the other $2^{\ell-1}$ vectors in $ck_\ell$. This can be done choosing  random $b'_{m-1},\ldots, b'_1\in\bits$ and aborting if the paths $(b'_{m-1},\ldots, b'_{1})$ and $(b_{m-1},\cdots, b_1)$ are distinct. Therefore, our security reduction will have a security loss factor of $1/2^{m-1}=2/\setsize$. We sample $ck_\ell\gets\distlin_1^{2^{\ell-1},\alpha_\ell}$, as defined on Section \ref{sec:mddh}, which implies that for every $\ell\in[m]$ there exists unique $\tilde{x}_{\ell},\tilde{r}_\ell\in\Z_q$ such that $\vecb{c}_\ell:=\tilde{x}_\ell\vecb{g}_{\ell,\alpha_\ell}+\tilde{r}_\ell\vecb{g}_{\ell,2^{\ell-1}+1}$.

It will be useful to prove the next lemma about $\alpha_\ell$.
\begin{lemma} Let $b_m,\ldots,b_1\in\bits$. For all $\ell\in[m]$, $\alpha_\ell = \alpha-\sfleft+1$ and, for all $\ell\in[m-1]$, $\alpha_{\ell+1}=\alpha_{\ell}+b_\ell2^{\ell-1}$.
\label{lemma:alpha}
\end{lemma}
\begin{proof}
Recall that $\sfleft=\sum_{i=\ell}^m b_i2^{i-1}+1$ is the index of the leftmost leaf under the path $(b_m,\cdots, b_\ell)$. It holds that
\begin{align*}
\alpha_\ell-1 &= \sum_{i=1}^m b_i2^{i-1} \mod 2^{\ell-1}\\
              &= \sum_{i=\ell}^m b_i 2^{i-1} + \sum_{i=1}^{\ell-1}b_i2^{i-1} \mod 2^{\ell-1}\\
              &= \sum_{i=1}^{\ell-1}b_i2^{i-1}\\
              &= \sum_{i=1}^m b_i2^{i-1} - \sfleft\\
              &= \alpha-\sfleft\\
&\Longleftrightarrow \alpha_\ell=\alpha-\sfleft+1.
\end{align*}
On the other hand
\begin{align*}
\alpha_{\ell+1}-1 &= \alpha-1 \mod 2^\ell\\
&= \sum_{i=1}^{\ell} b_i2^{i-1}\\
&= b_\ell2^{\ell-1}+\sum_{i=1}^{\ell-1}b_i 2^{i-1}\\
&= b_\ell2^{\ell-1} + (\alpha-1 \mod 2^{\ell-1})\\
&\Longleftrightarrow \alpha_{\ell+1}=\alpha_\ell+b_\ell2^{\ell-1}
\end{align*}
\end{proof}

We prove by induction on \(\ell\) that \(\vecb{c}_\ell= s_\alpha\vecb{g}_{\ell,\alpha_\ell}+\tilde{r}_{\ell}\vecb{g}_{\ell,2^{\ell-1}+1}\), for some $\tilde{r}_\ell\in\Z_q$. If this is the case $\vecb{c}_1=s_\alpha\vecb{g}_{1,1}+\tilde{r}_{1}\vecb{g}_{1,2}$. Soundness of proof (\ref{eq-log-6}) together with the fact that \(ck_1\) is perfectly binding implies that \(x=\tilde{x}_{1}=s_{\alpha}\in S\) which proves soundness.

In the base case ($\ell=m$), equation (\ref{eq-log-5}) and the fact that \(\vecb{g}_{m,i}\in\Span(\vecb{g}_{m,2^{m-1}+1})\) if \(i\neq \alpha_\ell\) together with Lemma \ref{lemma:alpha} implies that 
\begin{align*}
\vecb{c}_m &= (1-b_m)\sum_{i=1}^{2^{m-1}}s_i \vecb{g}_{m,i}+b_m\sum_{i=1}^{2^{m-1}}s_{i+2^{m-1}}\vecb{g}_{m,i}\\
&= (1-b_m)s_{\alpha_\ell}\vecb{g}_{m,\alpha_\ell} +b_ms_{\alpha_\ell+2^{m-1}}\vecb{g}_{m,\alpha_\ell}+\tilde{r}_1\vecb{g}_{m,2^{m-1}+1}\\
&= (1-b_m)s_{\alpha-\sfleft+1}\vecb{g}_{m,\alpha_\ell} +b_ms_{\alpha-\sfleft+1+2^{m-1}}\vecb{g}_{m,\alpha_\ell}+\tilde{r}_1\vecb{g}_{m,2^{m-1}+1}\\
&=\begin{cases}
    s_{\alpha-1+1}\vecb{g}_{1,\alpha_\ell}+\tilde{r}_1\vecb{g}_{1,t/2} & \text{ if } b_m=0 \ (\text{and thus }\sfleft=1) \\
    s_{\alpha-(2^{\ell-1}+1)+1+2^{\ell-1}}\vecb{g}_{1,\alpha_\ell}+\tilde{r}_1\vecb{g}_{1,t/2} & \text{ if } b_m=1 \ (\text{and thus }\sfleft=2^{\ell-1}+1) 
\end{cases}
\end{align*}
for some \(\tilde{r}_1\in\Z_q\). In both cases $\vecb{c}_1=s_\alpha\vecb{g}_{1,\alpha_\ell}+\tilde{r}_1\vecb{g}_{1,t/2}$.

In the inductive case we assume that \(\vecb{c}_{\ell+1}=s_{\alpha}\vecb{g}_{\ell+1,\alpha_{\ell+1}}+\tilde{r}_{\ell+1}\vecb{g}_{\ell+1,2^\ell+1}\) and we want to show that $\vecb{c}_\ell = s_\alpha\vecb{g}_{\ell,\alpha_\ell}+\tilde{r}_\ell\vecb{g}_{\ell,2^{\ell-1}+1}$. Since \(\vecb{g}_{\ell+1,\alpha_{\ell+1}}\) is linearly independent from the rest of vectors in \(ck_{\ell+1}\), any solution to equation (\ref{eq-log-split}) is equal to \(s_{\alpha}\) at position \(\alpha_{\ell+1}=\alpha_\ell+b_\ell2^{\ell-1}\) as depicted below.
\begin{align*}
\pmatri{\vecb{c}_{\ell+1}\\\vecb{c}_{\ell,1}\\\vecb{c}_{\ell,2}}=
\pmatri{
\cdots & \vecb{g}_{\ell+1,\alpha_\ell} & \cdots  & \vecb{g}_{\ell+1,\alpha_\ell+2^{\ell-1}} & \cdots\\
\cdots & \vecb{g}_{\ell,\alpha_\ell}     & \cdots  & \vecb{0}                           & \cdots\\
\cdots & \vecb{0}                        & \cdots  & \vecb{g}_{\ell,\alpha_\ell}        & \cdots
}
\pmatri{\vdots\\s_\alpha\\\vdots}
\end{align*}
If $b_{\ell}=0$, by Lemma \ref{lemma:alpha}, $\alpha_{\ell+1}=\alpha_\ell$. Therefore, any solution to equation (\ref{eq-log-split}) is equal to $s_\alpha$ at position $\alpha_\ell$ and thus $\vecb{c}_{\ell,1} = s_\alpha\vecb{g}_{\ell,\alpha_\ell}+\tilde{r}_{\ell,1}\vecb{g}_{\ell,2^{\ell-1}+1}$.
Equation \ref{eq-log-5} implies that
\begin{align*}
\vecb{c}_{\ell}=&(1-b_\ell)(s_\alpha\vecb{g}_{\ell,\alpha_\ell}+\tilde{r}_{\ell,1}\vecb{g}_{\ell,2^{\ell-1}+1})+b_\ell\vecb{c}_{\ell,2}+y_\ell\vecb{g}_{\ell,2^{\ell-1}+1}\\
               =& s_\alpha\vecb{g}_{\ell,\alpha_\ell}+(\tilde{r}_{\ell,1}+y_\ell)\vecb{g}_{\ell,2^{\ell-1}+1}.
\end{align*}
If $b_{\ell}=1$, then $\alpha_{\ell+1}=\alpha_\ell+2^{\ell-1}$ and similarly, $\vecb{c}_{\ell}=s_\alpha\vecb{g}_{\ell,\alpha_\ell}+(\tilde{r}_{\ell,2}+y_\ell)\vecb{g}_{\ell,2^{\ell-1}+1}$.




    \section{The Aggregated Case} \label{sec:log-set-memb-Z}
        
        Let $\setsize:=|S|$ and \(m:=\log \setsize \). The statement is now \([\grkb{\zeta}_1]=\GS.\Com_{ck_\GS}(x_1;r_1),\ldots,\allowbreak[\grkb{\zeta}_n]_1=\GS.\Com_{ck_\GS}(x_n;r_n)\), for some $n\in\mathbb{N}$, and the prover wants to show that \(x_i=s_{\alpha_i}\), for all \(i\in[n]\) and \(\alpha_i=\sum_{j=1}^m b_{i,j}2^{j-1}+1\), for some $b_{i,1},\ldots,b_{i,m}\in\bits$. We need reformulate equations (\ref{eq-log-2}) and (\ref{eq-log-3}) to take in count new variables. For \(\ell\in [m],j\in[n]\), define
\begin{align*}
&\vecb{x}_\ell^i:=
\begin{pmatrix}
\vecb{x}^i_{\ell,1}\\
\hline
\vecb{x}^i_{\ell,2}
\end{pmatrix}
:=
\begin{pmatrix}
x^i_{\ell,1}\\\vdots\\x^i_{\ell,2^{\ell-2}}\\
\hline
x^i_{\ell,2^{\ell-2}+1}\\\vdots\\x^i_{\ell,2^{\ell-1}}
\end{pmatrix},
&\vecb{x}^i_{m+1,1} := \begin{pmatrix}
s_1\\\vdots\\s_{t/2}
\end{pmatrix},\text{ and }
&\vecb{x}^i_{m+1,2} := \begin{pmatrix}
s_{t/2+1}\\\vdots\\s_{\setsize }
\end{pmatrix},
\end{align*}
and define new equations for each $\ell\in[m],i\in[n]$
\begin{align}
&\vecb{x}^i_\ell=({1}-{b}_{i,\ell})\vecb{x}^i_{\ell+1,1}+{b}_{i,\ell}\vecb{x}^i_{\ell+1,2},\label{eq-alog-1}\\
&x_i= \vecb{x}^{i}_1\label{eq-alog-2}
\end{align}

Next, we construct an aZKSMP for $S\subset\Z_q$ and in Section~\ref{sec:improved-aZKSMP-group-case} we show how to extend these ideas for the case of fixed \(S\subset\GG_1\).
The construction follows the intuition outlined before but it ``aggregates'' many instances on a single \(\Theta(\log \setsize )\) proof. From a high level this is done as follows.

We will rewrite equation (\ref{eq-alog-1}), which is a system of $m n$ equations, as $m$ equations of the form
\begin{equation}
\vecb{x}\vecb{y}^\top=\pmatri{0 & x_1y_2 & \cdots & x_1y_m\\x_2y_1& 0 & \cdots & x_2y_m\\\vdots&\vdots&\ddots& \vdots\\x_ny_1&x_ny_2&\cdots&0},
\label{eq-diag}
\end{equation}
where $\vecb{x}\in\Z_q^m,y\in\Z_q^n$ (i.e.~the diagonal of the matrix $\vecb{x}\vecb{y}^\top$ is $\vecb{0}$). We will use similar techniques to those of Chapter \ref{sec:bits} to give a constant size proof for the satisfiability of each of these equations. Therefore, to prove $m$ of this equation we will require $\Theta(m)=\Theta(\log t)$ group elements.

We can compute $\vecb{x}\vecb{y}^\top$ in the ``commitment space'' by means of $[\vecb{c}]_1[\vecb{d}^\top]_2$, where $[\vecb{c}]_1:=\MP.\Com_{ck_1}(\vecb{x};r_1)$ and $[\vecb{d}]_2:=\MP.\Com_{ck_2}(\vecb{y};r_2)$. Indeed, by the definition of MP commitments it holds that
\begin{align*}
[\vecb{c}]_1[\vecb{d}^\top]_2
=&\left(\sum_{i=1}^{m}x_i[\vecb{g}_i]_1+r_1[\vecb{g}_{m+1}]_1\right)\left(\sum_{j=1}^{n}y_j[\vecb{h}_j^\top]_2+r_2[\vecb{h}_{n+1}^\top]_2\right)\\
=&\sum_{i=1}^m\sum_{j=1}^n x_iy_j[\vecb{g}_i\vecb{h}_j^\top]_T+\sum_{i=1}^{m}x_ir_2[\vecb{g}_i\vecb{h}_{n+1}^\top]_T+\sum_{j=1}^{n+1}r_1y_j[\vecb{g}_{m+1}\vecb{h}_j^\top]_T
\end{align*}

Therefore, if the diagonal of $\vecb{x}\vecb{y}^\top$ is $\vecb{0}$, then $[\vecb{c}]_1[\vecb{d}^\top]_2$ is in the space spanned by $\{[\vecb{g}_i\vecb{h}_j^\top]:i\neq j \text{ or }i=m+1\text{ or } j=n+1\}$. Similarly as done in Section \ref{sec:bits-intuition}, equation (\ref{eq-diag}) can be proven computing two matrices $[\matr{\Theta}]_1\in\GG_1^{2\times2}$ and $[\matr{\Pi}]_2\in\GG_2^{2\times 2}$ and showing that $[\vecb{c}]_1[\vecb{d}^\top]_2=[\matr{\Theta}]_1[\matr{I}]_2+[\matr{I}]_1[\matr{\Pi}]_2$ and $\matr{\Theta}+\matr{\Pi}\in\Span(\{[\vecb{g}_i\vecb{h}_j^\top]:i=m+1\text{ or }j=n+1\})$.

Lets rewrite the right side of equation (\ref{eq-alog-1}) in the $\vecb{x}\vecb{y}^\top$ form.
\begin{align*}
\pmatri{\vecb{x}^1_{\ell+1,1}\\\vdots\\\vecb{x}^n_{\ell+1,1}}\left(\pmatri{b_{1,\ell}\\\vdots\\b_{n,\ell}}-\pmatri{1\\\vdots\\1}\right)^\top+
\pmatri{\vecb{x}^1_{\ell+1,1}\\\vdots\\\vecb{x}^n_{\ell+1,2}}\pmatri{b_{1,\ell}\\\vdots\\b_{n,\ell}}^\top
&=\\
\begin{pmatrix}
(1-b_{1,\ell})\vecb{x}^1_{\ell+1,1}+b_{1,\ell}\vecb{x}^1_{\ell+1,2} & \cdots & (1-b_{n,\ell})\vecb{x}^1_{\ell+1,1}+b_{n,\ell}\vecb{x}^1_{\ell+1,2}\\
\vdots & \ddots  & \vdots\\ 
(1-b_{1,\ell})\vecb{x}^n_{\ell+1,1}+b_{1,\ell}\vecb{x}^n_{\ell+1,2} & \cdots & (1-b_{n,\ell})\vecb{x}^n_{\ell+1,1}+b_{n,\ell}\vecb{x}^n_{\ell+1,2}
\end{pmatrix}.&
\end{align*}
If we view the previous matrix as one of size $n\times n$ where each entry is a vector from $\Z_q^{2^{\ell-1}}$, then the diagonal form the right sides of equation (\ref{eq-alog-1}). We rewrite the left side as
\begin{align*}
\pmatri{\vecb{x}^1_\ell\\\vdots\\\vecb{x}^n_\ell}\pmatri{1\\\vdots\\1}^\top
=
\begin{pmatrix}
\vecb{x}^1_\ell & \cdots & \vecb{x}^1_\ell\\
\vdots          &        & \vdots         \\
\vecb{x}^n_\ell & \cdots & \vecb{x}^n_\ell
\end{pmatrix}.
\end{align*}
and, again, the diagonal form the left sides of equation (\ref{eq-alog-1}). 

Now we show equation (\ref{eq-alog-1}) replacing variables with MP commitments and showing that
\begin{align*}
[\vecb{c}_\ell]_1\left(\sum_{j=1}^n[\vecb{h}_j]_2\right)^\top-[\vecb{c}_{\ell,1}]_1\left([\vecb{d}_\ell]_2-\sum_{j=1}^n[\vecb{h}_j]_2\right)^\top-[\vecb{c}_{\ell,2}]_1[\vecb{d}]_2^\top
=&\\
[\matr{\Theta}]_1[\matr{I}]_2+[\matr{I}]_1[\matr{\Pi}]_2&,
\end{align*}
where
\begin{align*}
&[\vecb{c}_\ell]_1:=\MP.\Com_{ck_\ell}\left(\pmatri{\vecb{x}^1_\ell\\\vdots\\\vecb{x}^n_\ell};r_\ell\right)
&[\vecb{d}_\ell]_1:=\MP.\Com_{ck}\left(\pmatri{b_{1,\ell}\\\vdots\\b_{n,\ell}};t_\ell\right)\\
&[\vecb{c}_{\ell,1}]_1:=\MP.\Com_{ck_\ell}\left(\pmatri{\vecb{x}^1_{\ell+1,1}\\\vdots\\\vecb{x}^n_{\ell+1,1}};r_{\ell,1}\right)
&[\vecb{c}_{\ell,2}]_1:=\MP.\Com_{ck_\ell}\left(\pmatri{\vecb{x}^1_{\ell+1,2}\\\vdots\\\vecb{x}^n_{\ell+1,2}}\right)\\
&ck_\ell:=[\pmatri{\vecb{G}^1_{\ell}&\cdots&\matr{G}^n_{\ell} & \vecb{g}_{\ell,n2^{\ell-1}+1}}]_1
&\matr{G}^i_{\ell}:=\pmatri{\vecb{g}_{\ell,(i-1)2^{\ell-1}+1}&\cdots&\vecb{g}_{\ell,i2^{\ell-1}}}\\
&ck:=[\matr{H}]_2 &
\matr{H}:=\pmatri{\vecb{h}_1&\cdots&\vecb{h}_n&\vecb{h}_{n+1}}.
\end{align*}
We need to show that $\matr{\Theta}+\matr{\Pi}$ is in the appropriate space, which is the one without components ``in the diagonal'' or with components in $\vecb{g}_{n2^{\ell-1}+1}\vecb{h}_j$ or $\vecb{g}_i\vecb{h}_{n+1}$ for any $i\in[n2^{\ell-1}],j\in[n]$. However, since we are working with matrices whose entries are vectors in $\Z_q^{\ell-1}$ we in fact need to show that
$$
\matr{\Theta}\in\Span(\vecb{g}_{\ell,i}\vecb{h}_j^\top:j\in[n+1],i\notin[(j-1)2^{\ell-1}+1,j2^{\ell-1}]\setminus[n2^{\ell-1}+1]).
$$

Finally, equation (\ref{eq-log-split}) is proven in the same way as in the non-aggregated case, but enlarging the matrix as consequence of the enlargement of commitment keys. Additionally, we prove equation (\ref{eq-alog-2}) with a proof that
\begin{align*}
[\grkb{\zeta}_1]_1,\ldots,[\grkb{\zeta}_n]_1 \text{ and } [\vecb{c}_1]_1 \text{ open to the same values.}
\end{align*}
\iffalse
We prove soundness in the same fashion as the protocols from Chapter \ref{sec:shuf-rp}, that is we guess the index $j^*$ of an element $x_{j^*}\notin S$ in order to choose $\vecb{h}_{j^*}$ linearly independent from the other vectors in $ck$. Similarly as in the non-aggregated case, we will guess the (sub-)path $(b_{j^*,m+1},\ldots, b_{j^*,1})$ (where $b_{j^*,1},\ldots,b_{j^*,m}$ are the uniques openings of, respectively, $[\vecb{d}_1]_2,\ldots,[\vecb{d}]_m$ at position $j^*$) and we will choose, for each $\ell\in[m]$, $\vecb{g}_{\ell,\alpha_{j^*,\ell}}$ linearly independent from the other vectors in $ck_\ell$, where $\alpha_{j^*,\ell}:=(\alpha_{j^*}-1 \mod 2^{\ell-1})+1$. With such choice of the commitment keys, we will be able to prove that $x_{j^*}=s_{\alpha_{j^*}}$ unless we can break some hardness assumption.
Consequently, the total security loss in the reduction will be of a factor of $\frac{2}{nt}$.  
\fi
\subsubsection{The Scheme}

\begin{description}

\item[{\(\algK_1(\gk, ck_\GS)\)}:]
For each \(\ell\in [m]\) let \(\matr{G}_\ell:=\matr{G}^1_{\ell}||\cdots||\matr{G}^n_{\ell}||\vecb{g}_{\ell,{n2^{\ell-1}+1}}\gets\distlin_1^{n2^{\ell-1}+1,0}\), where
\begin{align*}
&\matr{G}^i_{\ell}=
(\matr{G}^i_{\ell,1}|\matr{G}^i_{\ell,2})
=\\
&(\vecb{g}_{\ell,(i-1)2^{\ell-1}+1}\cdots\vecb{g}_{\ell, (i-1) 2^{\ell-1}+2^{\ell-2}}|\vecb{g}^\ell_{i,(i-1)2^{\ell-1}+2^{\ell-2}+1}\cdots\vecb{g}^\ell_{i,i2^{\ell-1}})
\in\Z_q^{2\times 2^{\ell-1}},
\end{align*}
 \(i\in  [n]\), and define \(ck_\ell:=[\matr{G}^\ell]_1\).
Let \(\matr{H}=\begin{pmatrix}\vecb{h}_1&\cdots&\vecb{h}_n&\vecb{h}_{n+1}\end{pmatrix}\gets\distlin_1^{n,0}\) and define \(ck:=[\matr{H}]_2\). 

Pick \(\matr{T}\gets\Z_q^{2\times 2}\) and for each \(\ell\in[m]\), \( i\in [n2^{\ell-1}+1]\), \(j\in  [n+1]\), such that \(i\notin [(j-1)2^{\ell-1}+1,j2^{\ell-1}]\) define matrices
\[\matr{M}^\ell_{i,j}:=([\matr{C}_{i,j}^\ell]_1,[\matr{D}_{i,j}^\ell]_2):=([\vecb{g}_{\ell,i}\vecb{h}_j^{\top}+\matr{T}]_1,[-\matr{T}]_2),\]
For \(\ell\in [m]\), let
$$
\mathcal{M}_\ell:=\{\matr{M}^\ell_{i,j}:j\in[n+1],i\notin[(j-1)2^{\ell-1}+1,j2^{\ell-1}]\setminus[n2^{\ell-1}+1]\}$$
and let \(\mathcal{C}_\ell\subset\Z_q^{2\times 2}\)
$$
\mathcal{C}_\ell:=\{\matr{C}^\ell_{i,j}:j\in[n+1],i\notin[(j-1)2^{\ell-1}+1,j2^{\ell-1}]\setminus[n2^{\ell-1}+1]\}.
$$

Let \(\Pi_\sfsum\) be the proof system for Sum in Subspace 
(Section~\ref{sec:sum}), \(\Pi_\mathsf{lin}\) the proof system for membership in linear subspaces from Section~\ref{sect:QANIZKlinspace}, \(\Pi_\sfbits\) the proof system for proving that many commitments open to bit-strings from section \ref{sec:matr-bits}, and \(\Pi_\sfcom\)
be an instance of the proof system for equal commitment opening (Section~\ref{sec:aggcommit}).

For each $\ell\in[m]$, let
\(\crs_{\sfsum,\ell} \gets \Pi_\sfsum.\algK_1(\gk, \mathcal{M}_\ell)\).\footnote{We identify
matrices in \(\GG_1^{2 \times 2}\) (respectively in \(\GG_2^{2 \times 2}\)) with vectors in \(\GG_1^{4}\) (resp. in \(\GG_2^{4}\)).}, let \(\crs_{\mathsf{lin},\ell}\gets \Pi_\mathsf{lin}.\algK_1(gk;\allowbreak[\matr{G}^\ell_{\mathsf{split}}]_1,n2^{\ell-1}+3)\), let \(\crs_\sfbits\gets\Pi_\sfbits.\algK_1(gk,[\matr{H}]_2,m)\), and let \(\crs_\sfcom \gets \Pi_\sfcom.\algK_1(\gk, ck_1,ck_\GS^n,m)\), where
\begin{align*}
&\matr{G}_\mathsf{split}^\ell:=\\
&\begin{pmatrix}
\matr{G}^1_{\ell+1,1} & \matr{G}^1_{\ell+1,2} & \cdots & \matr{G}^n_{\ell+1,1} & \matr{G}^n_{\ell+1,2} & \vecb{g}_{\ell+1,n2^\ell+1} & \matr{0}                       & \matr{0}\\
\matr{G}^1_{\ell,1}   & \matr{0}              & \cdots & \matr{G}^n_{\ell,n}   & \matr{0}              & \matr{0}                  & \vecb{g}_{\ell,n2^{\ell-1}+1}  & \matr{0}\\
\matr{0}              & \matr{G}^1_{\ell,1}   & \cdots & \matr{0}              & \matr{G}^n_{\ell,n}   & \matr{0}                  & \matr{0}                       & \vecb{g}_{\ell,n2^{\ell-1}+1}
\end{pmatrix},\\
&ck_\GS^n:=\pmatri{
    \vecb{u}_1 &        & \vecb{0}   & \vecb{u}_2 &        & \vecb{0}\\
               & \ddots &            &            & \ddots &         \\
    \vecb{0}   &        & \vecb{u}_1 & \vecb{0}   &        & \vecb{u}_2
}\in\GG_1^{2n\times2n}.
\end{align*}
The common reference string is given by:
\begin{eqnarray*}
\mathsf{crs}&:=&\left( gk, [\matr{G}]_1,
    [\matr{H}]_2, \{\mathcal{M}_\ell,\crs_{\sfsum,\ell},\crs_{\mathsf{lin},\ell}:\ell\in [m]\},\crs_\sfbits,\crs_\sfcom \right).
 \end{eqnarray*}


\item[{\(\algP(\mathsf{crs}, ([\grkb{\zeta}_1]_1, \ldots, [\grkb{\zeta}_n]_1,S), \langle (x_1,\ldots,x_n),(w_1,\ldots,w_n) \rangle)\)}:]
The prover compute commitments
\begin{align*}
&[\vecb{c}_\ell]_1:=\MP.\Com_{ck_\ell}({\vecb{x}_\ell^1}^\top,\ldots,{\vecb{x}_\ell^n}^\top;r_\ell), \text{ for } \ell \in [m],\\
&[\vecb{c}_{\ell,1}]_1:=\MP.\Com_{ck_\ell}({\vecb{x}^1_{\ell+1,1}}^\top,\ldots,{\vecb{x}^n_{\ell+1,1}}^\top;r_{\ell,1}),\\
&[\vecb{c}_{\ell,2}]_1:=\MP.\Com_{ck_\ell}({\vecb{x}^1_{\ell+1,2}}^\top,\ldots,{\vecb{x}^n_{\ell+1,2}}^\top;r_{\ell,2}), \text{ for } \ell\in[m-1]\\
&[\vecb{d}_\ell]_2:=\MP.\Com_{ck}(\vecb{b}_\ell;t_\ell),\text{ for } \ell\in[m]
\end{align*}
 where \(r_\ell,r_{\ell,1},r_{\ell,2},t_j\gets\Z_q\) and the variables \(\vecb{x}^i_\ell,\vecb{x}^i_{\ell,j},\vecb{b}_\ell\) are the ones defined in equation (\ref{eq-alog-1}). The prover computes a proof \(\pi_\sfbits\) that \([\vecb{d}_1]_2,\ldots,[\vecb{d}_m]_2\) open to bit-strings. Then, for \(\ell\in [m]\), the prover pick matrices \(\matr{R}_\ell\gets\Z_q^{2\times 2}\), computes
\begin{align*}
&([\matr{\Theta}_\ell]_1,[\matr{\Pi}_\ell]_2)  := \\
&\quad \sum_{i=1}^n\sum_{j\neq i}\sum_{k=1}^{2^{\ell-1}}(x^i_{\ell,k}-x^i_{\ell+1,k}(1-b_{j,\ell})-x^i_{\ell+1,2^{\ell-1}+k}b_{j,\ell})\matr{M}_{(i-1)2^{\ell-1}+k,j}^\ell\\
&\quad+ \sum_{i=1}^n\sum_{k=1}^{2^{\ell-1}}t_\ell(x^i_{\ell+1,k}-x^i_{\ell+1,2^{\ell-1}+k})\matr{M}_{(i-1)2^{\ell-1}+k,n+1}^\ell \\
&\quad+ \sum_{j=1}^n (r_\ell-r_{\ell,1}(1-b_{j,\ell})-r_{\ell,2}b_{j,\ell})\matr{M}_{n2^{\ell-1}+1,j}^\ell \\
&\quad+(r_{\ell,1}-r_{\ell,2})t_\ell\matr{M}^\ell_{n2^{\ell-1}+1,n+1}+([\matr{R}_\ell]_1,[-\matr{R}_\ell]_2),
\end{align*}
where \(r_{1,1}=r_{1,2}=0\), and computes proofs $\pi_{\mathsf{lin},\ell},\pi_{\mathsf{sum},\ell}$ that, respectively,
\begin{align*}
&\begin{pmatrix}
\vecb{c}_{\ell+1}\\\vecb{c}_{\ell,1}\\\vecb{c}_{\ell,2}
\end{pmatrix}\in
\Span(\matr{G}^{\ell}_\mathsf{split})\ (\text{ if }\ell< m), &\matr{\Theta_\ell}+\matr{\Pi_\ell}\in\Span(\mathcal{C}_\ell).
\end{align*}
Finally, it computes a proof \(\pi_\sfcom\) that \(([\grkb{\zeta}_1]_1,\ldots,[\grkb{\zeta}_n]_1)\) and \([\vecb{c}_1]_1\) open to the same value.

The proof is \(\pi:=(\{([\vecb{c}_\ell]_1,[\vecb{c}_{\ell,1}]_1,[\vecb{c}_{\ell,2}]_1,[\vecb{d}_\ell]_2,[\matr{\Theta}_\ell]_1,[\matr{\Pi}_\ell]_2,\pi_{\mathsf{lin},\ell},\pi_{\sfsum,\ell}):\ell\in [m]\},\pi_\sfbits,\pi_\sfcom)\).

\item[{\(\algV(\crs,([\grkb{\zeta}_1]_1, \ldots, [\grkb{\zeta}_n]_1,S),\pi)\)}:]
Let \([\vecb{c}_{m,1}]_1:=\MP.\Com_{ck_m}(s_1,\ldots,s_{\setsize /2};0),[\vecb{c}_{m,2}]:=\MP.\Com_{ck_m}(s_{\setsize /2+1},\ldots,s_{\setsize };0)\). The verifier checks the validity of \(\pi_\sfbits,\pi_\sfcom\) 
and, for each \(\ell\in [m]\), checks the validity of \(\pi_{\mathsf{lin},\ell},\pi_{\sfsum,\ell}\) and of equations
\begin{align}
&[\vecb{c}_\ell]_1\left(\sum_{j=1}^n [\vecb{h}_j]_2\right)^\top-
[\vecb{c}_{\ell,1}]_1\left(\sum_{j=1}^n[\vecb{h}_j]_2-[\vecb{d}_\ell]_2\right)^\top-
[\vecb{c}_{\ell,2}]_1[\vecb{d}_\ell]_2^\top = \nonumber\\
&[\matr{\Theta}_\ell]_1[\matr{I}]_2+[\matr{I}]_1[\matr{\Pi}_\ell]_2. \label{eq-alog-5}
\end{align}
If any of these checks fails, it rejects the proof.

\item[{\(\mathsf{S}_1({gk},ck_\GS)\):}] The simulator receives as input a description of an asymmetric bilinear group \({gk}\) and a GS commitment key $ck_\GS$. It generates and outputs the CRS in the same way as \(\algK_1\), but additionally it also  outputs the simulation trapdoor 
\(\tau:=(\matr{H},\tau_\sfcom,\tau_\sfbits,\{\tau_{\sfsum,\ell},\tau_{\mathsf{lin},\ell}:\ell\in [m]\})\),
where \(\tau_{\sfsum},\tau_{\sfbits},\tau_{\sfsum,\ell},\tau_{\mathsf{lin,\ell}}\) are, respectively, \({\Pi_\sfsum},{\Pi_\sfcom},\Pi_\sfsum,\Pi_\mathsf{lin}\) simulation trapdoors.

\item[{\(\mathsf{S}_2(\crs,([\grkb{\zeta}_1]_1,\ldots,[\grkb{\zeta}_n]_1,S),\tau)\):}] Define \(\vecb{x}_\ell^i:=\vecb{0}\) and \(\vecb{b}_\ell:=\vecb{0}\) for all \(\ell\in [m],i\in[n]\), and computes \([\vecb{c}_\ell]_1, [\vecb{c}_{\ell,1}]_1,[\vecb{c}_{\ell,2}]_1,[\vecb{d}_\ell]_2\) and \([\matr{\Theta}_\ell]_1,[\matr{\Pi}_\ell]_2\), as an honest prover would do (that is, with all variables set to 0).
Finally, simulate proofs \(\pi_\sfcom,\pi_\sfbits,\pi_{\sfsum,\ell},\pi_{\mathsf{lin},\ell}\) using the respective trapdoors.
\end{description}

We prove the following Theorem.

\begin{theorem} \label{theo:bits}
The proof system described above is a QA-NIZK proof system for the language \(\Lang_{ck_\GS,\mathsf{set}}^n\)
 with Perfect Completeness, Computational Soundness, and Perfect Zero-Knowledge.
\end{theorem}	

\subsubsection{Completeness}
Completeness follows from completeness of \(\Pi_\sfsum,\Pi_\mathsf{lin},\Pi_\sfbits,\Pi_\sfcom\), and from the fact that equation (\ref{eq-alog-5}) is satisfied for each \(\ell\in [m]\):
\begin{align*}
&\vecb{c}_\ell\left(\sum_{j=1}^n \vecb{h}_j\right)^\top-
\vecb{c}_{\ell,1}\left(\sum_{j=1}^n\vecb{h}_j-\vecb{d}_\ell\right)^\top-
\vecb{c}_{\ell,2}\vecb{d}_\ell^\top &= \\
&\sum_{i=1}^n\sum_{j=1}^n\matr{G}^i_{\ell}\vecb{x}^i_{\ell}\vecb{h}_j^\top+\sum_{j=1}^nr_\ell\vecb{g}_{\ell,n2^{\ell-1}+1}\vecb{h}_j^\top
-\sum_{i=1}^n\sum_{j=1}^n\matr{G}^i_{\ell}\vecb{x}^i_{\ell+1,1}(1-b_{j,\ell})\vecb{h}_j^\top\\
&+\sum_{i=1}^n\matr{G}^i_{\ell}\vecb{x}^i_{\ell+1,1} t_\ell\vecb{h}_{n+1}^\top-\sum_{j=1}^nr_{\ell,1}(1-b_{j,\ell})\vecb{g}_{\ell,n2^{\ell-1}+1}\vecb{h}_j^\top+ r_{\ell,1}t_\ell\vecb{g}_{\ell,n2^{\ell-1}+1}\vecb{h}_{n+1}^\top\\
&-\sum_{i=1}^n\sum_{j=1}^n\matr{G}^i_{\ell}\vecb{x}^i_{\ell+1,2}b_{j,\ell}\vecb{h}_j^\top-\sum_{i=1}^n\matr{G}^i_{\ell}\vecb{x}^i_{\ell+1,2} t_\ell\vecb{h}_{n+1}^\top-\sum_{j=1}^nr_{\ell,2}b_{j,\ell}\vecb{g}_{\ell,n2^{\ell-1}+1}\vecb{h}_j^\top\\
&- r_{\ell,2}t_\ell\vecb{g}_{\ell,n2^{\ell-1}+1}\vecb{h}_{n+1}^\top &=\\
&\sum_{i=1}^n\sum_{j\neq i}\matr{G}^i_\ell(\vecb{x}^i_\ell-\vecb{x}^i_{\ell+1,1}(1-b_{j,\ell})-\vecb{x}^i_{\ell+1,2}b_{j,\ell})\vecb{h}_j^\top+\\
&\sum_{i=1}^n\matr{G}^i_\ell(\vecb{x}^i_{\ell+1,1}-\vecb{x}^i_{\ell+1,2})t_\ell\vecb{h}_{n+1}^\top+\sum_{j=1}^n(r_\ell-r_{\ell,1}(1-b_{j,\ell})-r_{\ell,2}b_{j,\ell})\vecb{g}_{\ell,n2^{\ell-1}+1}\vecb{h}_{j}^\top\\
&+(r_{\ell,1}-r_{\ell,2})t_\ell\vecb{g}_{\ell,n2^{\ell-1}+1}\vecb{h}_{n+1}^\top &=\\
&\sum_{i=1}^n\sum_{j\neq i}\sum_{k=1}^{2^{\ell-1}}(x^i_{\ell,k}-x^i_{\ell+1,k}(1-b_{j,\ell})-x^i_{\ell+1,2^{\ell-1}+k}b_{j,\ell}))\vecb{g}_{\ell,(i-1)2^{\ell-1}+k}\vecb{h}_j^\top\\
&+\sum_{i=1}^n\sum_{k=1}^{2^{\ell-1}}t_\ell(x^i_{\ell+1,k}-x^i_{\ell+1,2^{\ell-1}+k}\vecb{g}_{\ell,(i-1)2^{\ell-1}+k}\vecb{h}_{n+1}^\top\\
&\sum_{j=1}^n(r_\ell-r_{\ell,1}(1-b_{j,\ell})-r_{\ell,2}b_{j,\ell})\vecb{g}_{\ell,n2^{\ell-1}+1}\vecb{h}_{j}^\top+(r_{\ell,1}-r_{\ell,2})t_\ell\vecb{g}_{\ell,n2^{\ell-1}+1}\vecb{h}_{n+1}^\top &=\\
&\matr{\Theta}\matr{I}+\matr{I}\matr{\Pi}.
\end{align*}

\subsubsection{Soundness}

The following theorem guarantees Soundness. 
 
\begin{theorem} Let \(\mathsf{Adv}_{{\Pi_\sfset}}(\advA)\) 
be the advantage of an adversary \(\advA\) against the soundness of 
the proof system  described above. There exist PPT adversaries
\(\advD_1,\advD_2,\advB_\sfbits,\advB_\sfcom,\advB_\sfsum,\advB_\mathsf{lin}\) such that 
\begin{align*}
\mathsf{Adv}_{{\Pi_\sfset}}(\advA) \leq 
n \left(\right.
    &\mathsf{Adv}_{\mathcal{L}_1,\Gr}(\advD_1) 
        + \setsize /2\left(4/q
            +  \mathsf{Adv}_{\Pi_\sfbits}(\advB_\sfbits)
            +  \mathsf{Adv}_{\mathcal{L}_1,\Hr}(\advB_2)\right. \\
    &+ \left.\left.\mathsf{Adv}_{{\Pi_\sfcom}}(\advB_\sfcom)
        + m\mathsf{Adv}_{{\Pi_\sfsum}}(\advB_\sfsum)
        + m\mathsf{Adv}_{{\Pi_\mathsf{lin}}}(\advB_\mathsf{lin})\right)\right).
\end{align*}
\label{teo:bitstr-soundness}
\end{theorem}

Recall that, given $b_1,\ldots,b_m\in\bits$, we defined $\alpha:=\sum_{i=1}^mb_i2^{i-1}+1$ and $\alpha_\ell:=(\alpha-1\mod 2^\ell-1)+1$. Recall also that, given a path $(b_m,\ldots, b_\ell)$ in the binary tree whose leafs are labeled from left to right by $s_1,\ldots,s_t$, we defined $\sfleft:=\sum_{i=\ell}^m b_i2^{i-1}+1$ and $\sfright:=\sfleft+2^{\ell-1}-1$.

The proof follows from the indistinguishability of the following games:
\begin{itemize}
\item[\(\mathsf{Real}\):] This is the real Soundness game. The output is 1 if the adversary submits some \(([\grkb{\zeta}_1]_1,\ldots,[\grkb{\zeta}_n]_1,S)\notin\Lang_{ck_\GS,\mathsf{set}}^n\) and the corresponding proof which is accepted by the verifier.
\item[\(\sfGame_0\):] This identical to \(\mathsf{Real}\), except that \(\algK_1\) does not receive \(ck_\GS\) as a input but
it samples \(ck_\GS\) itself together with its discrete logarithms.
\item[\(\sfGame_1\):] This game is identical to \(\sfGame_0\) except that now it chooses random \(j^*\in[n]\) and it aborts if \(x_{j^*}\notin S\).
\item[\(\sfGame_2\):] This game is identical to \(\sfGame_1\) except that now \(\matr{H}\gets\distlin^{n,j^*}_1\).
\item[\(\sfGame_3\):] This game is identical to \(\sfGame_2\) except that now it chooses a random (sub-)path $(b_m,\cdots, b_2)\gets\bits^{m-1}$ (which ignores the first edge) in the tree whose leafs are $s_1,\ldots,s_t$. This game aborts if \((b_{j^*,1},\ldots,b_{j^*,m})\notin\bits^m\) or \((b_2,\ldots, b_m)\neq(b_{j^*,2},\ldots, b_{j^*,m})\), where \(b_{j^*,2},\ldots,b_{j^*,m}\) are the openings of \([\vecb{d}_2]_2,\ldots,[\vecb{d}_m]_2\) at coordinate \(j^*\), respectively. Define $b_1:=b_{j^*,1}$
\item[\(\sfGame_4\):] This game is identical to \(\sfGame_3\) except that now \(\matr{G}_\ell\gets\distlin_1^{n2^{\ell-1},(j^*-1)2^{\ell-1}+\alpha_\ell}\), for \(\ell\in [m]\).
\end{itemize}

It is obvious that the first two games are indistinguishable. The rest of the argument goes as follows.

\begin{lemma}
\(\Pr\left[ \mathsf{Game}_1(\advA)=1\right]\geq\dfrac{1}{n}\Pr\left[\mathsf{Game}_0(\advA)=1\right].\)
\end{lemma}

\begin{proof}  The probability that
 \(\mathsf{Game}_1(\advA)=1\) is the probability that  a) \(\mathsf{Game}_0(\advA)=1\) and
b)  \(x_{j^*} \notin S\). The view of adversary \(\advA\) is independent of \(j^*\), while, if \(\mathsf{Game_0}(\advA)=1\), then there is at least one index \(j \in [n]\) such that  
such that  \(x_{j} \notin S\). Thus, 
the probability that the event described in b) occurs conditioned on \(\mathsf{Game_0}(\advA)=1\), is greater than or equal to \(1/n\) and the lemma follows.
\end{proof}

\begin{lemma} There exists a\ \(\distlin_1\)-\(\mddh_{\GG_2}\) adversary \(\advD_2\) such that
\(|\Pr\left[\allowbreak\mathsf{Game}_{1}(\advA)\allowbreak=1\right]\linebreak-\Pr\left[\mathsf{Game}_{2}(\advA)=1\right]|\) \(\leq \mathsf{Adv}_{\distlin_1,\ggen_a}(\advD_2).\)
\end{lemma}
\begin{proof}
We construct an adversary \(\advD_2\) that receives 
a challenge \(([\vecb{a}]_2,[\vecb{u}]_2)\) of the 
\(\distlin_1\)-\(\mddh_{\GG_2}\) Assumption. From this challenge, \(\advD_2\) just defines the matrix  \([\matr{H}]_2\in\GG_2^{2\times(n+1)}\) as the matrix whose last column is \([\vecb{a}]_2\), the ith column is \([\vecb{u}]_2\), and the rest of the columns are random vectors in the image of \([\vecb{a}]_2\). 
Obviously, when \([\vecb{u}]_2\) is sampled from 
the image of \([\vecb{a}]_2,\) \(\matr{H}\) follows the distribution \(\distlinizeroone\), while if \([\vecb{u}]_2\) is a uniform element of \(\GG^2_2\), \(\matr{H}\) follows the distribution \(\distlin_1^{n,j^*}\). 
 
Adversary \(\advD_2\) samples
\(\matr{G}^\ell \gets \distlin_1^{n2^{\ell-1},0}\). Given that \(\advD_2\) does not know the discrete logarithms of \([\matr{H}]_2\), it cannot compute the pairs \((\matr{C}^\ell_{i,j},\matr{D}^\ell_{i,j})\) exactly as in \(\sfGame_0\). Nevertheless, for each \(\ell\in[m],i\in[n2^{\ell-1}+1],j\in[n+1]\) such that $i\notin[(j-1)2^{\ell-1}+1,j2^{\ell-1}]$, it can compute identically distributed pairs by picking \(\matr{T}\gets\Z_q^{2\times 2}\) and defining
\[
([\matr{C}^\ell_{i,j}]_1,[\matr{D}^\ell_{i,j}]_2):=([\matr{T}]_1,\vecb{g}_{\ell,i}[\vecb{h}_j]_2^\top-[\matr{T}]_2).
\]

The rest of the elements of the CRS are honestly computed. When \(\matr{H}\gets\distlin_1^{n,0}\), \(\advD_2\) perfectly simulates \(\sfGame_0\), and when \(\matr{H}\gets\distlin_1^{n,j^*}\), \(\advD_2\) perfectly simulates \(\sfGame_1\), which concludes the proof. 
\end{proof}

\begin{lemma} There exists an adversary \(\advB_\sfbits\) against \(\Pi_\sfbits\) such that
\(\Pr\left[\allowbreak \mathsf{Game}_2(\advA)\allowbreak =1\right]\geq\dfrac{2}{\setsize }(\Pr\left[\mathsf{Game}_3(\advA)=1\right]+\adv_{\Pi_\sfbits}(\advB_\sfbits)).\)
\end{lemma}

\begin{proof}  The probability that
 \(\mathsf{Game}_3(\advA)=1\) is the probability that  a) \(\mathsf{Game}_2(\advA)=1\) and
b) \((b_{j^*,1},\ldots,b_{j^*,m})\notin\bits^m\) or \((b_2,\ldots, b_m) \neq (b_{j^*,2},\ldots, b_{j^*,m})\). If \((b_{j^*,1},\ldots,\allowbreak b_{j^*,m})\notin\bits^m\) we can build an adversary \(\advB_\sfbits\) against \(\Pi_\sfbits\) and thus, the probability that \((b_{j^*,1},\ldots,b_{j^*,m})\in\bits^m\) is less than \(\adv_{\Pi_\sfbits}(\advB_1)\). The view of adversary \(\advA\) is independent of \((b_{2},\ldots, b_{m})\), while, if \(\mathsf{Game_2}(\advA)=1\) and \((b_{j^*,1},\ldots,b_{j^*,m})\in\bits^{m}\), then \((b_{j^*,2}\cdots b_{j^*,m})\in[\setsize /2]\). Thus, 
the probability that the event described in b) occurs conditioned on \(\mathsf{Game_0}(\advA)=1\) and \((b_{j^*,1},\ldots,b_{j^*,m})\in\bits^{m}\), is greater than or equal to \(2/\setsize \) and the lemma follows.
\end{proof}

\begin{lemma} There exists a \(\distlin_1\)-\(\mddh_{\GG_1}\) adversary \(\advD_1\) such that
\(|\Pr\left[\mathsf{Game}_{3}(\advA)=1\right]\allowbreak-\Pr\left[\mathsf{Game}_{4}(\advA)=1\right]|\) $\leq
    \mathsf{Adv}_{\distlin_1,\GG_1}(\advD_1).$
\label{lemma:bits2}
\end{lemma}

\begin{proof}
We construct an adversary \(\advD_1\) that receives 
a challenge \(([\vecb{a}]_1,[\vecb{u}]_1)\) of the 
\(\distlin_1\)-\(\mddh_{\GG_1}\) Assumption. From this challenge, \(\advD_1\) defines for each \(\ell\in [m]\) the matrix  \([\matr{G}_\ell]_1\) as the matrix whose  \((j^*-1)2^{\ell-1}+\alpha_\ell\) th column is \([\vecb{u}]_1\), and the rest of the columns are random vectors in the image of \([\vecb{a}]_1\). 
Obviously, when \([\vecb{u}]_1\) is sampled from 
the image of \([\vecb{a}]_1\), \([\matr{G}_\ell]_1\) follows the distribution \(\distlin_1^{n2^{\ell-1},0}\), while if \([\vecb{u}]_1\) is a uniform element of \(\GG^2_1\), \([\matr{G}_\ell]_1\) follows the distribution \(\distlin_1^{n2^{\ell-1},(j^*-1)2^{\ell-1}+\alpha_\ell}\). 
 
The rest of the elements of the CRS are honestly computed. When \([\vecb{u}]_1\) is sampled from the image of \([\matr{a}]_1\), \(\advD_1\) perfectly simulates \(\sfGame_3\), and when \([\vecb{u}]_1\) is uniform, \(\advD_1\) perfectly simulates \(\sfGame_4\), which concludes the proof. 
\end{proof}


\begin{lemma}
There exist adversaries \(\advB_\sfcom\), against the strong soundness of \(\Pi_\sfcom\), \(\advB_\sfsum\), against the soundness of \(\Pi_\sfsum\), and an adversary \(\advB_\mathsf{lin}\) against the soundness of \(\Pi_\mathsf{lin}\), such that \(\Pr[\sfGame_4(\advA)=1]\leq 4/q+ \adv_{\Pi_\sfcom}(\advB_\sfcom)+m\adv_{\Pi_\sfsum}(\advB_\sfsum)+m\adv_{\Pi_\mathsf{lin}}(\advB_\mathsf{lin})\).
\end{lemma}
\begin{proof}
With probability \(1-4/q\), \(\{\vecb{g}_{\ell,(j^*-1)2^\ell-1+\alpha_\ell},\vecb{g}_{\ell,n2^{\ell-1}+1}\}\), \(\ell\in [m]\), and \(\{\vecb{h}_{j^*},\allowbreak \vecb{h}_{m+1}\}\) are bases of \(\Z_q^2\),
and, for each \(\ell\in [m],\mu\in\{1,2\}\), we can define \(\tilde{s}_\ell,\tilde{s}_{\ell,\mu},\tilde{r}_\ell,\tilde{r}_{\ell,\mu},b_{j^*,\ell},\tilde{t}_\ell\) as the unique coefficients in \(\Z_q\) such that \(\vecb{c}_\ell=\allowbreak \tilde{s}_\ell\vecb{g}_{\ell,(j^*-1)2^{\ell-1}+\alpha_\ell} + \tilde{r}_\ell \vecb{g}_{\ell,n2^{\ell-1}+1}, \vecb{c}_{\ell,\mu}=\tilde{s}_{\ell,\mu}\vecb{g}_{\ell,(j^*-1)2^{\ell-1}+\alpha_\ell} + \tilde{r}_{\ell,\mu} \vecb{g}_{\ell,n2^{\ell-1}+1},\) and \(\vecb{d}_\ell= b_{j^*,\ell} \vecb{h}_{j^*} + \tilde{t}_\ell \vecb{h}_{n+1}\).

Recall that if \(\sfGame_4(\advA)=1\) then \(x_{j^*}\notin S\). The adversary can win in \(\sfGame_4\) if one of the following events happen:
\begin{description}
\item[\(E_1\):] the adversary breaks soundness of \(\Pi_\sfcom\) and \(x_{j^*}\neq \tilde{s}_1\),
\item[\(E_2\):] the adversary breaks one of the \(m\)  instances of \(\Pi_\sfsum\) and \(\matr{\Theta}_\ell+\matr{\Pi}_\ell\notin\Span(\mathcal{C}_\ell)\),
\item[\(E_3\):] the adversary breaks one of the \(m\) instances of \(\Pi_\sflin\) and \((\vecb{c}_{\ell+1},\vecb{c}_{\ell,1},\vecb{c}_{\ell,2})\notin\Span(\matr{G}_\mathsf{split}^\ell)\),
\item[\(E_4\):] neither of \(E_1\),\(E_2\), or \(E_3\) happens, but \(x_{j^*}\notin S\) anyway.
\end{description}
By the law of total probabilities, \(\Pr[\sfGame_4(\advA)=1]\leq 4/q+\Pr[E_1]+\Pr[E_2]+\Pr[E_3]+\Pr[E_4]\), and is not hard to see that there exist adversaries \(\advB_\sfcom,\advB_\sfsum,\advB_\mathsf{lin}\) such that \(\Pr[E_1]=\adv_{\Pi_\sfcom}(\advB_\sfcom),\Pr[E_2]=m\adv_{\Pi_\sfsum}(\advB_\sfsum),\) and \(\Pr[E_3]=m\adv_{\Pi_\mathsf{lin}}(\advB_\mathsf{lin})\). Below we will show that \(\Pr[E_4]=0\) (using the same argument used in the non-aggregated case).

We prove by induction on \(\ell\) that \(\tilde{s}_\ell=s_{\alpha}\). If this is the case, the fact that \(\neg E_1\) implies that \(x_{j^*}=\tilde{s}_1=s_{\alpha}\in S\), which finish the proof.

In the base case ($\ell=m$), if we multiply equation (\ref{eq-alog-5}) on the right by a vector \(\vecb{k}\) such that \(\vecb{h}_j^\top\vecb{k}=1\) if \(j=j^*\) and \(0\) if not (which exists since \(\{\vecb{h}_{j^*},\vecb{h}_{n+1}\}\) is a basis of \(\Z_q^2\)), the fact that \(\matr{\Theta}_m+\matr{\Pi}_m\in\Span(\mathcal{C}_m)\), \(\vecb{g}_{m,i}\in\Span(\vecb{g}_{m,(j^*-1)2^{m-1}+1})\) if \(i\neq (j^*-1)2^{\ell-1}+\alpha_\ell\), together with Lemma \ref{lemma:alpha} implies that 
\begin{align*}
\vecb{c}_m &= (1-b_m)\sum_{i=1}^{2^{m-1}}s_i \vecb{g}_{m,i}+b_m\sum_{i=1}^{2^{m-1}}s_{i+2^{m-1}}\vecb{g}_{m,i}\\
&= (1-b_m)s_{\alpha_\ell}\vecb{g}_{m,\alpha_\ell} +b_ms_{\alpha_\ell+2^{m-1}}\vecb{g}_{m,\alpha_\ell}+\tilde{r}_1\vecb{g}_{m,2^{m-1}+1}\\
&= (1-b_m)s_{\alpha-\sfleft+1}\vecb{g}_{m,\alpha_\ell} +b_ms_{\alpha-\sfleft+1+2^{m-1}}\vecb{g}_{m,\alpha_\ell}+\tilde{r}_1\vecb{g}_{m,2^{m-1}+1}\\
&=\begin{cases}
    s_{\alpha-1+1}\vecb{g}_{1,\alpha_\ell}+\tilde{r}_1\vecb{g}_{1,t/2} & \text{ if } b_m=0 \ (\text{and thus }\sfleft=1) \\
    s_{\alpha-(2^{\ell-1}+1)+1+2^{\ell-1}}\vecb{g}_{1,\alpha_\ell}+\tilde{r}_1\vecb{g}_{1,t/2} & \text{ if } b_m=1 \ (\text{and thus }\sfleft=2^{\ell-1}+1) 
\end{cases}
\end{align*}
for some \(\tilde{r}_1\in\Z_q\). In both cases $\vecb{c}_1=s_\alpha\vecb{g}_{1,\alpha_\ell}+\tilde{r}_1\vecb{g}_{1,t/2}$.

In the inductive case we assume that \(\vecb{c}_{\ell+1}=s_{\alpha}\vecb{g}_{\ell+1,(j^*-1)2^\ell+\alpha_{\ell+1}}+\tilde{r}_{\ell+1}\vecb{g}_{\ell+1,n2^\ell+1}\) and we want to show that $\vecb{c}_\ell = s_\alpha\vecb{g}_{\ell,(j^*-1)2^{\ell-1}+\alpha_\ell}+\tilde{r}_\ell\vecb{g}_{\ell,n2^{\ell-1}+1}$. Since \(\vecb{g}_{\ell+1,\alpha_{\ell+1}}\) is linearly independent from the rest of vectors in \(ck_{\ell+1}\), any solution to 
\begin{equation}
\begin{pmatrix}\vecb{c}_{\ell+1}\\\vecb{c}_{\ell,1}\\\vecb{c}_{\ell,2}\end{pmatrix}=\matr{G}^{\ell}_{\mathsf{split}}\vecb{w} \label{eq-G-split}
\end{equation}
is equal to \(s_{\alpha}\) at position \((j^*-1)2^{\ell}+\alpha_{\ell+1}=(j^*-1)2^{\ell-1}+\alpha_\ell+b_\ell2^{\ell-1}\) as depicted below.
\begin{align*}
\pmatri{\vecb{c}_{\ell+1}\\\vecb{c}_{\ell,1}\\\vecb{c}_{\ell,2}}=
\pmatri{
\cdots & \vecb{g}_{\ell+1,(j^*-1)2^{\ell}+\alpha_\ell} & \cdots  & \vecb{g}_{\ell+1,(j^*-1)2^{\ell}+\alpha_\ell+2^{\ell-1}} & \cdots\\
\cdots & \vecb{g}_{\ell,(j^*-1)2^{\ell-1}+\alpha_\ell}     & \cdots  & \vecb{0}                           & \cdots\\
\cdots & \vecb{0}                        & \cdots  & \vecb{g}_{\ell,(j^*-1)2^{\ell-1}+\alpha_\ell}        & \cdots
}
\pmatri{\vdots\\s_\alpha\\\vdots}
\end{align*}
If $b_{\ell}=0$, by Lemma \ref{lemma:alpha}, $\alpha_{\ell+1}=\alpha_\ell$. Therefore, any solution to equation (\ref{eq-G-split})
 is equal to $s_\alpha$ at position $(j^*-1)2^{\ell}+\alpha_\ell$ and thus $\vecb{c}_{\ell,1} = s_\alpha\vecb{g}_{\ell,(j^*-1)2^{\ell-1}+\alpha_\ell}+\tilde{r}_{\ell,1}\vecb{g}_{\ell,n2^{\ell-1}+1}$.
Equation \ref{eq-alog-5} implies that
\begin{align*}
\vecb{c}_{\ell}=&(1-b_\ell)(s_\alpha\vecb{g}_{\ell,(j^*-1)2^{\ell-1}+\alpha_\ell}+\tilde{r}_{\ell,1}\vecb{g}_{\ell,n2^{\ell-1}+1})+b_\ell\vecb{c}_{\ell,2}+y_\ell\vecb{g}_{\ell,n2^{\ell-1}+1}\\
               =& s_\alpha\vecb{g}_{\ell,(j^*-1)2^{\ell-1}+\alpha_\ell}+(\tilde{r}_{\ell,1}+y_\ell)\vecb{g}_{\ell,n2^{\ell-1}+1}.
\end{align*}
If $b_{\ell}=1$, then $\alpha_{\ell+1}=\alpha_\ell+2^{\ell-1}$ and similarly, $\vecb{c}_{\ell}=s_\alpha\vecb{g}_{\ell,(j^*-1)2^{\ell-1}+\alpha_\ell}+(\tilde{r}_{\ell,2}+y_\ell)\vecb{g}_{\ell,n2^{\ell-1}+1}$.

\end{proof}
\subsubsection{Perfect Zero-Knowledge}
Note that the vector the vectors \([\vecb{c}_\ell],[\vecb{c}_{\ell,1}]_1,[\vecb{c}_{\ell,2}]_1,[\vecb{d}_\ell]_2\) and matrices \([\matr{\Theta}_\ell]_1,[\matr{\Pi}_\ell]_2\), \(1\leq\ell\leq m\), output by the prover and the simulator are, respectively, uniform vectors and uniform matrices conditioned on satisfying equation \ref{eq-alog-5}. This follows from the fact that \(ck,ck_1,\ldots,ck_\ell\) are all perfectly hiding commitment keys and that \([\matr{\Theta}_\ell]_1,[\matr{\Pi}_\ell]_1\) are the unique solutions of equation (\ref{eq-alog-5}) modulo the random choice of \(\matr{R}_\ell\). Finally, the rest of the proof follows from Zero-Knowledge of \(\Pi_\sfcom,\Pi_\sfbits,\Pi_\sfsum,\) and \(\Pi_\mathsf{lin}\).

\subsection{The case \(S\subset\GG_1\)} \label{sec:improved-aZKSMP-group-case}
We briefly justify that the case \(S\subset\GG_1\) follows directly from the case \(S\subset\Z_q\) when \(S\) is a fixed witness samplable set. That is, there is a fixed set $S$ for each CRS, and there is an efficient algorithm that samples \(s_1,\ldots,s_{\setsize }\in\Z_q\) such that \(S=\{[s_1]_1,\ldots,[s_{\setsize }]_1\}\). %Note that this is the same case of Section~\ref{sec:bits-applications} where the CRS depends on set.

The reason why is not clear how to compute proofs in this setting is that it requires to compute values of the type \([\vecb{s}_i\vecb{k}]_1\), where \([\vecb{k}]_\mu\), \(\mu\in\{1,2\}\), is a vector from one of the commitment keys. The solution is straightforward: use \(s_1,\ldots,s_{\setsize }\) to compute these values and add them to the CRS (with the consequent CRS growth). Therefore, the new CRS contains also, for each $\alpha\in[n],\ell\in[m],i\in[n],j\neq(i-1)2^{\ell-1}+\alpha_\ell$:
\begin{align*}
s_\alpha[\vecb{g}_{\ell,(i-1)2^{\ell-1}+\alpha_\ell}]_1 \text{ and }
 s_\alpha([\matr{C}^\ell_{(i-1)2^{\ell-1}+\alpha_\ell,j}]_1,[\matr{D}_{(j-1)2^{\ell-1},j}^\ell]_2).
\end{align*}




    \section{Application: Theoretical \(\Theta(\log n)\) Ring Signature} \label{sec:log-ring-signature}
        
        In this section we will describe how the improved aZKSMP can be used to obtain a \emph{theoretical} ring signature with signature size \(\Theta(\log n)\), where \(n\) is the size of the ring, in the standard model (i.e. without \emph{random oracles} nor \emph{non-falsifiable  assumptions}). In other words we show that there exist set-membership proofs of size \(\Theta(\log \setsize )\) even when \(S\subset \GG_1\) is not fixed (that is, prove membership in $\Lang_{ck,\mathsf{set}}^n\subset\GG_s^n$),\footnote{It seems that in the aggregated case this is also true. However, we leave this as an open problem.} and from this proof is direct to construct a ring signature using the techniques of Chandran et al.~\cite{ICALP:ChaGroSah07}.

We say that our construction is theoretical because, although the asymptotic signature size is \(\Theta(\log n)\), we use general results of NP-completeness and thus the ``real'' signature size is \(\Theta(\log n)+\mathsf{poly}(\lambda)\), where \(\lambda\) is the security parameter. Nonetheless, our result should be interpreted as a feasibility result and it poses the interesting question of whether the \(\mathsf{poly}(\lambda)\) part can be reduced to a practical value.

Lets see in a little more detail what is hidden in the asymptotics. When working on bilinear groups, the size of the proof is usually measured in number of group elements, and thus, if the proof is of size $f(n)$ groups elements its size in bits is $|\GG_s|f(n)$. Since $|\GG_s|$ is a linear function of the security parameter (and in practice of the order of one kilobit for $\lambda=128$) it is always ignored. In our case our proof additionally adds a constant number of group elements, with respect to $n$, but for which we only know that is upper bounded by some polynomial of $\lambda$. Therefore, although this part of the proof is independent of $n$, it is misleading to ignore its contribution to the total proof size.
 
We define an encoding function \(\mathcal{E}:\GG_1\cup\GG_2\to\bits^\ell\), where \(\ell=\mathsf{poly}(\lambda)\), which translates group elements to their natural bit encoding. Given a ring \(R=\{[vk_1]_1,\ldots,[vk_n]_1\}\) and \([vk]\in R\), we commit to \(\vecb{x}:=\mathcal{E}([vk]_1)\) using GS commitments and show that \(\mathcal{E}([vk]_1)\in\mathcal{E}(R)\), bit-by-bit, using the improved aZKSMP for \(S\subset \Z_q\). Let \([\vecb{c}]_1:=\GS.\Com_{ck}(\vecb{x}^\top;\matr{R}^\top)=([\vecb{c}_1]||\cdots||[\vecb{c}_\ell])\in\GG_1^{2\times \ell}\) and compute \([\vecb{d}]_1:=\GS.\Com_{ck}([vk]_1;(w_1,w_2)^\top)\), we would like to show that \(\mathcal{E}([vk]_1)=\vecb{x}\).

Given \(\vecb{x}\in\bits^\ell\), there exists a circuit \(C(\vecb{x},\vecb{w}_1,\vecb{w}_2,\vecb{y})\) that interprets \(\vecb{x},\vecb{w}_1,\vecb{w}_2\) as  elements \([x]_1,w_1,w_2\) of \(\GG_1,\Z_q,\Z_q\), respectively, and \(\vecb{y}\) as an element \([\vecb{d}]_1\) of \(\GG_1^2\), computes the bit-string
\begin{align*}
\vecb{s}:=
    &C_{\GG_1,-}(\\
        &\quad C_{\GG_1,-}(\\
            &\qquad C_{\GG_1,-}\left(
                \vecb{y},
                \mathcal{E}\left(
                        ({[x]_1},{[0]_1})^\top\right)\right),\\
            &\qquad C_{\GG_1,\cdot}(
                        \vecb{w}_1,
                        \mathcal{E}([\vecb{u}_1]_1)),\\
        &\quad C_{\GG_1,\cdot}(
                    \vecb{w}_2,
                    \mathcal{E}([\vecb{u}_2]_1)),
\end{align*}
where the circuit $C_{\GG_1,-}:\bits^{2\ell}\to\bits^\ell$ computes the subtraction (inverse and addition computation) of two elements of $\GG_1$, and the circuit $C_{\GG_1,\cdot}:\bits^{\log q+\ell}\to\bits^\ell$ computes the multiplication of an element from $\GG_1$ and an integer in $\Z_q$, and where \(([\vecb{u}_1]||[\vecb{u}_2]_1)=ck\). That is, $C$ is computing an encoding of $[\vecb{d}]_1-([x]_1,[0]_1)^\top-w_1[\vecb{u}]_1-w_2[\vecb{u}_2]$. Finally, $C$ returns 1 iff \(\vecb{s}=\mathcal{E}([0]_1)\). Using the NIZK proof system for circuit satisfiability of Groth et al.~\cite{EC:GroOstSah06} we can prove this statement with a proof of size \(\Theta(|C|)=\mathsf{poly}(\lambda)\), because \(|C|\) only depends on \(q=\poly(\lambda)\) and \(\ell=\poly(\lambda)\).

We conclude that we can prove that \([vk]_1\in R\) with a proof of size \(\Theta(\log n +\poly(\lambda))=\Theta(\log n)\), and thus we can construct a ring signature with signature size \(\Theta(\log n)\)


\chapter{Conclusions}

        In this thesis we constructed many new and more efficient non-interactive  zero-knowledge proofs. We showed that any set of quadratic equations of the type $b(b-1)=0$ and any set of linear equations with variables in $\Z_q$ have a proof whose size is independent of the number of equations. In the case of equations where variables are group elements, we showed that any set of linear equations has a proof whose size is independent of the number of equations, with the drawback that the CRS must be fixed to the specific set of equations.

Then we moved to the case of set-membership proofs, which can be equivalently seen as higher degree equations -- a proof of membership in the set of roots of a polynomial $p$ is a proof that $p(x)=0$ in the case where the variables are $\Z_q$ elements -- and has the advantage that can also be applied to the case where the variables are group elements (since it is not clear how to define higher degree equations where variables are group elements). First, we showed that for any fixed set $S$, the statement $x_1,\ldots,x_n\in S$ can be proven with a proof whose size is linear in the size of the set and but independent of the number of proofs. 
Finally, we considered the cases of non-fixed subsets of $\Z_q$ and (again) the case of fixed subset of $\GG_s$. For both cases we obtained even more efficient proofs. Specifically, we constructed proofs of size logarithmic in the size of the set and independent of the number of proofs.

With these results we constructed more efficient proofs of equal commitment opening, threshold Groth-Sahai proofs, ring signatures, proofs of correctness of a shuffle, and range proofs.

At the heart of most of our results was a new variant of Pedersen commitments and Groth-Sahai commitments, which we call extended multi-Pedersen commitments (MP commitments). MP commitments are length-reducing --they require less than $n$ group elements to commit to a vector in $\Z_q^n$-- which imply that they can not be perfectly binding. However, they can be perfectly binding at one coordinate (encoded in the commitment key) and behave as Groth-Sahai commitments at that coordinate (in fact, when $n=1$ MP commitments become Groth-Sahai commitments).

The major drawback of our results is its limited generality: (essentially) they only allow more efficient proofs for integer equations. When variables and constants are group elements, the results are much more limited. Indeed, they only work in the case of ``fixed equations'' or ``fixed sets'', and they do not apply at all for quadratic pairing product equations. The reason for this limitation is our dependency on MP commitments. Our extensions to group equations (or set-membership in $S\subset\GG_s$) essentially precomputes MP commitments for a fixed set of witness samplable group elements, that is, it is possible to sample the discrete logarithms and thus to compute MP commitments when setting up the CRS. This is of course not enough for general equations where group elements may be adversarially chosen and thus, there is no hope to compute its discrete logarithms. In fact, it can be shown that there does not exist an analogous of MP commitments for group elements. Indeed, Abe et al.~showed that is impossible to construct length-reducing group to group commitments \cite{EC:AbeHarOhk12} -- i.e.~commitment schemes that take $n$ group elements as arguments and return a commitment whose size is $o(n)$ group elements. 

This is in fact a practical limitation. Consider the case of ring signatures, where the central problem is to show that some secret verification key $vk\in R$, and $R$ is the set of all verification keys in the ring. This is just a (non-aggregated) set-membership proof, for which we constructed logarithmic proofs whenever the set is fixed. However, using our set-membership proofs the result is unsatisfactory: there is a single ring $R$ (or a constant number of rings) for which one can construct a logarithmic size ring signatures. This is not a ring signature.
%Consider for example the case of shuffles, where the central problem is to show that the set of input plain-texts is in the set of output plain-texts (this is an oversimplification but it suffices for our explanation) and the plain-text are group elements. One may thus have the hope to build a $\Theta(\log n)$, where $n$ is the number of plain-texts, if the set where we prove membership is fixed (and if we solve the problem that in fact we need to show that the decryption of the cyphertexts are). Then all the possible messages for which we can use the proof system is fixed. This is in fact impractical. Proofs of correctness of a shuffle are typically used within Mix-nets, which are in turn used within anonymous protocols such as ToR. Limiting the set of possible messages 

On the other hand, (quadratic) integer equations are general enough to encode any NP problem (quadratic integer equations can be shown NP-complete). In fact, any circuit $C:\bits^m\to \bits$ can be encoded into a set of quadratic equations which is satisfiable iff $C$ is satisfiable.
One may thus hope that our techniques could help on improving NIZK proofs for Circuit-Sat under falsifiable assumptions. The shortest proofs remains those of Groth et al.~\cite{EC:GroOstSah06}, which are of size within $\Theta(m)+\Theta(|C|)$. Essentially, Groth et al.'s proof computes a perfectly binding commitment to a satisfying assignment, requiring $\Theta(m)$ group elements, and computes perfectly binding commitments to the outputs of each gate and NIZK proofs that the output of each gate is correctly computed, requiring additional $\Theta(|C|)$ group elements. The correctness of the output of each gate is expressed as the satisfiability of an integer equation, so essentially Groth et al.'s proof is $\Theta(m)+\Theta(\#\mathrm{equations})$.
With our techniques we will obtain a proof of size $\Theta(m)+\Theta(|\pi|)$, where the $\Theta(m)$ term comes from commitments to variables and the $\Theta(|\pi|)$ term comes from the proof that those variables satisfy the equations. In this thesis we basically show that for many equations $|\pi|< \#\mathrm{equations}$, so this could be a good indication that we can beat Groth at al's proof. 

We discuss a little more about the generality/non-generality of integer equations. Indeed, we have said that our techniques are limited because they only work for integer equations, but then we pointed out that integer equations are general enough (in fact NP-complete). Can we encode general pairing product equations as integer equations? The answer is affirmative: group operations, pairings, exponentiations, etc.~can be written in terms of quadratic integer equations and thus, any pairing product equation can be written as a polynomial number of quadratic integer equations. Thus we may hope to encode pairing product equations into integer equations and use our results to improve proofs for pairing product equations. We can even use constant-size NIZK proofs for NP from Gennaro et al.~\cite{EC:GGPR13} to construct constant-size proofs of the satisfiability of any set of pairing-product equation.

However, we think that in this way the question is not properly answered. In fact, pairing product equations and Groth-Sahai proofs are usually used in the context of \emph{structure preserving cryptography}, which essentially  means that everything is done through the operations provided by the bilinear groups (essentially in the generic group model). If we reduce pairing product equations to an NP-complete problem, we are transforming group operations and all the group structure to operations over $\Z_q$. Furthermore, even from a practical point of view, the reduction to an NP-complete problem involves an overhead (at least in the prover's complexity) of reducing the instance to the satisfiability of a circuit which may be prohibitive. Therefore, we believe that it is worth to search for more efficient pairing product equations (or impossibility results) in the generic group model.

Finally we comment that there is also much room for optimization in our results. Maybe the worst part of our NIZK proofs that $b_1,\ldots b_n\in\bits$ and set-membership proofs is the quadratic sizes of the CRS. This usually implies that the prover's time complexity is also quadratic, since the CRS defines the set of generators of a vector space and the proof is a linear combination of of the generators. This in general requires to compute a quadratic number of exponentiations ($x[g]$)  and additions ($[g]+[h]$), but when the coefficients that define the witness are in $\bits$ (as in the proof that $b_1\ldots b_n\in\bits$), it only require a quadratic number of additions. We showed that for weight 1 this problem can be avoided, but in general it is an open question if one can do better than this.

Proof sizes are also a good candidate for optimization. Starting from our proofs for linear subspaces of $\GG_1\times\GG_2$, it is interesting to know if there is, even in the generic group model, a shorter proof or a lower bound on the proof size. The same applies for quadratic equations and set-membership proofs.

\iffalse
While set-membership proofs seems quite useful, there are protocols for which they are not enough. Consider the case of shuffles, there one would like to prove that the set of output plaintexts belongs to the set of input plaintexts but without repetitions. 


In Chapter \ref{sec:agg-asym} we constructed more efficient proofs for sets of linear equations over $\Z_q$ and $\GG_s$. Our results extends the result of Jutla and Roy \cite{C:JutRoy14} from one-sided to two-sided equations. We also showed that the same techniques allows to build QA-NIZK constant size proofs of membership in linear subspaces of $\GG_1\times\GG_2$, which extends the constant size QA-NIZK of membership in linear subspaces of $\GG_s$. Finally we showed how to extend the linearly homomorphic structure-Preserving signatures of Libert et al. \cite{EC:LPJY14} from the message space $\GG_s^n$ to $\GG_1^m\times\GG_2^n$.

In Chapter \ref{sec:bits} we constructed a constant-size QA-NIZK proof that a set of commitments open to a bit-string or equivalently, we showed that a set of equations of the type $b(b-1)=0$ has a proof of size linear in the number of variables and independent of the number of equations. In the first part of the chapter we devoted to case where the commitment is perfectly binding and then showed that it allowed more efficient ring signatures and threshold Groth-Sahai proofs. In the second part we devoted to case where the commitment is computationally binding and length-reducing. We introduced an extension of Pedersen commitments and showed that for these commitments we can build a QA-NIZK proof that there exists an opening in $\bits^n$.

In Chapter \ref{sec:shuf-rp} we constructed QA-NIZK proofs that many commitments open to elements in a set, where the proof size is linear in the size of the set and independent of the number of commitments. We then showed how to use this this primitive to construct more efficient proofs of correctness of a shuffle and range proof under standard assumptions.

In Chapter \ref{sec:extras} we showed how to construct a more efficient ring signature and also showed that the results from Chapter \ref{sec:shuf-rp} can be generalized and improved to logarithmic size proofs.

\section{Open Problems}

\subsubsection{Aggregation of any Groth-Sahai Proof}
One of the main problems that remains open is to extend (or give an impossibility result) the efficiency improvements to any kind of equation, that is, construct proofs for a set of pairing product equations of size linear in the number of variables and independent of the number of equations. One indication that this could be impossible is the fact that shrinking group to group commitments do not exists \cite{EC:AbeHarOhk12} and that in our results for quadratic equations over $\Z_q$ we used in a crucial way the multi-Pedersen commitments (which are shrinking commitments). A related or more simple question is whether linear equations over $\GG_s$ allow more efficient proofs without computing an ad-hoc CRS. We note that this case suffer the same problem, since for example the values 
$$\left\{\pmatri{{[a_{1,1}]_1}\\\vdots\\{[a_{1,k}]_1}},\ldots,\pmatri{{[a_{n,1}]_1}\\\vdots\\{[a_{n,k}]_1}}\right\}$$
included in the CRS can be thought as commitment keys of a length reducing commitment scheme (much like multi-Pedersen commitments). The fact that the discrete logarithms of the equation's constants appears multiplied by these commitment keys is to allow to compute a length reducing commitment of the constants of the equation.

The impossibility result of Abe et al. is proven when the key generation and commitment functions are {algebraic}, that is, the group elements are only ``manipulated'' using the group operations and its actual representation is ignored.
A possible avenue to bypass this impossibility is to take in count the actual representation of group elements. For example, we can commit to the binary representation of a group element using multi-Pedersen commitments. The problem now is that the commitment is not homomorphic and it is not straightforward to use use with Groth-Sahai proofs. However, it should be still theoretically possible if we rely on general results for NP-complete languages, as done in Section~\ref{sec:log-ring-signature}.
Indeed, the NP-completeness of quadratic equations over $\Z_q$ allows to translate any set of pairing product equations to a set of quadratic equations over $\Z_q$. For example to translate an equation $e([x]_1,[y]_2)=[0]_T$, we can consider the circuit which computes the function $C_e:\bits^{\ell_1}\times \bits^{\ell_2} \to\bits^{\ell_T}$, where $\ell_s$ is the size of the representation of an element of $\GG_s$, and express the circuit as a set of quadratic equations.

\subsubsection{More efficient proofs of quadratic equations}
Mostly all of our constructions derived from the proofs that many commitments open to a bit-string suffer from the same drawback: the CRS is quadratic in the size of the bit-string, which of course limits their applicability. In Section~\ref{sec:bits-extensions} we showed that when the bit-string is of weight 1 the CRS can be made linear, however this is not always the case and the problem is still open.
%Bootle et al. \cite{EC:BCCGP16} construct a proof system which, although in a different setting, would help to shed lights on how to solve this problem. There, at some of the protocol, it is computed a set of elements which can be arranged matrix such that the $i,j$ th is the coefficient of the monomial $x^{i-j}$. Thereby, all elements at positions $i,j$ such that $i-j=k$ are coefficients of the same monomial $x^k$. Despite this very informal description, we can apply this idea as follows: compute commitments keys $[\vecb{g}_i],[\vecb{h}_j]_2$ in such a way that $\vecb{g}_i\vecb{h}_j=\vecb{g}_{k+i} \vecb{h+k}^\top$ .
 
\subsubsection{Theoretical $\Theta(\log n)$ Shuffles}
It is interesting to explore if similar techniques to those from Section~\ref{sec:log-ring-signature} allow to construct a $\Theta(\log n)$ proof of correctness of a shuffle. The improved aZKSMP should allow to almost prove that a set of ciphertexts are a correct shuffle. We say almost, because the aZKSMP only allows to show that each output-plaintext is in the set of input-plaintexts and thus there might be repeated elements. The challenge is then to show that the aZKSMP do not contain repeated indices.

\subsubsection{Extend results for high-degree equations to circuits}
Gentry and Wichs showed that any sublinear NIZK proof must rely on non-falsifiable  assumptions if the language is ``hard'' (say an NP-hard language) \cite{STOC:GenWic11}. This result seems to rule out any sublinear NIZK proof if the assumption is any ``DDH-like'' assumption. However, many cryptographically interesting languages are seem not to be NP-hard. For example, the language of vectors in a subspace of $\GG_1$ seems not be NP-hard and has constant size proofs. In fact, what we believe that is happening is that the language is in fact ``easy'' given the appropriate trapdoor (in this case a basis of the kernel of the generating matrix) while an NP-hard language should not have such a trapdoor.

In general, in many languages the statement is the commitment to some value and we want to prove some assertion about the opening. Given a circuit $C$ and a commitment key $ck$, we define the following language
$$
\Lang_{ck,C}:= \{([\vecb{c}_1]_1,\ldots,[\vecb{c}_n]_1,a)\in\GG_1^2\times\bits^k:= \exists [x_i]_1\in\GG_1,w_i\in\Z_q \text{ s.t. } [\vecb{c}_i]_1=\GS.\Com_{ck}([x_i]_1;w_i)\text{ and }C([x]_1,a)=1\}.
$$
Many interesting problems can be instantiated in this way: in ring signatures $[x]_1$ is the verification key, $a$ the description of the ring, and $C$ outputs 1 iff $[x]_1$ is in the ring; b) in a shuffle $[\vecb{c}_1]_1,\ldots,[\vecb{c}_n]_1$, $n=2m$, are cyphertexts and $C([x]_1,\ldots,[x_n]_1)=1$ iff the first $m$ openings are a permutation of the las $m$ openings; c) in Groth-Sahai proofs the openings are the variables and the $C$ outputs 1 iff the variables satisfies the equations. In fact, the only interesting problem which is not of this type is a range proofs and is an interesting questions if there are other problems not of this type.

Using techniques from \cite{EC:GroOstSah06} one can construct NIZK proofs for $\Lang_{ck,C}$ of size $\Theta(|C|)=\Theta(n+|a|)$, since $|C|\leq n+|a|$, which boils down to compute linear shuffles or ring signatures. We have seen on Section~\ref{sec:log-ring-signature} that there exists ring signatures of size $\Theta(\log n)$, combining our results for aZKSMP and the same ``reduction to circuit'' approach. We believe that this is an indication that more efficient proofs for languages $\Lang_{ck,C}$ exists.

\fi



\appendix

\chapter{Dummy Appendix}

You can defer lengthy calculations that would otherwise only interrupt
the flow of your thesis to an appendix.


\backmatter

\bibliographystyle{alpha}
\bibliography{cryptobib/abbrev3,cryptobib/crypto,manualbib}

%\includepdf[pages={-}]{declaration-originality.pdf}

\end{document}
